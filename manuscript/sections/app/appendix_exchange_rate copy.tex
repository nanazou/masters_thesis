% !TeX root = ../../main.tex
% app/app_exchange_rate.tex

\chapter{割引因子ショックを含む為替レート決定式の導出}
\label{app:exchange_rate}

本稿の第5章における考察では、自国と外国で家計の時間選好(割引因子 \(\beta\))が異なる場合、為替レートの決定式に将来の \(\beta\) の格差が累積的に影響することを論じた。本付録では、その数理的な導出過程を、統計学的な恒等式と対数線形近似の性質、および前方展開(Forward Solving)の手法に焦点を当てて詳述する。

\section*{1. 基礎方程式:2つのオイラー方程式の比}

出発点として、代表的家計の最適化条件(オイラー方程式)を用いる。

\begin{itemize}
    \item \textbf{自国家計のオイラー方程式(\eqref{eq:final_euler_H}):}
    \begin{equation}
    \lambda_t^H = \beta_t^H (1+i_t^H) E_t [\lambda_{t+1}^H]
    \tag{\ref{eq:final_euler_H}}
    \end{equation}
    これに定義式 \(\lambda_t^{H/*} \equiv e_t^{/*} \lambda_t^H\) および一物一価の法則(UIP条件の基礎)を適用すると、自国家計による外貨評価式は以下のように記述できる。
    \begin{equation}
    \lambda_t^{H/*} = \beta_t^H (1+i_t^F) E_t [\lambda_{t+1}^{H/*}]
    \label{eq:app_euler_H_star}
    \end{equation}

    \item \textbf{外国家計のオイラー方程式(\eqref{eq:final_euler_F}):}
    \begin{equation}
    \lambda_t^{F/*} = \beta_t^F (1+i_t^F) E_t [\lambda_{t+1}^{F/*}]
    \tag{\ref{eq:final_euler_F}}
    \end{equation}
\end{itemize}

ここで、自国家計と外国家計による外貨評価の比率を \(\Psi_t\) と定義する。
\begin{equation}
\Psi_t \equiv \frac{\lambda_t^{H/*}}{\lambda_t^{F/*}}
\label{eq:psi_def}
\end{equation}
式 \eqref{eq:app_euler_H_star} と本文 \eqref{eq:final_euler_F} の辺々を割ると、共通項である金利 \((1+i_t^F)\) が相殺され、以下の関係が得られる。
\begin{equation}
\Psi_t = \frac{\beta_t^H}{\beta_t^F} \frac{E_t [\lambda_{t+1}^{H/*}]}{E_t [\lambda_{t+1}^{F/*}]}
\label{eq:psi_ratio_raw}
\end{equation}

\section*{2. 期待値の比の線形近似}

式 \eqref{eq:psi_ratio_raw} の右辺にある「期待値の比」を、計算可能な「比の期待値」へと変換する。
\begin{equation*}
\frac{E_t [\lambda_{t+1}^{H/*}]}{E_t [\lambda_{t+1}^{F/*}]} \approx E_t \left[ \frac{\lambda_{t+1}^{H/*}}{\lambda_{t+1}^{F/*}} \right]
\end{equation*}
この近似が対数線形近似(1次近似)のモデルにおいて正当化される理由を、一般的な確率変数の性質に基づいて数学的に証明する。

\subsection*{一般論としての変数の定義}
ある任意の確率変数 \(X_t\) について、その期待値(平均)を \(\bar{X} \equiv E[X_t]\) とし、期待値からの対数乖離(変化率)を \(\hat{x}_t\) と定義する。
\begin{equation*}
\hat{x}_t \equiv \frac{X_t - \bar{X}}{\bar{X}} \quad \Longleftrightarrow \quad X_t = \bar{X}(1+\hat{x}_t)
\end{equation*}
本稿のような線形近似モデルでは、この乖離 \(\hat{x}_t\) を「1次の微小量」として扱い、その2乗以上(\(\hat{x}_t^2\) や \(\hat{x}_t \hat{y}_t\))を無視できるほど小さい項(\(\approx 0\))とみなして計算を行う。

\subsection*{ステップ1:積の期待値の近似(共分散項の評価)}

一般に、2つの確率変数 \(X, Y\) の積の期待値は、「期待値の積」と「共分散(Covariance)」の和に分解できる。
\begin{equation*}
E_t[XY] = E_t[X]E_t[Y] + \text{Cov}_t(X, Y)
\end{equation*}
ここで、共分散の項を上記の定義に従って展開する。
\begin{align*}
\text{Cov}_t(X, Y) &= E_t \left[ (X - \bar{X}) (Y - \bar{Y}) \right] \\
&= E_t \left[ \left( \bar{X}(1+\hat{x}) - \bar{X} \right) \left( \bar{Y}(1+\hat{y}) - \bar{Y} \right) \right] \\
&= \bar{X}\bar{Y} E_t \left[ \hat{x}\hat{y} \right]
\end{align*}
ここで、右辺の期待値の中身は、微小変動同士の積(\(\hat{x} \times \hat{y}\) のオーダー)となっている。1次近似の枠組みでは、1次の微小量同士の積は2次の微小量となるため、無視することができる(\(\approx 0\))。したがって、以下の近似が成立する。
\begin{equation}
\text{Cov}_t(X, Y) \approx 0 \quad \Longrightarrow \quad E_t[XY] \approx E_t[X]E_t[Y]
\label{eq:approx_product}
\end{equation}

\subsection*{ステップ2:逆数の期待値の近似(分散項の評価)}

次に、変数 \(Z\) の逆数の期待値 \(E_t[1/Z]\) を考える。一般に、イェンゼンの不等式により \(E[1/Z] \neq 1/E[Z]\) であるが、その乖離幅は分散に依存する。
関数 \(f(Z) = 1/Z\) を、期待値 \(\bar{Z} = E_t[Z]\) の周りで2次までテイラー展開し、期待値をとる。
\begin{align*}
E_t\left[\frac{1}{Z}\right] &\approx E_t \left[ \frac{1}{\bar{Z}} - \frac{1}{\bar{Z}^2}(Z - \bar{Z}) + \frac{1}{\bar{Z}^3}(Z - \bar{Z})^2 \right] \\
&= \frac{1}{\bar{Z}} - \frac{1}{\bar{Z}^2}\underbrace{E_t[Z - \bar{Z}]}_{0} + \frac{1}{\bar{Z}^3}\underbrace{E_t[(Z - \bar{Z})^2]}_{\text{分散 Var}(Z)} \\
&= \frac{1}{E_t[Z]} + \frac{1}{(E_t[Z])^3} \text{Var}_t(Z)
\end{align*}
ここで第2項の分散 \(\text{Var}_t(Z)\) を展開すると、
\begin{equation*}
\text{Var}_t(Z) = \bar{Z}^2 E_t \left[ \hat{z}^2 \right]
\end{equation*}
なり、これも微小変動の2乗(\(\hat{z}^2\) のオーダー)を含むため、2次の微小量である。
したがって、1次近似においては分散項を無視することができ、以下の近似式が成立する。
\begin{equation}
E_t\left[\frac{1}{Z}\right] \approx \frac{1}{E_t[Z]}
\label{eq:approx_inverse}
\end{equation}

\subsection*{結論:比の期待値への変換}

式 \eqref{eq:approx_product} と 式 \eqref{eq:approx_inverse} の結果を組み合わせることで、期待値の比を比の期待値として扱うことが正当化される。
\begin{align*}
E_t \left[ \frac{X}{Y} \right] &= E_t \left[ X \cdot \frac{1}{Y} \right] \\
&\approx E_t [X] \cdot E_t \left[ \frac{1}{Y} \right] \quad (\because \text{共分散項} \approx 0) \\
&\approx E_t [X] \cdot \frac{1}{E_t [Y]} \quad (\because \text{分散項} \approx 0) \\
&= \frac{E_t [X]}{E_t [Y]}
\end{align*}

\section*{3. 将来に向けた前方展開と極限}

前節で示した一般論を本稿のモデルに適用する。本稿では、変数は定常状態(Steady State)の近傍で変動すると仮定しているため、変数の期待値(平均)\(\bar{X}\) は定常状態の値 \(X_{ss}\) と等しい(\(X_{ss} = E_t[X_t]\))。
この前提の下、式 \eqref{eq:psi_ratio_raw} の右辺に近似を適用すると、\(\Psi_t\) に関する以下の再帰式が得られる。
\begin{equation}
\Psi_t = \frac{\beta_t^H}{\beta_t^F} E_t \left[ \frac{\lambda_{t+1}^{H/*}}{\lambda_{t+1}^{F/*}} \right] = \frac{\beta_t^H}{\beta_t^F} E_t [\Psi_{t+1}]
\label{eq:psi_recursive}
\end{equation}

この再帰式を将来に向かって解き(Forward Solving)、現在の比率 \(\Psi_t\) を決定する。

\paragraph{① 再帰的代入と一般形の導出}
式 \eqref{eq:psi_recursive} の右辺にある \(\Psi_{t+1}\) に対して、1期先の関係式 \(\Psi_{t+1} = \frac{\beta_{t+1}^H}{\beta_{t+1}^F} E_{t+1}[\Psi_{t+2}]\) を代入する。この操作を \(k\) 回繰り返すと、以下の一般形が得られる。
\begin{equation}
\Psi_t = E_t \left[ \left( \prod_{j=0}^{k} \frac{\beta_{t+j}^H}{\beta_{t+j}^F} \right) \Psi_{t+k+1} \right]
\label{eq:psi_iterative}
\end{equation}

\paragraph{② 極限の適用}
次に、式 \eqref{eq:psi_iterative} の両辺について \(k \to \infty\) の極限をとる。
期待値の一般的な性質として、確率変数が有界であるなどの適切な条件下では、極限操作 \(\lim\) を期待値演算子 \(E_t\) の中に入れることができる(有界収束定理)。経済モデルにおいては、変数が定常状態近傍で安定的に推移することを前提とするため、この操作は正当化される。

したがって、
\begin{equation*}
\Psi_t = E_t \left[ \lim_{k \to \infty} \left( \left( \prod_{j=0}^{k} \frac{\beta_{t+j}^H}{\beta_{t+j}^F} \right) \Psi_{t+k+1} \right) \right]
\end{equation*}
となる。ここで、経済モデルの安定性条件として、無限遠方においてショックの影響は消失し、変数はショック後の新たな定常状態へ収束すると仮定する。すなわち、評価の比率 \(\Psi_{t+k+1}\) は、定常状態における比率 \(\Psi_{\infty}\) へと収束する。
\begin{equation*}
\lim_{k \to \infty} \Psi_{t+k+1} = \Psi_{\infty}
\end{equation*}
この \(\Psi_{\infty}\) は定数であるため、期待値および無限乗積の外に出すことができる。以上より、現在の \(\Psi_t\) は以下の形で確定する。
\begin{equation}
\Psi_t = E_t \left[ \prod_{j=0}^{\infty} \frac{\beta_{t+j}^H}{\beta_{t+j}^F} \right] \times \Psi_{\infty}
\label{eq:psi_final_solution}
\end{equation}

\section*{4. 結論:為替レートの決定式}

最後に、為替レートの定義式 \(e_t^{/*} = \lambda_t^{H/*} / \lambda_t^H\) に、\(\lambda_t^{H/*} = \Psi_t \lambda_t^{F/*}\) を代入する。
さらに、以下の関係式(消費のFOCおよびCPIの定義)を用いる。
\begin{equation*}
\lambda_t^H = \frac{1}{p_t^{H \to W} c_t^{H \to W}}, \quad \lambda_t^{F/*} = \frac{1}{p_t^{F \to W*} c_t^{F \to W}}
\end{equation*}
これらを代入して整理すると、最終的な為替レート決定式が得られる。

\begin{align}
e_t^{/*} &= \Psi_t \times \frac{\lambda_t^{F/*}}{\lambda_t^H} \nonumber \\
&= \Psi_t \times \frac{1 / (p_t^{F \to W*} c_t^{F \to W})}{1 / (p_t^{H \to W} c_t^{H \to W})} \nonumber \\
&= \underbrace{E_t \left[ \prod_{j=0}^{\infty} \frac{\beta_{t+j}^H}{\beta_{t+j}^F} \right]}_{\text{資本移動要因}} \times \underbrace{\frac{p_t^{H \to W} c_t^{H \to W}}{p_t^{F \to W*} c_t^{F \to W}}}_{\text{貿易均衡要因}} \times \Psi_{\infty}
\label{eq:app_exchange_rate_final}
\end{align}

ここで、第1項の「資本移動要因」について、定常状態における割引因子の設定が決定的な役割を果たす。

\begin{itemize}
    \item \textbf{定常状態で割引因子が等しい場合(\(\beta_{ss}^H = \beta_{ss}^F\)):}
    本稿の基本設定である。一時的なショック(需要ショック等)により \(\beta_t\) が変動しても、長期的には元の水準に戻る(平均回帰する)ため、無限乗積項は有限の値に収束する。このとき、一時的な \(\beta_t^H\) の上昇は、自国通貨安(\(e_t^{/*}\) 上昇)をもたらす。
    
    特に、全期間において \(\beta_t^H = \beta_t^F\) であるならば(例:生産性ショック分析など)、式 \eqref{eq:app_exchange_rate_final} の無限乗積項(資本移動要因)は完全に 1 となり消滅するため、為替レートは純粋に両国の名目支出比率のみで決定される。
    \begin{equation}
    e_t^{/*} = \Psi_{\infty} \times \frac{p_t^{H \to W} c_t^{H \to W}}{p_t^{F \to W*} c_t^{F \to W}} 
    \label{eq:exchange_rate_simplified}
    \end{equation}
    
    \item \textbf{定常状態で割引因子が異なる場合(\(\beta_{ss}^H \neq \beta_{ss}^F\)):}
    (本稿では扱わないが)もし恒久的に \(\beta\) が異なると仮定すると、無限乗積項は発散または消失し、安定的な均衡が存在しなくなる。したがって、本モデルのような無限期間モデルにおいて安定解を得るためには、定常状態において両国の割引因子が一致していることが前提条件となる。
\end{itemize}

\section*{5. 資源制約式と対外純資産の恒等的なゼロ均衡}

上記の為替レート決定メカニズムが、対外純資産(\(b_t^H\))の動学に与える影響を確認する。ここで、以下の2点を仮定する。

\begin{itemize}
    \item[①] 全期間において自国と外国の割引因子が等しい(\(\beta_t^H = \beta_t^F\))。
    \item[②] 初期時点およびそれ以前の対外純資産がゼロである(\(b_s^H = b_{s-1}^H = 0\))。
\end{itemize}

仮定①より、全期間において \(\beta_t^H = \beta_t^F\) であるため、\eqref{eq:exchange_rate_simplified} が成立する。

自国の資源制約式(\eqref{eq:final_resource_constraint_H})において、ショック発生前の \(t = s - 1\) の時点を考える。
\begin{equation}
p_{s-1}^{H \to W} c_{s-1}^{H \to W} + b_s^H = p_{s-1}^H y_{s-1}^H + (1+i_{s-2}^F) \frac{e_{s-1}^{/*}}{e_{s-2}^{/*}} b_{s-1}^H
\label{eq:app_resource_constraint_init}
\end{equation}
ここに仮定②(\(b_s^H = b_{s-1}^H = 0\))を代入すると、以下の関係が導出される。
\begin{equation}
p_{s-1}^{H \to W} c_{s-1}^{H \to W} = p_{s-1}^H y_{s-1}^H
\label{eq:app_trade_balance_init}
\end{equation}
これは、初期時点において貿易収支が均衡していたことを意味する。

次に、名目GDPの方程式を導出する。出発点として、自国の一人当たり財市場均衡条件(\eqref{eq:final_goods_market_eq_H})を用いる。
\begin{equation}
y_t^H = c_t^{H \to H} + \frac{M}{N} c_t^{F \to H}
\tag{\ref{eq:final_goods_market_eq_H}}
\end{equation}
この両辺に自国財価格 \(p_t^H\) を乗じると、名目GDPは各主体の需要内訳として以下のように展開される。
\begin{align*}
p_t^H y_t^H &= p_t^H c_t^{H \to H} + p_t^H \frac{M}{N} c_t^{F \to H} \\
&= p_t^H \left( \alpha^H \frac{p_t^{H \to W}}{p_t^H} c_t^{H \to W} \right) + p_t^H \frac{M}{N} \left( (1 - \alpha^F) \frac{p_t^{F \to W*}}{p_t^{H*}} c_t^{F \to W} \right) \\
&= \alpha^H (p_t^{H \to W} c_t^{H \to W}) + \frac{M}{N} (1 - \alpha^F) \frac{p_t^H}{p_t^{H*}} (p_t^{F \to W*} c_t^{F \to W})
\end{align*}
ここで、一物一価の法則 \(p_t^H = e_t^{/*} p_t^{H*}\) (\eqref{eq:lop})および、人口比 \(N=M=1\) の設定を適用すると、以下の名目GDPの方程式が得られる。
\begin{equation}
p_t^H y_t^H = \alpha^H(p_t^{H \to W} c_t^{H \to W}) + (1 - \alpha^F) e_t^{/*} (p_t^{F \to W*} c_t^{F \to W})
\label{eq:identity_gdp_consumption}
\end{equation}

この \eqref{eq:identity_gdp_consumption} に、為替レート決定式 \eqref{eq:app_exchange_rate_final} を代入して整理すると、以下の関係が得られる。
\begin{align*}
p_t^H y_t^H &= \alpha^H(p_t^{H \to W} c_t^{H \to W}) + (1 - \alpha^F) \left( \Psi_{\infty} \frac{p_t^{H \to W} c_t^{H \to W}}{p_t^{F \to W*} c_t^{F \to W}} \right) (p_t^{F \to W*} c_t^{F \to W}) \\
&= \alpha^H(p_t^{H \to W} c_t^{H \to W}) + (1 - \alpha^F) \Psi_{\infty} (p_t^{H \to W} c_t^{H \to W}) \\
&= \left[ \alpha^H + (1-\alpha^F)\Psi_{\infty} \right] p_t^{H \to W} c_t^{H \to W}
\end{align*}

経常収支均衡の要請 \eqref{eq:app_trade_balance_init} から、上式の係数が \(1\) でなければならない。
\[
1 = \alpha^H + (1-\alpha^F)\Psi_{\infty}
\]
これより、定常状態パラメータ \(\Psi_{\infty}\) は以下の一意な値として決定される。
\begin{equation}
\Psi_{\infty} = \frac{1-\alpha^H}{1-\alpha^F}
\label{eq:psi_infty_unity}
\end{equation}

\(\Psi_{\infty}\) は定数であるため、この値はショック後の \(t \ge s\) においても変化しない。したがって、名目GDPと名目消費の間には全期間を通じて以下の等価性が成立する。
\begin{equation}
p_t^H y_t^H = 1 \times p_t^{H \to W} c_t^{H \to W}
\label{eq:nominal_identity_unity}
\end{equation}

最後に、式 \eqref{eq:nominal_identity_unity} を自国の資源制約式(\ref{eq:final_resource_constraint_H})に代入すると、以下の資産蓄積に関する差分方程式が得られる。
\begin{equation}
b_{t+1}^H = (1+i_{t-1}^F) \frac{e_t^{/*}}{e_{t-1}^{/*}} b_t^H
\label{eq:asset_accumulation_final}
\end{equation}

初期条件 \(b_s^H = 0\) を用いると、帰納的にすべての \(t \ge s\) において \(b_{t+1}^H = 0\) であることが証明される。以上により、本モデルの Cole-Obstfeld 条件下では、為替レートが調整されることで常に貿易収支が均衡し、対外純資産は恒等的にゼロとなることが数学的に裏付けられた。