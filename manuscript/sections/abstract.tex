% !TeX root = ../main.tex
% sections/abstract.tex

本稿はゼロ金利に陥った経済を最も効果的に回復させる金融政策は名目総消費水準目標であると主張する。
名目総消費水準目標が最も効果的である理由は、過去の目標未達分は将来必ず埋め合わせるとの約束のもと
所得の限界効用の逆数である名目総消費を目標とすることで所得の限界効用をきわめて効率的に低下させ、
それにより貯蓄から消費への流れを強く促すためである。

所得の限界効用は追加1単位の貨幣で消費をおこなうことにより得られる効用である。
したがって将来における所得の限界効用が小さくなればなるほど
将来において1単位の貨幣で消費をおこなうことにより得られる効用も小さくなる。
そして(消費から得られる効用が消費の対数関数であるとき)所得の限界効用は
名目総消費の逆数となることが知られている。
よって将来において名目総消費を上昇させることすなわち所得の限界効用を減少させることを中央銀行が保証するならば、
家計は貯蓄して将来の消費から効用を得るよりも現在の消費を増やして効用を得る方が
より大きな生涯効用を得ることができる。

名目総消費水準目標は過去の目標未達分は将来必ず埋め合わせるとの約束により
人々の期待に強く働きかけようとする水準目標の一種である。
2008年の世界金融危機以降、深刻な景気後退に見舞われた各国の中央銀行は利下げを推し進めたが、
ついにはゼロ金利に達することとなり政策の有効性は大きく損なわれた。
ゼロ金利制約により利子率はゼロより下がることはなくなり
利子率をさらに引き下げて景気を刺激するという直接的な手段が失われたのである。
そこで残されたのは政策が人々の期待に働きかける効果であるが、伝統的な金融政策はこの期待形成力が弱いとされ、
この点で優位性のある水準目標に注目が集まった。
そうした水準目標の中でも名目GDP水準目標は少なからぬ学者に支持されその有効性がいくつかの研究で示されてきた。

では名目GDP水準目標よりも経済を効率的に回復させる金融政策はないのだろうか。
金融政策により経済を回復させるには名目利子率と将来の所得の限界効用を低下させ総需要曲線を上方に動かす必要がある。
伝統的なインフレ目標はインフレ率を目標とすることでこれをおこなってきた。
将来のインフレ率が上昇すると将来の物価が上昇するため、将来の名目総消費が増加する。
つまり名目総消費を構成する物価と消費のうち物価を上昇させることにより
将来の名目総消費を上昇させその逆数である将来の所得の限界効用を引き下げようとしたのだ。
これに対して名目総消費水準目標は名目総消費全体を目標とする。
こうすることにより将来の名目総消費全体がより効果的に押し上げられ
将来の所得の限界効用もより効率的に低下するのである。

この主張を検証するため本稿においては\textcite{Woodford1996}や\textcite{CespedesChangVelasco2002}
などを参考に自国と外国からなる開放経済モデルを構築する。
自国と外国においてはそれぞれの国内債券市場が完備であるとし、
これにより一国は代表的家計と呼ばれる仮想的な家計に代表され数学的分析が容易になる。
一方で自国と外国のあいだの国際債券市場は不完備であると仮定する。
このことにより国際債券の動きを通じた非対称な二国間の関係を分析することが可能となる。
このモデルにおいて、自国家計が我慢強くなるショック、
すなわち自国家計が将来を悲観することにより消費よりも貯蓄を好むようになるショックを与える。
このショックにより自国経済は深刻な不況とゼロ金利に陥るが、
その初期の落ち込みはどの程度か、またそこからの上昇の様子はどのようであるかを
名目総消費水準目標を含む11の金融政策について観察する。
このシミュレーションの結果、
名目総消費水準目標においては消費と生産の初期の落ち込みが最も小さく回復も最も早いことが示される。
また所得の限界効用についても初期の上昇が最も小さく下落も最も速やかであることが確認される。
その結果として家計の効用に基づいて計算された厚生の落ち込みも名目総消費水準目標において最小となる。

以上の分析により、本稿は名目総消費水準目標が
ゼロ金利下の経済を最も効果的に回復させる金融政策であると結論付ける。
