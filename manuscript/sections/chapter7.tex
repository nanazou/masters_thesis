% !TeX root = ../main.tex
% sections/chapter6.tex

\chapter{結論}
\label{chap:conclusion}
本稿はゼロ金利制約に直面した経済において名目総消費水準目標がいかに効率的に家計の期待に働きかけ
経済を迅速に回復させるかを負の \( \beta \) (主観的割引因子)ショックのシミュレーションにより確認した。
そこにおいて名目消費水準目標は名目利子率 \( i_t^H \) と
所得の期待限界効用 \( E_0[\lambda_t^H] \) に最も効果的に下落させ厚生の落ち込みを最小限にとどめた。
また負の \( a \) (生産性)ショックに対しても名目総消費水準目標の頑健性が示された。
本稿の物価硬直性の仮定においては \( a \) (生産性)ショックが起こっても物価と生産はほとんど動かない。
このとき名目総消費目標は静観の姿勢をとることで経済に余計な刺激を与えなかったのである。


実務面では指標集計と政策意図の浸透について名目総消費水準目標の優位性を論じた。
集計面においては、指標としての名目総消費は消費者物価指数や潜在生産に比べて正確な予測が容易であり、
また速報性の高い数値が既に提供されている点も適時適切な政策判断を下す上で大きな利点となる。
正確性と速報性の両立については現時点で依然として課題が残るものの、
決済データのリアルタイム集計やビッグデータの活用といったデジタル技術の進展により
段階的に解消されることが期待される。
くわえて家計にとって馴染み深い消費を目標に据えることは
中央銀行の政策意図を広く浸透させ円滑な期待形成を図る上で極めて有効である。


今後の本研究の課題としては効用関数の一般化、家計の異質性の導入、および経済の硬直性の緩和の3点が挙げられる。
効用関数の一般化や家計の異質性の導入はより多様な家計からなる現実的な経済の分析を可能にする。
また本稿の分析は物価が硬直的でありショックの持続性が強いという硬直的な自国の仮定にもとづいていた。
これは自国を日本と想定し設定したものであったが、
この仮定を緩めたときに依然として名目総消費水準目標の優位性が保たれるかの検証は今後の課題となる。
