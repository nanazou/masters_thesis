% !TeX root = ../../main.tex
% sections/chap4/sec_preparation_method.tex

\section{シミュレーション手法}
\label{sec:preparation_method}

本稿のシミュレーションではDynare 6.3を使用する。
Dynareを用いることによりショックに対する各変数の動的な反応をグラフとデータにより出力することができる。
またゼロ金利制約の再現のために\textcite{GuerrieriIacoviello2015}によって開発されたライブラリOccbinを用いる。
しかし初めに開発されたOccbinはDynare 6.3に対応していないため、
Johannes Pfeifer氏が最新のDynareとの互換性を維持するために修正した更新版リポジトリOccbin\_updateを使用する。

\subsection{Dynareによる数値計算}
\label{subsec:preparation_dynare}
シミュレーションにあたり3章の動学方程式系、定常状態、パラメータおよびショックをDynareに与える。
するとDynareはショックによって定常状態から飛ばされた経済が
動学方程式系をみたしながら元の定常状態へ戻ってくる様子をグラフに描いてくれる。
ここではこのシミュレーションの原理を直観的に説明する
動学方程式系 \( \mathbf{f} \) について、
定常状態 \( \mathbf{x}_{ss} \) はショックがない \( \mathbf{\epsilon} = 0 \) ときに
経済が永続的に留まる点として以下を満たすものと定義される。
\[ \mathbf{f}( \mathbf{x}_{ss}, \mathbf{x}_{ss}, 0 ) = 0 \]
この定常状態においてヤコビ行列が正則であれば陰関数定理によって点
\( ( \mathbf{x}_{t+1}, \mathbf{x}_t, \mathbf{\epsilon}_t ) = 
( \mathbf{x}_{ss}, \mathbf{x}_{ss}, 0 ) \)
の近傍において
\[ \mathbf{x}_{t+1} = \mathbf{h}( \mathbf{x}_t, \mathbf{\epsilon}_t ) \]
というベクトル値関数 \( \mathbf{h} \) の存在が数学的に保証される。
また定常状態の定義から \( \mathbf{x}_{ss} = \mathbf{h}( \mathbf{x}_{ss}, 0 ) \) という関係が成立する。
この関係を直観的に理解するため、経済の状態を一変数のスカラー \( x_t \) と仮定し、
ショックの値を固定した関数 \( h_{\epsilon}( x_t ) \equiv h( x_t, \epsilon ) \) を定義して
 \( (x_t, x_{t+1}) \) 平面上でその挙動を考察する。
ショックがない定常状態において、経済は \( x_{ss} = h_0( x_{ss} ) \) を満たす点に静止している。
これは2次元平面上では \( h_0 \) が45度線と交差している地点に相当する。
\( t=0 \) においてショック \( \epsilon_0 \) が発生すると、
グラフが \( h_{\epsilon_0} \) へと一時的に移動し、経済は定常状態 \( x_{ss} \) から
移動したグラフ上の点 \( x_0 = h_{\epsilon_0}( x_{ss} ) \) へとジャンプする。
ショックが収まる \( t \ge 1 \) 以降、グラフは元の \( h_0 \) に戻る。
経済は \( x_0 \) という地点から、以下の帰納的なプロセスによって定常状態へと戻っていく。
\[ x_1 = h_0( x_0 ), \quad x_2 = h_0( x_1 ), \quad \dots \]
この反復計算の結果 \( x_t \) は \( x_{ss} \) へと収束する。
この階段状の移動の軌跡を時間軸に沿ってプロットしたものがインパルス応答関数である。
実用上の計算においてはこの曲線 \( h_0 \) そのものを非線形のまま直接求めることは困難である。
そのため定常状態 \( (x_{ss}, x_{ss}) \) において曲線 \( h_0 \) に接する接線を導出する1次近似をおこなう。
この接線の傾きが1より小さければ経済は定常状態へと収束する。
ここで多次元の一般論へと議論を広げると以下の数理的な課題が生じる。
まず非線形なベクトル値関数 \( \mathbf{h} \) を厳密に特定することは実用上不可能である。
さらに9本の方程式系を満たしながら定常状態へ向かう関数 \( \mathbf{h} \) は数学的に複数存在する可能性があり、
その場合、定常状態の周りにおける1次近似( 接線の傾き )も一意に定まらない。
そのため1次近似を一意に決定するためのBlanchard-Kahn( BK )条件がある。
これは経済のジャンプ変数の数と
係数行列(接線の傾きに相当する)の固有値のうち1より大きいもの( 不安定根 )の数が
一致することを要求する条件である。
この条件が満たされるとき、発散する経路を排除し定常状態へと収束する唯一のサドルパス安定解として
接平面の傾きが一意に確定する。
シミュレーションにおいて、経済はこの確定した接線( 接平面 )の上を
ショックによって決まった初期値から帰納的に移動していくことで最終的な動学パスが算出されるのである。

\subsection{Occbinの役割}
\label{subsec:preparation_occbin}
前述の関数 \( \mathbf{h} \) は経済の自然な移動を記述するものであるが、
ゼロ金利制約は特定の区間においてその移動を壁のように遮断してしまう。
したがってすべての区間において \( \mathbf{h} \) により描写することは不可能である。
そこでOccbinは経済が自然に移動する区間とゼロ金利制約にかかる区間を別々のレジームとして定義する。
そしてそれぞれのレジームにおいて1次近似をおこない得られた結果を繋ぎ合わせることでこの問題を解決している。

\subsection{自国金融政策パラメータの決定に際するゲーム的状況の回避}
\label{subsec:preparation_game_theory}
シミュレーションをおこなうにあたり、
まず外国金融政策は消費者物価指数を用いたインフレ目標に固定し政策パラメータも標準的な値に固定する。
そのうえで自国の各金融政策については政策パラメータは
 \( \beta \) ショックのシミュレーションにおいて自国の厚生を最大化するものとする。
これらのパラメータを用いておこなった \( \beta \) ショックおよび \( a \) ショックのシミュレーション結果が
5章で表示するグラフである。
なお5章において説明されるが、本稿モデルのシミュレーションにおいては
自国金融政策は外国の生産者物価指数などの主要な外国関連変数に影響を与えない(遮断効果)。
そのため外国金融政策を生産者物価指数を用いたインフレ目標に固定することで
外国金融政策の最適パラメータが変わってしまうゲーム理論的状況を回避している。
これにより外国金融政策についてはパラメータを固定したまま自国金融政策の最適パラメータを探索することが正当化される。