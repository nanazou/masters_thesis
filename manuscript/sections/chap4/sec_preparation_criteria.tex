% !TeX root = ../../main.tex
% sections/chap4/sec_preparation_criteria.tex

\section{良い金融政策とは何か}
\label{sec:preparation_criteria}

本稿では経済に単発のショックを与えてシミュレーションをおこなう。
各金融政策の効果を比較するために家計の効用にもとづいて計算された厚生を用い、
この厚生が大きい金融政策が優れた金融政策ということになる。
しかしながら、長期的にみた場合にはどのような金融政策が厚生を大きくするのだろう。
ここでその理論的な考察をおこなっておく。

\subsection{経済の期待値は初期定常状態に一致する}

本稿のモデルにおいて経済の長期的な期待値は初期の定常状態と一致する。
これは以下の2つの理論的根拠にもとづく。
\paragraph{モデルの安定性}
本稿のモデルはBlanchard-Kahn条件を満たす安定的なサドルパス解を持つよう設計されている。
そのためショックによって経済が一時的に乖離したとしても時間の経過とともに必ず定常状態へと回帰する力が働く。
\paragraph{ショックの対称性}
経済が必ず定常状態へ回帰するといっても、それだけで経済の長期的な期待値が定常状態と一致するとはかぎらない。
たとえば経済を一時的に押し上げるショックばかりが継続的に起こるとすればどうだろう。
このとき経済が定常状態に戻るとしても、その期待値は定常状態よりも良いものになるはずである。
しかしながら、本稿のモデルにおいてはショックは期待値ゼロの正規分布に従うと仮定されている。
これにより正のショックと負のショックが同頻度で起こるため経済の期待値は定常状態となるのだ。

\subsection{定常状態は構造パラメータによって定まる}
この定常状態を決定するのは
生産性 \( a \)、労働人口 \( N \)、主観的割引因子 \( \beta \) といった経済の構造パラメータのみであり
金融政策は含まれない。
したがって金融政策の役割は平均を引き上げることではない。

\subsection{長期的にみて良い金融政策とは分散を最小化するもの}
ではどのような金融政策が長期的にみて良いのだろうか。
それは経済の期待値からの分散を小さくするものである。
以下ではなぜ分散が小さい方が望ましいのかを効用関数の非線形性にもとづき直観的に説明する。

\paragraph{効用は分散を嫌う}
家計の満足度を表す効用関数 \( U(c) \) は一般に上に凸の形状( \( U'' < 0 \) )をしている。
これは限界効用逓減の法則、すなわち、消費が増える喜びよりも減る痛みのほうが大きいという性質を表す。
たとえば定常状態の消費を \( c_{ss} \) とし、
変動がある場合の消費を \( c_t = c_{ss} + \delta_t \) とする( \( \delta_t \) は平均ゼロの変動 )。
2次近似を用いると期待効用は以下のように表せる。
\[
\operatorname{E}[U(c_t)] \approx U(c_{ss}) + U'(c_{ss})\operatorname{E}[\delta_t] + \frac{1}{2}U''(c_{ss})\operatorname{E}[\delta_t^2]
\]
ここで \( \operatorname{E}[\delta_t]=0 \) であるため第2項は消える。
しかし \( U'' < 0 \) であるため第3項の \( \frac{1}{2}U''(c_{ss})\operatorname{E}[\delta_t^2] \) は
常に負の値となる。
ここで \( \operatorname{E}[\delta_t^2] \) は消費の分散である。
したがって消費の平均値( \( c_{ss} \) )が同じであっても、
その分散( \( \operatorname{E}[\delta_t^2] \) )が大きくなればなるほど
平均的な満足度(期待効用)は低下するのである。


以上の議論より、長期的にみて良い金融政策とは経済の定常状態からの乖離を小さく抑えるものであることがわかった。
本稿ではどの金融政策が長期的にみて良いかの考察はおこなわないが、
グラフにおける乖離が大きいことは長期的な観点からは望ましくないという
