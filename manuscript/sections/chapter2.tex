% !TeX root = ../main.tex
% sections/chapter2.tex

\chapter{先行文献研究}
\label{chap:literature_review}

本稿はゼロ金利制約下における最適な金融政策を論じるものである。
こうした研究は1990年代に日本経済が長期停滞に入ってから始まった比較的新しいものであり、
2008年に世界金融危機が発生したことによりさらに活発におこなわれるようになってきた。

本章では本稿にいたる金融政策の歴史を期待という概念を軸に概観し本稿の位置づけを明確にする。
本稿の構成は以下の通りである。

まずフィリップス曲線が発見される以前の金融政策についてみる(第\ref{sec:review_before_phillips_curve}節)。
次にフィリップス曲線が発見されたものの期待については考慮されていなかった時代について述べる(第\ref{sec:review_discovery_of_phillips_curve}節)。
続いて期待によりフィリップス曲線が移動することが知られるようになったこととその原理を解説する(第\ref{sec:review_shifting_phillips_curve}節)。
そして裁量的金融政策が期待制御に失敗してフィリップス曲線の移動を止められず大インフレを招いたことをみる(第\ref{sec:review_failure_of_discretionary_policy}節)。
そしてこの大インフレの教訓からインフレ目標が採用され、
これが期待インフレ率を安定させ大いなる安定の時代を築いたことをみる(第\ref{sec:review_success_of_inflation_targeting}節)。
そしてこの時期に開発され政策運営を支えたマクロ経済モデルについて解説する(第\ref{sec:review_macro_model_development}節)。
続いて期待インフレ率の安定が副作用としてゼロ金利制約を生み出したことをみる(第\ref{sec:review_side_effects_of_low_inflation}節)。
またそれを受けて期待に強く働きかける水準目標が誕生したことについて述べる(第\ref{sec:review_zlb_challenge}節)。
さらに水準目標の有力候補として名目GDP水準目標が支持される理論的根拠について記述する(第\ref{sec:review_merits_of_ngdplt}節)。
続いて名目総消費水準目標の主要な関連研究を紹介する(第\ref{sec:review_nclt_studies}節)。
最後に本稿の貢献について説明する(第\ref{sec:review_our_contribution}節)。

\section{フィリップス曲線が発見されるまでの金融政策}
\label{sec:review_before_phillips_curve}

現代の金融政策論争は第二次世界大戦後に遡る \parencite[金融政策の歴史的概観については][などを参照]{ClaridaGaliGertler1999, Woodford2009}。
当時は失業率とインフレ率の負の相関関係が認識されていなかった。
そのためたとえば不況下において需要を刺激すると生産が増加して失業率が減るもののインフレ率は上昇しないと考えていた。
また完全雇用下において需要を刺激すると生産はもはや増加しないため失業率は減らず、
代わりにインフレ率のみが上昇すると考えられていたのである \parencite[この時代の政策思想については][に詳しい]{Friedman1968, GoodfriendKing1997}。

\section{フィリップス曲線の発見と期待の不在}
\label{sec:review_discovery_of_phillips_curve}

この認識を一変させたのが \textcite{Phillips1958} によるフィリップス曲線の発見であった。
彼は約100年間の英国のデータを用いて失業率と名目賃金上昇率(ひいてはインフレ率)の間に
安定的な負の相関関係が存在することを示した。
この発見は1960年代を通じて多くの政策担当者にインフレ率と失業率の間に
固定的な負の相関関係があるという信念を抱かせた。
インフレ率と失業率はどの組み合わせを選択するかという連続的なトレードオフの問題へと変貌したのである。
そこには人々の期待がフィリップス曲線を移動させるという現代的な考え方は存在しなかった。

\section{期待の発見と移動するフィリップス曲線}
\label{sec:review_shifting_phillips_curve}

この期待なき理論の弱点を \textcite{Friedman1968} と \textcite{Phelps1968} がそれぞれ独立に指摘した。
彼らによれば家計や生産者は期待インフレ率に基づいて現在の選択を決定するため、
期待インフレ率が変化すればフィリップス曲線そのものが移動するのである。
しかし初期の理論における期待がフィリップス曲線を移動させる原理の説明は
適応的期待や貨幣錯覚に依存するなど不合理な点が残されていた。
この原理はのちに \textcite{Fischer1977}、\textcite{Taylor1980}、\textcite{Calvo1983} といった
ニュー・ケインジアン経済学の先駆者たちにより価格の硬直性という概念を用いて
合理的期待形成の枠組みで説明された \parencite[この理論的変遷については][を参照]{ClaridaGaliGertler1999}。
彼らのモデルの中でも現在標準的に用いられているのが \textcite{Calvo1983} の提唱した
確率的な価格改定モデルである。
本節ではこのカルボ型価格改定と家計が生産者でもある
ヨーマン・ファーマー・モデルを用いてフィリップス曲線の仕組みを説明する。
カルボ型価格改定では毎期一定割合の家計のみが価格を変更することができる。
これは契約上の制約や、価格表の刷り直しに要する物理的な費用、
あるいは顧客への通知といった実務的な費用(メニュー・コスト)によるものと解釈される
\parencite{ClaridaGaliGertler1999, GoodfriendKing1997}。
さらに広義には値上げが顧客離れを招くのではないかという心理的な制約も価格が据え置かれる原因として考えられる。
ここで失業率が低下し生産が増加したとする。
このとき労働が増加するが、家計にとって労働の増加は労働の限界負効用を上昇させる。
価格改定の機会を得た家計はこの負効用の増大を補償し自身の効用を最適化するために価格を引き上げる。
これが失業率が下がればインフレ率は上がるという短期的なフィリップス曲線の理論的解釈である。
しかしこの短期的な関係はなぜ長続きしないのか。
ここで期待が決定的な役割を果たす。
カルボ型価格改定の下で価格改定の機会を得た家計は
一度価格を決めると次の改定機会までその価格を維持しなければならない。
したがって今日の価格決定は将来の経済状況に関する期待にもとづいて「前向き」におこなわれる。
たとえば中央銀行が短期フィリップス曲線における負の相関を利用し、
低失業率を維持するために高インフレを容認する姿勢を見せたとする。
ここで家計が「この中央銀行は将来的にも高インフレを容認するもの」と理解した場合、期待インフレ率が上昇する。
家計は毎期価格改定をできるわけではないため、期待インフレ率が上昇したならば、
将来の労働の負効用上昇分をあらかじめ現在の価格設定に上乗せする。
つまり期待が変化したことで、同じ失業率を維持するためにより高いインフレ率が必要になってしまったのである。
これは短期フィリップス曲線が上方移動したことを意味する。
もし中央銀行がインフレ容認の姿勢を取り続けるならば家計の期待インフレ率はますます上昇し
現在のインフレ率が加速度的に上昇していく。
これを打破するには中央銀行がインフレに対する断固たる姿勢を示し、
上昇したインフレ率を上回る幅で名目利子率を引き上げる必要がある(テイラー原理)。
この断固たる姿勢を見た家計が期待インフレ率を低下させればインフレ率上昇の悪循環が断ち切られるのである。

\section{裁量的金融政策による期待制御の失敗と大インフレの発生}
\label{sec:review_failure_of_discretionary_policy}

この理論的な懸念は1970年代の大インフレという形で現実のものとなった \parencite[][などを参照]{ClaridaGaliGertler1999, Mishkin1999, GoodfriendKing1997}。
多くの先進国が制御不能な高インフレと経済停滞が併存する深刻なスタグフレーションに見舞われ、
1960年代に観測された短期的なフィリップス曲線の関係は消失した。
この経験は規律なき裁量的政策がいかに経済を不安定化させうるかという教訓となった。
この政策失敗の核心は \textcite{ClaridaGaliGertler2000} による実証研究で説明されている。
彼らの分析によれば、1979年のボルカー議長就任以前のFRBの政策は
期待インフレ率が1\%上昇しても名目利子率を平均で1\%未満しか引き上げない受動的なものであった。
これはインフレが加速する局面で実質利子率の低下を許容してしまったことでインフレを抑制できなかったことを示している。
ではなぜ当時の政策当局はこうしたインフレを容認する政策を続けたのか。
\textcite{Bernanke2004} は、当時の政策当局は固定されたフィリップス曲線を信じる生産楽観主義に加えて
インフレ悲観主義に陥っていたと指摘する。
第一に、中央銀行はフィリップス曲線が固定されたものと信じ、
非常に低い失業率の達成を目標として緩和的な金融政策を運営した。
しかしこの政策は人々の期待インフレ率を上昇させフィリップス曲線を上方へ移動させた。
その結果、中央銀行が許容可能だと考えていた率をはるかに上回る高インフレが発生した。
第二に、この予期せぬインフレに直面した際、中央銀行は自らの緩和的な政策が原因であるとは考えず、
その主因を労働組合の賃金交渉圧力や石油価格の高騰といった
金融政策では制御不能なコストプッシュ要因にあると信じ込んだ。
さらにこの政策判断の誤りの根源にはより深刻な情報の問題があったことを \textcite{Orphanides2003} が示した。
彼の分析によれば1970年代の政策当局は経済の潜在生産を過大評価していた。
まずこの時期には生産性が低下しこれは負の供給ショックとしてフィリップス曲線を上方へ移動させた。
しかし中央銀行はこの構造変化に気づかず、
下方に位置していた古いフィリップス曲線とそれに基づく低い自然失業率を信じ続けていたのである。
観測された失業率はこの低い自然失業率を上回っていたため
誤った認識をもった政策当局は経済が不況にあると判断し緩和的な金融政策を実施した。
その結果、失業率は真の自然失業率を下回り期待インフレ率の上昇をもたらしたのである。

\section{インフレ目標による期待制御の成功}
\label{sec:review_success_of_inflation_targeting}
1970年代の大インフレへの反省から規律ある名目アンカーの確立が必要であるという合意が形成された。
その最初の試みは、貨幣の量が増えれば一定の割合のインフレが起きるという貨幣数量説にもとづく
貨幣供給目標であった。
しかし金融自由化やIT技術の発展により新しい種類の預金口座や金融商品が登場すると
マネーサプライとインフレ率の関係が不安定化し、貨幣供給目標は挫折した。
そこで多くの中央銀行は短期政策利子率を主要な政策手段とする
現代的なインフレ目標へと移行していったのである \parencite{Mishkin1999}。
このインフレ目標は人々の期待インフレ率を強力に安定させる名目アンカーとして機能し、
2000年代半ばまで続く「大いなる安定(The Great Moderation)」\parencite{Bernanke2004} の時代を築く
大きな成功を収めることとなった。

\section{マクロ経済モデルの発展}
\label{sec:review_macro_model_development}
こうした政策運営の変遷と並行し、経済学では期待がいかに現在のマクロ経済変数に影響を及ぼすかを
厳密に解明しようとする試みが進展した。
期待を考慮することの重要性はすでに認識されていたものの、
それを整合的に記述し政策評価に活用するための理論的枠組みは未だ不十分であった。
この課題を解消し期待管理に関する強固な理論的裏付けを与えたのがDSGE(動学的確率的一般均衡)モデルである。
DSGEモデルは1970年代以前のモデルが抱えていたミクロ的基礎の欠如という問題を克服し、
各主体の動学的な最適化行動から期待の役割を数学的に導出した。
まず \textcite{Fischer1977} や \textcite{Taylor1980} は名目契約の観点から価格の硬直性を理論化し、
さらに \textcite{Calvo1983} は生産者の確率的な価格改定行動を導入することで
将来の価格期待が現在の価格設定に直接影響を与える「前向き」な供給関数の導出に成功した。
さらに \textcite{Woodford2003} らはこれらのモデルが
家計や生産者の最適化行動にもとづいているというミクロ的な特性を活かし、
代表的家計の効用関数から直接的に社会的な厚生を定義する手法を確立した。
それまでのマクロ経済モデルでは政策の評価基準(損失関数)は
インフレ率や生産ギャップを恣意的に組み合わせたものに過ぎなかった。
しかしDSGEモデルの登場により、個別の家計の満足度という厳密なミクロ的根拠にもとづいた
客観的な政策評価が可能となったのである。
その後、DSGEモデルは中央銀行の実務における不可欠なシミュレーション・ツールとして活用されるようになり、
それが2000年代中盤までの大いなる安定の時代を支える
理論的背景を形作ったのである。

\section{低インフレ期待の副作用}
\label{sec:review_side_effects_of_low_inflation}

しかしインフレ目標の成功は負の側面ももたらした。
インフレ目標の浸透によって低インフレ環境が定着したことは経済が負のショックに見舞われた際に
期待インフレ率が容易に負の領域へと沈み込みやすくなる状況を生みだした。
期待インフレ率が十分高い水準にあれば負のショックが生じても正の領域に留まるが、
期待インフレ率が低ければわずかな負のショックによって負へと転じ
フィリップス曲線を下方へと押し下げてしまうのである。
加えてこうした低インフレ期待下では最適な価格から多少乖離していても生産者が被る損失が限定的となる。
そのためメニュー・コスト等の価格改定費用を支払ってまで頻繁に価格を変更する誘因が低下し
経済全体で価格の硬直性が強まることとなった。
これによりフィリップス曲線の傾きは著しく平坦化し金融政策による調整機能が大きく低下したのである。
通常であれば中央銀行が名目利子率を引き下げて需要を刺激し総需要(AD)曲線を上方へ移動させることで
物価を目標水準まで押し戻すことが可能である。
しかしフィリップス曲線が平坦な状況下では需要の拡大が物価の上昇に結びつきにくい。
そして物価を引き上げるためにさらなる名目利子率低下を模索し続けた結果、
名目利子率はゼロに達しこれ以上下がらなくなるゼロ金利制約の壁に突き当たったのである。
こうしたインフレ期待が負に転じやすい脆弱性と
平坦化したフィリップス曲線という二重の困難が現実のものとなったのが1990年代以降の日本経済であった。
大いなる安定と称賛された安定的構造は、ひとたびゼロ金利制約に直面すれば
利子率操作を通じたデフレ脱却を極めて困難にする強固な枠組みでもあったのである。

\section{ゼロ金利制約下における新たな期待制御への挑戦}
\label{sec:review_zlb_challenge}
1990年代の日本経済はバブル崩壊により深刻な需要不足に陥った。
これに対し日本銀行は断続的に名目利子率を引き下げたがついにゼロ近辺にまで到達した。
伝統的な利下げという手段が効力を失う状況は流動性の罠として知られ
当時の実務者にとって前例のない事態であった。
この難問に向き合うなかで、ゼロ金利制約に直面した中央銀行に残された手段は
人々の期待に働きかけることであるという認識が次第に築かれていった。
たとえば \textcite{EggertssonWoodford2003} は
中央銀行が「現在マイナス金利にできない分、将来景気回復後にも低金利を維持する」
と約束(コミットメント)することで人々の期待に働きかけ現在の経済を好転させられることを理論的に示した。
この埋め合わせの原理は歴史依存性と呼ばれ、その約束による期待の誘導はフォワード・ガイダンスと呼ばれる。
このフォワード・ガイダンスの効果は
 \textcite{ReifschneiderWilliams2000} や \textcite{JungTeranishiWatanabe2005} によっても裏付けられており
その詳しい波及経路は \textcite{ReifschneiderWilliams2000} などにより説明されている。
さらにこのフォワード・ガイダンスによる約束を制度化し信頼性を高めた仕組みが水準目標である。
水準目標はゼロ金利制約下でも期待への強力な働きかけをおこなえることが
\textcite{Woodford2012} などにより理論的に示されている。

\section{有力候補としての名目GDP水準目標とその改善可能性}
\label{sec:review_merits_of_ngdplt}
水準目標の中でも最も多くの支持を集めているのが名目GDP水準目標である。
この支持の理由は以下のような点にまとめられる
第一に供給ショックに対して頑健であること \parencite{Sumner2014, BhandariFrankel2017}。
第二に潜在生産といった観測が困難な情報に依存しないこと \parencite{Orphanides2003, BeckworthHendrickson2019}。
第三にDSGEモデルにおける厚生上優位であること \parencite{GarinLesterSims2016}
第四に名目GDPの安定化は効率的なリスク共有を促すこと \parencite{Sheedy2014}
こうした合理的な理由により支持を集めてきた名目GDP水準目標ではあるが、
これが本当に最善の目標であるかということに関して次のような疑問が生じてくる。
不況に陥った経済を回復させ厚生を高めるためには生産を増加させる必要がある。
のちに5章において示されるように、生産を増やすためには名目利子率および所得の期待限界効用を
下落させる必要がありそれが最も効率的にできる金融政策が最善のものといえる。
それでは名目利子率および所得の期待限界効用を名目GDP水準目標以上に効率的に下落させられる金融政策はないのだろうか。
この問いに対し本稿は名目総消費水準目標の優位性を主張する。
それにあたり次節では名目総消費水準目標に関するこれまでの学術的・実務的論争を整理し、
本稿の分析の前提となる既存の知見を確認する。

\section{名目総消費水準目標とその関連研究}
\label{sec:review_nclt_studies}
幾人かの研究者や実務家が異なる理由により名目総消費水準目標の賛否を論じてきた。
以下ではそのいくつかを紹介する。

\subsection{1. 労働市場の歪みを解消する:\textcite{Koenig1995}}
\label{sec:review_koenig_1995}
\textcite{Koenig1995} は代表的な競争的生産者と家計からなるモデルを用い
名目価格が硬直的である経済において名目消費の安定化が労働市場の歪みを解消することを論じた。
Koenig のモデルは第1期に期待にもとづいて第2期の名目賃金 $W$ が決定し、
第2期に負の生産性ショックなどが発生するという実質的な2期間の構成をとっている。
まず労働 $N$ の決定について考える。Koenig のモデルでは名目賃金 $W$ が事前に固定される一方で
生産者が雇用量を決定する権限を持つとされる。
したがって生産者は利潤最大化のために
労働の限界生産性と実質賃金 $\frac{W}{P}$ を一致させるように労働 $N$ を決定する。
\begin{equation}
N = \left( \frac{a}{W/P} \right)^{1/b}
\end{equation}
ここで $a$ は生産性パラメータ、$b$ は生産関数のパラメータである。
一方、理想的な「市場均衡値 $N^*$(競争値)」は生産者の利潤最大化行動による労働需要のみならず
家計の効用最大化行動にもとづく労働供給の条件も満たさなければならない。
家計は消費と労働の限界代替率が実質賃金 $\frac{W}{P}$ に一致するときに効用を最大化する労働 $N$ を得る。
\begin{equation}
\frac{W}{P} = N^{\alpha} C^{\beta}
\end{equation}
ここで $\alpha$ と $\beta$ は家計の選好を表すパラメータである。
賃金が伸縮的であればこれら需要と供給の条件を満たす点として $N^*$ が達成される。
次に負の生産性ショックなどが発生し生産性 $a$ が減少したとする。
もし物価 $P$ が不変であれば、需要の式から明らかなように
生産性 $a$ の低下により労働 $N$ が減少する。
すると家計の所得は減少するため消費 $C$ も減少する。
こうして労働 $N$ と消費 $C$ はともに減少するから
供給の式において(左辺)>(右辺)となり等号が保たれなくなる。
したがって労働 $N$ は市場均衡値 $N^*$ から乖離しておりこれは労働市場の歪みが発生している状態にほかならない。
ここで労働 $N$ を市場均衡値 $N^*$ に一致させるためには物価 $P$ が適切に上昇する必要がある。
物価 $P$ の上昇は実質賃金 $\frac{W}{P}$ を減少させ、また需要の式を通じて労働 $N$ を増加させる。
そして労働 $N$ の増加により消費 $C$ も増加する。
つまり実質賃金 $\frac{W}{P}$ は下落し、労働 $N$ と消費 $C$ は上昇するから
(左辺)>(右辺)となっていた供給の式は等号へ向かう。
この物価 $P$ 調整が適切であれば供給の式も満たされることとなり労働 $N$ は市場均衡値 $N^*$ となる。
\textcite{Koenig1995} は中央銀行が名目所得と名目消費の加重平均を目標とすれば
この適切な物価 $P$ 調整が自動的に達成されることを示した。
このように名目消費の安定化は名目賃金 $W$ の硬直性に由来する労働市場の歪みを解消し、
労働 $N$ を市場均衡値 $N^*$ へ導く有効な手段となる。
Koenig の議論は競争的均衡(市場均衡)が社会的に最も望ましいという前提に立脚しており、
賃金硬直性の影響を排して伸縮時の実現値を再現することを目的に置いている。
しかし本稿の5章における $a$ ショック(生産性ショック)の節で示すとおり
物価や賃金の伸縮性は必ずしも厚生の向上を保証するものではない。
特に生産性ショックに対して物価が伸縮的に反応することは
生産を大きく低下させる要因となり厚生を減少させる恐れがある。
こうした場合はむしろ、一定程度の物価の硬直性がショックの影響を緩和する緩衝材として機能し
より大きな厚生を実現しうる。

\subsection{2. 金融危機後の停滞を打開する実務的な手段とする:Sheets}
\label{sec:review_sheets}
実務の観点からは、元FRB局長の Nathan Sheets が
金融危機後の停滞を打開する強力な政策手段として名目総消費水準目標を提唱した \parencite{Harding2011}。
\textcite{Harding2011} が Financial Times 誌で紹介し
のちに \textcite{Chen2019} も学術的関心の対象として取り上げた Sheets の主張によれば
名目総消費水準目標には以下の5つの実務的利点がある。

\begin{enumerate}
  \item[(i)] 家計になじみ深く現在の金融政策でも扱われている消費者物価指数を用いるため、
  生産者物価指数を用いる名目 GDP 目標よりも政策意図の浸透が容易である。
  \item[(ii)] 消費の安定は「消費の平滑化」と同義でありこれは家計の厚生最大化に直結する。
  \item[(iii)] 消費は投資や外需を含む GDP よりも変動が低く目標として掲げやすい。
  \item[(iv)] 中央銀行が関与できない政府支出などを目標から除外することで金融政策の責任範囲が明確化される。
  \item[(v)] 消費が危機前の本来あるべき水準を下回っていると示すことで緩和政策に対する政治的・社会的支持を得やすい。
\end{enumerate}

\subsection{3. 金融市場の歪みを解消する:\textcite{Koenig2013}}
\label{sec:review_koenig_2013}
\textcite{Koenig2013} は資本家と労働者という二つの主体からなる2期間モデルを用い
名目債務が存在する経済において名目総消費水準目標が金融市場の歪みを解消することを論じた。
Koenig のモデルにおける社会全体の純利益(総資源)は
生産 $y$ から税 $g$ を差し引いた $c = y - g$ と定義される。
この利益 $c$ は外生的に決定される一定の分配率に基づき、
債務の清算が行われる前の基礎所得として資本家に $(1-\theta)c$、労働者に $\theta c$ の割合で配分される。
分析の焦点となる第2期はモデルの最終期であり資本家による新たな投資は想定されていない。
したがって各主体は基礎所得に前期から繰り越された実質的な債務 $\frac{DR}{P}$ の受け渡しを加減した額の
すべてを消費に充てることとなる。
このときの資本家の消費 $c_1$ および労働者の消費 $c_2$ は以下の通り記述される。
\begin{equation}
c_1 = (1-\theta)c + \frac{DR}{P}
\end{equation}
\begin{equation}
c_2 = \theta c - \frac{DR}{P}
\end{equation}
上式を合算すると $c_1 + c_2 = c$ が成立することから利益 $c$ は社会全体の総消費額と一致する。
ここで負の供給ショック等によって総消費 $c$ が 10\% 減少した局面を想定する。
この際、基礎所得である $(1-\theta)c$ および $\theta c$ も $c$ に比例しているため同様に 10\% 減少する。
しかしもし物価 $P$ が不変であり実質債務 $\frac{DR}{P}$ が固定されたままであるならば、
返済を受ける資本家の消費減少率は 10\% よりも小幅に留まる一方、
返済をおこなう労働者の減少率は 10\% を超過し、主体間で痛みの偏りが生じる。
両者の消費減少率を全体の減少率と一致させるためには基礎所得と同様に実質債務の項も 10\% 減少させる必要がある。
そのためには総消費 $c$ の減少を相殺する規模で物価 $P$ が上昇しなければならない。
\textcite{Koenig2013} は中央銀行が名目総消費水準目標($Pc = n^*$)を採用して
名目消費支出を一定に維持すればこの物価調整が自動的に達成されることを示した。
このように $c_1$ と $c_2$ の変化割合が社会全体の $c$ の変化と完全に同期することは
経済全体のショックの影響がすべての主体間で理想的に分散されていることを意味する。
これはあらゆるリスクに対する保険契約が完備された市場(完備市場)において実現される消費配分と等価であり、
名目総消費水準目標が不完備市場における歪みを解消する有効な手段であることを示唆している。

\subsection{4. マクロ経済を安定させない:\textcite{Chen2019}}
\label{sec:review_chen_2019}
経済の安定性分析の観点からは \textcite{Chen2019} が自己充足的な期待(アニマル・スピリット)に起因する
経済変動を抑制する効果について名目 GDP 水準目標と名目総消費水準目標の比較分析をおこなっている。
Chen は現金前払(CIA)制約を仮定している。これは次の 2 つの条件を合わせたものである

\begin{description}
  \item[(CIA-1)] 貨幣 \( M \) は消費 \( c \) の決済のみに使用され投資 \( i \) の決済には使用されない
  \item[(CIA-2)] 今期の消費 \( c \) のために家計は前期に貨幣 \( M \) を準備しなければならない
\end{description}

これら 2 つの条件により今期の名目消費 \( Pc \) は前期に家計が保有していた貨幣 \( M \) によって制限される( \( Pc \le M \) )。
Chenのモデルでは不確実性が存在せず変数の期待値と実現値は一致するため
今期の名目消費 $Pc$ は前期に期待していたものである。
したがってもし今期に $Pc < M$ となり余剰貨幣が存在するならば、
これを期待していた前期の家計は利息を生まない貨幣を投資 $i$ の購入に充てる。
この投資 $i$ の需要の増加は利子率の下落を招き、これは投資 $i$ の需要の増加幅を縮小させ消費 $c$ の需要を高める。
これにより物価 $P$ と消費 $c$ が上昇する。
これにより $Pc = M$ が成立するため今期に余剰貨幣は生まれない。
Chen のモデルにおいて金融政策とは中央銀行による $M$ の操作を指す。
中央銀行が $M$ を増加させれば、先ほどのように投資 $i$ と消費 $c$ が増加し生産 $y = c + i$ が拡大する。
一方で $M$ を減少させれば逆の動きが起き $y$ は減少する。
ここで自己充足的な期待(アニマル・スピリット)が上昇するとどうなるだろうか。
家計は $M$ の制限を受けることなく投資 $i$ を増大させることが可能であり、
期待が楽観に振れれば投資 $i$ 主導で生産 $y$ が増加し潜在パスを離れて上昇しようとする。
このとき名目総消費水準目標を採用していたならば
均衡条件 $Pc = M$ により中央銀行はマネーサプライ $M$ を目標値に固定し続けなければならない。
したがって中央銀行は $M$ を減じて生産 $y$ を抑制することができない。
一方で名目GDP水準目標を採用していたならば
中央銀行はこの $y$ の増加に反応して $M$ を減少させることで $y$ 押し戻すことができる。
Chenはこうした経済安定化の観点から名目総消費水準目標よりも名目GDP水準目標の方が望ましいと結論付けた。
しかしながら Chen によるこの結論は CIA 制約という強い仮定に依存していることに留意する必要がある。

\subsection{5. テイラールールに劣る:\textcite{Gross2023}}
\label{sec:review_gross_2023}
\textcite{Gross2023} はオーストラリア経済を対象とし
多セクターDSGEモデルを用いた金融政策の比較分析をおこなっている。
同研究ではインフレ率と生産ギャップの双方に反応する柔軟なインフレ目標や
物価乖離と生産ギャップの双方に反応する物価水準目標といったテイラールール型の政策が
名目総消費水準目標などの名目値目標よりも明らかに優れていると結論づけられている。
しかしこの成績を判定するために用いられた損失関数という計算式は各政策を公平に評価するように作られていない。
この式が最小化すべき対象として設定しているインフレや生産のギャップは
まさにテイラールールが目標としている変数そのものである。
そのため名目消費のようにこの評価式には入っていない別の変数を安定させようとする政策は
テイラールールと比較して必然的に成績が悪く判定されてしまう。
このような評価手法は、損失関数の開発元である \textcite{DorichMendesZhang2021} も、
特定の変数を含まない式で評価をおこなうことは他の政策を不当に低く評価することになるとして
推奨していないものである。

\section{本稿の貢献}
\label{sec:review_our_contribution}
前節において名目総消費水準目標に関する主要な研究を概観した。
しかしながらそれらはどれも家計の効用にもとづく厚生を基準として
名目総消費水準目標を他の政策と比較したものではなかった。
本稿は名目総消費水準目標やテイラールールを含む11の金融政策について
厚生を基準として比較分析をおこなう。
シミュレーションにあたっては \( \beta \) ショック(需要ショック)と
 \( a \) ショック(供給ショック)をそれぞれ個別に与える。
名目総消費水準目標は \( \beta \) ショックに対して厚生の落ち込みを最小にし、
 \( a \) ショックに対しても最善の政策とそん色ない。
さらに本稿はAS-AD分析による解釈をおこない、
結果が導かれた原因について詳細な理解を読者に提供する。