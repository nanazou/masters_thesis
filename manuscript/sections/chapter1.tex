% !TeX root = ../main.tex
% sections/chapter1.tex

\chapter{はじめに}
\label{chap:intro}

\section{研究の背景}
\label{sec:intro_background}
2008年の世界金融危機以降、各国の中央銀行は経済を回復させるため利子率を段階的にゼロ近辺まで引き下げた。
しかしその結果、それ以上の利下げが困難となるゼロ金利制約に直面したことで
利子率を低下させて総需要曲線を上へ移動させるという直接的な経済回復の手段が失われた。
このように利子率を下げられなくなった場合、所得の期待限界効用を下げることが残された手段となる。
しかしながら、伝統的な金融政策は期待への働きかけが十分ではなかったため手詰まり感がみられ
経済は容易に好転しなかった。
そうした中で注目を集めるようになったのが将来の景気回復後も緩和の継続を約束することにより
家計の期待に強く働きかけようとする水準目標である。
なかでも名目GDP水準目標はいくつかの理論的研究によりその効果が裏付けられており、
水準目標の中でもとりわけ多くの支持を集めている。
本稿はこのような政策環境を背景として議論を展開する。

\section{本稿の問い}
\label{sec:intro_question}
ではゼロ金利下の経済を最も効率的に回復させる金融政策は名目GDP水準目標なのだろうか。
あるいはより良い水準目標があるだろうか。
先ほど述べた所得の期待限界効用とは、
将来において追加1単位の貨幣で消費をおこなうことにより得られるであろう効用の期待値である。
そのためもし所得の期待限界効用が低下すれば、
家計は貯蓄して将来の消費から効用を得るよりも現在の消費を増やして効用を得る方が
より大きな生涯効用を得ることができるだろうと考えるため、貯蓄から消費への流れが生まれる。
一方で、消費から得られる効用が消費の対数関数であると仮定するなら、
所得の期待限界効用は期待名目総消費の逆数となる。
したがって将来において名目総消費を上昇させることを中央銀行が強く約束するなら、
期待名目総消費は直接の目標として上昇するため、
その逆数である所得の期待限界効用は最も効率的に低下するのではないか。
そしてその結果、経済は最も効率的に回復に向かうのではないか。

\section{本稿の貢献と結論}
\label{sec:intro_contribution}
本稿はゼロ金利下の経済を最も効率的に回復させる金融政策は名目総消費水準目標であると主張する。
この主張に説得力をもたせるため、
名目総消費水準目標を含む11の金融政策について
家計が財布のひもを固くする \( \beta \) ショックを与えて効果の分析をおこなう。
この需要ショックにより経済はゼロ金利に陥るが、名目総消費水準目標を用いたものはいち早く回復へと向かい、
家計の効用にもとづいて計算された厚生の落ち込みも最小となる。
また供給ショックへの効果も検証するため生産性が低下する \( a \) ショックを与えた分析もおこなう。
この \( a \) ショックについても名目総消費水準目標は頑健性を示す。
最後に名目総消費水準目標は指標集計や政策意図の浸透においても優れていることを説明し
現実の政策運営に適した目標であることを主張する。

\section{論文の構成}
\label{sec:intro_structure}
本稿の構成は以下の通りである。
第2章では、期待を軸に金融政策の歴史と先行研究を概観する。
第3章では、分析の基盤となる二国モデルを構築する。
第4章では、シミュレーションの設定を述べる。
第5章では、シミュレーション結果の提示とAS-AD分析をおこなう。
第6章では、名目総消費水準目標の効果についてまとめをおこない、実務的な点や今後の課題についても述べる。
第7章において本稿の総括をおこなう。

なお本稿のシミュレーションにおいて用いたプログラムは以下のリポジトリにおいて公開している。

\url{https://github.com/nanazou/masters_thesis}