% !TeX root = ../../main.tex
% sections/chap5/sec_results_overview.tex

本章では前章までの議論にもとづきシミュレーションをおこない結果を詳細に分析する。

本章の構成は以下のとおりである。
まず第 \ref{sec:results_asad} 節では、シミュレーション結果を解釈するための理論的枠組みとして
第 \ref{chap:model} 章の動学方程式系をもとに総供給(AS)曲線および総需要(AD)曲線を導出する。
続く第 \ref{sec:results_beta_shock} 節と第 \ref{sec:results_a_shock} 節においては、
家計が貯蓄志向を高める \( \beta \) ショックおよび
生産性が低下する \( a \) ショックのシミュレーションをおこなう。
それぞれの分析においては各変数のインパルス応答グラフを提示するとともに、
第 \ref{sec:results_asad} 節で構築したAS-AD曲線を用いて
各政策がどのような仕組みにより経済に影響を与えたかをみていく。

本章で提示するインパルス応答グラフの読み方について一点補足する。
各グラフにおいて横軸は期を表しており便宜的にショック発生期を \( t=0 \) と設定して描写している。
縦軸は各変数の定常状態からの水準乖離を示している。
これは対数乖離(ハット変数)による近似表示とは異なるため、
乖離の式とグラフの数値を照らし合わせる際には注意が必要である。