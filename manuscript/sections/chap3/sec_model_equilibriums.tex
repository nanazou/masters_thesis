% !TeX root = ../../main.tex
% sections/chap3/sec_model_equilibriums.tex

\section{市場均衡}
\label{sec:model_equilibriums}

\subsection{財市場の均衡}
\label{subsec:model_goods_market}
\paragraph{自国財市場}
家計 \( h \) が生産する財の市場は \( t \) 期におけるその財の生産 \( y_t^h \) が
全世界からの総需要と等しくなることで均衡する。
\begin{equation}
y_t^h = \sum_{h' \in H} c_t^{h' \to h} + \sum_{f \in F} c_t^{f \to h}
\label{eq:individual_goods_market_equilibrium_repeat}
\end{equation}

\paragraph{外国財市場}
同様に外国財市場の均衡は以下の関係式で記述される。
\begin{equation}
y_t^f = \sum_{h \in H} c_t^{h \to f} + \sum_{f' \in F} c_t^{f' \to f}
\label{eq:individual_goods_market_equilibrium_foreign}
\end{equation}

\subsection{一物一価の法則}
\label{subsec:model_lop}
国際間の財価格は名目為替レート \( e_t^{/*} \) を通じて以下の一物一価の法則が成立すると仮定する。
\begin{equation}
p_t^h = e_t^{/*} p_t^{h*} \quad , \quad p_t^f = e_t^{/*} p_t^{f*}
\label{eq:lop}
\end{equation}