% !TeX root = ../../main.tex
% sections/chap3/sec_model_steady_state.tex

\section{定常状態とショックの発生}
\label{sec:model_steady_state}

本モデルにおける経済の推移と、分析の出発点となる定常状態について以下のように定義する。

\paragraph{ショックの発生とタイムライン}
時点 $t \le s-1$ まで、経済は決定論的定常状態(Deterministic Steady State)にあり、外生ショックの実現値はすべてゼロであったと仮定する。
分析の対象となる外生的なショックは、時点 $t=s$ において発生する。
この $t=s$ におけるショックの実現により、経済は定常状態から乖離し、家計や企業は新たな情報に基づいて各変数の最適値を決定する。

\paragraph{定常状態における資産保有量}
時点 $t \le s-1$ の定常状態において、経済内に存在するすべての家計 $h$ は資産を保有しておらず、国内コンティンジェント債券 $d$ および外国債券 $b$ の保有量はともにゼロであると定義する。
\begin{equation}
d_s^h = 0 \quad , \quad b_s^h = 0 \quad (\text{for all } h \in H)
\label{eq:ss_assets}
\end{equation}
ここで、添え字 $s$ は $t=s$ の期首(すなわち $t=s-1$ の期末)において各家計が保有している資産残高を指している。

\paragraph{初期状態の同一性}
この仮定により、初期時点 $t=s$ においてショックが発生した直後の状態において、すべての家計は正味資産がゼロの状態でショックに直面することが担保される。
これにより、次節で述べる「国内完備市場とリスク共有」において、家計間の資産分布の異質性を考慮することなく、一貫性のある論理でリスク共有の導出を行うことが可能となる。