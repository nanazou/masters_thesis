% !TeX root = ../../main.tex
% sections/chap3/sec_model_overview.tex

\section{モデルの概要}
\label{sec:model_overview}

本章では本稿の分析の土台となる理論モデルを構築する。
本モデルは現代マクロ経済学の標準的な枠組み(DSGE)にもとづき以下を仮定する。

\begin{enumerate}
    \item 合理的経済人:
    家計は生涯効用を予算制約の下で最大化する。
    これは家計が高度な数学的計算能力をもち、無限の将来までを見据えた動的な最適化問題を解くことを意味する。
    この最適化問題を解くためには、家計は必然的に、
    将来の不確実な変数について何らかの期待(予測値)を形成する必要がある。
    またこの最適計画は時間整合的となり、将来新たな予想外のショックが発生しない限り、
    家計は将来のどの時点においても当初の計画を変更するインセンティブを持たずそのまま実行し続ける(付録A.1節参照)。
    
    \item 合理的期待:
    経済学における期待とは将来の不確実な変数に対して家計が抱く予測値のことである。
    この予測値を形成する原理には様々なものがありうる。
    たとえば適応的期待形成は、過去の実現値の加重平均を計算するなど
    何らかの(確率論的ではない)規則にもとづいて予測値を算出する。
    これに対し本稿が採用する合理的期待形成とは、
    家計がモデルの構造を完全に知っており、
    その知識にもとづいて数学的な期待値を計算して予測するという仮定である。
    したがって合理的期待とは、家計の主観的な予測値が客観的な数学的期待値と同一である、という仮定に他ならない。
    $t$ 期における期待値の計算は以下の2種類の情報を所与としておこなわれる。
    \begin{itemize}
        \item[a.] モデルの構造: 
        (i) すべての方程式体系、および
        (ii) すべての外生ショックの確率分布
        \item[b.] $t$ 期までのすべての変数の実現値の履歴(情報集合 $I_t$)。
    \end{itemize}
    これにより計算する期待値は $E[\cdot | I_t]$ となる。
    ここで情報集合 $I_t$ とは「$t$ 期までの変数が特定の値をとった」という事実を表す命題(式)、
    すなわち数学的な事象の集合を指す。
    これは、「$I_t$ に含まれる命題(例えば $c_t = c_t'$ という等式)が成立している」
    という条件の下での計算を意味する。
    その結果この期待値計算の中では $t$ 期およびそれ以前の変数($c_t, c_{t-1}$ など)は、もはや確率変数ではなく、
    その実現値と同一視される既知の数値(定数)として扱われる。
    本稿では慣例に従いこの $E[\cdot | I_t]$ を $E_t[\cdot]$ と省略して表記する。
    したがって合理的期待の仮定とは家計が適応的期待形成のような(確率論的ではない)ルールで予測するのではなく、
    モデルの構造(項目a.)と過去の履歴(項目b.)のすべてを駆使して、数学的に最適な予測($E_t[\cdot]$)を
    おこなうことを仮定するものである。
\end{enumerate}

これらの仮定を基礎に本モデルは価格の硬直性を特徴とする二国間ニュー・ケインジアンDSGEモデルとして構築される。
本モデルにおいては自国と外国それぞれにおいて国内の債券市場は完備であり家計間のリスク共有が完全におこわれる一方、
それらのあいだの国際債券市場は不完備でありリスク共有が限定的となる。

本章の構成は以下の通りである。
まず3.3節で、モデルの基本的な均衡条件である財市場の均衡と一物一価の法則を記述する。
3.4節では、モデルの基本的な意思決定主体である家計の最適化行動を定式化し消費や貯蓄に関する主要な関係式を導出する。
3.5節では、国内完備市場の仮定がもたらすリスク共有の含意と代表的家計への集計可能性について論じる。
3.6節では、家計が生産者として直面する価格設定問題を取り上げ、
カルボ型の価格硬直性を仮定した下での生産関数と価格分散の動学を導く。
3.7節では、政府の財政政策と、本稿で比較分析の対象となる3つの金融政策ルールを定義する。
3.8節で国全体の資源制約式を導出する。
最後に3.9節でこれらの方程式を集約した代表的家計モデルの最終的な方程式体系としてまとめる。