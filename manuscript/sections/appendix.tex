% !TeX root = ../main.tex
% sections/appendix.tex

\appendix
\label{chap:appendix} % 付録セクション全体の開始点

% =============================================================
% 1. プログラムと構成( 先頭に配置 )
% =============================================================
% !TeX root = ../../main.tex
% sections/app/appendix_programs_overview.tex

\chapter{数値計算プログラムの構成と実行環境}
\label{chap:appendix_programs_overview}

本付録では名目総消費水準目標のシミュレーションに使用したプログラムを提示する。
なお本稿のシミュレーションにおいて用いたプログラムの全体は以下のリポジトリにおいて公開している。

\url{https://github.com/nanazou/masters_thesis}

\section*{数値計算の実行環境とアルゴリズム}

シミュレーションの実行にあたっては、Dynare 6.3 および Octave 9.4.0 を使用した。特に、名目金利のゼロ金利制約( ZLB )を厳密に考慮するため、Guerrieri and Iacoviello (2015) によって開発された \texttt{Occbin} ツールキットを導入し、レジームスイッチングを伴う非線形動学を算出している。

なお、Octave 上で Dynare 等のパスを通すためのセットアップ・スクリプト( \texttt{env\_setup} 等 )を実行する場合は、まずコマンドウィンドウ上で該当するスクリプトが保存されているフォルダーまでカレントディレクトリを移動してから実行することに留意されたい。

\section*{プログラムの構成とディレクトリ構造}

本プロジェクト( \texttt{nclt\_project} )の主要なプログラムのディレクトリ構造は以下の通りである。なお、Dynare が自動生成する中間ファイルやシミュレーション結果の出力データについては、記述の簡潔さのため省略している。

\begin{verbatim}
nclt_project/
├── nclt_main.m                   % メイン・実行制御スクリプト
├── nclt_find_optimal_params.m    % 最適政策パラメータ探索スクリプト
└── src/                          % モデル定義・ソルバー関連
    ├── nclt_model.mod            % Dynareモデル定義(通常時)
    ├── nclt_model_zlb.mod        % Dynareモデル定義(ZLB考慮時)
    ├── nclt_model_declarations.inc      % 変数・パラメータの宣言
    ├── nclt_model_equations_common.inc  % モデルの共通方程式
    └── nclt_solve_core.m         % 均衡解法およびシミュレーション基幹部
\end{verbatim}

\section*{ソースコードの依存関係}

提示するプログラムは、以下の 3 つの役割に大別される。
\begin{enumerate}
    \item \textbf{モデル定義と共通方程式}: \texttt{src/} フォルダ内の \texttt{.mod} ファイルおよび \texttt{.inc} ファイルに、第 3 章で導出した非線形方程式体系を記述している。
    \item \textbf{シミュレーション実行スクリプト}: \texttt{nclt\_main.m} を通じて \texttt{nclt\_solve\_core.m} を呼び出し、Dynare を用いて各ショックに対するインパルス応答関数( IRF )を算出する。
    \item \textbf{政策パラメータの最適化}: \texttt{nclt\_find\_optimal\_params.m} を用い厚生を最大化する最適な政策パラメータの値をグリッドサーチによって特定する。
\end{enumerate}

次節以降に、各ソースコードの具体的内容を順次掲載する。

% =============================================================
% 2. 第 3 章( モデル構築 )で参照される基礎的な導出
% =============================================================
% !TeX root = ../../main.tex
% sections/app/appendix_cost_minimization.tex

\chapter{費用最小化問題からの需要関数と価格指数の導出( 第 1 段階 )}
\label{chap:appendix_cost_minimization}

本付録の目的は、総名目生産額が、集計産出量と真の価格指数の単純な積で表せること、そしてその関係を担保するために集計産出量が必然的にCES集計の形でなければならないことを証明することである。

\section*{定理 2:個別財への需要関数と真の価格指数}

家計 \( h \) が、ある時点 \( t \) において、所与の量の国内財バスケット \( c_t^{h \to H} = \left[ \sum_{h' \in H} (c_t^{h \to h'})^{\frac{\theta^H-1}{\theta^H}} \right]^{\frac{\theta^H}{\theta^H-1}} \) を、個別財の価格 \( \{p_t^{h'}\} \) の下で費用最小化的に購入するとき、以下の関係が成立する。
\begin{enumerate}
    \item 家計 \( h \) の個別財 \( h' \) への需要関数は、次式で与えられる。
    \[ c_t^{h \to h'} = \left( \frac{p_t^{h'}}{p_t^H} \right)^{-\theta^H} c_t^{h \to H} \]
    \item 上記の需要関数に現れる \( p_t^H \) は、国内財バスケットの「 真の価格指数 」であり、個々の財価格から以下のように定義される。この値は全ての国内家計で共通である。
    \[ p_t^H = \left[ \sum_{h' \in H} (p_t^{h'})^{1-\theta^H} \right]^{\frac{1}{1-\theta^H}} \]
\end{enumerate}

\section*{証明}

この定理を、自国家計 \( h \) の場合について証明する。外国の家計 \( f \) についても全く対称的な手順で証明可能である。

\subsection*{1. 費用最小化問題の設定}
家計が直面する問題は、時点 \( t \) において、所与の国内財バスケット \( c_t^{h \to H} \) を達成するための総費用を最小化することである。
\begin{itemize}
    \item \textbf{最小化対象}:
    \[ \min_{\{c_t^{h \to h'}\}_{h' \in H}} \sum_{h' \in H} p_t^{h'} c_t^{h \to h'} \]
    \item \textbf{制約条件}:
    \[ c_t^{h \to H} = \left[ \sum_{h' \in H} (c_t^{h \to h'})^{\frac{\theta^H-1}{\theta^H}} \right]^{\frac{\theta^H}{\theta^H-1}} \]
\end{itemize}

\subsection*{2. ラグランジュ関数と一階の条件( FOC )}
この問題を解くために、ラグランジュ関数 \( \mathcal{L}^{h \to H} \) を設定する。ラグランジュ乗数を \( \mu_t^{h \to H} \) とする。
\[
\mathcal{L}^{h \to H} = \sum_{h' \in H} p_t^{h'} c_t^{h \to h'} - \mu_t^{h \to H} \left( \left[ \sum_{h' \in H} (c_t^{h \to h'})^{\frac{\theta^H-1}{\theta^H}} \right]^{\frac{\theta^H}{\theta^H-1}} - c_t^{h \to H} \right)
\]
このラグランジュ関数を、任意の個別財 \( c_t^{h \to h'} \) で偏微分し、ゼロと置くことで一階の条件( FOC )が得られる。
\begin{align*}
\frac{\partial \mathcal{L}^{h \to H}}{\partial c_t^{h \to h'}} = p_t^{h'} - \mu_t^{h \to H} \cdot \frac{\theta^H}{\theta^H-1} \left[ \sum_{i \in H} (c_t^{h \to i})^{\frac{\theta^H-1}{\theta^H}} \right]^{\frac{\theta^H}{\theta^H-1}-1} \cdot \frac{\theta^H-1}{\theta^H} (c_t^{h \to h'})^{\frac{\theta^H-1}{\theta^H}-1} &= 0 \\
p_t^{h'} &= \mu_t^{h \to H} \cdot \left( c_t^{h \to H} \right)^{\frac{1}{\theta^H}} \cdot (c_t^{h \to h'})^{-\frac{1}{\theta^H}} \\
p_t^{h'} &= \mu_t^{h \to H} \left( \frac{c_t^{h \to H}}{c_t^{h \to h'}} \right)^{\frac{1}{\theta^H}}
\end{align*}

\subsection*{3. FOC からの需要関数の導出}
上記で得られた FOC を、個別財への需要量 \( c_t^{h \to h'} \) について解く。
\begin{align*}
    \frac{p_t^{h'}}{\mu_t^{h \to H}} &= \left( \frac{c_t^{h \to H}}{c_t^{h \to h'}} \right)^{\frac{1}{\theta^H}} \\
    \left(\frac{p_t^{h'}}{\mu_t^{h \to H}}\right)^{\theta^H} &= \frac{c_t^{h \to H}}{c_t^{h \to h'}}
\end{align*}
これにより、ラグランジュ乗数 \( \mu_t^{h \to H} \) を含む需要関数が導出される。
\[ c_t^{h \to h'} = \left( \frac{p_t^{h'}}{\mu_t^{h \to H}} \right)^{-\theta^H} c_t^{h \to H} \]

\subsection*{4. 真の価格指数 \( p_t^H \) の導出}
次に、ラグランジュ乗数 \( \mu_t^{h \to H} \) の具体的な形を導出する。ステップ 3 で得た需要関数を、制約条件である CES 集計式に代入する。
\[
c_t^{h \to H} = \left[ \sum_{h' \in H} \left( \left( \frac{p_t^{h'}}{\mu_t^{h \to H}} \right)^{-\theta^H} c_t^{h \to H} \right)^{\frac{\theta^H-1}{\theta^H}} \right]^{\frac{\theta^H}{\theta^H-1}}
\]
式を整理していく。
\begin{align*}
    c_t^{h \to H} &= \left[ \sum_{h' \in H} \left( \frac{p_t^{h'}}{\mu_t^{h \to H}} \right)^{1-\theta^H} (c_t^{h \to H})^{\frac{\theta^H-1}{\theta^H}} \right]^{\frac{\theta^H}{\theta^H-1}} \\
    &= \left[ (c_t^{h \to H})^{\frac{\theta^H-1}{\theta^H}} (\mu_t^{h \to H})^{\theta^H-1} \sum_{h' \in H} (p_t^{h'})^{1-\theta^H} \right]^{\frac{\theta^H}{\theta^H-1}} \\
    &= (c_t^{h \to H}) \cdot (\mu_t^{h \to H})^{\theta^H} \cdot \left[ \sum_{h' \in H} (p_t^{h'})^{1-\theta^H} \right]^{\frac{\theta^H}{\theta^H-1}}
\end{align*}
両辺の \( c_t^{h \to H} \) を消去し、 \( \mu_t^{h \to H} \) について解く。
\begin{align*}
    1 &= (\mu_t^{h \to H})^{\theta^H} \left[ \sum_{h' \in H} (p_t^{h'})^{1-\theta^H} \right]^{\frac{\theta^H}{\theta^H-1}} \\
    (\mu_t^{h \to H})^{-\theta^H} &= \left[ \sum_{h' \in H} (p_t^{h'})^{1-\theta^H} \right]^{\frac{\theta^H}{\theta^H-1}} \\
    \mu_t^{h \to H} &= \left[ \sum_{h' \in H} (p_t^{h'})^{1-\theta^H} \right]^{\frac{1}{1-\theta^H}}
\end{align*}
このラグランジュ乗数 \( \mu_t^{h \to H} \) は、個々の家計 \( h \) に依存しない共通の価格リストのみで決定されるため、全ての国内家計で共通の値をとる。
本稿では、この家計が直面する真の価格指数を \( p_t^H \) と定義する。
\[ p_t^H \equiv \mu_t^{h \to H} = \left[ \sum_{h' \in H} (p_t^{h'})^{1-\theta^H} \right]^{\frac{1}{1-\theta^H}} \]
この \( p_t^H \) をステップ 3 の需要関数に代入することで、定理の項目 1 が証明される( 証明終 )。      % 家計の需要関数
% !TeX root = ../../main.tex
% sections/app/appendix_cpi_derivation.tex

\chapter{総消費価格指数と財バスケットへの需要関数の導出( 第 2 段階 )}
\label{chap:appendix_cpi_derivation}

\section*{定理 3:財バスケットへの需要関数と総消費価格指数}

家計 \( h \) が、ある時点 \( t \) において、所与の量の総消費バスケット \( c_t^{h \to W} \) を、国内財バスケットの真の価格指数 \( p_t^H \) と外国財バスケットの真の価格指数 \( p_t^{F*} \) の下で費用最小化的に購入するとき、以下の関係が成立する。
\begin{enumerate}
    \item 国内財バスケット \( c_t^{h \to H} \) および外国財バスケット \( c_t^{h \to F} \) への需要関数は、それぞれ次式で与えられる。
    \[ c_t^{h \to H} = \alpha^H \frac{p_t^{H \to W}}{p_t^H} c_t^{h \to W} \]
    \[ c_t^{h \to F} = (1-\alpha^H) \frac{p_t^{H \to W}}{e_t p_t^{F*}} c_t^{h \to W} \]
    \item 上記の需要関数に現れる \( p_t^{H \to W} \) は、総消費バスケットの価格指数( CPI )であり、各財バスケットの真の価格指数から以下のように定義される。
    \[ p_t^{H \to W} = (p_t^H)^{\alpha^H} (e_t p_t^{F*})^{1-\alpha^H} \]
\end{enumerate}

\section*{証明}

この定理を、自国家計 \( h \) の場合について証明する。

\subsection*{1. 費用最小化問題の設定}
この第 2 段階では、家計が時点 \( t \) において、所与の総消費量 \( c_t^{h \to W} \) を達成するために、国内財バスケット \( c_t^{h \to H} \) と海外財バスケット \( c_t^{h \to F} \) をどのように組み合わせれば総費用を最小化できるかを分析する。
\begin{itemize}
    \item \textbf{最小化すべき総費用}:
    \[ \min_{c_t^{h \to H}, c_t^{h \to F}} \quad p_t^H c_t^{h \to H} + e_t p_t^{F*} c_t^{h \to F} \]
    \item \textbf{制約条件}( 目標とする正規化されたバスケットの量 ):
    \[ c_t^{h \to W} \equiv \frac{(c_t^{h \to H})^{\alpha^H} (c_t^{h \to F})^{1-\alpha^H}}{(\alpha^H)^{\alpha^H} (1-\alpha^H)^{1-\alpha^H}} \]
\end{itemize}

\subsection*{2. ラグランジュ関数と一階の条件( FOC )}
制約式の対数を取ると扱いやすい。ラグランジュ関数 \( \mathcal{L}^{h \to W} \) を設定する( この段階のラグランジュ乗数を \( \eta_t^h \) とする )。
\[ \mathcal{L}^{h \to W} = p_t^H c_t^{h \to H} + e_t p_t^{F*} c_t^{h \to F} - \eta_t^h \left( \alpha^H \ln c_t^{h \to H} + (1-\alpha^H) \ln c_t^{h \to F} - \ln c_t^{h \to W} - \text{const.} \right) \]
これを \( c_t^{h \to H} \) と \( c_t^{h \to F} \) でそれぞれ偏微分してゼロと置くと、以下の一階の条件( FOC )が得られる。
\begin{align*}
\frac{\partial \mathcal{L}^{h \to W}}{\partial c_t^{h \to H}} = p_t^H - \eta_t^h \frac{\alpha^H}{c_t^{h \to H}} = 0 \quad &\implies \quad p_t^H c_t^{h \to H} = \alpha^H \eta_t^h \\
\frac{\partial \mathcal{L}^{h \to W}}{\partial c_t^{h \to F}} = e_t p_t^{F*} - \eta_t^h \frac{1-\alpha^H}{c_t^{h \to F}} = 0 \quad &\implies \quad e_t p_t^{F*} c_t^{h \to F} = (1-\alpha^H) \eta_t^h
\end{align*}

\subsection*{3. 需要関数の導出( \( \eta_t^h \) を含む形 )}
上記の一階の条件をそれぞれ \( c_t^{h \to H} \) と \( c_t^{h \to F} \) について解くと、ラグランジュ乗数 \( \eta_t^h \) を含む形で各バスケットへの需要関数が得られる。
\[ c_t^{h \to H} = \frac{\alpha^H \eta_t^h}{p_t^H} \quad , \quad c_t^{h \to F} = \frac{(1-\alpha^H) \eta_t^h}{e_t p_t^{F*}} \]

\subsection*{4. ラグランジュ乗数 \( \eta_t^h \) の導出}
ステップ 3 で得た需要関数を、制約条件である総消費バスケットの定義式に代入する。
\begin{align*}
c_t^{h \to W} &= \frac{1}{(\alpha^H)^{\alpha^H} (1-\alpha^H)^{1-\alpha^H}} \left( \frac{\alpha^H \eta_t^h}{p_t^H} \right)^{\alpha^H} \left( \frac{(1-\alpha^H) \eta_t^h}{e_t p_t^{F*}} \right)^{1-\alpha^H} \\
&= \frac{(\eta_t^h)^{\alpha^H + (1-\alpha^H)}}{(\alpha^H)^{\alpha^H} (1-\alpha^H)^{1-\alpha^H}} \cdot \frac{(\alpha^H)^{\alpha^H} (1-\alpha^H)^{1-\alpha^H}}{(p_t^H)^{\alpha^H} (e_t p_t^{F*})^{1-\alpha^H}} \\
c_t^{h \to W} &= \eta_t^h \cdot \frac{1}{(p_t^H)^{\alpha^H} (e_t p_t^{F*})^{1-\alpha^H}}
\end{align*}
この式をラグランジュ乗数 \( \eta_t^h \) について解く。
\[ \eta_t^h = c_t^{h \to W} (p_t^H)^{\alpha^H} (e_t p_t^{F*})^{1-\alpha^H} \]

\subsection*{5. 総合物価指数 \( p_t^{H \to W} \) と最終的な需要関数の導出}
名目総消費は、最小化すべき総費用 \( p_t^H c_t^{h \to H} + e_t p_t^{F*} c_t^{h \to F} \) として定義される。ステップ 2 の FOC より、これは \( \alpha^H \eta_t^h + (1-\alpha^H) \eta_t^h = \eta_t^h \) に等しい。一方で、名目総消費は \( p_t^{H \to W} c_t^{h \to W} \) とも定義されるため、 \( p_t^{H \to W} c_t^{h \to W} = \eta_t^h \) という関係が成立しなければならない。

この関係式にステップ 4 で導出した \( \eta_t^h \) の式を代入する。
\[ p_t^{H \to W} c_t^{h \to W} = c_t^{h \to W} (p_t^H)^{\alpha^H} (e_t p_t^{F*})^{1-\alpha^H} \]
両辺の \( c_t^{h \to W} \) を消去することで、定理の項目 2 で示された\textbf{総合物価指数 \( p_t^{H \to W} \)} が導かれる。
\[ p_t^{H \to W} = (p_t^H)^{\alpha^H} (e_t p_t^{F*})^{1-\alpha^H} \]
最後に、この \( \eta_t^h = p_t^{H \to W} c_t^{h \to W} \) という関係をステップ 3 で得た需要関数に代入することで、定理の項目 1 で示された最終的な財バスケットへの需要関数が完成する。
\[ c_t^{h \to H} = \frac{\alpha^H (p_t^{H \to W} c_t^{h \to W})}{p_t^H} = \alpha^H \frac{p_t^{H \to W}}{p_t^H} c_t^{h \to W} \]
\[ c_t^{h \to F} = \frac{(1-\alpha^H) (p_t^{H \to W} c_t^{h \to W})}{e_t p_t^{F*}} = (1-\alpha^H) \frac{p_t^{H \to W}}{e_t p_t^{F*}} c_t^{h \to W} \]
( 証明終 )         % CPI の定義
% !TeX root = ../../main.tex
% sections/app/appendix_production_aggregation.tex

\chapter{生産関数の集計と価格分散}
\label{chap:appendix_production_aggregation}

本付録では、個々の家計の生産関数から出発し、価格の異質性が存在する経済における国全体の集計生産関数を導出する。

\section*{定理 5:価格分散を考慮した集計生産関数}

個人の生産関数が \( y_t^h = a_t^H l_t^h \) であり、個別財への需要が \( y_t^h = ( p_t^h / p_t^H )^{-\theta^H} Y_t^H \) で与えられる経済において、国全体の集計生産関数は以下のように表される。
\[
Y_t^H = \frac{a_t^H L_t^H}{\Delta_t^H}
\]
ここで、 \( L_t^H \) は総労働投入量 \( \sum_{h \in H} l_t^h \) であり、 \( \Delta_t^H \) は経済全体の非効率性を示す価格分散項であり、次式で定義される。
\[
\Delta_t^H \equiv \sum_{h \in H} \left( \frac{p_t^h}{p_t^H} \right)^{-\theta^H}
\]

\section*{証明}

この定理を証明するために、総労働投入量 \( L_t^H \) の定義式から出発し、集計生産関数を導出する。

\subsection*{1. 総労働の定義式からの展開}
国全体の総労働投入量 \( L_t^H \) は、全ての家計の労働投入量の合計である。
\[
L_t^H = \sum_{h \in H} l_t^h
\]
この式に、個人の生産関数 \( y_t^h = a_t^H l_t^h \) を \( l_t^h \) について解いた \( l_t^h = y_t^h / a_t^H \) を代入する。国全体の生産性 \( a_t^H \) は全ての家計で共通であるため、総和の外に出すことができる。
\begin{align*}
L_t^H &= \sum_{h \in H} \frac{y_t^h}{a_t^H} \\
&= \frac{1}{a_t^H} \sum_{h \in H} y_t^h
\end{align*}
次に、個別財への需要関数 \( y_t^h = \left( p_t^h / p_t^H \right)^{-\theta^H} Y_t^H \) を代入する。集計産出量 \( Y_t^H \) は個々の家計 \( h \) に依存しないため、これも総和の外に出すことができる。
\begin{align*}
L_t^H &= \frac{1}{a_t^H} \sum_{h \in H} \left[ \left( \frac{p_t^h}{p_t^H} \right)^{-\theta^H} Y_t^H \right] \\
&= \frac{Y_t^H}{a_t^H} \sum_{h \in H} \left( \frac{p_t^h}{p_t^H} \right)^{-\theta^H}
\end{align*}

\subsection*{2. 価格分散項の定義と結論}
ここで、総和の部分を価格分散項 \( \Delta_t^H \) として定義する。
\[
\Delta_t^H \equiv \sum_{h \in H} \left( \frac{p_t^h}{p_t^H} \right)^{-\theta^H}
\]
この定義をステップ 1 で得られた式に代入すると、総労働 \( L_t^H \) と総生産 \( Y_t^H \) の間に以下の厳密な関係式が成立する。
\[
L_t^H = \frac{Y_t^H}{a_t^H} \Delta_t^H
\]
この式を \( Y_t^H \) について解くことで、定理で示された集計生産関数が得られる( 証明終 )。
\[
Y_t^H = \frac{a_t^H L_t^H}{\Delta_t^H}
\] % 生産関数の集計
% !TeX root = ../../main.tex
% sections/app/appendix_aggregate_output.tex

\chapter{集計産出量と名目所得の関係}
\label{chap:appendix_aggregate_output}

本付録の目的は、総名目生産額が、集計産出量と真の価格指数の単純な積で表せること、そしてその関係を担保するために集計産出量が必然的にCES集計の形でなければならないことを証明することである。

\section*{定理4:総名目所得の分解と集計産出量の整合性}

財市場の均衡において、ある時点 \( t \) で、以下の 2 つの関係が成立する。
\begin{enumerate}
    \item 国全体の総名目所得は、集計産出量 \( Y_t^H \) と真の価格指数 \( p_t^H \) の積に等しい。
    \[ \sum_{h \in H} p_t^h y_t^h = p_t^H Y_t^H \]
    \item 上記の関係式が成立するためには、集計産出量 \( Y_t^H \) は、個々の財の生産量 \( y_t^h \) のCES集計関数でなければならない。
    \[ Y_t^H = \left[ \sum_{h \in H} (y_t^h)^{\frac{\theta^H-1}{\theta^H}} \right]^{\frac{\theta^H}{\theta^H-1}} \]
\end{enumerate}

\section*{証明}

\subsection*{1. 個別生産量と集計量の関係式の導出}
まず、個別財 \( h \) の生産量 \( y_t^h \) と集計産出量 \( Y_t^H \) の間の関係を厳密に導出する。財市場の均衡では、個別財 \( h \) の生産量( 供給 )は、その財への全世界からの総需要と等しい。
\[ y_t^h = \sum_{h' \in H} c_t^{h' \to h} + \sum_{f \in F} c_t^{f \to h} \]
ここに、各消費者( 国内家計 \( h' \)、外国の家計 \( f \) )の個別財への需要関数を代入する。買い手が国内か国外かに関わらず、彼らが直面する自国財の価格体系は共通であるため、需要関数 \( c_t^{\cdot \to h} = (p_t^h/p_t^H)^{-\theta^H} c_t^{\cdot \to H} \) の価格に関する項は共通となる。
\begin{align*}
y_t^h &= \sum_{h' \in H} \left[ \left( \frac{p_t^h}{p_t^H} \right)^{-\theta^H} c_t^{h' \to H} \right] + \sum_{f \in F} \left[ \left( \frac{p_t^h}{p_t^H} \right)^{-\theta^H} c_t^{f \to H} \right] \\
&= \left( \frac{p_t^h}{p_t^H} \right)^{-\theta^H} \left[ \sum_{h' \in H} c_t^{h' \to H} + \sum_{f \in F} c_t^{f \to H} \right]
\end{align*}
ここで、大括弧の中の項は「 世界全体からの国内財バスケットへの総需要 」を意味する。財市場全体の均衡において、これは国内財の集計産出量 \( Y_t^H \) と定義上一致する。
\[ Y_t^H \equiv \sum_{h' \in H} c_t^{h' \to H} + \sum_{f \in F} c_t^{f \to H} \]
したがって、以下の厳密な関係式が導かれる。
\[ y_t^h = \left( \frac{p_t^h}{p_t^H} \right)^{-\theta^H} Y_t^H \]

\subsection*{2. 総名目所得の計算}
次に、総名目所得 \( \sum_{h \in H} p_t^h y_t^h \) を計算する。ステップ 1 で導出した関係式を代入する。
\begin{align*}
    \sum_{h \in H} p_t^h y_t^h &= \sum_{h \in H} p_t^h \left[ \left( \frac{p_t^h}{p_t^H} \right)^{-\theta^H} Y_t^H \right] \\
    &= Y_t^H (p_t^H)^{\theta^H} \sum_{h \in H} (p_t^h)^{1-\theta^H}
\end{align*}
ここで、付録 \ref{chap:appendix_cost_minimization} で導出した真の価格指数 \( p_t^H \) の定義式の両辺を \( 1-\theta^H \) 乗すると、 \( \sum_{h \in H} (p_t^h)^{1-\theta^H} = (p_t^H)^{1-\theta^H} \) という関係が得られる。これを上式に代入する。
\begin{align*}
    \sum_{h \in H} p_t^h y_t^h &= Y_t^H (p_t^H)^{\theta^H} (p_t^H)^{1-\theta^H} \\
    &= p_t^H Y_t^H
\end{align*}
これにより、定理の項目 1 が証明された。

\subsection*{3. 集計産出量 \( Y_t^H \) の定義の整合性}
最後に、この関係式が成り立つために、集計産出量 \( Y_t^H \) が必然的にCES集計の形でなければならないことを示す。ステップ 1 の関係式の両辺を \( \frac{\theta^H-1}{\theta^H} \) 乗し、全ての \( h \) について合計を取る。
\begin{align*}
\sum_{h \in H} (y_t^h)^{\frac{\theta^H-1}{\theta^H}} &= \sum_{h \in H} \left[ \left( \frac{p_t^h}{p_t^H} \right)^{-\theta^H} Y_t^H \right]^{\frac{\theta^H-1}{\theta^H}} \\
&= \sum_{h \in H} \left( \frac{p_t^h}{p_t^H} \right)^{1-\theta^H} (Y_t^H)^{\frac{\theta^H-1}{\theta^H}} \\
&= \frac{(Y_t^H)^{\frac{\theta^H-1}{\theta^H}}}{(p_t^H)^{1-\theta^H}} \sum_{h \in H} (p_t^h)^{1-\theta^H}
\end{align*}
再び価格指数の定義 \( (p_t^H)^{1-\theta^H} = \sum_{h \in H} (p_t^h)^{1-\theta^H} \) を使うと、右辺の価格項は相殺される。
\[
\sum_{h \in H} (y_t^h)^{\frac{\theta^H-1}{\theta^H}} = (Y_t^H)^{\frac{\theta^H-1}{\theta^H}}
\]
この両辺の \( \frac{\theta^H}{\theta^H-1} \) 乗を取ると、 \( Y_t^H \) が定理の項目 2 で示されたCES集計関数でなければならないことが示される( 証明終 )。      % 集計生産量と資源制約
% !TeX root = ../../main.tex
% sections/app/appendix_equivalence.tex

\chapter{効用最大化問題と 2 段階最適化の等価性}
\label{chap:appendix_equivalence}

本付録では、本稿のモデル構築において採用している「 個別財の費用最小化問題を解き、そこで得られた価格指数を用いて効用最大化問題を解く 」という 2 段階の最適化アプローチが、全ての個別財の消費量を一度に選択するという単一の効用最大化問題と数学的に完全に等価であることを証明する。



\section*{定理 1:最適化問題の等価性}

家計 \( h \) が、ある時点 \( t \) において所与の総支出額 \( E_t^h \) の下で、全ての個別財 \( \{c_t^{h \to h'}\} \), \( \{c_t^{h \to f'}\} \) の消費量を直接選択する単一の効用最大化問題は、以下の 2 段階の最適化問題と等価である。
\begin{enumerate}
    \item \textbf{第 1 段階( 費用最小化 )}: 所与の財バスケット量 \( c_t^{h \to H} \), \( c_t^{h \to F} \) を達成するための最小費用と、その際の個別財への需要を求める。この過程で、真の価格指数 \( p_t^H \), \( p_t^{F*} \) が導出される。
    \item \textbf{第 2 段階( 効用最大化 )}: 第 1 段階で導出された価格指数を所与として、予算制約 \( p_t^H c_t^{h \to H} + e_t p_t^{F*} c_t^{h \to F} = E_t^h \) の下で、財バスケット \( c_t^{h \to H} \), \( c_t^{h \to F} \) の最適な組み合わせを選択し、効用を最大化する。
\end{enumerate}

\section*{証明}

この定理を証明するために、出発点となる厳密な単一の効用最大化問題を、数学的に等価な変形を施すことで、 2 段階の最適化問題へと帰着させる。

\subsection*{1. 出発点:単一の効用最大化問題}
厳密な問題設定は、時点 \( t \) において、以下の通りである。
\begin{itemize}
    \item \textbf{最大化対象}:
    \[
    \max_{\{c_t^{h \to h'}\}, \{c_t^{h \to f'}\}} \ln(c_t^{h \to W})
    \]
    ただし、 \( c_t^{h \to W} \), \( c_t^{h \to H} \), \( c_t^{h \to F} \) は各個別財消費量の関数である。

    \item \textbf{制約条件}:
    \[
    \sum_{h' \in H} p_t^{h'} c_t^{h \to h'} + \sum_{f' \in F} p_t^{f'} c_t^{h \to f'} = E_t^h
    \]
\end{itemize}

\subsection*{2. 問題の分解}
この最大化問題は、総支出 \( E_t^h \) を国内財への支出 \( E_t^{h \to H} \) と海外財への支出 \( E_t^{h \to F} \) に分割する、以下のネストした( 入れ子構造の )問題と等価である。
\[
\max_{E_t^{h \to H}, E_t^{h \to F}} \left( \max_{\{c_t^{h \to h'}\}, \{c_t^{h \to f'}\}} \ln(c_t^{h \to W}) \quad \text{s.t.} \sum p_t^{h'}c_t^{h \to h'} = E_t^{h \to H}, \sum p_t^{f'}c_t^{h \to f'} = E_t^{h \to F} \right)
\]
\[
\text{subject to} \quad E_t^{h \to H} + E_t^{h \to F} = E_t^h
\]

\subsection*{3. 双対性( Duality )の利用}
ここで、内側の最大化問題に注目する。例えば国内財については、「 所与の支出 \( E_t^{h \to H} \) で、バスケット \( c_t^{h \to H} \) の量を最大化する問題 」である。
\[
\max_{\{c_t^{h \to h'}\}} c_t^{h \to H}(\{c_t^{h \to h'}\}) \quad \text{s.t.} \quad \sum_{h' \in H} p_t^{h'}c_t^{h \to h'} = E_t^{h \to H}
\]
ミクロ経済学の双対性( duality )の原理により、この問題は、「 所与のバスケット量 \( \bar{c}_t^{h \to H} \) を、最小の費用で達成する問題 」と完全に等価である。
\[
\min_{\{c_t^{h \to h'}\}} \sum_{h' \in H} p_t^{h'}c_t^{h \to h'} \quad \text{s.t.} \quad c_t^{h \to H}(\{c_t^{h \to h'}\}) = \bar{c}_t^{h \to H}
\]
この費用最小化問題こそが、 2 段階アプローチにおける第 1 段階に他ならない。本稿の付録 \ref{chap:appendix_cost_minimization} で示したように、この問題を解くことで、所与のバスケット量 \( c_t^{h \to H} \) を達成するための最小費用( 支出関数 )は \( p_t^H c_t^{h \to H} \) となることが導かれる。同様に、外国財バスケットの最小費用は \( e_t p_t^{F*} c_t^{h \to F} \) となる。

\subsection*{4. 問題の再定式化と結論}
この支出関数の関係 \( E_t^{h \to H} = p_t^H c_t^{h \to H} \) と \( E_t^{h \to F} = e_t p_t^{F*} c_t^{h \to F} \) を、ステップ 2 で分解した問題に代入する。すると、問題は以下のように書き換えられる。
\begin{itemize}
    \item \textbf{最大化}:
    \[
    \max_{c_t^{h \to H}, c_t^{h \to F}} \ln \left( \frac{(c_t^{h \to H})^{\alpha^H} (c_t^{h \to F})^{1-\alpha^H}}{(\alpha^H)^{\alpha^H} (1-\alpha^H)^{1-\alpha^H}} \right)
    \]
    \item \textbf{制約条件}:
    \[
    p_t^H c_t^{h \to H} + e_t p_t^{F*} c_t^{h \to F} = E_t^h
    \]
\end{itemize}
この書き換えられた問題は、まさしく 2 段階アプローチにおける第 2 段階そのものである。

以上により、厳密な単一の効用最大化問題が、本稿で採用した 2 段階の最適化アプローチと数学的に完全に等価であることが証明された( 証明終 )。           % 変数の同値関係( 再帰的価格等 )
% !TeX root = ../../main.tex
% sections/app/appendix_resource_constraint.tex

\chapter{国全体の資源制約式の導出}
\label{chap:appendix_resource_constraint}

本付録では、個々の家計の予算制約式を集計し、国内市場の均衡条件を適用することで、国全体の資源制約式を導出する。

\section*{定理 7:国全体の資源制約式}

国内金融市場が完備であり、政府が均衡予算を達成する経済において、全ての個人の予算制約式を集計すると、以下の国全体の資源制約式が得られる。
\[
p_t^{H \to W} C_t^{H \to W} + B_{t+1}^{H} = p_t^H Y_t^H + (1+i_{t-1}^F) \frac{e_t}{e_{t-1}} B_t^{H}
\]
ここで、 \( C_t^{H \to W} \) は国全体の総消費、 \( Y_t^H \) は集計産出量、 \( B_{t+1}^{H} \) は期末の対外純資産( 自国通貨建て )を表す。

\section*{証明}

\subsection*{1. 国全体での集計}
まず、家計 \( h \) の期ごとの名目予算制約式を、国内の全ての家計 \( h \in H \) について足し合わせる( \( \sum_{h \in H} \) )。全ての変数は時点 \( t \) に依存する。
\[
\begin{split}
    & \sum_{h \in H} \left( \sum_{j' \in J} q_{t+1}(j') d_{t+1}^{h \to H}(j') + b_{t+1}^{h \to F} + p_t^{H \to W} c_t^{h \to W} \right) \\
    & \qquad = \sum_{h \in H} \left( d_t^{h \to H} + (1+i_{t-1}^F) \frac{e_t}{e_{t-1}} b_t^{h \to F} + (1-\tau_t^H) p_t^h y_t^h + t_t^H \right)
\end{split}
\]

\subsection*{2. 国内取引の相殺}
次に、国全体で集計するとゼロになる国内完結の取引を相殺する。

\paragraph{国内債券市場の均衡}
国内で取引される状態コンティンジェント債券は、国内の誰かの負債が他の誰かの資産となるゼロサム取引であるため、純供給はゼロである。
\[ \sum_{h \in H} d_{t}^{h \to H} = 0 \quad \text{and} \quad \sum_{h \in H} \sum_{j'} q_{t+1}(j') d_{t+1}^{h \to H}(j') = 0 \]
これにより、集計した式から国内債券に関する項( \( d \) )はすべて消去される。

\paragraph{政府部門の予算制約}
政府は均衡予算を達成し、税収のすべてを家計への移転に使うため、移転の総額と税収の総額は一致する。
\[
\sum_{h \in H} t_t^H = \sum_{h \in H} \tau_t^H p_t^h y_t^h
\]
この関係を用いると、集計後の予算制約式の右辺にある移転項 \( \sum t_t^H \) は、所得項の一部である税金部分 \( \sum \tau_t^H p_t^h y_t^h \) と相殺される。

\subsection*{3. 集計変数への書き換え}
国内取引が相殺された結果、式に残るのは以下の項のみである。
\[
\sum_{h \in H} \left( b_{t+1}^{h \to F} + p_t^{H \to W} c_t^{h \to W} \right) = \sum_{h \in H} \left( (1+i_{t-1}^F) \frac{e_t}{e_{t-1}} b_t^{h \to F} + p_t^h y_t^h \right)
\]
この式を、国全体の集計変数( 大文字の変数 )を使って書き換える。

\paragraph{消費と対外資産の集計}
価格指数 \( p_t^{H \to W} \) は全ての家計で共通であるため、総和 \( \sum_{h \in H} \) の外に出すことができる。
\[
\sum_{h \in H} p_t^{H \to W} c_t^{h \to W} = p_t^{H \to W} \sum_{h \in H} c_t^{h \to W} \equiv p_t^{H \to W} C_t^{H \to W}
\]
対外純資産( 自国通貨建て )については、単純な総和として定義される。
\[
\sum_{h \in H} b_{t+1}^{h \to F} \equiv B_{t+1}^{H}
\]
同様に、前期からの対外資産の償還額も \( (1+i_{t-1}^F) \frac{e_t}{e_{t-1}} B_t^{H} \) となる。

\paragraph{名目総生産の集計}
付録 \ref{chap:appendix_aggregate_output} で証明した通り、総名目所得は \( \sum_{h \in H} p_t^h y_t^h = p_t^H Y_t^H \) という関係が厳密に成立する。

\subsection*{4. 結論}
以上の集計結果をまとめると、定理で示された国全体の資源制約式が導出される( 証明終 )。
\[
p_t^{H \to W} C_t^{H \to W} + B_{t+1}^{H} = p_t^H Y_t^H + (1+i_{t-1}^F) \frac{e_t}{e_{t-1}} B_t^{H}
\]    % 資源制約の線形化
% !TeX root = ../../main.tex
% sections/app/appendix_risk_sharing.tex

\chapter{完備市場における消費の共通化と貯蓄の個別化}
\label{chap:appendix_risk_sharing}

本付録では、国内完備市場とカルボ型価格設定を仮定したモデルにおいて、なぜ全ての家計の消費が共通化される一方、貯蓄( 資産ポートフォリオ )は家計ごとに個別化されるのかを厳密に証明する。

\section*{定理 9:リスク共有の結果}

国内金融市場が完備である経済において、以下の関係が成立する。
\begin{enumerate}
    \item 全ての国内家計 \( h \in H \) の所得の限界効用は、いかなる時点 \( t \) 、いかなる状態 \( j \) においても常に一致する。
    \[ \lambda_t^h = \lambda_t^{h'} \quad \forall h, h' \in H \]
    \item 全ての国内家計 \( h \in H \) の総消費指数は、常に一致する。
    \[ c_t^{h \to W} = c_t^{h' \to W} \quad \forall h, h' \in H \]
    \item 各家計が購入する次期のための資産ポートフォリオの総価値は、家計が当期に価格改定の機会を得たか否かによって異なるため、一般に一致しない。
\end{enumerate}

\section*{証明}

\subsection*{1. 所得の限界効用 \( \lambda_t \) の一致( 定理 9-1 の証明 )}

\paragraph{ステップ A:限界効用の比率の不変性}
家計の最適化行動は全ての時点・状態で成立するため、任意の 2 つの国内家計 \( h \) と \( h' \) の国内コンティンジェント債券に関する一階の条件( FOC )は、常に成立する。
\[
\begin{aligned}
\lambda_t^h q_{t+1} &= \beta_t^H \pi \lambda_{t+1}^h \\
\lambda_t^{h'} q_{t+1} &= \beta_t^H \pi \lambda_{t+1}^{h'}
\end{aligned}
\]
これら 2 式の比を取ると共通項が消去され、以下の関係が得られる。
\[
\frac{\lambda_t^h}{\lambda_t^{h'}} = \frac{\lambda_{t+1}^h}{\lambda_{t+1}^{h'}} = k
\]
この比率 \( k \) は、時間や状態に依存しない普遍的な定数である。

\paragraph{ステップ B:定数 \( k \) の値の特定}
第 \ref{sec:model_overview} 節で定義した「 事前的対称性 」の仮定を用いる。初期時点 \( s \) において、全ての家計は同一の選好をもち、かつ初期貯蓄がゼロ( \( d_s^h = d_s^{h'} = 0 \) )である。このとき、各家計が直面する最適化問題は数学的に完全に同一であるため、初期の所得の限界効用は全ての家計で一致しなければならない。
\[
\lambda_s^h = \lambda_s^{h'} \quad \Longrightarrow \quad k = \frac{\lambda_s^h}{\lambda_s^{h'}} = 1
\]

\paragraph{ステップ C:結論}
普遍的な定数が \( k=1 \) であることから、将来のすべての時点・すべての状態において \( \lambda_t^h = \lambda_t^{h'} \) が成立する。

\subsection*{2. 総消費指数 \( c_t^{h \to W} \) の一致( 定理 9-2 の証明 )}
全ての家計の \( \lambda_t \) が一致するという結果を、消費に関する FOC( 式 \ref{eq:model_foc_consumption_home} )に適用する。
\[
\lambda_t = \frac{1}{p_t^{H \to W} c_t^{h \to W}}
\]
左辺の \( \lambda_t \) と右辺の物価指数 \( p_t^{H \to W} \) は全ての家計で共通であるため、総消費指数 \( c_t^{h \to W} \) もまた、全ての家計間で完全に一致しなければならない。

\subsection*{3. ポートフォリオ購入総額の個別化( 定理 9-3 の証明 )}
家計 \( h \) の予算制約式を、次期のために購入する資産ポートフォリオの総価値 \( v_{t+1}^h \equiv \sum_{j'} q_{t+1} d_{t+1}^{h \to H} + e_t b_{t+1}^{h \to F*} \) について整理する。
\[
v_{t+1}^h = \underbrace{\left( d_t^{h \to H} + (1+i_{t-1}^F)e_t b_t^{h \to F*} + (1 - \tau_t^H) p_t^h y_t^h + t_t^H \right)}_{\text{当期の総収入}} - \underbrace{p_t^{H \to W} c_t^{h \to W}}_{\text{当期の消費支出}}
\]
定理 9-2 より消費支出は共通であるが、当期の総収入、特に労働所得 \( (1 - \tau_t^H) p_t^h y_t^h \) はカルボ型の価格設定( \( p_t^h \) の異質性 )によって家計ごとに異なる。支出が共通で収入が異なる以上、その差額を埋める資産保有額 \( v_{t+1}^h \) は、各家計が直面したショックに応じて個別化される。( 証明終 )          % リスクシェアリング条件

% =============================================================
% 2. 第 4 章( モデル構築 )で参照される基礎的な導出
% =============================================================
% !TeX root = ../../main.tex
% sections/app/appendix_optimal_price_derivation.tex

\chapter{ニューケインジアン・フィリップス曲線の詳細な導出}
\label{chap:appendix_optimal_price_derivation}

本付録では、本文第 \ref{chap:results} 章で用いたニューケインジアン・フィリップス曲線を一切の省略なく導出する。

\section{価格決定家計のラグランジアンと最適化の前提}
\label{sec:appendix_optimal_price_derivation_lagrangian}

価格改定の機会を得た家計は、任意の時点 \( t \) において以下の目的関数を最大化する価格 \( \widetilde{p}_t^h \) を選択する。

\[
\begin{aligned}
\max_{\widetilde{p}_t^h} & \Biggl[ \left( \log c_t^{h \to W} - \frac{\phi^H}{2}(l_t^{h})^2 \right) \\
& \quad + \lambda_{t}^{H} \biggl( \Bigl( d_{t}^{h \to H} + (1+i_{t-1}^F)e_t^{/*} b_{t}^{h \to F} + (1 - \tau_{t}^{H}) \widetilde{p}_t^h y_{t}^{h} + t_{t}^{H} \Bigr) \\
& \quad - \Bigl( \sum_{j' \in J} q_{t,t+1}(j') d_{t+1}^{h \to H}(j') + e_{t}^{/*} b_{t+1}^{h \to F} + p_t^{H \to W} c_t^{h \to W} \Bigr) \biggr) \Biggr] \\
& + E_t \Biggl[ \sum_{k=1}^{\infty} (\xi^H)^k \left(\prod_{j=0}^{k-1} \beta_{t+j}^H \right) \Biggl\{ \left( \log c_{t+k}^{h \to W} - \frac{\phi^H}{2}(l_{t+k}^{h})^2 \right) \\
& \quad + \lambda_{t+k}^{H} \biggl( \Bigl( d_{t+k}^{h \to H} + (1+i_{t+k-1}^F)e_{t+k}^{/*} b_{t+k}^{h \to F} + (1 - \tau_{t+k}^{H}) \widetilde{p}_t^h y_{t+k}^{h} + t_{t+k}^{H} \Bigr) \\
& \quad - \Bigl( \sum_{j' \in J} q_{t+k,t+k+1}(j') d_{t+k+1}^{h \to H}(j') + e_{t+k}^{/*} b_{t+k+1}^{h \to F} + p_{t+k}^{H \to W} c_{t+k}^{h \to W} \Bigr) \biggr) \Biggr\} \Biggr]
\end{aligned}
\]

この目的関数を \( \widetilde{p}_t^h \) について偏微分し、 1 階の条件を求める際に、以下の点を考慮する。

\begin{itemize}
    \item 個別家計 \( h \) の価格 \( \widetilde{p}_t^h \) がマクロ変数( \( p_{t+k}^{H \to W}, \lambda_{t+k}^H, \bar{p}_{t+k}^H \) など )に与える影響は無視できるほど小さいと仮定する。
    \item 価格 \( \widetilde{p}_t^h \) が所得を通じて消費 \( c_{t+k}^{h \to W} \) に与える影響は、予算制約として織り込まれているため、効用関数内の \( c_{t+k}^{h \to W} \) は \( \widetilde{p}_t^h \) から独立しているものとして扱う。
    \item 一方で、制約 \( l_{t+k}^h = y_{t+k}^h / a_{t+k}^H \) と \( y_{t+k}^h = (\frac{\widetilde{p}_t^h}{\bar{p}_{t+k}^H})^{-\theta^H} y_{t+k}^H \) は、効用関数内の労働 \( l_{t+k}^h \) と所得項の生産 \( y_{t+k}^h \) に反映される。これにより \( l_{t+k}^h \) と \( y_{t+k}^h \) は \( \widetilde{p}_t^h \) の関数となり、微分対象になる。
\end{itemize}

これらの仮定のもとで 1 階の条件を計算すると、効用関数の \( \log c_{t+k}^{h \to W} \) の項と、予算制約の大部分の項の微分がゼロとなり、労働の非効用と収入の項のみが残る。

\section{時点 \( t \) における価格 \( \widetilde{p}_t^h \) の最適化条件の導出}
\label{sec:appendix_optimal_price_derivation_foc}

カルボ型の価格設定では、価格改定家計は、一度設定した価格が将来全ての期間にわたり有効であり続ける可能性を考慮し、期待効用の割引現在価値の合計を最大化する単一の価格 \( \widetilde{p}_t^h \) を選択する。

この最適化問題の 1 階の条件は、目的関数を \( \widetilde{p}_t^h \) で偏微分し、ゼロと置くことで得られる。
\[
E_t \sum_{k=0}^{\infty} (\xi^H)^k \left(\prod_{j=0}^{k-1} \beta_{t+j}^H \right) \left[ \frac{\partial}{\partial \widetilde{p}_t^h} \left\{ \left( - \frac{\phi^H}{2}(l_{t+k}^h)^2 \right) + \lambda_{t+k}^H \left( (1 - \tau_{t+k}^H) \widetilde{p}_t^h y_{t+k}^h \right) \right\} \right] = 0
\]

次に、中括弧内の各項の偏微分を、前提となる関係式を用いて詳細に計算する。

\subsection*{前提となる関係式}
\begin{itemize}
    \item \textbf{生産関数}: \( y_{t+k}^h = a_{t+k}^H l_{t+k}^h \implies l_{t+k}^h = y_{t+k}^h / a_{t+k}^H \)
    \item \textbf{需要関数}: \( y_{t+k}^h = \left( \frac{\widetilde{p}_t^h}{\bar{p}_{t+k}^H} \right)^{-\theta^H} y_{t+k}^H \)
\end{itemize}

\subsection*{ステップ 1:第 1 項( 労働の非効用 )の偏微分}
連鎖律( chain rule )を用いると、
\[
\frac{\partial}{\partial \widetilde{p}_t^h} \left( - \frac{\phi^H}{2}(l_{t+k}^h)^2 \right) = -\phi^H l_{t+k}^h \frac{\partial l_{t+k}^h}{\partial \widetilde{p}_t^h}
\]
ここで、 \( \frac{\partial l_{t+k}^h}{\partial \widetilde{p}_t^h} \) を求めるために、前提となる関係式から順に計算する。

\begin{enumerate}
    \item まず、生産関数の関係から、 \( l_{t+k}^h \) の \( \widetilde{p}_t^h \) に関する偏微分は次のようになる。
    \[
    \frac{\partial l_{t+k}^h}{\partial \widetilde{p}_t^h} = \frac{1}{a_{t+k}^H} \frac{\partial y_{t+k}^h}{\partial \widetilde{p}_t^h}
    \]

    \item 次に、需要関数を \( \widetilde{p}_t^h \) で偏微分する。
    \[
    \begin{aligned}
    \frac{\partial y_{t+k}^h}{\partial \widetilde{p}_t^h} &= (-\theta^H) \left( \frac{\widetilde{p}_t^h}{\bar{p}_{t+k}^H} \right)^{-\theta^H-1} \cdot \frac{1}{\bar{p}_{t+k}^H} \cdot y_{t+k}^H \\
    &= -\frac{\theta^H}{\widetilde{p}_t^h} \cdot \left( \frac{\widetilde{p}_t^h}{\bar{p}_{t+k}^H} \right)^{-\theta^H} \cdot y_{t+k}^H \\
    &= -\theta^H \frac{y_{t+k}^h}{\widetilde{p}_t^h} \quad ( \because \text{元の需要関数の定義より} )
    \end{aligned}
    \]
\end{enumerate}

上記 2 つの関係を組み合わせると、 \( \frac{\partial l_{t+k}^h}{\partial \widetilde{p}_t^h} \) は以下のように求められる。
\[
\frac{\partial l_{t+k}^h}{\partial \widetilde{p}_t^h} = \frac{1}{a_{t+k}^H} \left( -\theta^H \frac{y_{t+k}^h}{\widetilde{p}_t^h} \right) = -\theta^H \frac{l_{t+k}^h}{\widetilde{p}_t^h} \quad ( \because y_{t+k}^h/a_{t+k}^H = l_{t+k}^h )
\]
これを最初の式に代入すると、第 1 項の偏微分は、
\[
-\phi^H l_{t+k}^h \left( -\theta^H \frac{l_{t+k}^h}{\widetilde{p}_t^h} \right) = \frac{\phi^H \theta^H (l_{t+k}^h)^2}{\widetilde{p}_t^h}
\]
となる。

\subsection*{ステップ 2:第 2 項( 所得 )の偏微分}
積の微分法則( product rule )を用いると、
\[
\frac{\partial}{\partial \widetilde{p}_t^h} \left( \widetilde{p}_t^h y_{t+k}^h \right) = 1 \cdot y_{t+k}^h + \widetilde{p}_t^h \frac{\partial y_{t+k}^h}{\partial \widetilde{p}_t^h}
\]
ステップ 1 で求めた需要関数の微分 \( \frac{\partial y_{t+k}^h}{\partial \widetilde{p}_t^h} = -\theta^H \frac{y_{t+k}^h}{\widetilde{p}_t^h} \) を代入すると、
\[
y_{t+k}^h + \widetilde{p}_t^h \left( -\theta^H \frac{y_{t+k}^h}{\widetilde{p}_t^h} \right) = y_{t+k}^h - \theta^H y_{t+k}^h = (1-\theta^H)y_{t+k}^h
\]
したがって、第 2 項の偏微分は、
\[
\lambda_{t+k}^H (1 - \tau_{t+k}^H) (1-\theta^H) y_{t+k}^h
\]
となる。

\subsection*{ステップ 3: 1 階の条件式と共通最適価格 \( \widetilde{p}_t^H \) の定義}

ステップ 1 とステップ 2 で求めた偏微分を元の 1 階の条件式に代入し、結合すると以下のようになる。
\[
E_t \sum_{k=0}^{\infty} (\xi^H)^k \left(\prod_{j=0}^{k-1} \beta_{t+j}^H \right) \left[ \frac{\phi^H \theta^H (l_{t+k}^h)^2}{\widetilde{p}_t^h} + \lambda_{t+k}^H (1 - \tau_{t+k}^H) (1-\theta^H) y_{t+k}^h \right] = 0
\]
この 1 階の条件式を \( \widetilde{p}_t^h \) について解くと、個々の家計 \( h \) の最適価格が得られる。
\[
\widetilde{p}_t^h = \frac{\theta^H}{\theta^H-1} \frac{ E_t \sum_{k=0}^{\infty} (\xi^H)^k \left(\prod_{j=0}^{k-1} \beta_{t+j}^H \right) \left[ \phi^H (l_{t+k}^{h})^2 \right] }{ E_t \sum_{k=0}^{\infty} (\xi^H)^k \left(\prod_{j=0}^{k-1} \beta_{t+j}^H \right) \left[ \lambda_{t+k}^H (1 - \tau_{t+k}^{H}) y_{t+k}^{h} \right] }
\]
右辺に含まれる個別変数 \( l_{t+k}^h \) と \( y_{t+k}^h \) を、需要関数と生産関数の関係式を用いて \( \widetilde{p}_t^h \) の関数として展開する。

\begin{itemize}
    \item \( y_{t+k}^h = \left( \frac{\widetilde{p}_t^h}{\bar{p}_{t+k}^H} \right)^{-\theta^H} y_{t+k}^H \)
    \item \( l_{t+k}^h = y_{t+k}^h / a_{t+k}^H = \frac{1}{a_{t+k}^H} \left( \frac{\widetilde{p}_t^h}{\bar{p}_{t+k}^H} \right)^{-\theta^H} y_{t+k}^H \)
\end{itemize}

これらを代入して \( \widetilde{p}_t^h \) の項を整理すると、以下の関係式が得られる。
\[
(\widetilde{p}_t^h)^{1+\theta^H} = \frac{\theta^H}{\theta^H-1} \frac{ E_t \sum_{k=0}^{\infty} (\xi^H)^k \left(\prod_{j=0}^{k-1} \beta_{t+j}^H \right) \left[ \phi^H \frac{(y_{t+k}^H)^2}{(a_{t+k}^H)^2} (\bar{p}_{t+k}^H)^{2\theta^H} \right] }{ E_t \sum_{k=0}^{\infty} (\xi^H)^k \left(\prod_{j=0}^{k-1} \beta_{t+j}^H \right) \left[ \lambda_{t+k}^H (1 - \tau_{t+k}^{H}) y_{t+k}^H (\bar{p}_{t+k}^H)^{\theta^H} \right] }
\]
この式の右辺に出てくる全ての変数は、価格改定を行う全ての家計 \( h \) にとって共通である( 完備市場の仮定より \( \lambda_{t+k}^h = \lambda_{t+k}^H \) )。

したがって、この方程式の解である \( \widetilde{p}_t^h \) も、全ての価格改定を行う家計 \( h \) にとって完全に同一の値となる。この全ての家計にとって共通の最適価格を \( \widetilde{p}_t^H \) と表記する。

\subsection*{ステップ 4:最適価格式の再帰形式への変換( 詳細導出 )}

分子と分母の無限和の部分をそれぞれ新しい補助変数 \( v_t \) と \( w_t \) で定義する。

\paragraph{分子 \( v_t \) の再帰式の導出}

\begin{enumerate}
    \item \textbf{定義式から出発する。}
    \[ v_t = E_t \sum_{k=0}^{\infty} (\xi^H)^k \left(\prod_{j=0}^{k-1} \beta_{t+j}^H \right) \left[ \phi^H \frac{(y_{t+k}^H)^2}{(a_{t+k}^H)^2} (\bar{p}_{t+k}^H)^{2\theta^H} \right] \]
    \item \textbf{無限和を「 今日の項( k = 0 ) 」と「 明日以降の項( k \ge 1 ) 」に分解する。}
    ( \( k=0 \) のとき、 \( \prod \) の部分は空積なので 1 となる )
    \[ v_t = \underbrace{\phi^H \frac{(y_t^H)^2}{(a_t^H)^2} (\bar{p}_t^H)^{2\theta^H}}_{k=0 \text{の項}} + E_t \underbrace{\sum_{k=1}^{\infty} (\xi^H)^k \left(\prod_{j=0}^{k-1} \beta_{t+j}^H \right) \left[ \dots \right]}_{k \ge 1 \text{の項}} \]
    \item \textbf{明日以降の項の和から、共通の因子である \( \beta_t^H \xi^H \) を外に出す。}
    \[ v_t = \phi^H \frac{(y_t^H)^2}{(a_t^H)^2} (\bar{p}_t^H)^{2\theta^H} + E_t \left[ \xi^H \beta_t^H \sum_{k=1}^{\infty} (\xi^H)^{k-1} \left(\prod_{j=1}^{k-1} \beta_{t+j}^H \right) \left[ \dots \right] \right] \]
    \item \textbf{総和の添え字を \( k' = k-1 \) と置き換える。} すると \( k=1 \) は \( k'=0 \) に対応し、累積積も \( \prod_{j=1}^{k-1} \beta_{t+j}^H = \prod_{j'=0}^{k'-1} \beta_{t+1+j'}^H \) となる。
    \[ v_t = \phi^H \frac{(y_t^H)^2}{(a_t^H)^2} (\bar{p}_t^H)^{2\theta^H} + \xi^H E_t \left[ \beta_t^H \sum_{k'=0}^{\infty} (\xi^H)^{k'} \left(\prod_{j'=0}^{k'-1} \beta_{t+1+j'}^H \right) \left[ \dots \right]_{t+1+k'} \right] \]
    \item \textbf{期待値の法則( Law of Iterated Expectations )を適用する。}
    時点 \( t \) の期待値の中にある \( \beta_t^H \) と明日以降の項の和は、 \( t+1 \) 時点の期待値 \( E_{t+1} \) で書き換えられる。
    \[ v_t = \phi^H \frac{(y_t^H)^2}{(a_t^H)^2} (\bar{p}_t^H)^{2\theta^H} + \xi^H E_t \left[ \beta_t^H E_{t+1} \left[ \sum_{k'=0}^{\infty} (\xi^H)^{k'} \left( \dots \right) \right] \right] \]
    \( \beta_t^H \) は \( t \) 期の情報なので \( E_t \) の外に出せ、また \( E_t[E_{t+1}[\cdot]] = E_t[\cdot] \) なので、角括弧の中身は \( v_{t+1} \) の定義そのものになる。
    \[ v_t = \phi^H \frac{(y_t^H)^2}{(a_t^H)^2} (\bar{p}_t^H)^{2\theta^H} + \beta_t^H \xi^H E_t[v_{t+1}] \]
\end{enumerate}

\paragraph{分母 \( w_t \) の再帰式の導出}

\begin{enumerate}
    \item \textbf{定義式から出発し、「 今日の項 」と「 明日以降の項 」に分解する。}
    \[ w_t = \lambda_{t}^H (1 - \tau_{t}^{H}) y_{t}^H (\bar{p}_{t}^H)^{\theta^H} + E_t \sum_{k=1}^{\infty} (\xi^H)^k \left(\prod_{j=0}^{k-1} \beta_{t+j}^H \right) \left[ \dots \right] \]
    \item \textbf{分子と同様の手順で、共通因子 \( \beta_t^H \xi^H \) の括り出しと添え字の変換を行う。}
    \[ w_t = \lambda_{t}^H (1 - \tau_{t}^{H}) y_{t}^H (\bar{p}_{t}^H)^{\theta^H} + \beta_t^H \xi^H E_t \left[ \sum_{k'=0}^{\infty} (\xi^H)^{k'} \left(\prod_{j'=0}^{k'-1} \beta_{t+1+j'}^H \right) \left[ \dots \right]_{t+1+k'} \right] \]
    \item \textbf{角括弧の中身を \( w_{t+1} \) に置き換える。}
    \[ w_t = \lambda_t^H (1 - \tau_t^{H}) y_t^H (\bar{p}_t^H)^{\theta^H} + \beta_t^H \xi^H E_t[w_{t+1}] \]
\end{enumerate}

以上の詳細な導出により、最適価格は以下の 3 本の連立方程式に集約される。
\begin{align}
(\widetilde{p}_t^H)^{1+\theta^H} &= \frac{\theta^H}{\theta^H-1} \frac{v_t}{w_t} \\
v_t &= \phi^H \frac{(y_t^H)^2}{(a_t^H)^2} (\bar{p}_t^H)^{2\theta^H} + \beta_t^H \xi^H E_t[v_{t+1}] \\
w_t &= \lambda_t^H (1 - \tau_t^{H}) y_t^H (\bar{p}_t^H)^{\theta^H} + \beta_t^H \xi^H E_t[w_{t+1}]
\end{align}

\section{対数線形化の手法:定義と一般法則}
\label{sec:appendix_optimal_price_derivation_log_linearization_rules}

\subsection*{1. ハット変数の定義}
ある変数 \( x_t \) の定常状態の値を \( x_{ss} \) とするとき、対数乖離 \( \hat{x}_t \) を次のように定義する。
\[ \hat{x}_t \equiv \log(x_t) - \log(x_{ss}) = \log\left(\frac{x_t}{x_{ss}}\right) \]
 \( x=0 \) の周りでの \( \log(1+x) \approx x \) というマクローリン展開を用いることで、乖離が小さい場合にはパーセント乖離率とほぼ等しくなる。
\[ \hat{x}_t = \log\left(1 + \frac{x_t - x_{ss}}{x_{ss}}\right) \approx \frac{x_t - x_{ss}}{x_{ss}} \]

\subsection*{2. 対数線形化の一般法則}
\begin{itemize}
    \item \textbf{積のルール}: \( z_t = x_t y_t \implies \hat{z}_t = \hat{x}_t + \hat{y}_t \)
    \item \textbf{商のルール}: \( z_t = x_t / y_t \implies \hat{z}_t = \hat{x}_t - \hat{y}_t \)
    \item \textbf{べき乗のルール}: \( z_t = x_t^a \implies \hat{z}_t = a \hat{x}_t \)
    \item \textbf{定数倍のルール}: \( z_t = a x_t \implies \hat{z}_t = \hat{x}_t \)
    \item \textbf{和のルール}: \( z_t = a x_t + b y_t \implies z_{ss} \hat{z}_t = a x_{ss} \hat{x}_t + b y_{ss} \hat{y}_t \)
\end{itemize}

\section{最適価格式の対数線形化}
\label{sec:appendix_optimal_price_derivation_optimal_price_log_linearization}

\subsection*{ステップ 1:出発点となる方程式群}
前章の方程式を、一般法則を用いてそれぞれ対数線形化する。

\begin{enumerate}
    \item \textbf{最適価格の定義式の対数線形化}
    \[ (1+\theta^H) \hat{\widetilde{p}}_t^H = \hat{v}_t - \hat{w}_t \]

    \item \textbf{\( v_t \) の再帰式の対数線形化}
    和のルールを適用する。定常状態では \( v_{flow,ss} = (1-\beta_{ss}^H \xi^H)v_{ss} \), \( v_{stock,ss} = \beta_{ss}^H \xi^H v_{ss} \) の関係があるため、
    \[ \hat{v}_t = (1-\beta_{ss}^H \xi^H) \left( 2\hat{y}_t^H - 2\hat{a}_t^H + 2\theta^H \hat{\bar{p}}_t^H \right) + \beta_{ss}^H \xi^H \left( \hat{\beta}_t^H + E_t[\hat{v}_{t+1}] \right) \]

    \item \textbf{\( w_t \) の再帰式の対数線形化}
    \[ \hat{w}_t = (1-\beta_{ss}^H \xi^H) \left( \hat{\lambda}_t^H + \widehat{(1-\tau_t^H)} + \hat{y}_t^H + \theta^H \hat{\bar{p}}_t^H \right) + \beta_{ss}^H \xi^H \left( \hat{\beta}_t^H + E_t[\hat{w}_{t+1}] \right) \]
\end{enumerate}

\subsection*{ステップ 2:式の結合と最終的な関係式の導出}

\begin{enumerate}
    \item \textbf{\( \hat{v}_t - \hat{w}_t \) を計算し、最適価格 \( \hat{\widetilde{p}}_t^H \) を代入する}
    \[ (1+\theta^H) \hat{\widetilde{p}}_t^H = (1-\beta_{ss}^H \xi^H) \left[ \dots \right] + \beta_{ss}^H \xi^H E_t[(1+\theta^H) \hat{\widetilde{p}}_{t+1}^H] \]
    ここで \( \left[ \dots \right] \) の部分は、 \( \hat{v}_t \) のフロー部分から \( \hat{w}_t \) のフロー部分を引いた、以下の駆動項 \( \hat{\psi}_t \) となる。
    \[ \hat{\psi}_t = \hat{y}_t^H - 2\hat{a}_t^H + \theta^H \hat{\bar{p}}_t^H - \hat{\lambda}_t^H - \widehat{(1-\tau_t^H)} \]

    \item \textbf{最終的な関係式}
    両辺を \( 1+\theta^H \) で割ることで、以下の式が導かれる。
    \[ \hat{\widetilde{p}}_t^H = \frac{1-\beta_{ss}^H \xi^H}{1+\theta^H} \hat{\psi}_t + \beta_{ss}^H \xi^H E_t[\hat{\widetilde{p}}_{t+1}^H] \]
\end{enumerate}

\section{価格設定の導出( 詳細解説 )}
\label{sec:appendix_optimal_price_derivation_pricing_dynamics}

\subsection*{ステップ 1:全体の物価指数の対数線形化}
物価指数 PPI の動学を表す次式を対数線形化する。
\[ (\bar{p}_t^H)^{1-\theta^H} = (1-\xi^H)(\widetilde{p}_t^H)^{1-\theta^H} + \xi^H(\bar{p}_{t-1}^H)^{1-\theta^H} \]
和のルールとべき乗のルールを適用し整理すると、以下の関係が得られる。
\[ \hat{\bar{p}}_t^H = (1-\xi^H) \hat{\widetilde{p}}_t^H + \xi^H \hat{\bar{p}}_{t-1}^H \]

\textbf{インフレと相対価格の関係式の導出:}
インフレ率 \( \hat{\pi}_t^H \equiv \hat{\bar{p}}_t^H - \hat{\bar{p}}_{t-1}^H \) について変形する。まず、両辺から \( \hat{\bar{p}}_{t-1}^H \) を引く。
\[ \hat{\bar{p}}_t^H - \hat{\bar{p}}_{t-1}^H = (1-\xi^H) \hat{\widetilde{p}}_t^H + \xi^H \hat{\bar{p}}_{t-1}^H - \hat{\bar{p}}_{t-1}^H \]
右辺を整理し \( \hat{\bar{p}}_{t-1}^H = \hat{\bar{p}}_t^H - \hat{\pi}_t^H \) を代入する。
\[ \hat{\pi}_t^H = (1-\xi^H)(\hat{\widetilde{p}}_t^H - (\hat{\bar{p}}_t^H - \hat{\pi}_t^H)) \]
項を整理し、インフレ \( \hat{\pi}_t^H \) の項を左辺にまとめると、以下の関係式が得られる。
\[ (1-(1-\xi^H)) \hat{\pi}_t^H = (1-\xi^H)(\hat{\widetilde{p}}_t^H - \hat{\bar{p}}_t^H) \]
\[ \xi^H \hat{\pi}_t^H = (1-\xi^H)(\hat{\widetilde{p}}_t^H - \hat{\bar{p}}_t^H) \quad \text{--- ( 式 A )} \]

\subsection*{ステップ 2: 2 つの式の結合と整理}
最適価格設定のルールを \( \hat{\widetilde{p}}_t^H = (1-\beta_{ss}^H \xi^H) \hat{\psi}_t' + \beta_{ss}^H \xi^H E_t[\hat{\widetilde{p}}_{t+1}^H] \) ( ただし \( \hat{\psi}_t' = \frac{\hat{\psi}_t}{1+\theta^H} \) )とする。

\paragraph{手順 A:式の準備}
式 A を変形して得られる相対価格の式を代入できるよう、最適価格ルールの両辺から \( \hat{\bar{p}}_t^H \) を引く。
\[ \hat{\widetilde{p}}_t^H - \hat{\bar{p}}_t^H = (1-\beta_{ss}^H \xi^H) \hat{\psi}_t' - \hat{\bar{p}}_t^H + \beta_{ss}^H \xi^H E_t[\hat{\widetilde{p}}_{t+1}^H] \]

\paragraph{手順 B:期待値の展開}
右辺の期待値 \( E_t[\hat{\widetilde{p}}_{t+1}^H] \) に \( E_t[\hat{\bar{p}}_{t+1}^H] \) を足して引く。
\[ \hat{\widetilde{p}}_t^H - \hat{\bar{p}}_t^H = (1-\beta_{ss}^H \xi^H) \hat{\psi}_t' - \hat{\bar{p}}_t^H + \beta_{ss}^H \xi^H E_t\left[(\hat{\widetilde{p}}_{t+1}^H - \hat{\bar{p}}_{t+1}^H) + \hat{\bar{p}}_{t+1}^H\right] \]
 \( \hat{\bar{p}}_{t+1}^H = \hat{\bar{p}}_t^H + \hat{\pi}_{t+1}^H \) を代入し、 \( \hat{\bar{p}}_t^H \) は時点 \( t \) で既知であるため期待値の外に出す。
\[ \hat{\widetilde{p}}_t^H - \hat{\bar{p}}_t^H = (1-\beta_{ss}^H \xi^H) \hat{\psi}_t' - \hat{\bar{p}}_t^H + \beta_{ss}^H \xi^H E_t\left[\hat{\widetilde{p}}_{t+1}^H - \hat{\bar{p}}_{t+1}^H\right] + \beta_{ss}^H \xi^H(\hat{\bar{p}}_t^H + E_t[\hat{\pi}_{t+1}^H]) \]

\paragraph{手順 C:式の整理}
相対価格をインフレ率で書き換え、期待値の項をまとめる。
\[ \frac{\xi^H}{1-\xi^H} \hat{\pi}_t^H = (1-\beta_{ss}^H \xi^H)(\hat{\psi}_t' - \hat{\bar{p}}_t^H) + \frac{\beta_{ss}^H \xi^H}{1-\xi^H} E_t[\hat{\pi}_{t+1}^H] \]
両辺に \( \frac{1-\xi^H}{\xi^H} \) を掛けることで、インフレの方程式が得られる。
\[ \hat{\pi}_t^H = \beta_{ss}^H E_t[\hat{\pi}_{t+1}^H] + \frac{(1-\xi^H)(1-\beta_{ss}^H \xi^H)}{\xi^H}(\hat{\psi}_t' - \hat{\bar{p}}_t^H) \]

\subsection*{最終ステップ:ニューケインジアン・フィリップス曲線の導出}
中間式に \( \hat{\psi}_t' \) および \( \hat{\psi}_t \) の定義を代入し、係数 \( \kappa^H = \frac{(1-\xi^H)(1-\beta_{ss}^H \xi^H)}{\xi^H(1+\theta^H)} \) を定義して整理すると、最終的なフィリップス曲線が得られる。
\[ \hat{\pi}_t^H = \beta_{ss}^H E_t[\hat{\pi}_{t+1}^H] + \kappa^H \left( \hat{y}_t^H - 2\hat{a}_t^H - \hat{\bar{p}}_t^H - \hat{\lambda}_t^H - \widehat{(1-\tau_t^H)} \right) \]
( 証明終 )          % フィリップス曲線

% =============================================================
% 3. 第 5 章( シミュレーション結果 )で参照される詳細な分析
% =============================================================
% !TeX root = ../../main.tex
% sections/app/appendix_exchange_rate.tex

\chapter{割引因子ショックを含む為替レート決定式の導出}
\label{chap:appendix_exchange_rate}

本稿の第 5 章における考察では、自国と外国で家計の時間選好( 割引因子 \( \beta \) )が異なる場合、為替レートの決定式に将来の \( \beta \) の格差が累積的に影響することを論じた。本付録では、その数理的な導出過程を、統計学的な恒等式と対数線形近似の性質、および前方展開( Forward Solving )の手法に焦点を当て詳述する。

\section*{1. 期待値の比に関する線形近似の証明}

本稿のような線形近似モデルにおいて、期待値の比 \( E_t[X_t] / E_t[Y_t] \) が「 比の期待値 」 \( E_t[X_t / Y_t] \) で近似できる理由を、一般的な確率変数の性質に基づいて数学的に証明する。

\subsection*{一般論としての変数の定義}
ある任意の確率変数 \( X_t \) について、その期待値( 平均 )を \( \bar{X} \equiv E[X_t] \) とし、期待値からの対数乖離( 変化率 )を \( \hat{x}_t \) と定義する。
\begin{equation*}
    \hat{x}_t \equiv \frac{X_t - \bar{X}}{\bar{X}} \quad \Longleftrightarrow \quad X_t = \bar{X}(1+\hat{x}_t)
\end{equation*}
本稿のような線形近似モデルでは、この乖離 \( \hat{x}_t \) を「 1 次の微小量 」として扱い、その 2 乗以上( \( \hat{x}_t^2 \) や \( \hat{x}_t \hat{y}_t \) )を無視できるほど小さい項( \( \approx 0 \) )として計算を行う。

\subsection*{ステップ 1:積の期待値の近似( 共分散項の評価 )}

一般に、 2 つの確率変数 \( X, Y \) の積の期待値は、「 期待値の積 」と「 共分散( Covariance ) 」の和に分解できる。
\begin{equation*}
    E_t[XY] = E_t[X]E_t[Y] + \text{Cov}_t(X, Y)
\end{equation*}
ここで、共分散の項を上記の定義に従って展開する。
\begin{align*}
    \text{Cov}_t(X, Y) &= E_t \left[ (X - \bar{X}) (Y - \bar{Y}) \right] \\
    &= E_t \left[ \left( \bar{X}(1+\hat{x}) - \bar{X} \right) \left( \bar{Y}(1+\hat{y}) - \bar{Y} \right) \right] \\
    &= \bar{X}\bar{Y} E_t \left[ \hat{x}\hat{y} \right]
\end{align*}
ここで、右辺の期待値の中身は、微小変動同士の積( \( \hat{x} \times \hat{y} \) のオーダー )となっている。 1 次近似の枠組みでは、 1 次の微小量同士の積は 2 次の微小量となるため、無視することができる( \( \approx 0 \) )。したがって、以下の近似が成立する。
\begin{equation}
    \text{Cov}_t(X, Y) \approx 0 \quad \Longrightarrow \quad E_t[XY] \approx E_t[X]E_t[Y]
    \label{eq:appendix_exchange_rate_approx_product}
\end{equation}

\subsection*{ステップ 2:逆数の期待値の近似( 分散項の評価 )}

次に、変数 \( Z \) の逆数の期待値 \( E_t[1/Z] \) を考える。一般に、イェンゼンの不等式により \( E[1/Z] \neq 1/E[Z] \) であるが、その乖離幅は分散に依存する。
関数 \( f(Z) = 1/Z \) を、期待値 \( \bar{Z} = E_t[Z] \) の周りで 2 次までテイラー展開し、期待値をとる。
\begin{align*}
    E_t\left[\frac{1}{Z}\right] &\approx E_t \left[ \frac{1}{\bar{Z}} - \frac{1}{\bar{Z}^2}(Z - \bar{Z}) + \frac{1}{\bar{Z}^3}(Z - \bar{Z})^2 \right] \\
    &= \frac{1}{\bar{Z}} - \frac{1}{\bar{Z}^2}\underbrace{E_t[Z - \bar{Z}]}_{0} + \frac{1}{\bar{Z}^3}\underbrace{E_t[(Z - \bar{Z})^2]}_{\text{分散 Var}(Z)} \\
    &= \frac{1}{E_t[Z]} + \frac{1}{(E_t[Z])^3} \text{Var}_t(Z)
\end{align*}
ここで分散 \( \text{Var}_t(Z) = \bar{Z}^2 E_t \left[ \hat{z}^2 \right] \) は微小変動の 2 乗を含むため、 1 次近似においてはこれを無視できる。
したがって、以下の近似式が成立する。
\begin{equation}
    E_t\left[\frac{1}{Z}\right] \approx \frac{1}{E_t[Z]}
    \label{eq:appendix_exchange_rate_approx_inverse}
\end{equation}

\subsection*{結論:比の期待値への変換}

式 \eqref{eq:appendix_exchange_rate_approx_product} と 式 \eqref{eq:appendix_exchange_rate_approx_inverse} の結果を組み合わせることで、期待値の比を比の期待値として扱うことが正当化される。
\begin{align*}
    \frac{E_t [X]}{E_t [Y]} &= E_t [X] \cdot \frac{1}{E_t [Y]} \\
    &\approx E_t [X] \cdot E_t \left[ \frac{1}{Y} \right] \quad ( \because \text{分散項} \approx 0 ) \\
    &\approx E_t \left[ X \cdot \frac{1}{Y} \right] \quad ( \because \text{共分散項} \approx 0 ) \\
    &= E_t \left[ \frac{X}{Y} \right]
\end{align*}
本付録の以降の導出では、この一般的な近似式 \( E_t[X_t]/E_t[Y_t] \approx E_t[X_t/Y_t] \) を利用する。

\section*{2. 修正UIP条件とオイラー方程式の比}

名目為替レートの決定プロセスを明らかにするため、まず両国の基礎方程式を整理する。

\begin{itemize}
    \item \textbf{自国代表的家計の修正UIP条件:}
    本文第 5 章( 式 \ref{eq:results_asad_modified_uip_H} )で確認した通り、自国の UIP 条件 \eqref{eq:uip_condition_home} に対し、自国家計にとっての外貨 1 単位の限界効用の定義式 \( \lambda_t^{H/*} \equiv e_t^{/*} \lambda_t^H \) を適用すると、以下の「 修正UIP条件 」が得られる。
    \begin{equation}
        \lambda_t^{H/*} = \beta_t^H (1+i_t^F) E_t [\lambda_{t+1}^{H/*}]
        \tag{\ref{eq:results_asad_modified_uip_H}}
    \end{equation}

    \item \textbf{外国代表的家計の修正UIP条件:}
    同様の手順を、外国代表的家計の UIP 条件 \eqref{eq:model_uip_F} に対しても適用する。外国代表的家計にとっての外貨( この場合、自国通貨 1 単位 )の限界効用は \( \lambda_t^{F/*} \) である。外国のオイラー方程式から、以下の関係が得られる。
    \begin{equation}
        \lambda_t^{F/*} = \beta_t^F (1+i_t^F) E_t [\lambda_{t+1}^{F/*}]
        \label{eq:appendix_exchange_rate_modified_uip_F}
    \end{equation}
\end{itemize}

ここで、自国代表的家計と外国代表的家計による外貨評価の比率を \( \Lambda_t \) と定義する。
\begin{equation}
    \Lambda_t \equiv \frac{\lambda_t^{H/*}}{\lambda_t^{F/*}}
    \label{eq:appendix_exchange_rate_lambda_def}
\end{equation}
式 \eqref{eq:results_asad_modified_uip_H} と 式 \eqref{eq:appendix_exchange_rate_modified_uip_F} の辺々を割ると、共通項である外国の名目金利 \( (1+i_t^F) \) が相殺され、以下の関係が得られる。
\begin{equation}
    \Lambda_t = \frac{\beta_t^H}{\beta_t^F} \frac{E_t [\lambda_{t+1}^{H/*}]}{E_t [\lambda_{t+1}^{F/*}]}
    \label{eq:appendix_exchange_rate_lambda_ratio_raw}
\end{equation}

第 1 節で証明した一般的な近似関係を用いることで、式 \eqref{eq:appendix_exchange_rate_lambda_ratio_raw} の右辺にある「 期待値の比 」を「 比の期待値 」へと変換し、以下の再帰式を得る。
\begin{equation}
    \Lambda_t \approx \frac{\beta_t^H}{\beta_t^F} E_t \left[ \frac{\lambda_{t+1}^{H/*}}{\lambda_{t+1}^{F/*}} \right] = \frac{\beta_t^H}{\beta_t^F} E_t [\Lambda_{t+1}]
    \label{eq:appendix_exchange_rate_lambda_recursive}
\end{equation}

\section*{3. 将来に向けた前方展開と極限}

前節で導出した再帰式 \eqref{eq:appendix_exchange_rate_lambda_recursive} を将来に向かって逐次代入( Forward Solving )することで、現在の比率 \( \Lambda_t \) を決定する。

\paragraph{① 再帰的代入と一般形の導出}
式 \eqref{eq:appendix_exchange_rate_lambda_recursive} の右辺にある \( \Lambda_{t+1} \) に対して、 1 期先の関係式 \( \Lambda_{t+1} = \frac{\beta_{t+1}^H}{\beta_{t+1}^F} E_{1+t}[\Lambda_{t+2}] \) を代入する。この操作を \( k \) 回繰り返すと、反復期待値の法則により、以下の一般形が得られる。
\begin{equation}
    \Lambda_t = E_t \left[ \left( \prod_{j=0}^{k} \frac{\beta_{t+j}^H}{\beta_{t+j}^F} \right) \Lambda_{t+k+1} \right]
    \label{eq:appendix_exchange_rate_lambda_iterative}
\end{equation}

\paragraph{② 極限の適用}
次に、式 \eqref{eq:appendix_exchange_rate_lambda_iterative} の両辺について \( k \to \infty \) の極限をとる。
本稿のモデルにおいては、全ての変数が定常状態近傍で安定的に推移( 有界性 )することを前提としている。このとき、ルベーグの優収束定理( Dominated Convergence Theorem )により、期待値演算子 \( E_t \) と極限操作 \( \lim \) の順序を入れ替えることが正当化される。すなわち、期待値の極限は極限の期待値に等しい。

\begin{align*}
    \Lambda_t &= \lim_{k \to \infty} E_t \left[ \left( \prod_{j=0}^{k} \frac{\beta_{t+j}^H}{\beta_{t+j}^F} \right) \Lambda_{t+k+1} \right] \\
    &= E_t \left[ \lim_{k \to \infty} \left( \left( \prod_{j=0}^{k} \frac{\beta_{t+j}^H}{\beta_{t+j}^F} \right) \Lambda_{t+k+1} \right) \right]
\end{align*}

ここで、経済の安定性条件より、無限遠方において一時的なショックの影響は消失し、変数はショック後の新たな定常状態へ収束すると仮定する。したがって、評価の比率 \( \Lambda_{t+k+1} \) は、定常状態における比率 \( \Lambda_{\infty} \) へと収束する。
\begin{equation*}
    \lim_{k \to \infty} \Lambda_{t+k+1} = \Lambda_{\infty}
\end{equation*}
不完備市場モデルにおいては、この \( \Lambda_{\infty} \) はショックの履歴に依存する確率変数としての性質を持つため、期待値演算子の中に留まる。

ここで、資本移動要因 \( \mathcal{B}_t \) を以下のように定義する。
\begin{equation}
    \mathcal{B}_t \equiv \prod_{j=0}^{\infty} \frac{\beta_{t+j}^H}{\beta_{t+j}^F}
    \label{eq:appendix_exchange_rate_bt_definition}
\end{equation}

以上より、現在の \( \Lambda_t \) は以下の形式で確定する。
\begin{equation}
    \Lambda_t = E_t \left[ \mathcal{B}_t \Lambda_{\infty} \right]
    \label{eq:appendix_exchange_rate_lambda_final_solution}
\end{equation}

\section*{4. 結論:為替レートの決定式}

為替レートの定義式 \( e_t^{/*} = \lambda_t^{H/*} / \lambda_t^H \) に、 \( \lambda_t^{H/*} = \Lambda_t \lambda_t^{F/*} \) を代入する。
さらに、代表的家計の所得の限界効用の定義 \eqref{eq:model_lambda_H} および外国の同様の関係式を用いる。
\begin{equation*}
    \lambda_t^H = \frac{1}{p_t^{H \to W} c_t^{H \to W}}, \quad \lambda_t^{F/*} = \frac{1}{p_t^{F \to W*} c_t^{F \to W}}
\end{equation*}
これらを代入して整理すると、最終的な為替レート決定式が得られる。

\begin{align}
    e_t^{/*} &= \Lambda_t \times \frac{\lambda_t^{F/*}}{\lambda_t^H} \nonumber \\
    &= \Lambda_t \times \frac{1 / (p_t^{F \to W*} c_t^{F \to W})}{1 / (p_t^{H \to W} c_t^{H \to W})} \nonumber \\
    &= E_t \left[ \mathcal{B}_t \Lambda_{\infty} \right] \times \frac{p_t^{H \to W} c_t^{H \to W}}{p_t^{F \to W*} c_t^{F \to W}}
    \label{eq:appendix_exchange_rate_final}
\end{align}

ここで、期待値項 \( E_t \left[ \mathcal{B}_t \Lambda_{\infty} \right] \) について、定常状態における割引因子の設定が決定的な役割を果たす。

\begin{itemize}
    \item \textbf{定常状態で割引因子が等しい場合( \( \beta_{ss}^H = \beta_{ss}^F \) ):}
    本稿の基本設定である。一時的なショックにより \( \beta_t \) が変動しても、長期的には元の水準に戻るため、無限乗積項 \( \mathcal{B}_t \) は有限の値に収束する。
    特に、全期間において \( \beta_t^H = \beta_t^F \) であるならば、式 \eqref{eq:appendix_exchange_rate_final} の期待値項は \( E_t [\Lambda_{\infty}] \) となり、為替レートは純粋に両国の名目支出比率のみで決定される。
    \begin{equation}
        e_t^{/*} = E_t [\Lambda_{\infty}] \times \frac{p_t^{H \to W} c_t^{H \to W}}{p_t^{F \to W*} c_t^{F \to W}} 
        \label{eq:appendix_exchange_rate_simplified}
    \end{equation}
    
    \item \textbf{定常状態で割引因子が異なる場合( \( \beta_{ss}^H \neq \beta_{ss}^F \) ):}
    もし恒久的に \( \beta \) が異なると仮定すると、無限乗積項は発散または消失し、安定的な均衡が存在しなくなる。したがって、本モデルのような無限期間モデルにおいて安定解を得るためには、定常状態において両国の割引因子が一致していることが前提条件となる。
\end{itemize}

\section*{5. 資源制約式と対外純資産の恒等的なゼロ均衡}

上記の為替レート決定メカニズムが、対外純資産( \( b_t^H \) )の動学に与える影響を確認する。

\subsection*{1. 名目GDP方程式の導出と為替レート決定式の代入}

自国の一人当たり財市場均衡条件 \eqref{eq:model_goods_market_eq_H} から出発する。
\begin{equation}
    y_t^H = c_t^{H \to H} + \frac{M}{N} c_t^{F \to H}
    \label{eq:appendix_exchange_rate_goods_market_raw}
\end{equation}
この両辺に自国財価格 \( p_t^H \) を乗じると、名目GDPは各主体の需要内訳として以下のように展開される。
\begin{align*}
    p_t^H y_t^H &= p_t^H c_t^{H \to H} + p_t^H \frac{M}{N} c_t^{F \to H} \\
    &= p_t^H \left( \alpha^H \frac{p_t^{H \to W}}{p_t^H} c_t^{H \to W} \right) + p_t^H \frac{M}{N} \left( (1 - \alpha^F) \frac{p_t^{F \to W*}}{p_t^{H*}} c_t^{F \to W} \right) \\
    &= \alpha^H (p_t^{H \to W} c_t^{H \to W}) + \frac{M}{N} (1 - \alpha^F) \frac{p_t^H}{p_t^{H*}} (p_t^{F \to W*} c_t^{F \to W})
\end{align*}
ここで、一物一価の法則 \( p_t^H = e_t^{/*} p_t^{H*} \) \eqref{eq:model_lop} および、人口比 \( N=M=1 \) の設定を適用すると、以下の一般的な名目GDPの方程式が得られる。
\begin{equation}
    p_t^H y_t^H = \alpha^H(p_t^{H \to W} c_t^{H \to W}) + (1 - \alpha^F) e_t^{/*} (p_t^{F \to W*} c_t^{F \to W})
    \label{eq:appendix_exchange_rate_gdp_general}
\end{equation}

さらに、式 \eqref{eq:appendix_exchange_rate_gdp_general} の右辺にある為替レート \( e_t^{/*} \) に、先に導出した決定式 \eqref{eq:appendix_exchange_rate_final} を代入する。
\begin{align}
    p_t^H y_t^H &= \alpha^H(p_t^{H \to W} c_t^{H \to W}) + (1 - \alpha^F) \left( E_t \left[ \mathcal{B}_t \Lambda_{\infty} \right] \frac{p_t^{H \to W} c_t^{H \to W}}{p_t^{F \to W*} c_t^{F \to W}} \right) (p_t^{F \to W*} c_t^{F \to W}) \nonumber \\
    &= \alpha^H(p_t^{H \to W} c_t^{H \to W}) + (1 - \alpha^F) E_t \left[ \mathcal{B}_t \Lambda_{\infty} \right] (p_t^{H \to W} c_t^{H \to W}) \nonumber \\
    &= \left[ \alpha^H + (1 - \alpha^F) E_t \left[ \mathcal{B}_t \Lambda_{\infty} \right] \right] p_t^{H \to W} c_t^{H \to W}
    \label{eq:appendix_exchange_rate_gdp_substituted}
\end{align}

ここで第 \ref{sec:model_steady_state} 節において仮定されたとおり
初期時点およびそれ以前の対外純資産はゼロ( \( b_s^H = b_{s-1}^H = 0 \) )である。

自国の資源制約式 \eqref{eq:model_resource_constraint_H} においてショック発生前の \( t = s - 1 \) の時点を考える。
\begin{equation}
    p_{s-1}^{H \to W} c_{s-1}^{H \to W} + b_s^H = p_{s-1}^H y_{s-1}^H + (1+i_{s-2}^F) \frac{e_{s-1}^{/*}}{e_{s-2}^{/*}} b_{s-1}^H
    \label{eq:appendix_exchange_rate_resource_constraint_init}
\end{equation}
ここに仮定 \( b_s^H = b_{s-1}^H = 0 \) を代入すると、以下の貿易収支の均衡条件が導かれる。
\begin{equation}
    p_{s-1}^H y_{s-1}^H = p_{s-1}^{H \to W} c_{s-1}^{H \to W}
    \label{eq:appendix_exchange_rate_trade_balance_init}
\end{equation}
式 \eqref{eq:appendix_exchange_rate_gdp_substituted} を初期時点 \( t = s - 1 \) で評価し、貿易収支均衡 \eqref{eq:appendix_exchange_rate_trade_balance_init} と比較すると、右辺の名目消費にかかる係数部分は \( 1 \) でなければならないことがわかる。すなわち、
\begin{equation}
    1 = \alpha^H + (1 - \alpha^F) E_{s-1} \left[ \mathcal{B}_{s-1} \Lambda_{\infty} \right]
    \label{eq:appendix_exchange_rate_coefficient_unity_condition}
\end{equation}
が成立する。

\subsection*{2. a ショックが発生したときの均衡}

次に、全期間において自国と外国の割引因子が等しく( \( \beta_t^H = \beta_t^F \) )、生産性ショック( \( a \) ショック )のみが発生するケースを検討する。

まず、本稿の第 5 章 「 生産性ショックにおける国際的波及( 遮断効果 ) 」で論じたように、外貨建ての限界効用 \( \lambda_t^{H/*} \) および \( \lambda_t^{F/*} \) は生産性ショックの影響を受けない方程式系によって決定される。したがって、ショックの前後でこれらの値は変化せず、その比率である \( \Lambda_t = \lambda_t^{H/*} / \lambda_t^{F/*} \) も不変に維持される。すなわち、
\begin{equation}
    \Lambda_{s-1} = \Lambda_s = \dots = \Lambda_{\infty}
\end{equation}
が成立する。この性質により、評価の比率の極限値 \( \Lambda_{\infty} \) は、ショックの履歴に依存しない確定的定数とみなすことができ、期待値演算子 \( E_t \) の外に出すことが可能となる。

また、本ケースでは全期間において割引因子が一定であるため、定義式 \eqref{eq:appendix_exchange_rate_bt_definition} より資本移動要因は恒等的に \( \mathcal{B}_t = 1 \) となる。

この条件を式 \eqref{eq:appendix_exchange_rate_coefficient_unity_condition} に適用すると、
\begin{equation}
    1 = \alpha^H + (1 - \alpha^F) \Lambda_{\infty}
    \label{eq:appendix_exchange_rate_coefficient_unity_condition_substituted}
\end{equation}
となり、これを整理すると以下が得られる。
\begin{equation}
    \Lambda_{\infty} = \frac{1 - \alpha^H}{1 - \alpha^F}
    \label{eq:appendix_exchange_rate_lambda_steady_state_ratio}
\end{equation}

この \( \Lambda_{\infty} \) と \( \mathcal{B}_t = 1 \) を式 \eqref{eq:appendix_exchange_rate_gdp_substituted} に代入すると、以下の名目貿易収支の均衡式が得られる。
\begin{equation}
    p_t^H y_t^H = p_t^{H \to W} c_t^{H \to W}
    \label{eq:appendix_exchange_rate_nominal_trade_balance_identity}
\end{equation}

最後に、式 \eqref{eq:appendix_exchange_rate_nominal_trade_balance_identity} を自国の資源制約式 \eqref{eq:model_resource_constraint_H} に代入すると、資産蓄積に関する以下の差分方程式が得られる。
\begin{equation}
    b_{t+1}^H = (1+i_{t-1}^F) \frac{e_t^{/*}}{e_{t-1}^{/*}} b_t^H
    \label{eq:appendix_exchange_rate_nfa_difference_equation}
\end{equation}

初期条件 \( b_s^H = 0 \) を用いると、帰納的にすべての \( t \ge s \) において \( b_{t+1}^H = 0 \) であることが証明される。以上により、Cole-Obstfeld 条件下では為替レートの調整により貿易収支が常に均衡し、対外純資産が恒等的にゼロとなることが数学的に裏付けられた。          % 為替レートと GDP の恒等式
% !TeX root = ../../main.tex
% sections/app/appendix_log_linearization.tex

\chapter{対数線形近似の定義と幾何学的解釈}
\label{chap:appendix_log_linearization}

本付録では、動学的確率的一般均衡( DSGE )モデルの標準的な手法に従い、非線形の方程式系を定常状態近傍で対数線形近似( Log-linear approximation )することで、線形の連立差分方程式系へと変換する手法について詳述する。本節では、本稿で用いるハット変数( \( \hat{x}_t \) )の定義と、その数学的・幾何学的な意味について補足する。

\section{対数線形化の目的と定義}
\label{sec:appendix_log_linearization_purpose}

経済モデルの均衡条件は、通常、コブ=ダグラス型生産関数 \( Y_t = A_t K_t^\alpha L_t^{1-\alpha} \) のように、変数の積や冪乗を含む非線形方程式として記述される。これらの式をそのまま解くことは困難であるため、対数変換を行うことで乗算を加算( 線形 )の形に変換し、計算上の便宜を図ることが対数線形化の主たる目的である。

本稿では、ある変数 \( x_t \) の定常状態の値を \( x \) とするとき、定常状態からの対数乖離 \( \hat{x}_t \) を以下のように定義する。
\begin{equation}
    \hat{x}_t \equiv \ln x_t - \ln x
    \label{eq:appendix_log_linearization_hat_definition}
\end{equation}
ここで、 \( \ln x_t \) という値そのものには、具体的な経済学的単位( 円や数量など )としての意味はない。これはあくまで、積の形を和の形に変換するために導入された数学的な操作上の値に過ぎない。
しかし、この定義式に対して定常状態近傍での 1 次近似( テイラー展開 )を適用することで、 \( \hat{x}_t \) に具体的な経済学的意味( 変化率 )が付与される。

\section{変化率との関係}
\label{sec:appendix_log_linearization_rate}

関数 \( f(z) = \ln z \) を定常状態 \( z = x \) の周りで 1 次テイラー展開すると、以下の近似式が得られる。
\begin{equation}
    f(x_t) \approx f(x) + f'(x)(x_t - x)
\end{equation}
 \( f(z) = \ln z \) より \( f'(z) = 1/z \) であるため、上式は以下のように書き換えられる。
\begin{equation}
    \ln x_t \approx \ln x + \frac{1}{x}(x_t - x)
\end{equation}
これを変形すると、対数乖離 \( \hat{x}_t \) と変化率の関係が導かれる。
\begin{equation}
    \underbrace{\ln x_t - \ln x}_{\hat{x}_t} \approx \frac{x_t - x}{x}
    \label{eq:appendix_log_linearization_approx_formula}
\end{equation}
右辺は「 定常状態からの変化率( % ) 」そのものである。したがって、本稿における \( \hat{x}_t \) は、数学的には対数の差であるが、経済学的には「 定常状態からのパーセント乖離 」として解釈される。

\section{幾何学的解釈:曲線と接線の高さ}
\label{sec:appendix_log_linearization_geometry}

式 \eqref{eq:appendix_log_linearization_approx_formula} の両辺が近似的に等しい( \( \approx \) )ことの意味は、幾何学的には「 曲線の高さ 」と「 接線の高さ 」の関係として理解できる。



横軸に変数 \( z \)、縦軸に \( y = \ln z \) をとったグラフを考える。
\begin{itemize}
    \item \textbf{左辺( \( \ln x_t - \ln x \) ):}
    これは、対数曲線 \( y = \ln z \) 上における、定常状態 \( x \) から \( x_t \) までの\textbf{「 実際の高さ( 縦軸 )の変化量 」}を表している。
    
    \item \textbf{右辺( \( \frac{x_t - x}{x} \) ):}
    これは、定常状態 \( x \) において曲線に引いた\textbf{「 接線 」上での高さの変化量}を表している。
    接線の傾きは \( f'(x) = 1/x \) であるため、横方向の変化量 \( (x_t - x) \) に対する縦方向の変化量は、
    \[
    \text{縦の変化} = \text{傾き} \times \text{横の変化} = \frac{1}{x} \times (x_t - x)
    \]
    となる。
\end{itemize}

すなわち、対数線形近似とは、定常状態の近傍において「 対数曲線上の高さの変化( 左辺 ) 」を「 接線上の高さの変化( 右辺 ) 」で代用することに他ならない。
 \( x_t \) が \( x \) に十分に近ければ、曲線と接線はほぼ重なり合うため、この近似の精度は保たれる。シミュレーションにおいては、この近似関係を等号として扱うことで、巨大な連立方程式系を行列演算によって効率的に解くことが可能となる。

\section{積の線形化( \( \widehat{x_t y_t} = \hat{x}_t + \hat{y}_t \) の導出 )}
\label{sec:appendix_log_linearization_product}

変数の積 \( x_t y_t \) の対数線形近似が、それぞれのハット変数の和 \( \hat{x}_t + \hat{y}_t \) となることは、前述のハット変数の定義を用いることで、以下のように計算できる。

\begin{enumerate}
    \item \textbf{積の変数の定義:}
    まず、積の変数 \( x_t y_t \) について、その対数乖離 \( \widehat{x_t y_t} \) を定義に従って記述する。対数線形モデルにおいては、これを線形近似した厳密な等号として扱う。
    \begin{equation}
        \widehat{x_t y_t} = \ln(x_t y_t) - \ln(xy)
    \end{equation}
    
    \item \textbf{個別の変数の定義:}
    同様に、個別の変数 \( x_t, y_t \) についても、それぞれの対数乖離を定義する。
    \begin{align}
        \hat{x}_t &= \ln x_t - \ln x \\
        \hat{y}_t &= \ln y_t - \ln y
    \end{align}

    \item \textbf{展開と代入:}
    第 1 式の右辺に対し、対数の性質 \( \ln(AB) = \ln A + \ln B \) を適用して展開する。
    \begin{equation}
        \widehat{x_t y_t} = (\ln x_t + \ln y_t) - (\ln x + \ln y)
    \end{equation}
    項を並べ替えて整理する。
    \begin{equation}
        \widehat{x_t y_t} = (\ln x_t - \ln x) + (\ln y_t - \ln y)
    \end{equation}
    ここに第 2 項の個別変数の定義を代入すると、以下の関係が得られる。
    \begin{equation}
        \widehat{x_t y_t} = \hat{x}_t + \hat{y}_t
    \end{equation}
\end{enumerate}

このように、対数線形近似を用いることで、変数の乗算( 非線形 )の関係式は、ハット変数の加算( 線形 )の関係式へと変換される。      % 線形化フィリップス曲線の導出
% !TeX root = ../../main.tex
% sections/app/appendix_welfare_correction.tex

\chapter{厚生評価における近似の整合性と補正項の導出}
\label{chap:appendix_welfare_correction}

本付録では、本稿のシミュレーション分析で採用している「 1 次近似( 対数線形近似 )による動学算出 」と「 2 次近似的視点に基づく厚生評価 」の間の論理的整合性、および具体的な補正ロジックについて詳述する。

\section{近似による情報の欠落と整合性の問題}
\label{sec:appendix_welfare_correction_consistency}

本稿のシミュレーションは対数線形近似( 1 次近似 )を用いているが、この手法で得られた結果を厚生評価に用いる際には、「 非線形な関数 」と「 近似された変数 」の取り扱いに細心の注意が必要である。なぜなら、\textbf{「 非線形な関数 」に「 1 次近似された変数 」をそのまま代入して計算すると、計算の整合性が取れなくなる( 偽の精度が生じる )}からである。

この問題を理解するために、簡単な「 A + B 」の例を考えてみよう。
ある真の値 \( X \) が、主要な変動成分である「 1 次の項 \( A \) 」と、微細な補正成分である「 2 次の項 \( B \) 」から成るとする( \( X = A + B \) )。
一方、評価したい関数が \( F(X) = X + X^2 \) という非線形( 2 次 )の関数だとする。

真の値をこの関数に入力して、 2 次の精度まで正しく計算すると、以下のようになる( 3 次以上の微小項は無視する )。
\[
\begin{aligned}
F(A+B) &= (A+B) + (A+B)^2 \\
&= A + B + (A^2 + 2AB + B^2) \\
&\approx A + B + A^2
\end{aligned}
\]
ここで重要なのは、\textbf{「 入力に含まれる 2 次の項 \( B \) 」と「 関数によって生成される 2 次の項 \( A^2 \) 」の両方が、計算結果として残る}という点である。

しかし、もしシミュレーションが 1 次近似で行われていたらどうなるだろうか。シミュレーション結果 \( X^{sim} \) は、微細な \( B \) を無視して \( X^{sim} \approx A \) と出力される。これをそのまま非線形関数に代入してしまうと、
\[
F(X^{sim}) = A + A^2
\]
となる。一見もっともらしい値が出るが、ここには重大な欠陥がある。本来足されるべきだった \textbf{「 入力由来の 2 次の項 \( B \) 」が欠落している一方で、「 関数由来の 2 次の項 \( A^2 \) 」だけが計算されている}のである。同じ重要度を持つはずの要素のうち、片方だけを計算し、片方を無視するのは、計算として不整合であり、結果の信頼性を損なう。

本稿の分析においても、これと全く同じ問題が発生する。

\paragraph{① 指数関数によるレベル変数の復元を行ってはならない理由}
シミュレーション結果として得られる対数乖離 \( \hat{x}_t \) から、レベル変数 \( x_t \) を復元する際、定義式である指数関数 \( x_t = x_{ss}\exp(\hat{x}_t) \) を用いて計算してはいけない。
なぜなら、\textbf{非線形な指数関数をマクローリン展開すると} \( x_{ss}(1 + \hat{x}_t + \frac{1}{2}\hat{x}_t^2 + \dots) \) となり、\textbf{2 次以上の項が含まれる}からである。入力である \( \hat{x}_t \) 自体が 2 次の情報( 上例の \( B \) )を欠落させているにも関わらず、計算式だけで 2 次以上の項を生成するのは、上述の不整合を引き起こす。

\paragraph{② 線形復元した変数を非線形な効用関数に代入してはならない理由}
では、 2 次以上の項が出ないように \( x_t = x_{ss}(1 + \hat{x}_t) \) と線形で復元すればよいかというと、それも不十分である。
もし、そうして復元した \( x_t \) ( すなわち \( c_t, l_t \) )を、以下のような対数や二乗を含む非線形な効用関数
\[
U(c_t^{H\to W}, l_t^H) = \ln c_t^{H\to W} - \frac{\phi^H}{2}(l_t^H)^2
\]
に直接代入してしまえば、結局は効用関数側が 2 次以上の項( \( A^2 \) など )を生成することになり、入力情報の欠落( \( B \) の無視 )との不整合が生じるからである。

\paragraph{③ 正しい対処法:効用関数自体の線形近似}
整合性を保つための正しい手順は、変数を代入する前に、まず効用関数 \( U \) 自体を定常状態近傍で 1 次近似( 線形化 )し、その式に出てくる \( \hat{c}_t, \hat{l}_t \) に、シミュレーションで得られた 1 次近似値 \( \hat{c}_t, \hat{l}_t \) を代入することである。
このように関数を線形化しておけば、「 線形 」対「 線形 」の対応となり、整合性が保たれる。

\paragraph{④ 価格分散コスト \( \Delta \) の手動補正}
ただし、この線形化の手順をとると、本来は 2 次以上の項として評価されるべき重要な要素、すなわち「 価格分散コスト \( \Delta_t \) 」が式から消滅してしまう( 1 次近似の世界では \( \Delta_t \approx 0 \) となるため )。
そこで本稿では、この消えてしまったコスト \( \Delta_t \) を、 \textcite{Woodford2003} にならい別途手計算で導出し、線形近似された厚生の値から事後的に差し引くという補正を行う。これにより、シミュレーションの整合性を保つとともに、価格のばらつきによる経済的損失を正確に評価に反映させることが可能となる。

次節より、その具体的な導出過程を示す。

\section{価格分散コストの補正式の導出}
\label{sec:appendix_welfare_correction_derivation}

補正の根拠は、線形近似モデルにおいてゼロとなる価格分散の影響を、効用関数の計算に復元することにある。以下にその導出過程を示す。

\paragraph{1. 効用関数の線形近似}

まず、前節で示した期間効用関数 \( U(c_t^{H \to W}, l_t^H) \) を再掲する。
\begin{equation}
    U(c_t^{H \to W}, l_t^H) = \ln c_t^{H \to W} - \frac{\phi^H}{2}(l_t^H)^2
\end{equation}
この関数の値が、定常状態 \( (c_{ss}^{H \to W}, l_{ss}^H) \) からどの程度乖離するかを、 1 次テイラー展開( 線形近似 )を用いて評価する。
多変数関数の近似公式を適用すると、効用の変動部分は以下のように記述できる。
\begin{align}
    U(c_t^{H \to W}, l_t^H) - U(c_{ss}^{H \to W}, l_{ss}^H) &\approx \frac{\partial U}{\partial c_t^{H \to W}}(c_{ss}^{H \to W}, l_{ss}^H) \cdot (c_t^{H \to W} - c_{ss}^{H \to W}) \nonumber \\
    &\quad + \frac{\partial U}{\partial l_t^H}(c_{ss}^{H \to W}, l_{ss}^H) \cdot (l_t^H - l_{ss}^H)
\end{align}
ここで、定常状態における各偏微分係数は以下の通りである。
\begin{align}
    \frac{\partial U}{\partial c_t^{H \to W}}(c_{ss}^{H \to W}, l_{ss}^H) &= \frac{1}{c_{ss}^{H \to W}} \\
    \frac{\partial U}{\partial l_t^H}(c_{ss}^{H \to W}, l_{ss}^H) &= -\phi^H l_{ss}^H
\end{align}
また、ここでの変数 \( \hat{x}_t \) を定常状態からの乖離 \( \hat{x}_t = \frac{x_t - x_{ss}}{x_{ss}} \) と定義すると、レベル変数の乖離は \( x_t - x_{ss} = x_{ss} \hat{x}_t \) と変換できる。
これらを上式に代入して整理すると、以下の線形近似された効用の変動式が得られる。
\begin{align}
    U(c_t^{H \to W}, l_t^H) - U(c_{ss}^{H \to W}, l_{ss}^H) &\approx \frac{1}{c_{ss}^{H \to W}} \cdot (c_{ss}^{H \to W} \hat{c}_t^{H \to W}) + (-\phi^H l_{ss}^H) \cdot (l_{ss}^H \hat{l}_t^H) \nonumber \\
    &= \hat{c}_t^{H \to W} - \phi^H (l_{ss}^H)^2 \hat{l}_t^H \label{eq:appendix_welfare_correction_utility_base}
\end{align}

\paragraph{2. 労働投入量の近似と補正項の導出}

次に、労働投入量 \( \hat{l}_t^H \) を生産関数から導出する。
第 3 章で示した集計生産関数 \( y_t^H = a_t^H l_t^H / \Delta_t^H \) を、まず労働投入量 \( l_t^H \) について解く。
\begin{equation}
    l_t^H = \frac{y_t^H \Delta_t^H}{a_t^H}
\end{equation}
この式の両辺について定常状態からの乖離( ハット変数 )をとると、以下の関係が得られる。
\begin{equation}
    \hat{l}_t^H = \hat{y}_t^H - \hat{a}_t^H + \hat{\Delta}_t^H \label{eq:appendix_welfare_correction_labor_approx}
\end{equation}

この式 \eqref{eq:appendix_welfare_correction_labor_approx} を、そのまま式 \eqref{eq:appendix_welfare_correction_utility_base} に代入すると、以下の式が得られる。
\begin{align}
    U(c_t^{H \to W}, l_t^H) - U(c_{ss}^{H \to W}, l_{ss}^H) &\approx \hat{c}_t^{H \to W} - \phi^H (l_{ss}^H)^2 (\hat{y}_t^H - \hat{a}_t^H + \hat{\Delta}_t^H) \nonumber \\
    &= \hat{c}_t^{H \to W} - \phi^H (l_{ss}^H)^2 (\hat{y}_t^H - \hat{a}_t^H) - \phi^H (l_{ss}^H)^2 \hat{\Delta}_t^H \label{eq:appendix_welfare_correction_utility_combined}
\end{align}

シミュレーションにおいては、この第 3 項 \( -\phi^H (l_{ss}^H)^2 \hat{\Delta}_t^H \) が重要な役割を果たす。標準的な線形近似モデルではこの項が欠落してしまうため、本分析では特別に 2 次近似を用いて \( \hat{\Delta}_t^H \) を導出し、手動で厚生計算に反映させる。

以下に、その導出過程を記述する。

\paragraph{価格分散の関数の定義と近似方針}

まず、第 3 章で導出した\textbf{物価指数の動学}( 式 \ref{eq:final_price_index_dynamics_H} )と\textbf{価格分散の動学}( 式 \ref{eq:final_dispersion_dynamics_H} )を出発点とする。ここで議論を簡潔にするため人口を \( N=1 \) とする。
\begin{equation}
    1 = (1-\xi^H) \left( \frac{\widetilde{p}_t^H}{p_t^H} \right)^{1-\theta^H} + \xi^H (\pi_t^H)^{\theta^H-1} \label{eq:appendix_welfare_correction_pi_dynamics}
\end{equation}
\begin{equation}
    \Delta_t^H = (1-\xi^H) \left( \frac{\widetilde{p}_t^H}{p_t^H} \right)^{-\theta^H} + \xi^H (\pi_t^H)^{\theta^H} \Delta_{t-1}^H \label{eq:appendix_welfare_correction_delta_dynamics}
\end{equation}
式 \eqref{eq:appendix_welfare_correction_pi_dynamics} を相対価格 \( \widetilde{p}_t^H / p_t^H \) について解き、それを式 \eqref{eq:appendix_welfare_correction_delta_dynamics} に代入して相対価格を消去すると、価格分散 \( \Delta_t^H \) はインフレ率 \( \pi_t^H \) と前期の分散 \( \Delta_{t-1}^H \) のみの関数として、以下のように定義できる。
\begin{equation}
    \Delta_t^H = (1-\xi^H)^{\frac{1}{1-\theta^H}} \left( 1 - \xi^H (\pi_t^H)^{\theta^H-1} \right)^{\frac{\theta^H}{\theta^H-1}} + \xi^H (\pi_t^H)^{\theta^H} \Delta_{t-1}^H \label{eq:appendix_welfare_correction_delta_combined}
\end{equation}

この関数を、定常状態 \( (\pi_{ss}^H, \Delta_{ss}^H) = (1, 1) \) の周りで 2 次テイラー展開( 近似 )する。
一般に、 2 変数関数 \( f(x, y) \) の点 \( (x_0, y_0) \) 周りでの 2 次近似公式は以下のように記述される。
\begin{align}
    f(x, y) &\approx f(x_0, y_0) + \frac{\partial f}{\partial x}(x_0, y_0)(x - x_0) + \frac{\partial f}{\partial y}(x_0, y_0)(y - y_0) \nonumber \\
    &\quad + \frac{1}{2} \left[ \frac{\partial^2 f}{\partial x^2}(x_0, y_0)(x - x_0)^2 + 2\frac{\partial^2 f}{\partial x \partial y}(x_0, y_0)(x - x_0)(y - y_0) + \frac{\partial^2 f}{\partial y^2}(x_0, y_0)(y - y_0)^2 \right]
\end{align}
この公式を本モデルに適用する。ここで、インフレ率と価格分散の定常状態値は \( \pi_{ss}^H = 1, \Delta_{ss}^H = 1 \) であるため、レベル変数の乖離 \( x_t - 1 \) は、対数乖離 \( \hat{x}_t \) と以下のように一致する。
\begin{equation}
    x_t - 1 = \frac{x_t - 1}{1} = \frac{x_t - x_{ss}}{x_{ss}} = \hat{x}_t
\end{equation}
また、変数 \( \pi_t^H \) と \( \Delta_{t-1}^H \) は第 2 項において分離した形( 積の形 )で入っており、第 2 項は \( \Delta_{t-1}^H \) について 1 次式であるため、 \( \Delta_{t-1}^H \) に関する 2 階微分はゼロとなる。また交差項 \( \hat{\pi}_t^H \hat{\Delta}_{t-1}^H \) は 3 次の微小量となるため無視できる。

したがって、求める近似式は以下の形となる。
\begin{equation}
    \hat{\Delta}_t^H \approx \frac{\partial \Delta_t^H}{\partial \Delta_{t-1}^H}(1, 1) \hat{\Delta}_{t-1}^H + \frac{\partial \Delta_t^H}{\partial \pi_t^H}(1, 1) \hat{\pi}_t^H + \frac{1}{2} \frac{\partial^2 \Delta_t^H}{\partial (\pi_t^H)^2}(1, 1) (\hat{\pi}_t^H)^2
\end{equation}

\paragraph{偏微分係数の導出}

まず、式 \eqref{eq:appendix_welfare_correction_delta_combined} に基づき、全ての偏導関数を計算する。

\textbf{1. \( \Delta_{t-1}^H \) に関する 1 階偏微分} \\
第 2 項のみを微分する。
\begin{equation}
    \frac{\partial \Delta_t^H}{\partial \Delta_{t-1}^H} = \xi^H (\pi_t^H)^{\theta^H}
\end{equation}

\textbf{2. \( \pi_t^H \) に関する 1 階偏微分} \\
合成関数の微分公式を用いて計算する。
\begin{align}
    \frac{\partial \Delta_t^H}{\partial \pi_t^H} &= (1-\xi^H)^{\frac{1}{1-\theta^H}} \frac{\theta^H}{\theta^H-1} \left( 1 - \xi^H (\pi_t^H)^{\theta^H-1} \right)^{\frac{\theta^H}{\theta^H-1}-1} \left( -\xi^H (\theta^H-1) (\pi_t^H)^{\theta^H-2} \right) \nonumber \\
    &\quad + \xi^H \theta^H (\pi_t^H)^{\theta^H-1} \Delta_{t-1}^H \nonumber \\
    &= -\theta^H \xi^H (1-\xi^H)^{\frac{1}{1-\theta^H}} \left( 1 - \xi^H (\pi_t^H)^{\theta^H-1} \right)^{\frac{1}{\theta^H-1}} (\pi_t^H)^{\theta^H-2} \nonumber \\
    &\quad + \xi^H \theta^H (\pi_t^H)^{\theta^H-1} \Delta_{t-1}^H
\end{align}

\textbf{3. \( \pi_t^H \) に関する 2 階偏微分} \\
上記の 1 階偏微分をさらにもう一度 \( \pi_t^H \) で微分する。第 1 項には積の微分公式を適用する。
\begin{align}
    \frac{\partial^2 \Delta_t^H}{\partial (\pi_t^H)^2} &= -\theta^H \xi^H (1-\xi^H)^{\frac{1}{1-\theta^H}} \Bigg[ \frac{1}{\theta^H-1} \left( 1 - \xi^H (\pi_t^H)^{\theta^H-1} \right)^{ \frac{1}{\theta^H-1}-1} (-\xi^H (\theta^H-1) (\pi_t^H)^{\theta^H-2}) \cdot (\pi_t^H)^{\theta^H-2} \nonumber \\
    &\quad + \left( 1 - \xi^H (\pi_t^H)^{\theta^H-1} \right)^{\frac{1}{\theta^H-1}} \cdot (\theta^H-2)(\pi_t^H)^{\theta^H-3} \Bigg] \nonumber \\
    &\quad + \xi^H \theta^H (\theta^H-1) (\pi_t^H)^{\theta^H-2} \Delta_{t-1}^H
\end{align}

次に、計算したこれらの偏導関数を定常状態 \( (\pi_{ss}^H, \Delta_{ss}^H) = (1, 1) \) で評価する。

\textbf{1. ラグ項の係数の評価}
\begin{equation}
    \frac{\partial \Delta_t^H}{\partial \Delta_{t-1}^H}(1, 1) = \xi^H (1)^{\theta^H} = \xi^H
\end{equation}

\textbf{2. 1 次の係数の評価}
\begin{align}
    \frac{\partial \Delta_t^H}{\partial \pi_t^H}(1, 1) &= -\theta^H \xi^H (1-\xi^H)^{\frac{1}{1-\theta^H}} \left( 1 - \xi^H \right)^{\frac{1}{\theta^H-1}} (1) + \xi^H \theta^H (1) (1) \nonumber \\
    &= -\theta^H \xi^H (1-\xi^H)^{\frac{1}{1-\theta^H} + \frac{1}{\theta^H-1}} + \xi^H \theta^H \nonumber \\
    &= -\theta^H \xi^H (1-\xi^H)^0 + \xi^H \theta^H \nonumber \\
    &= -\theta^H \xi^H + \theta^H \xi^H = 0
\end{align}

\textbf{3. 2 次の係数の評価}
\begin{align}
    \frac{\partial^2 \Delta_t^H}{\partial (\pi_t^H)^2}(1, 1) &= -\theta^H \xi^H (1-\xi^H)^{\frac{1}{1-\theta^H}} \Bigg[ -\xi^H (1-\xi^H)^{\frac{1}{\theta^H-1}-1} (1) + (1-\xi^H)^{\frac{1}{\theta^H-1}} (\theta^H-2) \Bigg] \nonumber \\
    &\quad + \xi^H \theta^H (\theta^H-1) (1) (1) \nonumber \\
    &= \theta^H (\xi^H)^2 (1-\xi^H)^{-1} - \theta^H \xi^H (\theta^H-2) (1) + \xi^H \theta^H (\theta^H-1) \nonumber \\
    &= \frac{\theta^H (\xi^H)^2}{1-\xi^H} + \theta^H \xi^H \left[ -(\theta^H-2) + (\theta^H-1) \right] \nonumber \\
    &= \frac{\theta^H (\xi^H)^2}{1-\xi^H} + \theta^H \xi^H \nonumber \\
    &= \theta^H \xi^H \left( \frac{\xi^H}{1-\xi^H} + 1 \right) = \frac{\theta^H \xi^H}{1-\xi^H}
\end{align}

\paragraph{結論:価格分散の 2 次近似式と線形モデルでの含意}

以上の計算結果を近似式に代入することで、以下の最終的な関係式が得られる。
\begin{align}
    \hat{\Delta}_t^H &\approx \xi^H \hat{\Delta}_{t-1}^H + 0 \cdot \hat{\pi}_t^H + \frac{1}{2} \left( \frac{\theta^H \xi^H}{1-\xi^H} \right) (\hat{\pi}_t^H)^2 \nonumber \\
    &= \xi^H \hat{\Delta}_{t-1}^H + \frac{\theta^H \xi^H}{2(1-\xi^H)} (\hat{\pi}_t^H)^2
    \label{eq:appendix_welfare_correction_delta_approx_final}
\end{align}
これが最終的な価格分散の 2 次近似式である。

この結果( 式 \ref{eq:appendix_welfare_correction_delta_approx_final} )は、線形近似モデルにおいて価格分散項がゼロとなることの数学的な証明となっている。具体的には、上式の導出過程で確認した通り、インフレ率 \( \hat{\pi}_t^H \) に関する 1 次の係数( 偏微分係数 )は計算の結果ちょうど \( 0 \) となる。したがって、もし方程式系全体を 1 次のオーダーで近似( 線形化 )した場合、 2 次以上の微小量である右辺第 2 項( \( (\hat{\pi}_t^H)^2 \) を含む項 )は無視され、式は以下のように退化する。
\[
    \hat{\Delta}_t^H \approx \xi^H \hat{\Delta}_{t-1}^H
\]
定常状態から出発する場合、初期値は \( \hat{\Delta}_0^H = 0 \) であるため、この再帰式に従えば、将来にわたり常に \( \hat{\Delta}_t^H = 0 \) となる。これが、標準的な線形近似シミュレーションにおいて価格分散による厚生ロスが消失してしまう理由である。

\paragraph{厚生計算に用いる最終的な効用関数}

最後に、導出された価格分散の近似式 \eqref{eq:appendix_welfare_correction_delta_approx_final} を、先述の効用の近似式 \eqref{eq:appendix_welfare_correction_utility_combined} に代入することで、本分析のプログラムにおいて実際に計算される最終的な効用関数の式が得られる。

\begin{equation}
    U(c_t^{H \to W}, l_t^H) - U(c_{ss}^{H \to W}, l_{ss}^H) \approx \hat{c}_t^{H \to W} - \phi^H (l_{ss}^H)^2 (\hat{y}_t^H - \hat{a}_t^H) - \phi^H (l_{ss}^H)^2 \left( \xi^H \hat{\Delta}_{t-1}^H + \frac{\theta^H \xi^H}{2(1-\xi^H)} (\hat{\pi}_t^H)^2 \right)
    \label{eq:appendix_welfare_correction_final_utility}
\end{equation}     % 厚生補正( 価格分散の動学 )
\input{sections/app/appendix_insulation_effect.tex}      % 遮断効果の数学的証明