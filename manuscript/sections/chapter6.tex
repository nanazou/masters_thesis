% !TeX root = ../main.tex
% sections/chapter6.tex

\chapter{考察}
\label{chap:discussion}
第5章のシミュレーション分析はゼロ金利制約が存在する物価が硬直的な経済において
名目総消費水準目標がマクロ経済を安定させることを明らかにした。
本章ではこの名目総消費水準目標について AS-AD 分析によるまとめの解釈を与え実務面から補足もおこなう。

\section{AS-AD 分析による名目総消費水準目標の働きのまとめ}
\label{sec:discussion_as_ad}
ここでは AS-AD 分析を用いて名目総消費水準目標の効果について総括する。
AS 曲線は期待生産者物価指数 \( E_t [p_{t+1}^H] \) によって動かされ、
AD 曲線は名目利子率 \( i_t^H \) と所得の期待限界効用 \( E_t [\lambda_{t+1}^H] \) によって動かされるのであった。

\subsection{\( \beta \) ショック(需要ショック)への反応}
\label{sec:discussion_as_ad_beta_shock}
負の \( \beta^H \) ショックにおいて名目総消費水準目標が優れた成績を残した要因として
名目利子率 \( i_t^H \) と所得の期待限界効用 \( E_t [\lambda_{t+1}^H] \) を
効果的に下落させたことが考えられる。

\paragraph{1. 名目利子率 \( i_t^H \)}
負の \( \beta^H \) ショックは自国家計の財布のひもを固くするショックであるから
とりわけ総消費指数 \( c_t^{H \to W} \) を大幅に低下させる。
これにより名目総消費 \( p_t^{H \to W} c_t^{H \to W} \) も大幅に低下するため
名目総消費水準目標により自国の利子率 \( i_t^H \) は大きく下げられ
ゼロ金利に達してからの目標未達分の累積も大きくなる。
これにより AD 曲線は右方に移動する。
\paragraph{2. 所得の期待限界効用 \( E_t [\lambda_{t+1}^H] \)}
1階の条件(FOC)より名目総消費 \( p_{t+1}^{H \to W} c_{t+1}^{H \to W} \) は
所得の限界効用 \( \lambda_{t+1}^H \) の逆数であった。
付録Aの「割引因子ショックを含む為替レート決定式の導出」の期待値の議論より、
この逆数の関係はこれらの期待値についても近似的に成立する。
したがって名目総消費水準目標は期待名目総消費 \( E_t [p_{t+1}^{H \to W} c_{t+1}^{H \to W}] \) を上昇、
よって所得の期待限界効用 \( E_t [\lambda_{t+1}^H] \) を下落させる。
これにより AD 曲線は右方へと移動する。
\paragraph{3. 期待生産者物価指数 \( E_t [p_{t+1}^H] \)}
名目総消費指数 \( p_t^{H \to W} c_t^{H \to W} \) には消費者物価指数 \( p_t^{H \to W} \) の一部として
生産者物価指数 \( p_t^H \) が含まれる。
付録Aの「割引因子ショックを含む為替レート決定式の導出」の期待値の議論より、
この逆数の関係はこれらの期待値についても近似的に成立する。
そのため名目総消費水準目標は期待名目総消費 \( E_t [p_{t+1}^{H \to W} c_{t+1}^{H \to W}] \) を上昇、
よって期待生産者物価指数 \( E_t [p_{t+1}^H] \) の下落を抑えた。
これにより AS 曲線はほとんど動かない。


こうして AD 曲線を大きく右方移動させた一方で、AS 曲線はおおよそ初期位置にとどめたため
経済を強力に回復へと向かった。


\subsection{\( a \) ショック(供給ショック)のまとめ}
\label{sec:discussion_as_ad_a_shock}
第 5 章の分析結果から負の \( a^H \) ショックに対しても名目総消費水準目標は高い頑健性を示すことが明らかになった。
付録 \ref{app:appendix_exchange_rate} 式 \eqref{eq:appendix_exchange_rate_final} で証明したとおり
本稿のモデルでは \( a^H \) ショック時には名目総消費と名目 GDP は一致する。
これにより名目総消費水準目標は名目 GDP 水準目標と同値な政策となる。
\paragraph{1. 名目利子率  \( i_t^H \)}
負の \( a^H \) ショックは生産性を低下させるショックであるから生産 \( y_t^H \) を低下させるかのようにみえる。
しかしながら本稿モデルの価格硬直性の仮定により AS 曲線の傾き \( \kappa = 0 \) となり
\( a^H \) 自体の変動は AS 曲線を動かさない。
これにより生産 \( y_t^H \) や生産者物価指数 \( p_t^H \) は動かない。
したがって名目総消費水準目標は名目金利 \( i_t^H \) を動かさない。
これにより AD 曲線は動かない。
\paragraph{2. 所得の期待限界効用 \( E_t [\lambda_{t+1}^H] \)}
前節において名目消費目標に関する主要な研究を概観した段落同様、負の \( a^H \) ショックでは生産 \( y_t^H \) や生産者物価指数 \( p_t^H \) はほとんど動かず
したがって名目総消費水準目標は名目金利 \( i_t^H \) を動かさない。
よって生産 \( y_t^H \) および生産者物価指数 \( p_t^H \) は将来においても動かないことが予想されるため
期待生産 \( E_t [y_t^H] \) および期待生産者物価指数 \( E_t [p_{t+1}^H] \) も変動しない。
したがって期待名目総消費 \( E_t [p_{t+1}^{H \to W} c_{t+1}^{H \to W} = \bar{p}_{t+1}^H y_{t+1}^H] \) も変動しないから
その逆数である所得の期待限界効用 \( E_t [\lambda_{t+1}^H] \) も変動しない。
これにより AD 曲線は動かない。
\paragraph{3. 期待生産者物価指数 \( E_t [p_{t+1}^H] \)}
全段落で述べたように、本稿の価格が硬直的なモデルにおいては
\( a^H \) ショックでは生産 \( y_t^H \) や生産者物価指数 \( p_t^H \) は動かず
したがって名目総消費水準目標は名目金利 \( i_t^H \) を動かさない。
よって生産者物価指数 \( p_t^H \) は将来においても動かないことが予想されるため
期待生産者物価指数 \( E_t [p_{t+1}^H] \) も変動しない。
これにより AS 曲線は動かない。


このように名目総消費水準目標は負の \( a^H \) ショックに際しては不必要な政策介入をおこなわず
生産の維持された経済をただ静観する。


\section{実務面からみた名目総消費水準目標の実現可能性}
\label{sec:discussion_practicality}
実務において金融政策を採用できるかどうかは目標とする指標の速報性および正確性に依存する。
本節では指標作成の実務的観点から名目総消費水準目標の実現可能性を考察する。

\subsection{主要統計の作成主体と現状の課題}
まず本稿が比較対象とする主要3指標の作成過程を整理する(表 \ref{tab:stat_comparison})。
\begin{table}[H]
\centering
\caption{主要マクロ統計の特性比較}
\label{tab:stat_comparison}
\small
\begin{tabular}{lllp{20em}}
\toprule
指標 & 作成主体 & 公表頻度 & 特徴と実務上の課題 \\
\midrule
名目 GDP & 内閣府 & 四半期 & 確定に時間を要し、大幅な遡及改定が頻発する。 \\
消費者物価指数 (CPI) & 総務省 & 月次 & 品質調整(ヘドニック法)等の統計的加工が複雑。 \\
消費活動指数(名目) & 日本銀行 & 月次 & 速報性は高いが、既存統計に基づく推計値の側面が強い。 \\
\bottomrule
\end{tabular}
\end{table}
日本銀行が公表する「消費活動指数」は、供給側(商業動態統計等)と需要側(家計調査等)のデータを統合した速報性の高い指標であるが、現時点では後に算出される国民経済計算(GDP 統計)の確定値に比して正確性で劣るという課題がある。

\subsection{デジタル技術による正確性と速報性の両立}
しかし近年のデジタル技術の発展は統計の正確性を劇的に向上させる可能性を秘めている。
\begin{itemize}
    \item \textbf{ビッグデータの活用:} クレジットカードや電子マネーの決済データ(キャッシュレスデータ)のリアルタイム集計により、サンプル調査に頼らない全数に近い消費動向の把握が可能となる。
    \item \textbf{POS データの連携:} 小売店の販売時点情報管理(POS)データは、従来のアンケート方式よりも迅速かつ正確に「実際に支払われた名目額」を記録する。
\end{itemize}
これらの技術により現在の日銀統計が持つ「速報性」を維持したまま「正確性」が補完されることで
名目総消費水準目標を現実的な政策ルールとして運用するための技術的基盤が整いつつある。

\subsection{他政策に対する集計上の優位性}
以上の実務的背景を踏まえ、名目総消費水準目標がもつ相対的な優位性を以下の通り整理する。
\paragraph{1. 名目 GDP 水準目標(NGDPLT)に対する優位性}
GDP は消費以外に投資、政府支出、在庫変動、純輸出を網羅する必要がある。
特に設備投資や在庫の推計は測定誤差が大きく、速報値と確報値の乖離(遡及改定)の主因となる。
対して名目総消費は、取引頻度が高くデジタル決済との親和性が強いため、GDP 全体よりも早期かつ正確な集計が可能である。
\paragraph{2. インフレ目標(IT)およびテーラールール(TR)に対する優位性}
名目総消費水準目標は伝統的なインフレ目標(IT)やテーラールール(TR)に対しても集計上の優位性をもつ。
まず実務上のインフレ目標やテイラールールが参照する消費者物価指数(CPI)の算出には
製品の性能向上を価格下落とみなす「品質調整」や「代替バイアス」の処理など高度に専門的な統計的調整を要する。
このような統計的加工の複雑さはデータの公表に際して不可避なタイムラグを生じさせるだけでなく
推計の前提となる仮定の置き方次第で数値が変動しうるという不確実性を伴う。
これに対して名目総消費は取引された名目額そのものであり確実性が高い。
さらにテーラールールは潜在生産(potential output)という推定の困難な情報にも依存している。
第 2 章でも述べた通り潜在生産の誤推計は政策判断を誤らせ
経済を不安定化させる主因となりうる \parencite{Orphanides2003, BeckworthHendrickson2019}。
対して名目総消費は実際に観測可能な指標であるため、潜在産出のような誤推計の問題は生じにくい。


\section{政策意図の浸透}
\label{sec:discussion_communication}
実務における金融政策の効果は中央銀行の意図が家計に正しく伝わり適切な期待を形成できるかに依存する。
この観点からも名目総消費水準目標は名目 GDP 水準目標に対して明確な優位性をもつ。
元 FRB 局長の Nathan Sheets が指摘したように
 GDP は家計にとってはなじみが浅いためその概念が十分に理解されているとは言えない。
対照的に消費は家計の生活に直結した指標である。
そのため「中央銀行が皆さんの名目的な消費水準を将来にわたって保証する」という約束は
家計の貯蓄・消費判断に対して極めて具体的かつ直感的な指針を与える。
こうして消費を標的とすることは、とりわけゼロ金利下において
中央銀行の景気回復への約束を家計に浸透させ期待を形成するための強力な手段となりうる。


\section{モデルの限界と今後の研究課題}
\label{sec:discussion_limitations}
本稿の分析はゼロ金利制約下における名目総消費水準目標の優位性を理論的、実務的に明らかにしたが、
理論的評価の厳密性と実務的適用可能性をさらに高めるためには以下の課題を検討していく必要がある。

\subsection{効用関数の一般化}
\label{sec:discussion_limitations_preference}
本稿では家計の自国財と外国財の代替弾力性 \( \eta = 1 \) および
対数効用(相対的リスク回避度 \( \sigma = 1 \))を仮定した。
これらの仮定により 1 階の条件(FOC)において所得の限界効用 \( \lambda^H \) が
名目総消費支出 \( p^{H \to W} c^{H \to W} \) の逆数となり、さらに外国関連変数への遮断効果が生じた。
しかしこれらのパラメータが 1 ではない効用関数の下では
名目総消費目標のもつ期待への働きかけの効果が変化し、外国関連変数の動きも思わぬ波及をもたらす可能性がある。
これらの効用パラメータを一般化した場合の政策評価は理論の汎用性を高める上での課題となる。

\subsection{家計の異質性の導入}
\label{sec:discussion_limitations_heterogeneity}
効用関数の一般化と関連して家計の異質性の導入が考えられる。
本稿は家計パラメータの同一性と国内債券市場の完備性を仮定することで
消費や最適生産者物価を全家計で共通化し代表的家計モデルによる分析をおこなった。
しかし現実には家計の選好や物価改定確率は一様ではなく、
またあらゆるリスクに対して債券が完備されているわけでもない。
そこでたとえば主観的割引因子 \( \beta \) を高めた我慢強い家計、
低めたその日暮らしの家計が混在するモデルを作ることもできる。
また国内債券市場が完備でないモデルを考えれば
価格改定機会の運により生まれた生産所得の格差が保険によって均一化されないため
可処分所得が一致せず、消費に違いが生まれる。
こうした多様な家計が存在する経済における政策評価も今後の課題である。

\subsection{経済の硬直性の緩和}
\label{sec:discussion_limitations_rigidity}
本分析では日本経済の実証データ等にもとづき物価の強い硬直性( \( \xi^H = 0.99 \) )や
ショックの強い持続性を仮定している。
これは名目総消費目標のもつ期待への強い働きかけの能力を最大限に活かすための設定であった。
経済にこうした硬さがないのであれば
名目総消費目標のもつ \( \beta \) ショックへの優位性は薄れる可能性がある。
また物価が硬直的でないとすると \( a \) ショックが起こった際に価格はより上昇するだろうから
名目総水準目標においては生産の下落が予想される。
このようなときに名目利子率に動きは出るのか、あるいは依然として静観するのだろうか。
こうした検証をおこなうには、より広範なパラメータ空間で分析をおこなう必要がある。
