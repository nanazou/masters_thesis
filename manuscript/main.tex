%%%%%%%%%%%%%%%%%%%%%%%%%%%%%%%%%%%%%%%%%
% マスターファイル: main.tex
% 2025年度 立教大学大学院 経済学研究科 修士学位申請論文(改訂版)
%%%%%%%%%%%%%%%%%%%%%%%%%%%%%%%%%%%%%%%%%

% --- 1. メタデータ (main.xmpdata) の同期設定 ---
% ご提示いただいた内容を正確に反映しています。
\begin{filecontents*}{\jobname.xmpdata}
\Title{ゼロ金利制約下における名目総消費水準目標の優位性(改訂版)}
\Author{板倉 俊}
\Keywords{金融政策, ゼロ金利制約, 名目総消費水準目標}
\Publisher{立教大学大学院}
\Subject{修士学位申請論文}
\Copyright{Copyright (c) 2026 Shun Itakura. All rights reserved.}
\end{filecontents*}

% --- 2. 設定の読み込み ---
% !TeX root = main.tex
% preamble.tex

\documentclass[book]{jlreq}

% ----- 必須パッケージ -----
\usepackage{amsmath}
\usepackage{amssymb}
\usepackage{booktabs}
\usepackage{graphicx}
\usepackage{float}
\usepackage{longtable}
\usepackage{multirow}
\usepackage{subcaption}
\usepackage{pdfpages}
\usepackage{url}
\usepackage{color}

% ★ PDF/A 形式の設定 (hyperrefの機能を含みます) ★
\usepackage[a-1b]{pdfx} 

% ----- 参考文献の設定 -----
\usepackage[
  style=authoryear,
  backend=biber,
  uniquename=false,
  uniquelist=false,
  maxcitenames=1
]{biblatex}
\addbibresource{references.bib}

% ----- プログラムコード表示用パッケージ -----
\usepackage{listings}
\usepackage{xcolor}

\lstset{
  basicstyle={\ttfamily\small},
  keywordstyle={\color{blue}\bfseries},
  commentstyle={\color{green!50!black}},
  stringstyle={\color{purple}},
  frame=single,
  numbers=left,
  numberstyle={\tiny\color{gray}},
  breaklines=true,
  captionpos=t,
  keepspaces=true,
  showstringspaces=false,
  upquote=true,
  columns=[l]{fullflexible}
}

% 「要旨」の定義
\newcommand{\abstractname}{要旨}
\newenvironment{abstract}
 {\begin{center}%
  \bfseries\abstractname
 \end{center}%
 \quotation}
{\endquotation}

\begin{document}

% ===== 立教大学様式 表紙 =====
\begin{titlepage}
    \centering
    \vspace*{10mm}
    {\Large 2025年度 修士学位申請論文} \\
    \vfill % 余白を自動調整し、1ページに収めます
    {\huge ゼロ金利制約下における \\ 名目総消費水準目標の優位性} \\
    \vspace{1em}
    {\Large (改訂版)} \\
    \vfill
    {\Large 立教大学大学院 \\ 経済学研究科経済学専攻 \\ 博士課程前期課程} \\
    \vfill
    {\Large 氏名 板倉 俊} \\
%    {\Large 学生番号 24KK009H} \\ % ★ご自身の学生番号をご入力ください
    \vfill
    {\Large 修士論文作成指導教員} \\
    {\Large 山本 周吾} \\ 
    \vspace*{20mm}
\end{titlepage}

\clearpage

% ===== 前付 (要旨・謝辞・目次) =====
\pagenumbering{roman} 

\begin{abstract}
\setlength{\parindent}{0em}
\setlength{\parskip}{1em}
\noindent
% !TeX root = ../main.tex
% sections/abstract.tex

本稿はゼロ金利に陥った経済を最も効果的に回復させる金融政策は名目総消費水準目標であると主張する。
名目総消費水準目標が最も効果的である理由は、過去の目標未達分は将来必ず埋め合わせるとの約束のもと
所得の限界効用の逆数である名目総消費を目標とすることで所得の限界効用をきわめて効率的に低下させ、
それにより貯蓄から消費への流れを強く促すためである。

所得の限界効用は追加1単位の貨幣で消費をおこなうことにより得られる効用である。
したがって将来における所得の限界効用が小さくなればなるほど
将来において1単位の貨幣で消費をおこなうことにより得られる効用も小さくなる。
そして(消費から得られる効用が消費の対数関数であるとき)所得の限界効用は
名目総消費の逆数となることが知られている。
よって将来において名目総消費を上昇させることすなわち所得の限界効用を減少させることを中央銀行が保証するならば、
家計は貯蓄して将来の消費から効用を得るよりも現在の消費を増やして効用を得る方が
より大きな生涯効用を得ることができる。

名目総消費水準目標は過去の目標未達分は将来必ず埋め合わせるとの約束により
人々の期待に強く働きかけようとする水準目標の一種である。
2008年の世界金融危機以降、深刻な景気後退に見舞われた各国の中央銀行は利下げを推し進めたが、
ついにはゼロ金利に達することとなり政策の有効性は大きく損なわれた。
ゼロ金利制約により利子率はゼロより下がることはなくなり
利子率をさらに引き下げて景気を刺激するという直接的な手段が失われたのである。
そこで残されたのは政策が人々の期待に働きかける効果であるが、伝統的な金融政策はこの期待形成力が弱いとされ、
この点で優位性のある水準目標に注目が集まった。
そうした水準目標の中でも名目GDP水準目標は少なからぬ学者に支持されその有効性がいくつかの研究で示されてきた。

では名目GDP水準目標よりも経済を効率的に回復させる金融政策はないのだろうか。
金融政策により経済を回復させるには名目利子率と将来の所得の限界効用を低下させ総需要曲線を上方に動かす必要がある。
伝統的なインフレ目標はインフレ率を目標とすることでこれをおこなってきた。
将来のインフレ率が上昇すると将来の物価が上昇するため、将来の名目総消費が増加する。
つまり名目総消費を構成する物価と消費のうち物価を上昇させることにより
将来の名目総消費を上昇させその逆数である将来の所得の限界効用を引き下げようとしたのだ。
これに対して名目総消費水準目標は名目総消費全体を目標とする。
こうすることにより将来の名目総消費全体がより効果的に押し上げられ
将来の所得の限界効用もより効率的に低下するのである。

この主張を検証するため本稿においては\textcite{Woodford1996}や\textcite{CespedesChangVelasco2002}
などを参考に自国と外国からなる開放経済モデルを構築する。
自国と外国においてはそれぞれの国内債券市場が完備であるとし、
これにより一国は代表的家計と呼ばれる仮想的な家計に代表され数学的分析が容易になる。
一方で自国と外国のあいだの国際債券市場は不完備であると仮定する。
このことにより国際債券の動きを通じた非対称な二国間の関係を分析することが可能となる。
このモデルにおいて、自国家計が我慢強くなるショック、
すなわち自国家計が将来を悲観することにより消費よりも貯蓄を好むようになるショックを与える。
このショックにより自国経済は深刻な不況とゼロ金利に陥るが、
その初期の落ち込みはどの程度か、またそこからの上昇の様子はどのようであるかを
名目総消費水準目標を含む11の金融政策について観察する。
このシミュレーションの結果、
名目総消費水準目標においては消費と生産の初期の落ち込みが最も小さく回復も最も早いことが示される。
また所得の限界効用についても初期の上昇が最も小さく下落も最も速やかであることが確認される。
その結果として家計の効用に基づいて計算された厚生の落ち込みも名目総消費水準目標において最小となる。

以上の分析により、本稿は名目総消費水準目標が
ゼロ金利下の経済を最も効果的に回復させる金融政策であると結論付ける。

\end{abstract}
\clearpage

% !TeX root = ../main.tex
% sections/acknowledgments.tex

\chapter*{謝辞}
\addcontentsline{toc}{chapter}{謝辞}

% --- ここから謝辞の内容を記入してください ---

本稿の作成にあたり指導教員の山本周吾氏には長時間にわたり理論の基礎から丁寧なご指導をいただきました。
また副指導教員の岡部桂史氏には授業の内外を通じて
論文の執筆姿勢や経済学的な思考法について有益なご助言をいただきました。
そして所属先の東京保安通信株式会社には本稿作成のためにお時間いただくなど多方面にわたりお世話になりました。
皆様のご助力に心より感謝申し上げます。

% -------------------------------------------
\clearpage

\tableofcontents
\clearpage

% ===== 本文 =====
\pagenumbering{arabic}
\setcounter{page}{1}

% !TeX root = ../main.tex
% sections/chapter1.tex

\chapter{はじめに}
\label{chap:intro}

\section{研究の背景}
\label{sec:intro_background}
2008年の世界金融危機以降、各国の中央銀行は経済を回復させるため利子率を段階的にゼロ近辺まで引き下げた。
しかしその結果、それ以上の利下げが困難となるゼロ金利制約に直面したことで
利子率を低下させて総需要曲線を上へ移動させるという直接的な経済回復の手段が失われた。
このように利子率を下げられなくなった場合、所得の期待限界効用を下げることが残された手段となる。
しかしながら、伝統的な金融政策は期待への働きかけが十分ではなかったため手詰まり感がみられ
経済は容易に好転しなかった。
そうした中で注目を集めるようになったのが将来の景気回復後も緩和の継続を約束することにより
家計の期待に強く働きかけようとする水準目標である。
なかでも名目GDP水準目標はいくつかの理論的研究によりその効果が裏付けられており、
水準目標の中でもとりわけ多くの支持を集めている。
本稿はこのような政策環境を背景として議論を展開する。

\section{本稿の問い}
\label{sec:intro_question}
ではゼロ金利下の経済を最も効率的に回復させる金融政策は名目GDP水準目標なのだろうか。
あるいはより良い水準目標があるだろうか。
先ほど述べた所得の期待限界効用とは、
将来において追加1単位の貨幣で消費をおこなうことにより得られるであろう効用の期待値である。
そのためもし所得の期待限界効用が低下すれば、
家計は貯蓄して将来の消費から効用を得るよりも現在の消費を増やして効用を得る方が
より大きな生涯効用を得ることができるだろうと考えるため、貯蓄から消費への流れが生まれる。
一方で、消費から得られる効用が消費の対数関数であると仮定するなら、
所得の期待限界効用は期待名目総消費の逆数となる。
したがって将来において名目総消費を上昇させることを中央銀行が強く約束するなら、
期待名目総消費は直接の目標として上昇するため、
その逆数である所得の期待限界効用は最も効率的に低下するのではないか。
そしてその結果、経済は最も効率的に回復に向かうのではないか。

\section{本稿の貢献と結論}
\label{sec:intro_contribution}
本稿はゼロ金利下の経済を最も効率的に回復させる金融政策は名目総消費水準目標であると主張する。
この主張に説得力をもたせるため、
名目総消費水準目標を含む11の金融政策について
家計が財布のひもを固くする \( \beta \) ショックを与えて効果の分析をおこなう。
この需要ショックにより経済はゼロ金利に陥るが、名目総消費水準目標を用いたものはいち早く回復へと向かい、
家計の効用にもとづいて計算された厚生の落ち込みも最小となる。
また供給ショックへの効果も検証するため生産性が低下する \( a \) ショックを与えた分析もおこなう。
この \( a \) ショックについても名目総消費水準目標は頑健性を示す。
最後に名目総消費水準目標は指標集計や政策意図の浸透においても優れていることを説明し
現実の政策運営に適した目標であることを主張する。

\section{論文の構成}
\label{sec:intro_structure}
本稿の構成は以下の通りである。
第2章では、期待を軸に金融政策の歴史と先行研究を概観する。
第3章では、分析の基盤となる二国モデルを構築する。
第4章では、シミュレーションの設定を述べる。
第5章では、シミュレーション結果の提示とAS-AD分析をおこなう。
第6章では、名目総消費水準目標の効果についてまとめをおこない、実務的な点や今後の課題についても述べる。
第7章において本稿の総括をおこなう。

なお本稿のシミュレーションにおいて用いたプログラムは以下のリポジトリにおいて公開している。

\url{https://github.com/nanazou/masters_thesis}
% !TeX root = ../main.tex
% sections/chapter2.tex

\chapter{先行文献研究}
\label{chap:literature_review}

本稿はゼロ金利制約下における最適な金融政策を論じるものである。
こうした研究は1990年代に日本経済が長期停滞に入ってから始まった比較的新しいものであり、
2008年に世界金融危機が発生したことによりさらに活発におこなわれるようになってきた。

本章では本稿にいたる金融政策の歴史を期待という概念を軸に概観し本稿の位置づけを明確にする。
本稿の構成は以下の通りである。

まずフィリップス曲線が発見される以前の金融政策についてみる(第\ref{sec:review_before_phillips_curve}節)。
次にフィリップス曲線が発見されたものの期待については考慮されていなかった時代について述べる(第\ref{sec:review_discovery_of_phillips_curve}節)。
続いて期待によりフィリップス曲線が移動することが知られるようになったこととその原理を解説する(第\ref{sec:review_shifting_phillips_curve}節)。
そして裁量的金融政策が期待制御に失敗してフィリップス曲線の移動を止められず大インフレを招いたことをみる(第\ref{sec:review_failure_of_discretionary_policy}節)。
そしてこの大インフレの教訓からインフレ目標が採用され、
これが期待インフレ率を安定させ大いなる安定の時代を築いたことをみる(第\ref{sec:review_success_of_inflation_targeting}節)。
そしてこの時期に開発され政策運営を支えたマクロ経済モデルについて解説する(第\ref{sec:review_macro_model_development}節)。
続いて期待インフレ率の安定が副作用としてゼロ金利制約を生み出したことをみる(第\ref{sec:review_side_effects_of_low_inflation}節)。
またそれを受けて期待に強く働きかける水準目標が誕生したことについて述べる(第\ref{sec:review_zlb_challenge}節)。
さらに水準目標の有力候補として名目GDP水準目標が支持される理論的根拠について記述する(第\ref{sec:review_merits_of_ngdplt}節)。
続いて名目総消費水準目標の主要な関連研究を紹介する(第\ref{sec:review_nclt_studies}節)。
最後に本稿の貢献について説明する(第\ref{sec:review_our_contribution}節)。

\section{フィリップス曲線が発見されるまでの金融政策}
\label{sec:review_before_phillips_curve}

現代の金融政策論争は第二次世界大戦後に遡る \parencite[金融政策の歴史的概観については][などを参照]{ClaridaGaliGertler1999, Woodford2009}。
当時は失業率とインフレ率の負の相関関係が認識されていなかった。
そのためたとえば不況下において需要を刺激すると生産が増加して失業率が減るもののインフレ率は上昇しないと考えていた。
また完全雇用下において需要を刺激すると生産はもはや増加しないため失業率は減らず、
代わりにインフレ率のみが上昇すると考えられていたのである \parencite[この時代の政策思想については][に詳しい]{Friedman1968, GoodfriendKing1997}。

\section{フィリップス曲線の発見と期待の不在}
\label{sec:review_discovery_of_phillips_curve}

この認識を一変させたのが \textcite{Phillips1958} によるフィリップス曲線の発見であった。
彼は約100年間の英国のデータを用いて失業率と名目賃金上昇率(ひいてはインフレ率)の間に
安定的な負の相関関係が存在することを示した。
この発見は1960年代を通じて多くの政策担当者にインフレ率と失業率の間に
固定的な負の相関関係があるという信念を抱かせた。
インフレ率と失業率はどの組み合わせを選択するかという連続的なトレードオフの問題へと変貌したのである。
そこには人々の期待がフィリップス曲線を移動させるという現代的な考え方は存在しなかった。

\section{期待の発見と移動するフィリップス曲線}
\label{sec:review_shifting_phillips_curve}

この期待なき理論の弱点を \textcite{Friedman1968} と \textcite{Phelps1968} がそれぞれ独立に指摘した。
彼らによれば家計や生産者は期待インフレ率に基づいて現在の選択を決定するため、
期待インフレ率が変化すればフィリップス曲線そのものが移動するのである。
しかし初期の理論における期待がフィリップス曲線を移動させる原理の説明は
適応的期待や貨幣錯覚に依存するなど不合理な点が残されていた。
この原理はのちに \textcite{Fischer1977}、\textcite{Taylor1980}、\textcite{Calvo1983} といった
ニュー・ケインジアン経済学の先駆者たちにより価格の硬直性という概念を用いて
合理的期待形成の枠組みで説明された \parencite[この理論的変遷については][を参照]{ClaridaGaliGertler1999}。
彼らのモデルの中でも現在標準的に用いられているのが \textcite{Calvo1983} の提唱した
確率的な価格改定モデルである。
本節ではこのカルボ型価格改定と家計が生産者でもある
ヨーマン・ファーマー・モデルを用いてフィリップス曲線の仕組みを説明する。
カルボ型価格改定では毎期一定割合の家計のみが価格を変更することができる。
これは契約上の制約や、価格表の刷り直しに要する物理的な費用、
あるいは顧客への通知といった実務的な費用(メニュー・コスト)によるものと解釈される
\parencite{ClaridaGaliGertler1999, GoodfriendKing1997}。
さらに広義には値上げが顧客離れを招くのではないかという心理的な制約も価格が据え置かれる原因として考えられる。
ここで失業率が低下し生産が増加したとする。
このとき労働が増加するが、家計にとって労働の増加は労働の限界負効用を上昇させる。
価格改定の機会を得た家計はこの負効用の増大を補償し自身の効用を最適化するために価格を引き上げる。
これが失業率が下がればインフレ率は上がるという短期的なフィリップス曲線の理論的解釈である。
しかしこの短期的な関係はなぜ長続きしないのか。
ここで期待が決定的な役割を果たす。
カルボ型価格改定の下で価格改定の機会を得た家計は
一度価格を決めると次の改定機会までその価格を維持しなければならない。
したがって今日の価格決定は将来の経済状況に関する期待にもとづいて「前向き」におこなわれる。
たとえば中央銀行が短期フィリップス曲線における負の相関を利用し、
低失業率を維持するために高インフレを容認する姿勢を見せたとする。
ここで家計が「この中央銀行は将来的にも高インフレを容認するもの」と理解した場合、期待インフレ率が上昇する。
家計は毎期価格改定をできるわけではないため、期待インフレ率が上昇したならば、
将来の労働の負効用上昇分をあらかじめ現在の価格設定に上乗せする。
つまり期待が変化したことで、同じ失業率を維持するためにより高いインフレ率が必要になってしまったのである。
これは短期フィリップス曲線が上方移動したことを意味する。
もし中央銀行がインフレ容認の姿勢を取り続けるならば家計の期待インフレ率はますます上昇し
現在のインフレ率が加速度的に上昇していく。
これを打破するには中央銀行がインフレに対する断固たる姿勢を示し、
上昇したインフレ率を上回る幅で名目利子率を引き上げる必要がある(テイラー原理)。
この断固たる姿勢を見た家計が期待インフレ率を低下させればインフレ率上昇の悪循環が断ち切られるのである。

\section{裁量的金融政策による期待制御の失敗と大インフレの発生}
\label{sec:review_failure_of_discretionary_policy}

この理論的な懸念は1970年代の大インフレという形で現実のものとなった \parencite[][などを参照]{ClaridaGaliGertler1999, Mishkin1999, GoodfriendKing1997}。
多くの先進国が制御不能な高インフレと経済停滞が併存する深刻なスタグフレーションに見舞われ、
1960年代に観測された短期的なフィリップス曲線の関係は消失した。
この経験は規律なき裁量的政策がいかに経済を不安定化させうるかという教訓となった。
この政策失敗の核心は \textcite{ClaridaGaliGertler2000} による実証研究で説明されている。
彼らの分析によれば、1979年のボルカー議長就任以前のFRBの政策は
期待インフレ率が1\%上昇しても名目利子率を平均で1\%未満しか引き上げない受動的なものであった。
これはインフレが加速する局面で実質利子率の低下を許容してしまったことでインフレを抑制できなかったことを示している。
ではなぜ当時の政策当局はこうしたインフレを容認する政策を続けたのか。
\textcite{Bernanke2004} は、当時の政策当局は固定されたフィリップス曲線を信じる生産楽観主義に加えて
インフレ悲観主義に陥っていたと指摘する。
第一に、中央銀行はフィリップス曲線が固定されたものと信じ、
非常に低い失業率の達成を目標として緩和的な金融政策を運営した。
しかしこの政策は人々の期待インフレ率を上昇させフィリップス曲線を上方へ移動させた。
その結果、中央銀行が許容可能だと考えていた率をはるかに上回る高インフレが発生した。
第二に、この予期せぬインフレに直面した際、中央銀行は自らの緩和的な政策が原因であるとは考えず、
その主因を労働組合の賃金交渉圧力や石油価格の高騰といった
金融政策では制御不能なコストプッシュ要因にあると信じ込んだ。
さらにこの政策判断の誤りの根源にはより深刻な情報の問題があったことを \textcite{Orphanides2003} が示した。
彼の分析によれば1970年代の政策当局は経済の潜在生産を過大評価していた。
まずこの時期には生産性が低下しこれは負の供給ショックとしてフィリップス曲線を上方へ移動させた。
しかし中央銀行はこの構造変化に気づかず、
下方に位置していた古いフィリップス曲線とそれに基づく低い自然失業率を信じ続けていたのである。
観測された失業率はこの低い自然失業率を上回っていたため
誤った認識をもった政策当局は経済が不況にあると判断し緩和的な金融政策を実施した。
その結果、失業率は真の自然失業率を下回り期待インフレ率の上昇をもたらしたのである。

\section{インフレ目標による期待制御の成功}
\label{sec:review_success_of_inflation_targeting}
1970年代の大インフレへの反省から規律ある名目アンカーの確立が必要であるという合意が形成された。
その最初の試みは、貨幣の量が増えれば一定の割合のインフレが起きるという貨幣数量説にもとづく
貨幣供給目標であった。
しかし金融自由化やIT技術の発展により新しい種類の預金口座や金融商品が登場すると
マネーサプライとインフレ率の関係が不安定化し、貨幣供給目標は挫折した。
そこで多くの中央銀行は短期政策利子率を主要な政策手段とする
現代的なインフレ目標へと移行していったのである \parencite{Mishkin1999}。
このインフレ目標は人々の期待インフレ率を強力に安定させる名目アンカーとして機能し、
2000年代半ばまで続く「大いなる安定(The Great Moderation)」\parencite{Bernanke2004} の時代を築く
大きな成功を収めることとなった。

\section{マクロ経済モデルの発展}
\label{sec:review_macro_model_development}
こうした政策運営の変遷と並行し、経済学では期待がいかに現在のマクロ経済変数に影響を及ぼすかを
厳密に解明しようとする試みが進展した。
期待を考慮することの重要性はすでに認識されていたものの、
それを整合的に記述し政策評価に活用するための理論的枠組みは未だ不十分であった。
この課題を解消し期待管理に関する強固な理論的裏付けを与えたのがDSGE(動学的確率的一般均衡)モデルである。
DSGEモデルは1970年代以前のモデルが抱えていたミクロ的基礎の欠如という問題を克服し、
各主体の動学的な最適化行動から期待の役割を数学的に導出した。
まず \textcite{Fischer1977} や \textcite{Taylor1980} は名目契約の観点から価格の硬直性を理論化し、
さらに \textcite{Calvo1983} は生産者の確率的な価格改定行動を導入することで
将来の価格期待が現在の価格設定に直接影響を与える「前向き」な供給関数の導出に成功した。
さらに \textcite{Woodford2003} らはこれらのモデルが
家計や生産者の最適化行動にもとづいているというミクロ的な特性を活かし、
代表的家計の効用関数から直接的に社会的な厚生を定義する手法を確立した。
それまでのマクロ経済モデルでは政策の評価基準(損失関数)は
インフレ率や生産ギャップを恣意的に組み合わせたものに過ぎなかった。
しかしDSGEモデルの登場により、個別の家計の満足度という厳密なミクロ的根拠にもとづいた
客観的な政策評価が可能となったのである。
その後、DSGEモデルは中央銀行の実務における不可欠なシミュレーション・ツールとして活用されるようになり、
それが2000年代中盤までの大いなる安定の時代を支える
理論的背景を形作ったのである。

\section{低インフレ期待の副作用}
\label{sec:review_side_effects_of_low_inflation}

しかしインフレ目標の成功は負の側面ももたらした。
インフレ目標の浸透によって低インフレ環境が定着したことは経済が負のショックに見舞われた際に
期待インフレ率が容易に負の領域へと沈み込みやすくなる状況を生みだした。
期待インフレ率が十分高い水準にあれば負のショックが生じても正の領域に留まるが、
期待インフレ率が低ければわずかな負のショックによって負へと転じ
フィリップス曲線を下方へと押し下げてしまうのである。
加えてこうした低インフレ期待下では最適な価格から多少乖離していても生産者が被る損失が限定的となる。
そのためメニュー・コスト等の価格改定費用を支払ってまで頻繁に価格を変更する誘因が低下し
経済全体で価格の硬直性が強まることとなった。
これによりフィリップス曲線の傾きは著しく平坦化し金融政策による調整機能が大きく低下したのである。
通常であれば中央銀行が名目利子率を引き下げて需要を刺激し総需要(AD)曲線を上方へ移動させることで
物価を目標水準まで押し戻すことが可能である。
しかしフィリップス曲線が平坦な状況下では需要の拡大が物価の上昇に結びつきにくい。
そして物価を引き上げるためにさらなる名目利子率低下を模索し続けた結果、
名目利子率はゼロに達しこれ以上下がらなくなるゼロ金利制約の壁に突き当たったのである。
こうしたインフレ期待が負に転じやすい脆弱性と
平坦化したフィリップス曲線という二重の困難が現実のものとなったのが1990年代以降の日本経済であった。
大いなる安定と称賛された安定的構造は、ひとたびゼロ金利制約に直面すれば
利子率操作を通じたデフレ脱却を極めて困難にする強固な枠組みでもあったのである。

\section{ゼロ金利制約下における新たな期待制御への挑戦}
\label{sec:review_zlb_challenge}
1990年代の日本経済はバブル崩壊により深刻な需要不足に陥った。
これに対し日本銀行は断続的に名目利子率を引き下げたがついにゼロ近辺にまで到達した。
伝統的な利下げという手段が効力を失う状況は流動性の罠として知られ
当時の実務者にとって前例のない事態であった。
この難問に向き合うなかで、ゼロ金利制約に直面した中央銀行に残された手段は
人々の期待に働きかけることであるという認識が次第に築かれていった。
たとえば \textcite{EggertssonWoodford2003} は
中央銀行が「現在マイナス金利にできない分、将来景気回復後にも低金利を維持する」
と約束(コミットメント)することで人々の期待に働きかけ現在の経済を好転させられることを理論的に示した。
この埋め合わせの原理は歴史依存性と呼ばれ、その約束による期待の誘導はフォワード・ガイダンスと呼ばれる。
このフォワード・ガイダンスの効果は
 \textcite{ReifschneiderWilliams2000} や \textcite{JungTeranishiWatanabe2005} によっても裏付けられており
その詳しい波及経路は \textcite{ReifschneiderWilliams2000} などにより説明されている。
さらにこのフォワード・ガイダンスによる約束を制度化し信頼性を高めた仕組みが水準目標である。
水準目標はゼロ金利制約下でも期待への強力な働きかけをおこなえることが
\textcite{Woodford2012} などにより理論的に示されている。

\section{有力候補としての名目GDP水準目標とその改善可能性}
\label{sec:review_merits_of_ngdplt}
水準目標の中でも最も多くの支持を集めているのが名目GDP水準目標である。
この支持の理由は以下のような点にまとめられる
第一に供給ショックに対して頑健であること \parencite{Sumner2014, BhandariFrankel2017}。
第二に潜在生産といった観測が困難な情報に依存しないこと \parencite{Orphanides2003, BeckworthHendrickson2019}。
第三にDSGEモデルにおける厚生上優位であること \parencite{GarinLesterSims2016}
第四に名目GDPの安定化は効率的なリスク共有を促すこと \parencite{Sheedy2014}
こうした合理的な理由により支持を集めてきた名目GDP水準目標ではあるが、
これが本当に最善の目標であるかということに関して次のような疑問が生じてくる。
不況に陥った経済を回復させ厚生を高めるためには生産を増加させる必要がある。
のちに5章において示されるように、生産を増やすためには名目利子率および所得の期待限界効用を
下落させる必要がありそれが最も効率的にできる金融政策が最善のものといえる。
それでは名目利子率および所得の期待限界効用を名目GDP水準目標以上に効率的に下落させられる金融政策はないのだろうか。
この問いに対し本稿は名目総消費水準目標の優位性を主張する。
それにあたり次節では名目総消費水準目標に関するこれまでの学術的・実務的論争を整理し、
本稿の分析の前提となる既存の知見を確認する。

\section{名目総消費水準目標とその関連研究}
\label{sec:review_nclt_studies}
幾人かの研究者や実務家が異なる理由により名目総消費水準目標の賛否を論じてきた。
以下ではそのいくつかを紹介する。

\subsection{1. 労働市場の歪みを解消する:\textcite{Koenig1995}}
\label{sec:review_koenig_1995}
\textcite{Koenig1995} は代表的な競争的生産者と家計からなるモデルを用い
名目価格が硬直的である経済において名目消費の安定化が労働市場の歪みを解消することを論じた。
Koenig のモデルは第1期に期待にもとづいて第2期の名目賃金 $W$ が決定し、
第2期に負の生産性ショックなどが発生するという実質的な2期間の構成をとっている。
まず労働 $N$ の決定について考える。Koenig のモデルでは名目賃金 $W$ が事前に固定される一方で
生産者が雇用量を決定する権限を持つとされる。
したがって生産者は利潤最大化のために
労働の限界生産性と実質賃金 $\frac{W}{P}$ を一致させるように労働 $N$ を決定する。
\begin{equation}
N = \left( \frac{a}{W/P} \right)^{1/b}
\end{equation}
ここで $a$ は生産性パラメータ、$b$ は生産関数のパラメータである。
一方、理想的な「市場均衡値 $N^*$(競争値)」は生産者の利潤最大化行動による労働需要のみならず
家計の効用最大化行動にもとづく労働供給の条件も満たさなければならない。
家計は消費と労働の限界代替率が実質賃金 $\frac{W}{P}$ に一致するときに効用を最大化する労働 $N$ を得る。
\begin{equation}
\frac{W}{P} = N^{\alpha} C^{\beta}
\end{equation}
ここで $\alpha$ と $\beta$ は家計の選好を表すパラメータである。
賃金が伸縮的であればこれら需要と供給の条件を満たす点として $N^*$ が達成される。
次に負の生産性ショックなどが発生し生産性 $a$ が減少したとする。
もし物価 $P$ が不変であれば、需要の式から明らかなように
生産性 $a$ の低下により労働 $N$ が減少する。
すると家計の所得は減少するため消費 $C$ も減少する。
こうして労働 $N$ と消費 $C$ はともに減少するから
供給の式において(左辺)>(右辺)となり等号が保たれなくなる。
したがって労働 $N$ は市場均衡値 $N^*$ から乖離しておりこれは労働市場の歪みが発生している状態にほかならない。
ここで労働 $N$ を市場均衡値 $N^*$ に一致させるためには物価 $P$ が適切に上昇する必要がある。
物価 $P$ の上昇は実質賃金 $\frac{W}{P}$ を減少させ、また需要の式を通じて労働 $N$ を増加させる。
そして労働 $N$ の増加により消費 $C$ も増加する。
つまり実質賃金 $\frac{W}{P}$ は下落し、労働 $N$ と消費 $C$ は上昇するから
(左辺)>(右辺)となっていた供給の式は等号へ向かう。
この物価 $P$ 調整が適切であれば供給の式も満たされることとなり労働 $N$ は市場均衡値 $N^*$ となる。
\textcite{Koenig1995} は中央銀行が名目所得と名目消費の加重平均を目標とすれば
この適切な物価 $P$ 調整が自動的に達成されることを示した。
このように名目消費の安定化は名目賃金 $W$ の硬直性に由来する労働市場の歪みを解消し、
労働 $N$ を市場均衡値 $N^*$ へ導く有効な手段となる。
Koenig の議論は競争的均衡(市場均衡)が社会的に最も望ましいという前提に立脚しており、
賃金硬直性の影響を排して伸縮時の実現値を再現することを目的に置いている。
しかし本稿の5章における $a$ ショック(生産性ショック)の節で示すとおり
物価や賃金の伸縮性は必ずしも厚生の向上を保証するものではない。
特に生産性ショックに対して物価が伸縮的に反応することは
生産を大きく低下させる要因となり厚生を減少させる恐れがある。
こうした場合はむしろ、一定程度の物価の硬直性がショックの影響を緩和する緩衝材として機能し
より大きな厚生を実現しうる。

\subsection{2. 金融危機後の停滞を打開する実務的な手段とする:Sheets}
\label{sec:review_sheets}
実務の観点からは、元FRB局長の Nathan Sheets が
金融危機後の停滞を打開する強力な政策手段として名目総消費水準目標を提唱した \parencite{Harding2011}。
\textcite{Harding2011} が Financial Times 誌で紹介し
のちに \textcite{Chen2019} も学術的関心の対象として取り上げた Sheets の主張によれば
名目総消費水準目標には以下の5つの実務的利点がある。

\begin{enumerate}
  \item[(i)] 家計になじみ深く現在の金融政策でも扱われている消費者物価指数を用いるため、
  生産者物価指数を用いる名目 GDP 目標よりも政策意図の浸透が容易である。
  \item[(ii)] 消費の安定は「消費の平滑化」と同義でありこれは家計の厚生最大化に直結する。
  \item[(iii)] 消費は投資や外需を含む GDP よりも変動が低く目標として掲げやすい。
  \item[(iv)] 中央銀行が関与できない政府支出などを目標から除外することで金融政策の責任範囲が明確化される。
  \item[(v)] 消費が危機前の本来あるべき水準を下回っていると示すことで緩和政策に対する政治的・社会的支持を得やすい。
\end{enumerate}

\subsection{3. 金融市場の歪みを解消する:\textcite{Koenig2013}}
\label{sec:review_koenig_2013}
\textcite{Koenig2013} は資本家と労働者という二つの主体からなる2期間モデルを用い
名目債務が存在する経済において名目総消費水準目標が金融市場の歪みを解消することを論じた。
Koenig のモデルにおける社会全体の純利益(総資源)は
生産 $y$ から税 $g$ を差し引いた $c = y - g$ と定義される。
この利益 $c$ は外生的に決定される一定の分配率に基づき、
債務の清算が行われる前の基礎所得として資本家に $(1-\theta)c$、労働者に $\theta c$ の割合で配分される。
分析の焦点となる第2期はモデルの最終期であり資本家による新たな投資は想定されていない。
したがって各主体は基礎所得に前期から繰り越された実質的な債務 $\frac{DR}{P}$ の受け渡しを加減した額の
すべてを消費に充てることとなる。
このときの資本家の消費 $c_1$ および労働者の消費 $c_2$ は以下の通り記述される。
\begin{equation}
c_1 = (1-\theta)c + \frac{DR}{P}
\end{equation}
\begin{equation}
c_2 = \theta c - \frac{DR}{P}
\end{equation}
上式を合算すると $c_1 + c_2 = c$ が成立することから利益 $c$ は社会全体の総消費額と一致する。
ここで負の供給ショック等によって総消費 $c$ が 10\% 減少した局面を想定する。
この際、基礎所得である $(1-\theta)c$ および $\theta c$ も $c$ に比例しているため同様に 10\% 減少する。
しかしもし物価 $P$ が不変であり実質債務 $\frac{DR}{P}$ が固定されたままであるならば、
返済を受ける資本家の消費減少率は 10\% よりも小幅に留まる一方、
返済をおこなう労働者の減少率は 10\% を超過し、主体間で痛みの偏りが生じる。
両者の消費減少率を全体の減少率と一致させるためには基礎所得と同様に実質債務の項も 10\% 減少させる必要がある。
そのためには総消費 $c$ の減少を相殺する規模で物価 $P$ が上昇しなければならない。
\textcite{Koenig2013} は中央銀行が名目総消費水準目標($Pc = n^*$)を採用して
名目消費支出を一定に維持すればこの物価調整が自動的に達成されることを示した。
このように $c_1$ と $c_2$ の変化割合が社会全体の $c$ の変化と完全に同期することは
経済全体のショックの影響がすべての主体間で理想的に分散されていることを意味する。
これはあらゆるリスクに対する保険契約が完備された市場(完備市場)において実現される消費配分と等価であり、
名目総消費水準目標が不完備市場における歪みを解消する有効な手段であることを示唆している。

\subsection{4. マクロ経済を安定させない:\textcite{Chen2019}}
\label{sec:review_chen_2019}
経済の安定性分析の観点からは \textcite{Chen2019} が自己充足的な期待(アニマル・スピリット)に起因する
経済変動を抑制する効果について名目 GDP 水準目標と名目総消費水準目標の比較分析をおこなっている。
Chen は現金前払(CIA)制約を仮定している。これは次の 2 つの条件を合わせたものである

\begin{description}
  \item[(CIA-1)] 貨幣 \( M \) は消費 \( c \) の決済のみに使用され投資 \( i \) の決済には使用されない
  \item[(CIA-2)] 今期の消費 \( c \) のために家計は前期に貨幣 \( M \) を準備しなければならない
\end{description}

これら 2 つの条件により今期の名目消費 \( Pc \) は前期に家計が保有していた貨幣 \( M \) によって制限される( \( Pc \le M \) )。
Chenのモデルでは不確実性が存在せず変数の期待値と実現値は一致するため
今期の名目消費 $Pc$ は前期に期待していたものである。
したがってもし今期に $Pc < M$ となり余剰貨幣が存在するならば、
これを期待していた前期の家計は利息を生まない貨幣を投資 $i$ の購入に充てる。
この投資 $i$ の需要の増加は利子率の下落を招き、これは投資 $i$ の需要の増加幅を縮小させ消費 $c$ の需要を高める。
これにより物価 $P$ と消費 $c$ が上昇する。
これにより $Pc = M$ が成立するため今期に余剰貨幣は生まれない。
Chen のモデルにおいて金融政策とは中央銀行による $M$ の操作を指す。
中央銀行が $M$ を増加させれば、先ほどのように投資 $i$ と消費 $c$ が増加し生産 $y = c + i$ が拡大する。
一方で $M$ を減少させれば逆の動きが起き $y$ は減少する。
ここで自己充足的な期待(アニマル・スピリット)が上昇するとどうなるだろうか。
家計は $M$ の制限を受けることなく投資 $i$ を増大させることが可能であり、
期待が楽観に振れれば投資 $i$ 主導で生産 $y$ が増加し潜在パスを離れて上昇しようとする。
このとき名目総消費水準目標を採用していたならば
均衡条件 $Pc = M$ により中央銀行はマネーサプライ $M$ を目標値に固定し続けなければならない。
したがって中央銀行は $M$ を減じて生産 $y$ を抑制することができない。
一方で名目GDP水準目標を採用していたならば
中央銀行はこの $y$ の増加に反応して $M$ を減少させることで $y$ 押し戻すことができる。
Chenはこうした経済安定化の観点から名目総消費水準目標よりも名目GDP水準目標の方が望ましいと結論付けた。
しかしながら Chen によるこの結論は CIA 制約という強い仮定に依存していることに留意する必要がある。

\subsection{5. テイラールールに劣る:\textcite{Gross2023}}
\label{sec:review_gross_2023}
\textcite{Gross2023} はオーストラリア経済を対象とし
多セクターDSGEモデルを用いた金融政策の比較分析をおこなっている。
同研究ではインフレ率と生産ギャップの双方に反応する柔軟なインフレ目標や
物価乖離と生産ギャップの双方に反応する物価水準目標といったテイラールール型の政策が
名目総消費水準目標などの名目値目標よりも明らかに優れていると結論づけられている。
しかしこの成績を判定するために用いられた損失関数という計算式は各政策を公平に評価するように作られていない。
この式が最小化すべき対象として設定しているインフレや生産のギャップは
まさにテイラールールが目標としている変数そのものである。
そのため名目消費のようにこの評価式には入っていない別の変数を安定させようとする政策は
テイラールールと比較して必然的に成績が悪く判定されてしまう。
このような評価手法は、損失関数の開発元である \textcite{DorichMendesZhang2021} も、
特定の変数を含まない式で評価をおこなうことは他の政策を不当に低く評価することになるとして
推奨していないものである。

\section{本稿の貢献}
\label{sec:review_our_contribution}
前節において名目総消費水準目標に関する主要な研究を概観した。
しかしながらそれらはどれも家計の効用にもとづく厚生を基準として
名目総消費水準目標を他の政策と比較したものではなかった。
本稿は名目総消費水準目標やテイラールールを含む11の金融政策について
厚生を基準として比較分析をおこなう。
シミュレーションにあたっては \( \beta \) ショック(需要ショック)と
 \( a \) ショック(供給ショック)をそれぞれ個別に与える。
名目総消費水準目標は \( \beta \) ショックに対して厚生の落ち込みを最小にし、
 \( a \) ショックに対しても最善の政策とそん色ない。
さらに本稿はAS-AD分析による解釈をおこない、
結果が導かれた原因について詳細な理解を読者に提供する。
% !TeX root = ../main.tex
% sections/chapter3.tex

\chapter{モデル}
\label{chap:model}

% !TeX root = ../../main.tex
% sections/chap3/sec_model_overview.tex

\section{モデルの概要}
\label{sec:model_overview}

本章では本稿の分析の土台となる理論モデルを構築する。
本モデルは現代マクロ経済学の標準的な枠組み(DSGE)にもとづき以下を仮定する。

\begin{enumerate}
    \item 合理的経済人:
    家計は生涯効用を予算制約の下で最大化する。
    これは家計が高度な数学的計算能力をもち、無限の将来までを見据えた動的な最適化問題を解くことを意味する。
    この最適化問題を解くためには、家計は必然的に、
    将来の不確実な変数について何らかの期待(予測値)を形成する必要がある。
    またこの最適計画は時間整合的となり、将来新たな予想外のショックが発生しない限り、
    家計は将来のどの時点においても当初の計画を変更するインセンティブを持たずそのまま実行し続ける(付録A.1節参照)。
    
    \item 合理的期待:
    経済学における期待とは将来の不確実な変数に対して家計が抱く予測値のことである。
    この予測値を形成する原理には様々なものがありうる。
    たとえば適応的期待形成は、過去の実現値の加重平均を計算するなど
    何らかの(確率論的ではない)規則にもとづいて予測値を算出する。
    これに対し本稿が採用する合理的期待形成とは、
    家計がモデルの構造を完全に知っており、
    その知識にもとづいて数学的な期待値を計算して予測するという仮定である。
    したがって合理的期待とは、家計の主観的な予測値が客観的な数学的期待値と同一である、という仮定に他ならない。
    $t$ 期における期待値の計算は以下の2種類の情報を所与としておこなわれる。
    \begin{itemize}
        \item[a.] モデルの構造: 
        (i) すべての方程式体系、および
        (ii) すべての外生ショックの確率分布
        \item[b.] $t$ 期までのすべての変数の実現値の履歴(情報集合 $I_t$)。
    \end{itemize}
    これにより計算する期待値は $E[\cdot | I_t]$ となる。
    ここで情報集合 $I_t$ とは「$t$ 期までの変数が特定の値をとった」という事実を表す命題(式)、
    すなわち数学的な事象の集合を指す。
    これは、「$I_t$ に含まれる命題(例えば $c_t = c_t'$ という等式)が成立している」
    という条件の下での計算を意味する。
    その結果この期待値計算の中では $t$ 期およびそれ以前の変数($c_t, c_{t-1}$ など)は、もはや確率変数ではなく、
    その実現値と同一視される既知の数値(定数)として扱われる。
    本稿では慣例に従いこの $E[\cdot | I_t]$ を $E_t[\cdot]$ と省略して表記する。
    したがって合理的期待の仮定とは家計が適応的期待形成のような(確率論的ではない)ルールで予測するのではなく、
    モデルの構造(項目a.)と過去の履歴(項目b.)のすべてを駆使して、数学的に最適な予測($E_t[\cdot]$)を
    おこなうことを仮定するものである。
\end{enumerate}

これらの仮定を基礎に本モデルは価格の硬直性を特徴とする二国間ニュー・ケインジアンDSGEモデルとして構築される。
本モデルにおいては自国と外国それぞれにおいて国内の債券市場は完備であり家計間のリスク共有が完全におこわれる一方、
それらのあいだの国際債券市場は不完備でありリスク共有が限定的となる。

本章の構成は以下の通りである。
まず3.3節で、モデルの基本的な均衡条件である財市場の均衡と一物一価の法則を記述する。
3.4節では、モデルの基本的な意思決定主体である家計の最適化行動を定式化し消費や貯蓄に関する主要な関係式を導出する。
3.5節では、国内完備市場の仮定がもたらすリスク共有の含意と代表的家計への集計可能性について論じる。
3.6節では、家計が生産者として直面する価格設定問題を取り上げ、
カルボ型の価格硬直性を仮定した下での生産関数と価格分散の動学を導く。
3.7節では、政府の財政政策と、本稿で比較分析の対象となる3つの金融政策ルールを定義する。
3.8節で国全体の資源制約式を導出する。
最後に3.9節でこれらの方程式を集約した代表的家計モデルの最終的な方程式体系としてまとめる。
% !TeX root = ../../main.tex
% sections/chap3/sec_model_variable_definitions.tex

\section{変数の定義}
\label{sec:model_variable_definitions}

\subsection*{表記の規則と確率変数の扱い}

本稿では、不確実性を伴う変数の表記について以下の規則を採用する。

\begin{enumerate}
    \item \textbf{確率変数と実現値の同一視:} \\
    本章で記述される方程式(予算制約式、市場均衡条件、最適化条件など)は特段の断りがないかぎり、時点 \( t \) における情報集合 \( I_t \)(時点 \( t \) までのショックの実現値を含む)が所与とされた下での関係式を表す。
    したがって式中の変数(例:\( c_t, p_t \))は、その時点 \( t \) における確定した実現値として記述する。状態 \( j \) への依存を示す表記(\( (j) \))は記述の煩雑さを避けるため省略する。
    
    \item \textbf{期待値:} \\
    将来の変数(例:\( c_{t+1} \))は、時点 \( t \) から見て不確実な確率変数である。これらを含む項については、時点 \( t \) の情報集合に基づく条件付き期待値演算子 \( E_t[\cdot] \) を用いて記述する。
    
    \item \textbf{通貨建て (Currency):} \\
    アスタリスク * の有無で示す。
    \begin{itemize}
        \item もともと自国通貨建てである変数は * をつけず \( x_t \) のように表す。
        \item もともと外国通貨建てである変数は * をつけ \( x_t^* \) のように表す。
        \item もともと自国通貨建てである変数 \( x_t \) について、\( x_t^* \equiv (1/e_t^{/*}) x_t \) により \( x_t^* \) を定義する。
        \item もともと外国通貨建てである変数 \( x_t^* \) について、\( x_t \equiv e_t^{/*} x_t^* \) により \( x_t \) を定義する。
    \end{itemize}
    ここで \( e_t^{/*} \) は名目為替レート(自国通貨/外国通貨)である。(例: \( \bar{p}_t^{F*}, b_t^{h \to F} \))
\end{enumerate}

\subsection*{内生変数}

\subsubsection*{家計の変数}
\begin{itemize}
    \item \( c_t^{h \to h'} \) :家計 \( h \) による、家計 \( h' \) が生産する財の消費。
    \item \( l_t^h \) :家計 \( h \) の労働。
    \item \( y_t^h \) :家計 \( h \) の財の生産。
    \item \( p_t^h \) :家計 \( h \) の財の価格。
    \item \( \widetilde{p}_t^h \) :価格改定機会を得た家計 \( h \) が設定する最適価格。
    \item \( v_t, w_t \) : 最適価格の補助変数。
    \item \( d_{t+1}^{h \to H} \) :家計 \( h \) が時点 \( t \) に購入する、\( t+1 \) 期の状態に依存する国内コンティンジェント債券の数量。
    \item \( b_{t+1}^{h \to F} \) :家計 \( h \) が時点 \( t \) に保有する、外国リスクフリー債券の自国通貨建ての保有額(名目値)。
    \item \( \lambda_t^h \) :家計 \( h \) の予算制約に関するラグランジュ乗数(所得の限界効用)。
    \item (外国も同様に \( c_t^{f \to f'}, l_t^f, y_t^f, p_t^{f*}, \widetilde{p}_t^{f*}, v_t^F, w_t^F, d_{t+1}^{f \to F*}, b_{t+1}^{f \to H*}, \lambda_t^{f/*} \) が定義される。)
\end{itemize}

\subsubsection*{消費指数}
\begin{itemize}
    \item \( c_t^{h \to H} \) :家計 \( h \) の自国財消費指数。
    \begin{equation}
    c_t^{h \to H} \equiv \left[ \sum_{h' \in H} (c_t^{h \to h'})^{\frac{\theta^H-1}{\theta^H}} \right]^{\frac{\theta^H}{\theta^H-1}}
    \label{eq:def_domestic_consumption_index}
    \end{equation}
    \item \( c_t^{h \to F} \) :家計 \( h \) の外国財消費指数。
    \begin{equation}
    c_t^{h \to F} \equiv \left[ \sum_{f \in F} (c_t^{h \to f})^{\frac{\theta^F-1}{\theta^F}} \right]^{\frac{\theta^F}{\theta^F-1}}
    \label{eq:def_foreign_consumption_index}
    \end{equation}
    \item \( c_t^{h \to W} \) :家計 \( h \) の総消費指数。
    \begin{equation}
    c_t^{h \to W} \equiv \frac{(c_t^{h \to H})^{\alpha^H} (c_t^{h \to F})^{1-\alpha^H}}{(\alpha^H)^{\alpha^H} (1-\alpha^H)^{1-\alpha^H}}
    \label{eq:def_world_consumption_index}
    \end{equation}
    \item (外国も同様に \( c_t^{f \to F}, c_t^{f \to H}, c_t^{f \to W} \) が定義される。)
\end{itemize}

\subsubsection*{国全体の変数}
\begin{itemize}
    \item \( q_{t, t+1} \) : 時点 \( t \) における、\( t+1 \) 期の状態コンティンジェント債券の価格(確率的割引因子)。
    \item \( e_t^{/*} \) : 名目為替レート(自国通貨/外国通貨)。
    \item \( p_t^H \) :自国の生産者物価指数 (PPI)。
    \begin{equation}
    p_t^H \equiv \left[ \sum_{h' \in H} (p_t^{h'})^{1-\theta^H} \right]^{\frac{1}{1-\theta^H}}
    \label{eq:def_price_index_H_unnorm}
    \end{equation}
    \item \( \bar{p}_t^H \) :自国の正規化された生産者物価指数 (PPI)。
    \begin{equation}
    \bar{p}_t^H \equiv \left[ \frac{1}{N}\sum_{h' \in H} (p_t^{h'})^{1-\theta^H} \right]^{\frac{1}{1-\theta^H}} = N^{-\frac{1}{1-\theta^H}} p_t^H
    \label{eq:def_ppi_H}
    \end{equation}
    \item \( p_t^{F*} \) :外国の生産者物価指数 (PPI, 外国通貨建て)。
    \begin{equation}
    p_t^{F*} \equiv \left[ \sum_{f' \in F} (p_t^{f'*})^{1-\theta^F} \right]^{\frac{1}{1-\theta^F}}
    \label{eq:def_price_index_F_unnorm}
    \end{equation}
    \item \( \bar{p}_t^{F*} \) :外国の正規化された生産者物価指数 (PPI, 外国通貨建て)。
    \begin{equation}
    \bar{p}_t^{F*} \equiv \left[ \frac{1}{M}\sum_{f' \in F} (p_t^{f'*})^{1-\theta^F} \right]^{\frac{1}{1-\theta^F}} = M^{-\frac{1}{1-\theta^F}} p_t^{F*}
    \label{eq:def_ppi_F}
    \end{equation}
    \item \( p_t^{H \to W} \) :自国の消費者物価指数 (CPI)。
    \begin{equation}
    p_t^{H \to W} \equiv (p_t^H)^{\alpha^H} (p_t^F)^{1-\alpha^H}
    \label{eq:def_cpi_H}
    \end{equation}
    \item \( \bar{p}_t^{H \to W} \) :自国の正規化された消費者物価指数 (CPI)。
    \begin{equation}
    \bar{p}_t^{H \to W} \equiv (\bar{p}_t^H)^{\alpha^H} (\bar{p}_t^F)^{1-\alpha^H}
    \label{eq:def_norm_cpi_H}
    \end{equation}
    \item \( p_t^{F \to W*} \) :外国の消費者物価指数 (CPI, 外国通貨建て)。
    \begin{equation}
    p_t^{F \to W*} \equiv (p_t^{F*})^{\alpha^F} (p_t^{H*})^{1-\alpha^F}
    \label{eq:def_cpi_F}
    \end{equation}
    \item \( \bar{p}_t^{F \to W*} \) :外国の正規化された消費者物価指数 (CPI, 外国通貨建て)。
    \begin{equation}
    \bar{p}_t^{F \to W*} \equiv (\bar{p}_t^{F*})^{\alpha^F} (\bar{p}_t^{H*})^{1-\alpha^F}
    \label{eq:def_norm_cpi_F}
    \end{equation}
    \item \( \pi_t^H \) : 自国の生産者物価指数グロス・インフレ率 (\( \bar{p}_t^H / \bar{p}_{t-1}^H \))。
    \item \( \pi_t^{F*} \) : 外国の生産者物価指数グロス・インフレ率 (\( \bar{p}_t^{F*} / \bar{p}_{t-1}^{F*} \))。
    \item \( \pi_t^{H \to W} \) : 自国の消費者物価指数グロス・インフレ率 (\( \bar{p}_t^{H \to W} / \bar{p}_{t-1}^{H \to W} \))。
    \item \( \pi_t^{F \to W*} \) : 外国の消費者物価指数グロス・インフレ率 (\( \bar{p}_t^{F \to W*} / \bar{p}_{t-1}^{F \to W*} \))。
    \item \( Y_t^H \) :自国全体の総生産指数。
    \begin{equation}
    Y_t^H \equiv \left[ \sum_{h \in H} (y_t^h)^{\frac{\theta^H-1}{\theta^H}} \right]^{\frac{\theta^H}{\theta^H-1}}
    \label{eq:def_output_index_H}
    \end{equation}
    \item \( L_t^H \) :自国全体の総労働。
    \begin{equation}
    L_t^H \equiv \sum_{h \in H} l_t^h
    \label{eq:def_total_labor_H}
    \end{equation}
    \item \( C_t^{H \to W} \) :自国全体の総消費。
    \begin{equation}
    C_t^{H \to W} \equiv \sum_{h \in H} c_t^{h \to W}
    \label{eq:def_total_consumption_H}
    \end{equation}
    \item \( B_t^{H} \) :自国全体の対外純資産(自国通貨建て)。
    \begin{equation}
    B_t^{H} \equiv \sum_{h \in H} b_t^{h \to F}
    \label{eq:def_total_assets_H}
    \end{equation}
    \item \( B_t^{F*} \) :自国全体の対外純資産(外国通貨建て)。
    \begin{equation}
    B_t^{F*} \equiv -(N/M)(B_t^H / e_t^{/*})
    \label{eq:def_total_assets_F}
    \end{equation}
    (世界全体の債券市場均衡 \( N b_t^{h \to F} + M e_t^{/*} b_t^{f \to F*} = 0 \) より導出される定義)
    \item \( \Delta_t^H \) :自国全体の総価格分散。
    \begin{equation}
    \Delta_t^H \equiv \sum_{h \in H} \left(\frac{p_t^h}{p_t^H}\right)^{-\theta^H}
    \label{eq:def_total_dispersion_H}
    \end{equation}
    \item (外国も同様に \( Y_t^F, L_t^F, C_t^{F \to W}, \Delta_t^F, p_t^{F*}, \bar{p}_t^{F*}, p_t^{F \to W*}, \pi_t^{F*} \) などが定義される。)
\end{itemize}

\subsubsection*{代表的家計の変数}
3.4節のリスク共有の議論に基づき、代表的家計の変数(\( H \) 付きなど)を以下のように定義する。
\begin{itemize}
    \item \( c_t^{H \to W} \) :代表的家計の総消費指数。
    \begin{equation}
    c_t^{H \to W} \equiv c_t^{h \to W} \quad (\text{※3.4節より全家計で共通のため})
    \label{eq:def_ra_consumption_world}
    \end{equation}
    \item \( c_t^{H \to H} \) :代表的家計の自国財消費指数。
    \begin{equation}
    c_t^{H \to H} \equiv c_t^{h \to H} \quad (\text{※3.4節より全家計で共通のため})
    \label{eq:def_ra_consumption_domestic}
    \end{equation}
    \item \( c_t^{H \to F} \) :代表的家計の外国財消費指数。
    \begin{equation}
    c_t^{H \to F} \equiv c_t^{h \to F} \quad (\text{※3.4節より全家計で共通のため})
    \label{eq:def_ra_consumption_foreign}
    \end{equation}
    \item \( l_t^H \) :代表的家計の労働。
    \begin{equation}
    l_t^H \equiv \frac{1}{N} \sum_{h \in H} l_t^h
    \label{eq:def_ra_labor}
    \end{equation}
    \item \( y_t^H \) :代表的家計の生産指数。
    \begin{equation}
    y_t^H \equiv Y_t^H / N
    \label{eq:def_ra_output}
    \end{equation}
    \item \( \lambda_t^H \) :所得の限界効用。
    \begin{equation}
    \lambda_t^H \equiv \lambda_t^h \quad (\text{※3.4節より全家計で共通のため})
    \label{eq:def_ra_marginal_utility}
    \end{equation}
    \item \( b_t^{H} \) :代表的家計の対外純資産。
    \begin{equation}
    b_t^{H} \equiv B_t^{H} / N
    \label{eq:def_ra_assets}
    \end{equation}
    \item \( \widetilde{p}_t^H \) :代表的家計の最適価格。
    \begin{equation}
    \widetilde{p}_t^H \equiv \widetilde{p}_t^h \quad (\text{※3.5.2節より全価格改定家計で共通のため})
    \label{eq:def_ra_optimal_price}
    \end{equation}
    \item (外国も同様に \( c_t^{F \to W}, c_t^{F \to F}, c_t^{F \to H}, l_t^F, y_t^F, \lambda_t^{F/*}, b_t^{F*}, \widetilde{p}_t^{F*} \) が定義される。)
\end{itemize}

\subsubsection*{政策変数}
\begin{itemize}
    \item \( i_t^H, i_t^F \) : 自国と外国の政策金利(無リスク名目金利)。
    \item \( \tau_t^H, \tau_t^F \) : 自国と外国の所得税率。
    \item \( t_t^H, t_t^{F*} \) : 自国と外国の一括移転。
    \item \( \gamma_t^H, \gamma_t^F \) : 水準目標政策における過去の目標乖離の累積項。
\end{itemize}

\subsection*{外生変数(ショック)}
\begin{itemize}
    \item \( a_t^H, a_t^F \) : 生産性(技術水準)。
    \item \( \beta_t^H, \beta_t^F \) : 時間割引因子。
    \item \( \chi_t^H, \chi_t^F \) : 金融政策の目標パス。
    \item \( \varepsilon_t^{\tau,H}, \varepsilon_{t}^{i,H}, \dots \) : 各外生変数の確率過程に従うショック項。(外国も同様に \( \varepsilon_t^{\tau,F}, \varepsilon_{t}^{i,F} \) など)
\end{itemize}

\subsection*{パラメータ}
\subsubsection*{家計の選好}
\begin{itemize}
    \item \( \phi^H, \phi^F \) : 労働の非効用の重み。
    \item \( \alpha^H, \alpha^F \) : 消費バスケットにおける自国財(内需)への嗜好の強さ(ホームバイアス)。
\end{itemize}

\subsubsection*{生産と価格設定}
\begin{itemize}
    \item \( \theta^H, \theta^F \) : 個別財間の代替の弾力性。
    \item \( \xi^H, \xi^F \) : 価格を改定できない企業の割合(カルボ・パラメータ)。
\end{itemize}

\subsubsection*{政策ルール}
\begin{itemize}
    \item \( \phi_{\pi}, \phi_{y}, \phi_{gap}, \phi_{level} \) : 金融政策ルールにおける各目標への反応係数。(自国・外国共通の場合もあるが、区別する場合は \( \phi_{\pi}^H, \phi_{\pi}^F \) などとする。)
    \item \( \rho_i, \rho_{\tau}, \dots \) : 各外生変数の確率過程における自己回帰係数。(自国・外国共通の場合もあるが、区別する場合は \( \rho_i^H, \rho_i^F \) などとする。)
\end{itemize}

\subsubsection*{その他}
\begin{itemize}
    \item \( N, M \) : 自国と外国の人口(家計の数)。
\end{itemize}

\subsection*{不確実性の源泉と確率変数}
\label{subsec:model_stochastic_nature}
本節で定義した変数のうち、どれが確率変数となり、なぜそうなるのかをここで明記する。

\paragraph{1. 不確実性の源泉}
本稿のモデルにおけるすべての不確実性は、「外生変数(ショック)」の節で定義された変数
(例:生産性 $a_t^H$, 割引因子 $\beta_t^H$, 税率 $\tau_t^H$ など)から生じる。
これらの外生変数は第4章などで定義されるAR(1)(1次自己回帰)のような確率過程に従う。
たとえば $\beta_t^H$ は
\begin{equation}
\log(\beta_t^H) = (1-\rho_{\beta}^H)\log(\beta_{ss}^H) + \rho_{\beta}^H \log(\beta_{t-1}^H) + \epsilon_t^{\beta,H}
\label{eq:stochastic_process_beta_ra}
\end{equation}
のように記述される。
この式の最後にある $\epsilon_t^{\beta,H}$ が毎期新たにランダムに発生する新規ショックであり、
これが不確実性の唯一の源泉である。

\paragraph{2. 確率の仮定}
本稿ではすべての新規ショック項($\epsilon_t^{\beta,H}, \epsilon_t^{a,H}, \epsilon_t^{\tau,H}, ...$)を
まとめたベクトル $\boldsymbol{\epsilon}_t$ は時間を通じて独立かつ同一の分布(i.i.d.)に従うと仮定する。
この仮定は $t$ 期において、次期 $t+1$ にいかなる新規ショック $\boldsymbol{\epsilon}_{t+1}$ が発生する確率も、
現在および過去の状態やそこに至る経路には一切依存しないこと(経路非依存性)を意味する。

\paragraph{3. 確率変数となる変数}
上記の仮定から、確率変数となるのは以下の2種類である。
\begin{itemize}
    \item \textbf{外生変数(ショック変数):}
    $\epsilon_t$ がランダムであるため、$\epsilon_t$ に依存する $a_t^H$, $\beta_t^H$, $\tau_t^H$ などの
    外生変数が確率変数となる。
    
    \item \textbf{すべての内生変数:}
    モデルのすべての内生変数(例:消費 $c_t^H$, 生産 $y_t^H$, 所得の限界効用 $\lambda_t^H$,
    価格 $p_t^H$, 為替レート $e_t^{/*}$ など)はモデルの方程式系(第3.8節)を通じて
    確率的な外生変数の現在値と期待される将来経路の関数として決定される。
    したがって入力(外生変数)が確率変数であるため、その出力であるすべての解(内生変数)もまた確率変数となる。
\end{itemize}
このため家計や企業は将来の確率変数(例:$\lambda_{t+1}^H$, $e_{t+1}$ など)の値を
現時点 $t$ で正確に知ることはできず、その期待値($E_t[\cdot]$)にもとづいて最適化行動をとる必要がある。
% !TeX root = ../../main.tex
% sections/chap3/sec_model_equilibriums.tex

\section{市場均衡}
\label{sec:model_equilibriums}

\subsection{財市場の均衡}
\label{subsec:model_goods_market}
\paragraph{自国財市場}
家計 \( h \) が生産する財の市場は \( t \) 期におけるその財の生産 \( y_t^h \) が
全世界からの総需要と等しくなることで均衡する。
\begin{equation}
y_t^h = \sum_{h' \in H} c_t^{h' \to h} + \sum_{f \in F} c_t^{f \to h}
\label{eq:individual_goods_market_equilibrium_repeat}
\end{equation}

\paragraph{外国財市場}
同様に外国財市場の均衡は以下の関係式で記述される。
\begin{equation}
y_t^f = \sum_{h \in H} c_t^{h \to f} + \sum_{f' \in F} c_t^{f' \to f}
\label{eq:individual_goods_market_equilibrium_foreign}
\end{equation}

\subsection{一物一価の法則}
\label{subsec:model_lop}
国際間の財価格は名目為替レート \( e_t^{/*} \) を通じて以下の一物一価の法則が成立すると仮定する。
\begin{equation}
p_t^h = e_t^{/*} p_t^{h*} \quad , \quad p_t^f = e_t^{/*} p_t^{f*}
\label{eq:lop}
\end{equation}
% !TeX root = ../../main.tex
% sections/chap3/sec_model_households.tex

\section{家計}
\label{sec:model_households}
本節では自国の家計 \( h \in H \) の最適化行動を記述する。
外国の家計 \( f \in F \) の行動も対称的に定式化される。

\subsection{最適化問題}
\label{subsec:model_optimization_problem}

家計が直面する問題はマクロ経済全体の不確実性(状態 \( j \))と
価格改定の可否に関する個人的な不確実性(状態 \( \Xi \))の同時確率分布の下で
生涯効用を最大化する問題として厳密に定式化される。
しかしこれを一度に解こうとすると記述が複雑になるため、
本稿では多くの先行研究にならいこの問題を「価格以外の変数の決定問題」と「価格の決定問題」の2つに分割して分析する。
このように分割しても問題として同値であることは付録A「効用最大化問題と2段階最適化の等価性」で示される。

\subsubsection{価格以外の変数の決定問題}
まず価格 \( p_t^h \) を所与とした上で家計の最適化問題を考える。
家計 \( h \) は各期において以下の名目予算制約の下で生涯効用関数を最大化する。
本稿では \textcite{GarinLesterSims2016} など多くの先行研究にならい
この生涯効用関数の値を家計 \( h \) の厚生と定義する。
このとき操作変数は消費 \( c_t^{h \to h'}, c_t^{h \to f} \)、労働 \( l_t^h \)、
および各種債権の保有量 \( d_{t+1}^{h \to H}, b_{t+1}^{h \to F} \) である。

\paragraph{自国家計の生涯効用関数:}
\begin{equation}
\operatorname{E}_s \sum_{t=s}^{\infty} \left( \prod_{k=0}^{t-s-1} \beta_{s+k}^H \right) \left( \log \left( \frac{\left( \left[ \sum_{h' \in H} (c_t^{h \to h'})^{\frac{\theta^H-1}{\theta^H}} \right]^{\frac{\theta^H}{\theta^H-1}} \right)^{\alpha^H} \left( \left[ \sum_{f \in F} (c_t^{h \to f})^{\frac{\theta^F-1}{\theta^F}} \right]^{\frac{\theta^F}{\theta^F-1}} \right)^{1-\alpha^H}}{(\alpha^H)^{\alpha^H} (1-\alpha^H)^{1-\alpha^H}} \right) - \frac{\phi^H}{2}(l_t^h)^2 \right)
\label{eq:lifetime_utility_home_original}
\end{equation}

\paragraph{自国家計の名目予算制約:}
\begin{equation}
\begin{split}
& \sum_{j' \in J} q_{t, t+1}(j') d_{t+1}^{h \to H}(j') + b_{t+1}^{h \to F} + \sum_{h' \in H} p_t^{h'} c_t^{h \to h'} + \sum_{f \in F} p_t^{f} c_t^{h \to f} \\
& \qquad = d_t^{h \to H} + (1+i_{t-1}^F) \frac{e_t^{/*}}{e_{t-1}^{/*}} b_t^{h \to F} + (1-\tau_t^H) p_t^h y_t^h + t_t^H
\end{split}
\label{eq:nominal_budget_home_original}
\end{equation}

この最適化問題は家計が多種多様な個別財の消費 \( c_t^{h \to h'}, c_t^{h \to f} \) をすべて同時に
選択しなければならないため非常に複雑である。
そこでこの問題をより扱いやすくするために標準的な方法である2段階の最適化に問題を分割する。
これは元の問題と同値であり、家計の最適化を
「①所与の総消費指数(log の引数全体。のちに \( c_t^{h \to W} \) をこの値に一致するように定義する)を
どのような消費を組み合わせて最小費用で実現するか」という期内の問題と
「②どのように総消費指数 \( c_t^{h \to W} \)、労働供給 \( l_t^h \)、
および各種債権の保有量 \( d_{t+1}^{h \to H}, b_{t+1}^{h \to F} \) を組み合わせて
名目予算制約のもと生涯効用関数を最大化するか」という期をまたぐ問題に分けて考えることに相当する。
この同値性の証明は付録A「効用最大化問題と2段階最適化の等価性」に譲る。

\paragraph{外国家計の最適化問題}
同様に外国家計 \( f \) は外国通貨建ての名目予算制約の下で生涯効用関数を最大化する。

\paragraph{外国家計の生涯効用関数:}
\begin{equation}
\operatorname{E}_s \sum_{t=s}^{\infty} \left( \prod_{k=0}^{t-s-1} \beta_{s+k}^F \right) \left( \log \left( \frac{\left( \left[ \sum_{f' \in F} (c_t^{f \to f'})^{\frac{\theta^F-1}{\theta^F}} \right]^{\frac{\theta^F}{\theta^F-1}} \right)^{\alpha^F} \left( \left[ \sum_{h' \in H} (c_t^{f \to h'})^{\frac{\theta^H-1}{\theta^H}} \right]^{\frac{\theta^H}{\theta^H-1}} \right)^{1-\alpha^F}}{(\alpha^F)^{\alpha^F} (1-\alpha^F)^{1-\alpha^F}} \right) - \frac{\phi^F}{2}(l_t^f)^2 \right)
\label{eq:lifetime_utility_foreign_original}
\end{equation}

\paragraph{外国家計の名目予算制約 (外国通貨建て):}
\begin{equation}
\begin{split}
& \sum_{j' \in J} q_{t, t+1}^*(j') d_{t+1}^{f \to F*}(j') + b_{t+1}^{f \to H*} + \sum_{f' \in F} p_t^{f'*} c_t^{f \to f'} + \sum_{h' \in H} p_t^{h'*} c_t^{f \to h'} \\
& \qquad = d_t^{f \to F*} + (1+i_{t-1}^H) \frac{e_{t-1}^{/*}}{e_t^{/*}} b_t^{f \to H*} + (1-\tau_t^F) p_t^{f*} y_t^f + t_t^{F*}
\end{split}
\label{eq:nominal_budget_foreign_original}
\end{equation}

\subsection{消費指数と物価指数}
\label{subsec:model_consumption_baskets}
本節では上記の2段階最適化法で用いる各種の消費指数とそれに対応する物価指数を定義していく。
家計が多様な財から効用を得る構造をモデル化するため各財の消費をもとに消費指数と呼ばれる集計量を定義する。
この消費指数を定義するCES(Constant Elasticity of Substitution)関数は
家計の選好に含まれる2種類のパラメータ \( \alpha \) と \( \theta \) の役割を理解する上で中心的な役割を果たす。

一般にCES型関数は、2財の場合、次のように書かれる。
\begin{equation}
U = \left( \alpha_1 c_1^{\frac{\theta-1}{\theta}} + \alpha_2 c_2^{\frac{\theta-1}{\theta}} \right)^{\frac{\theta}{\theta-1}}
\label{eq:ces_general_form_households}
\end{equation}

ここで \( \alpha \) は財への根源的な好みの度合い(バイアス)を示し \( \theta \) は
代替の弾力性を示すパラメータである。
内側の指数 \( \frac{\theta-1}{\theta} \) の値は財の代替のしやすさを決定する。
指数が小さい(\( \theta \)が1に近い)場合、ある財の限界的な効用の減少が早いため、
各財に消費を分散させる方が効用が高まり代替が難しい状況となる。
逆に指数が大きい(\( \theta \)が大きい)場合、限界効用の減少が緩やかであるため各財は代替が容易な状況となる。
なお全体にかかっている外側の指数 \( \frac{\theta}{\theta-1} \) は
内側の指数による数学的な変換を元に戻す調整項である。
これにより、集計量である消費指数が元の消費という経済的に意味のある単位を維持することが保証される。
具体的には、消費の単位を「単位」とすると
\begin{equation}
\left[ (\text{単位})^{\frac{\theta-1}{\theta}} \right]^{\frac{\theta}{\theta-1}} = \text{単位}^{(\frac{\theta-1}{\theta}) \cdot (\frac{\theta}{\theta-1})} = \text{単位}^1
\label{eq:unit_consistency_ces}
\end{equation}
となり消費指数は元の消費と同じ次元をもつ。

このCES型関数から限界代替率(\( MRS_{1,2} \))が次のように導かれる。
\begin{equation}
MRS_{1,2} = \frac{\alpha_1}{\alpha_2} \left( \frac{c_1}{c_2} \right)^{-1/\theta}
\label{eq:mrs_ces_households}
\end{equation}

\( MRS_{1,2} \)は特定の消費時点において財1が財2の何倍の価値があるかを表すため、
1財の代替のしにくさを表す指標と考えることができる。
この\( MRS_{1,2} \)の水準は式の通り \( \alpha_1, \alpha_2, c_1, c_2, \theta \) によって決定される。
しかし \( \alpha_1, \alpha_2, \theta \) は家計の選好を表す固定されたパラメータである。
そこで \( \alpha \)の影響を取り除き、
変動する変数である消費の比率(\( c_2/c_1 \))と\( MRS_{1,2} \) の純粋な関係を分析する必要が生じる。

その際に変化率の関係、すなわち弾力性を考えることで2つの変数の関係を分析することができる。
具体的には \( MRS_{1,2} \) の変化の割合が消費の比率 \( c_2/c_1 \) の変化の割合の何倍かは一定値となり、
それが代替の弾力性 \( \theta \) となる。

以下では、家計 \( h \) の効用の源となる総消費指数を合成する。
この総消費指数の対数(log)をとったものが \( h \) が消費から得られる効用となる。
総消費指数を合成するため、まずそのもととなる自国財消費指数と外国財消費指数を定義する。
これらの消費指数はすべて先ほど説明したCES型関数の特殊なものである。
どのように特殊であるかといえば、
自国財および外国財消費指数は個別財の消費を\( \alpha \)を均一化(対称性を仮定)して
CES型関数により合成したものである。
また総消費指数は自国財および外国財消費指数を\( \theta=1 \)としてCES型関数により合成したものである。
以上の設計思想は以下の表のようにまとめられる。

\begin{table}[H]
\centering
\caption{消費指数の設計思想の比較}
\begin{tabular}{lll}
\toprule
 & 自国財および外国財消費指数 (\( c_t^{h \to H}, c_t^{h \to F} \)) & 総消費指数 (\( c_t^{h \to W} \)) \\
\midrule
\(\alpha\)の扱い & 対称性を仮定するため不要 & 非対称性(ホームバイアス)のため残す \\
\(\theta\)の扱い & パラメータとして残す & \(\theta=1\)に固定(コブ=ダグラス型) \\
分析の主役 & \(\theta\) (代替の弾力性) & \(\alpha^H\) (ホームバイアス) \\
\bottomrule
\end{tabular}
\label{tab:consumption_index_philosophy}
\end{table}

このモデルの設計が持つ意義は分析の焦点を明確に分離している点にある。
自国財および外国財消費指数においては個別財間の代替の弾力性 \( \theta \) が
生産者の価格設定行動や物価の粘着性を左右する。
この \( \theta \) を分析の主役とするため各財好みを表す \( \alpha \) は平準化して消去する。
一方で総消費指数においては2国間の消費・貿易の基本的なパターン
(どれだけ輸入し、どれだけ国内で消費するか)を決定するのは
家計の根源的な内外財への選好の偏り、すなわちホームバイアス \( \alpha^H \) である。
代替の弾力性 \( \theta \) は為替レートの変化などに対する反応の仕方を決めるが、
それは \( \alpha^H \) が設定した基本的な貿易パターン上での変動に過ぎない。
したがって国際的なパターンを分析する上では \( \alpha^H \) がより根源的なパラメータであると考え、
この \( \alpha \) を分析の主役とするため \( \theta=1 \) という
「最も普通の」場合を考えて \( \theta \) を消去する。

以降の節でこれらの消費指数の厳密な定義とそこから導かれる関係式を詳述する。

\subsubsection{個別財消費から自国財・外国財消費指数への集計}
\label{subsubsec:model_consumption_stage1a}
ここでは2段階最適化の第1段階(期内費用最小化)のうち、さらに1つ目のステップを考える。
まず自国家計 \( h \) が消費する自国財 \( c_t^{h \to h'} \) を集計して
自国財消費指数 \( c_t^{h \to H} \) を定義する。
これは各個別財への選好が対称的である (すべての\( h' \in H \)について\(\alpha_{h'}=1 \)) 
としたCES関数の形をとる。
\begin{equation}
c_t^{h \to H} \equiv \left[ \sum_{h' \in H} (c_t^{h \to h'})^{\frac{\theta^H-1}{\theta^H}} \right]^{\frac{\theta^H}{\theta^H-1}}
\label{eq:domestic_basket_def_households}
\end{equation}

同様に外国財を集計した外国財消費指数 \( c_t^{h \to F} \) も定義される。
\begin{equation}
c_t^{h \to F} \equiv \left[ \sum_{f \in F} (c_t^{h \to f})^{\frac{\theta^F-1}{\theta^F}} \right]^{\frac{\theta^F}{\theta^F-1}}
\label{eq:foreign_basket_def_households}
\end{equation}

\paragraph{費用最小化と需要関数(自国家計 \(\to\) 自国財消費指数)}
家計 \( h \) が所与の自国財消費指数 \( c_t^{h \to H} \) を達成するために
各個別財 \( c_t^{h \to h'} \) の購入費用 \( \sum_{h'} p_t^{h'} c_t^{h \to h'} \) を最小化する問題を考える。
この問題のラグランジアンは以下のように記述される。
\begin{equation}
\mathcal{L}^{h \to H} = \sum_{h' \in H} p_t^{h'} c_t^{h \to h'} + \mu_t^{h \to H} \left( c_t^{h \to H} - \left[ \sum_{h' \in H} (c_t^{h \to h'})^{\frac{\theta^H-1}{\theta^H}} \right]^{\frac{\theta^H}{\theta^H-1}} \right)
\label{eq:lagrangian_domestic_basket}
\end{equation}
ここでラグランジュ乗数 \( \mu_t^{h \to H} \) は
家計 \( h \) にとっての自国財消費指数 \( c_t^{h \to H} \) の限界価格(生産者物価指数(PPI))を表す。

この費用最小化問題から家計 \( h' \) が生産する個別財への需要関数が導出される。
\begin{equation}
c_t^{h \to h'} = \left( \frac{p_t^{h'}}{\mu_t^{h \to H}} \right)^{-\theta^H} c_t^{h \to H}
\label{eq:demand_individual_home_good}
\end{equation}

この需要関数を名目自国財消費 \( \sum_{h'} p_t^{h'} c_t^{h \to h'} \) に
代入して計算すると \( \sum_{h'} p_t^{h'} c_t^{h \to h'} = \mu_t^{h \to H} c_t^{h \to H} \) という
関係が得られる(詳細は付録A参照)。
これは \( \mu_t^{h \to H} \) が自国財消費指数の限界価格であることから予想された結果である。

この最適化問題を解く過程で、限界価格 \( \mu_t^{h \to H} \) は
個々の財価格のみで構成される以下の形で決定される(詳細は付録A参照)。
\begin{equation}
\mu_t^{h \to H} = \left[ \sum_{h' \in H} (p_t^{h'})^{1-\theta^H} \right]^{\frac{1}{1-\theta^H}}
\label{eq:mu_h_H_definition}
\end{equation}
この式からわかるように \( \mu_t^{h \to H} \) は個々の財価格 \( p_t^{h'} \) のみで決定され、
家計 \( h \) 自身の選択(\( c_t^{h \to h'} \) など)には依存しない。
したがって \( \mu_t^{h \to H} \) はすべての自国家計で共通の値をとる。

\paragraph{他の物価指数(\(\mu\))の導出}
上記と全く同様の費用最小化問題を以下の3つのケースについて考えることで
それぞれの消費指数に対応する物価指数(ラグランジュ乗数)が導出される。

\begin{enumerate}
    \item \textbf{自国家計 \( h \) が外国財消費指数 \( c_t^{h \to F} \) を最小費用で達成する問題}: \\
    この問題から自国家計が直面する外国財消費指数の生産者物価指数(PPI、自国通貨建て)
    \( \mu_t^{h \to F} \) が以下のように導かれる。
    \begin{equation}
    \mu_t^{h \to F} \equiv \left[ \sum_{f' \in F} (p_t^{f'})^{1-\theta^F} \right]^{\frac{1}{1-\theta^F}}
    \label{eq:mu_h_F_definition}
    \end{equation}
    (ここで \( p_t^{f' } \) は外国財 \( f' \) の自国通貨建て価格) 

    \item \textbf{外国家計 \( f \) が外国財消費指数 \( c_t^{f \to F} \) を最小費用で達成する問題}: \\
    この問題から外国家計が直面する外国財消費指数の生産者物価指数(PPI、外国通貨建て)
    \( \mu_t^{f \to F} \) が以下のように導かれる。
    \begin{equation}
    \mu_t^{f \to F} \equiv \left[ \sum_{f' \in F} (p_t^{f' *})^{1-\theta^F} \right]^{\frac{1}{1-\theta^F}}
    \label{eq:mu_f_F_definition}
    \end{equation}
    (ここで \( p_t^{f' *} \) は外国財 \( f' \) の外国通貨建て価格) 

    \item \textbf{外国家計 \( f \) が自国財消費指数 \( c_t^{f \to H} \) を最小費用で達成する問題}: \\
    この問題から外国家計が直面する自国財消費指数の生産者物価指数(PPI、外国通貨建て)
    \( \mu_t^{f \to H} \) が以下のように導かれる。
    \begin{equation}
    \mu_t^{f \to H} \equiv \left[ \sum_{h' \in H} (p_t^{h' *})^{1-\theta^H} \right]^{\frac{1}{1-\theta^H}}
    \label{eq:mu_f_H_definition}
    \end{equation}
    (ここで \( p_t^{h' *} \) は自国財 \( h' \) の外国通貨建て価格) 
\end{enumerate}

一物一価の法則(\( p_t^h = e_t^{/*} p_t^{h*} \) および \( p_t^f = e_t^{/*} p_t^{f*} \))を用いると、
異なる通貨建ての物価指数間に以下の厳密な関係が導出される。
\begin{equation}
\mu_t^{h \to H} = \left[ \sum (e_t^{/*} p_t^{h'*})^{1-\theta^H} \right]^{\frac{1}{1-\theta^H}} = e_t^{/*} \left[ \sum (p_t^{h'*})^{1-\theta^H} \right]^{\frac{1}{1-\theta^H}} = e_t^{/*} \mu_t^{f \to H}
\label{eq:price_relation_mu_h_f_H}
\end{equation}

同様に
\begin{equation}
\mu_t^{h \to F} = \left[ \sum (e_t^{/*} p_t^{f'*})^{1-\theta^F} \right]^{\frac{1}{1-\theta^F}} = e_t^{/*} \left[ \sum (p_t^{f'*})^{1-\theta^F} \right]^{\frac{1}{1-\theta^F}} = e_t^{/*} \mu_t^{f \to F}
\label{eq:price_relation_mu_h_f_F}
\end{equation}

したがって自国通貨建ての生産者物価指数(PPI)は
外国通貨建ての生産者物価指数(PPI)に名目為替レートを乗じたものと一致する。

\paragraph{記法の簡略化とマクロ物価指数}
今後の記述を簡潔にするため、家計が直面する2つの基本的な生産者物価指数(PPI)を
以下のように \( p_t^H \) と \( p_t^{F*} \) で表記する。
\begin{align}
p_t^H &\equiv \mu_t^{h \to H} \quad \text{(自国財の自国通貨建て生産者物価指数(PPI)}
\label{eq:define_pH} \\
p_t^{F*} &\equiv \mu_t^{f \to F} \quad \text{(外国財の外国通貨建て生産者物価指数(PPI)}
\label{eq:define_pFstar}
\end{align}

この記法を用いると4つの物価指数の関係は
 \( \mu_t^{h \to H} = p_t^H \),
 \( \mu_t^{h \to F} = p_t^{F} \),
 \( \mu_t^{f \to F} = p_t^{F*} \),
 \( \mu_t^{f \to H} = p_t^{H*} \) と整理される。

これらは家計が意思決定で直面する生産者物価指数(PPI)である。
これとは別にマクロ分析のため経済規模の影響(\( N, M \))を取り除いた
「正規化された」生産者物価指数(PPI)
 \( \bar{p}_t^H \) と \( \bar{p}_t^{F*} \) を以下のように定義する。
\begin{align}
\bar{p}_t^H &\equiv \left[ \frac{1}{N} \sum_{h' \in H} (p_t^{h'})^{1-\theta^H} \right]^{\frac{1}{1-\theta^H}} = N^{-\frac{1}{1-\theta^H}} p_t^H
\label{eq:define_pbarH} \\
\bar{p}_t^{F*} &\equiv \left[ \frac{1}{M} \sum_{f' \in F} (p_t^{f'*})^{1-\theta^F} \right]^{\frac{1}{1-\theta^F}} = M^{-\frac{1}{1-\theta^F}} p_t^{F*}
\label{eq:define_pbarFstar}
\end{align}

\paragraph{費用最小化に対応する需要関数}
各費用最小化問題から導出される個別財への需要関数は上記の記法を用いると以下のように整理される。
\begin{itemize}
    \item 自国家計による自国財 \( h' \) への需要:
    \begin{equation}
    c_t^{h \to h'} = \left( \frac{p_t^{h'}}{p_t^H} \right)^{-\theta^H} c_t^{h \to H}
    \label{eq:demand_h_h_prime}
    \end{equation}
    \item 自国家計による外国財 \( f' \) への需要:
    \begin{equation}
    c_t^{h \to f'} = \left( \frac{p_t^{f'}}{p_t^{F}} \right)^{-\theta^F} c_t^{h \to F}
    \label{eq:demand_h_f_prime}
    \end{equation}
    \item 外国家計による外国財 \( f' \) への需要:
    \begin{equation}
    c_t^{f \to f'} = \left( \frac{p_t^{f'*}}{p_t^{F*}} \right)^{-\theta^F} c_t^{f \to F} \label{eq:demand_f_f_prime}
    \end{equation}
    \item 外国家計による自国財 \( h' \) への需要:
    \begin{equation}
    c_t^{f \to h'} = \left( \frac{p_t^{h'*}}{p_t^{H*}} \right)^{-\theta^H} c_t^{f \to H} \label{eq:demand_f_h_prime}
    \end{equation}
\end{itemize}

また各最小化問題における名目消費と消費指数の関係は以下の通りである。
\begin{itemize}
    \item \begin{equation} \sum_{h'} p_t^{h'} c_t^{h \to h'} = p_t^H c_t^{h \to H} \label{eq:nominal_domestic_consumption_households} \end{equation}
    \item \begin{equation} \sum_{f'} p_t^{f'} c_t^{h \to f'} = p_t^{F} c_t^{h \to F} \label{eq:nominal_foreign_consumption_households} \end{equation}
    \item \begin{equation} \sum_{f'} p_t^{f'*} c_t^{f \to f'} = p_t^{F*} c_t^{f \to F} \label{eq:nominal_foreign_consumption_foreign_households} \end{equation}
    \item \begin{equation} \sum_{h'} p_t^{h'*} c_t^{f \to h'} = p_t^{H*} c_t^{f \to H} \label{eq:nominal_domestic_consumption_foreign_households} \end{equation}
\end{itemize}

\subsubsection{自国・外国財消費指数から総消費指数への集計}
\label{subsubsec:model_consumption_stage1b}
次に2段階最適化の第1段階(期内費用最小化)のうち2つ目のステップを考える。
総消費指数 \( c_t^{h \to W} \) は
自国財消費指数 \( c_t^{h \to H} \) と外国財消費指数 \( c_t^{h \to F} \) から合成される。
これはCES型指数において代替の弾力性パラメータ \( \theta \) が1である特別な場合(コブ=ダグラス型)に対応する。

\paragraph{自国家計の費用最小化}
総消費指数は以下のように定義される。
\begin{equation}
c_t^{h \to W} \equiv \frac{(c_t^{h \to H})^{\alpha^H} (c_t^{h \to F})^{1-\alpha^H}}{(\alpha^H)^{\alpha^H} (1-\alpha^H)^{1-\alpha^H}}
\label{eq:world_consumption_index_def}
\end{equation}

この定義はパラメータ \( \alpha^H \) を用いた加重幾何平均の形をとる。
この形式がもつ一つの重要な含意は単位の整合性である。
仮に消費指数の単位を「単位」とすると
\begin{equation}
(\text{単位})^{\alpha^H} \times (\text{単位})^{1-\alpha^H} = \text{単位}^{\alpha^H + 1-\alpha^H} = \text{単位}^1
\label{eq:unit_consistency_world_basket}
\end{equation}
となり総消費指数 \( c_t^{h \to W} \) はその構成要素と同じ次元を持つ。
これにより \( c_t^{h \to W} \) は家計の総体的な消費水準を示す指標として成立する。

\( 1 / \left( (\alpha^H)^{\alpha^H} (1-\alpha^H)^{1-\alpha^H} \right) \) という定数項は
総消費指数 \( c_t^{h \to W} \) を調整するための正規化項である。
モデルの式中では消費者物価指数 \( p_t^{H \to W} \) が多用されるため、
その定義式が簡潔になるよう、あらかじめ \( c_t^{h \to W} \) の定義にこの項を入れておくのが一般的である。
この正規化により後続する消費者物価指数 \( p_t^{H \to W} \) の定義式から
定数項 \( (\alpha^H)^{\alpha^H} (1-\alpha^H)^{1-\alpha^H} \) が消え、
\( p_t^{H \to W} = (p_t^H)^{\alpha^H} (p_t^{F})^{1-\alpha^H} \) という簡潔な形で表現される。

次に家計 \( h \) は所与の総消費指数 \( c_t^{h \to W} \) を達成するために
自国財消費指数 \( c_t^{h \to H} \) と外国財消費指数 \( c_t^{h \to F} \) の
組み合わせ費用 \( p_t^H c_t^{h \to H} + p_t^{F} c_t^{h \to F} \) を最小化する問題を考える。
この問題のラグランジアンは以下のように書ける。
\begin{equation}
\mathcal{L}^{h \to W} = p_t^H c_t^{h \to H} + p_t^{F} c_t^{h \to F} + \eta_t^h \left( c_t^{h \to W} - \frac{(c_t^{h \to H})^{\alpha^H} (c_t^{h \to F})^{1-\alpha^H}}{(\alpha^H)^{\alpha^H} (1-\alpha^H)^{1-\alpha^H}} \right)
\label{eq:lagrangian_world_basket}
\end{equation}

ここでラグランジュ乗数 \( \eta_t^h \) は
家計 \( h \) にとっての総消費指数 \( c_t^{h \to W} \) の限界価格を表す。

この費用最小化問題の一階条件から各消費指数への需要関数は \( \eta_t^h \) を用いて以下のように導かれる。
\begin{align}
p_t^H c_t^{h \to H} &= \alpha^H \eta_t^h c_t^{h \to W}
\label{eq:demand_domestic_basket_pre_nominal} \\
p_t^{F} c_t^{h \to F} &= (1-\alpha^H) \eta_t^h c_t^{h \to W}
\label{eq:demand_foreign_basket_pre_nominal}
\end{align}

この結果を名目総消費 \( p_t^H c_t^{h \to H} + p_t^{F} c_t^{h \to F} \) に代入すると
\begin{equation}
p_t^H c_t^{h \to H} + p_t^{F} c_t^{h \to F} = \alpha^H \eta_t^h c_t^{h \to W} + (1-\alpha^H) \eta_t^h c_t^{h \to W} = \eta_t^h c_t^{h \to W}
\label{eq:nominal_total_expenditure_relation_pre}
\end{equation}
という関係が得られる。
これは第1段階と同様に \( \eta_t^h \) が総消費指数の限界価格であることから予想された結果である。

この最適化問題を解く過程で限界価格 \( \eta_t^h \) は以下の形で決定される(詳細は付録A参照)。
\begin{equation}
\eta_t^h = (p_t^H)^{\alpha^H} (p_t^{F})^{1-\alpha^H}
\label{eq:marginal_price_world_index}
\end{equation}

この式からわかるように \( \eta_t^h \) は各国の生産者物価指数 \( p_t^H, p_t^{F} \) のみで決定され
家計 \( h \) 自身の選択には依存しない。
したがって \( \eta_t^h \) はすべての家計で共通の値をとる。
この共通の限界価格を \( p_t^{H \to W} \) と定義し本稿では消費者物価指数 (CPI) と呼ぶ。
\begin{equation}
p_t^{H \to W} \equiv \eta_t^h = (p_t^H)^{\alpha^H} (p_t^{F})^{1-\alpha^H}
\label{eq:cpi_definition}
\end{equation}

これとは別にマクロ分析のため経済規模の影響(\( N, M \))を取り除いた
「正規化された」消費者物価指数 \( \bar{p}_t^{H \to W} \) を次のように定義する。
\begin{equation}
\bar{p}_t^{H \to W} \equiv (\bar{p}_t^H)^{\alpha^H} (\bar{p}_t^F)^{1-\alpha^H}
\label{eq:normalized_cpi_definition}
\end{equation}
この \( \bar{p}_t^{H \to W} \) は個々の企業の価格を直接集計した統計上の消費者物価指数に相当する。
中央銀行が消費者物価指数インフレ目標を採用する場合、
目標とするインフレ率はこの正規化された消費者物価指数 \( \bar{p}_t^{H \to W} \) の変化率として算出される。

この関係 \( \eta_t^h = p_t^{H \to W} \) を式
\eqref{eq:demand_domestic_basket_pre_nominal} と
\eqref{eq:demand_foreign_basket_pre_nominal} に代入し、
\( c_t^{h \to H} \) と \( c_t^{h \to F} \) について解くことで最終的な需要関数が得られる。
\begin{align}
c_t^{h \to H} &= \alpha^H \frac{p_t^{H \to W}}{p_t^H} c_t^{h \to W}
\label{eq:demand_domestic_basket} \\
c_t^{h \to F} &= (1-\alpha^H) \frac{p_t^{H \to W}}{p_t^{F}} c_t^{h \to W}
\label{eq:demand_foreign_basket}
\end{align}

ここで定義上は集計の際の重みとして導入されたパラメータ \( \alpha^H \) が
家計の最適化の結果として支出シェアという経済学的な意味を持つことを示す。
式\eqref{eq:demand_domestic_basket_pre_nominal} は
自国財消費指数への名目支出額 \( p_t^H c_t^{h \to H} \) が総消費指数への名目総支出額
 \( \eta_t^h c_t^{h \to W} = p_t^{H \to W} c_t^{h \to W} \) の \( \alpha^H \) 倍に等しくなることを
直接的に示している。
同様に式\eqref{eq:demand_foreign_basket_pre_nominal} は
外国財消費指数への名目支出額 \( p_t^{F} c_t^{h \to F} \) が
名目総支出額の \( (1-\alpha^H) \) 倍になることを示している。
したがって家計が最適な消費の組み合わせを選択した結果、
定義における重みパラメータ \( \alpha^H \) と \( 1-\alpha^H \) は
必然的に総支出に占める自国財消費指数と外国財消費指数それぞれへの支出シェアと一致する。
また名目総消費と指数の間には
 \( p_t^H c_t^{h \to H} + p_t^{F} c_t^{h \to F} = p_t^{H \to W} c_t^{h \to W} \) という関係が成り立つ。

\paragraph{外国家計の費用最小化}
同様に外国家計 \( f \) は所与の総消費指数 \( c_t^{f \to W} \) を達成するために
外国財消費指数 \( c_t^{f \to F} \) と自国消費指数 \( c_t^{f \to H} \) の組み合わせ費用
 \( p_t^{F*} c_t^{f \to F} + p_t^{H*} c_t^{f \to H} \) を最小化する。
 この結果、外国の消費者物価指数 (CPI, 外国通貨建て) \( p_t^{F \to W*} \) が
 ラグランジュ乗数 \( \eta_t^f \) と等しいものとして以下のように定義される。
\begin{equation}
p_t^{F \to W*} \equiv \eta_t^f = (p_t^{F*})^{\alpha^F} (p_t^{H*})^{\alpha^F}
\label{eq:cpi_foreign_definition}
\end{equation}

これとは別にマクロ分析のため経済規模の影響(\( N, M \))を取り除いた
「正規化された」外国の消費者物価指数 \( \bar{p}_t^{F \to W*} \) を次のように定義する。
\begin{equation}
\bar{p}_t^{F \to W*} \equiv (\bar{p}_t^{F*})^{\alpha^F} (\bar{p}_t^{H*})^{1-\alpha^F}
\label{eq:normalized_cpi_foreign_definition}
\end{equation}
この \( \bar{p}_t^{F \to W*} \) は個々の企業の価格を直接集計した統計上の消費者物価指数に相当する。
中央銀行が消費者物価指数インフレ目標を採用する場合、
目標とするインフレ率はこの正規化された消費者物価指数 \( \bar{p}_t^{F \to W*} \) の変化率として算出される。

また外国家計の各消費指数への需要関数は以下の通りである。
\begin{align}
c_t^{f \to F} &= \alpha^F \frac{p_t^{F \to W*}}{p_t^{F*}} c_t^{f \to W}
\label{eq:demand_foreign_basket_f} \\
c_t^{f \to H} &= (1-\alpha^F) \frac{p_t^{F \to W*}}{p_t^{H*}} c_t^{f \to W}
\label{eq:demand_domestic_basket_f}
\end{align}
名目総消費と指数の間には \( p_t^{F*} c_t^{f \to F} + p_t^{H*} c_t^{f \to H} = p_t^{F \to W*} c_t^{f \to W} \) という関係が成り立つ。

\subsubsection{最適化の一階条件(期をまたぐ問題)}
ここからは2段階最適化の第2段階、すなわち期をまたぐ最適化問題を扱う。
第一段階である期内費用最小化の結果、
家計の名目総支出 \( \sum_{h'} p_t^{h'} c_t^{h \to h'} + \sum_{f} p_t^f c_t^{h \to f} \) は
最終財の物価指数 \( p_t^{H \to W} \) と消費指数 \( c_t^{h \to W} \) の積で簡潔に表現できることがわかる
(証明は付録Aに譲る)。
\begin{equation}
\sum_{h'} p_t^{h'} c_t^{h \to h'} + \sum_{f} p_t^f c_t^{h \to f} = p_t^{H \to W} c_t^{h \to W}
\label{eq:nominal_total_expenditure_identity}
\end{equation}

また生涯効用関数の \( \log \) の引数全体も \( c_t^{h \to W} \) に一致する。
\begin{equation}
\frac{(c_t^{h \to H})^{\alpha^H} (c_t^{h \to F})^{1-\alpha^H}}{(\alpha^H)^{\alpha^H} (1-\alpha^H)^{1-\alpha^H}} = c_t^{h \to W}
\label{eq:world_index_equivalence}
\end{equation}

これらの関係を用いることで、3.2.1節で提示した複雑な予算制約を持つ問題は
より単純な操作変数を用いた問題に書き換えることができる。
この第2段階の問題のラグランジアンは以下のように記述される。
ここでの操作変数は、総消費指数 \( c_t^{h \to W} \)、労働供給 \( l_t^h \)、
および各種債権の保有量 \( d_{t+1}^{h \to H}, b_{t+1}^{h \to F} \) である。
\begin{equation}
\begin{aligned}
\mathcal{L}^h = & \operatorname{E}_s \Biggl[ \sum_{t=s}^{\infty} \left( \prod_{k=0}^{t-s-1} \beta_{s+k}^H \right) \Biggl\{ \left( \log c_t^{h \to W} - \frac{\phi^H}{2}(l_t^h)^2 \right) \\
& \quad + \lambda_t^h \biggl( \Bigl( d_{t}^{h \to H} + (1+i_{t-1}^{F}) \frac{e_t^{/*}}{e_{t-1}^{/*}} b_{t}^{h \to F} + (1 - \tau_t^H) p_t^{h} y_t^{h} + t_t^H \Bigr) \\
& \quad - \Bigl( \sum_{j' \in J} q_{t, t+1}(j') d_{t+1}^{h \to H}(j') + b_{t+1}^{h \to F} + p_t^{H \to W} c_t^{h \to W} \Bigr) \biggr) \Biggr\} \Biggr]
\end{aligned}
\label{eq:lagrangian_intertemporal_households}
\end{equation}

このラグランジアンから以下の主要な一階の条件(FOC)が導かれる。
導出の厳密な過程(状態 \( j \) に依存した変数での偏微分、反復期待値の法則の適用、係数除去の論理)は付録Aに譲る。

\paragraph{自国家計のFOC}
総消費指数 \( c_t^{h \to W} \) に関するFOCから時点 \( t \) における
所得の限界効用 \( \lambda_t^h \) は以下のように決定される。
\begin{equation}
\lambda_t^h = \frac{1}{p_t^{H \to W} c_t^{h \to W}}
\label{eq:foc_consumption_home}
\end{equation}

国内コンティンジェント債券 \( d_{t+1}^{h \to H} \) に関するFOCを集計することで
国内の異時点間の最適化条件(オイラー方程式)が得られる。
\begin{equation}
\lambda_t^h = \beta_t^H (1+i_t^H) \operatorname{E}_t [ \lambda_{t+1}^h ]
\label{eq:euler_domestic_home}
\end{equation}
ここで \( i_t^H \) は国内の無リスク名目金利である。

国際リスクフリー債券 \( b_{t+1}^{h \to F*} \) に関するFOCから
国際的な資産選択の条件であるUIP条件が導かれる。
\begin{equation}
\lambda_t^h e_t^{/*} = (1+i_t^F) \beta_t^H \operatorname{E}_t \left[ \lambda_{t+1}^h e_{t+1}^{/*} \right]
\label{eq:uip_condition_home}
\end{equation}
ここで \( i_t^F \) は外国の無リスク名目金利である。

\paragraph{外国家計のFOC}
同様に外国家計 \( f \) のラグランジアン(簡略化版)から以下の一階の条件が導かれる。
\begin{equation}
\begin{aligned}
\mathcal{L}^f = & \operatorname{E}_s \Biggl[ \sum_{t=s}^{\infty} \left( \prod_{k=0}^{t-s-1} \beta_{s+k}^F \right) \Biggl\{ \left( \log c_t^{f \to W} - \frac{\phi^F}{2}(l_t^f)^2 \right) \\
& \quad + \lambda_t^{f/*} \biggl( \Bigl( d_{t}^{f \to F*} + (1+i_{t-1}^{H}) \frac{e_{t-1}^{/*}}{e_t^{/*}} b_{t}^{f \to H*} + (1 - \tau_t^F) p_t^{f*} y_t^f + t_t^{F*} \Bigr) \\
& \quad - \Bigl( \sum_{j' \in J} q_{t, t+1}^*(j') d_{t+1}^{f \to F*}(j') + b_{t+1}^{f \to H*} + p_t^{F \to W*} c_t^{f \to W} \Bigr) \biggr) \Biggr\} \Biggr]
\end{aligned}
\label{eq:lagrangian_intertemporal_foreign_households}
\end{equation}

\begin{equation}
\lambda_t^{f/*} = \frac{1}{p_t^{F \to W*} c_t^{f \to W}}
\label{eq:foc_consumption_foreign}
\end{equation}

\begin{equation}
\lambda_t^{f/*} = \beta_t^F (1+i_t^F) \operatorname{E}_t [ \lambda_{t+1}^{f/*} ]
\label{eq:euler_foreign}
\end{equation}

\begin{equation}
\frac{\lambda_t^{f/*}}{e_t^{/*}} = (1+i_t^H) \beta_t^F \operatorname{E}_t \left[ \lambda_{t+1}^{f/*} \frac{1}{e_{t+1}^{/*}} \right]
\label{eq:uip_condition_foreign}
\end{equation}
% sections/chap3/sec_model_risk_sharing.tex

\section{国内完備市場とリスク共有}
\label{sec:model_risk_sharing}

前節では家計 \( h \) の最適化問題を定式化した。
本節ではモデルの仮定である国内完備市場が家計間のリスク共有にどのような含意をもたらすかを説明する。
これによりなぜ所得の限界効用と消費が家計間で共通化されるのかを明らかにする。

\subsection{所得の限界効用 \(\lambda_t^h\) の共通化}
本モデルでは自国と外国のそれぞれにおいて国内の債券市場は完備であると仮定しており
家計はあらゆる不確実性に対応した国内コンティンジェント債券 \( d_{t+1}^{h \to H} \) を取引できる。

任意の2つの家計 \( h \) と \( h' \) を考える。
両者ともすべての時点・すべての状態で最適化行動をとるため、
国内コンティンジェント債券の購入に関する一階の条件(FOC)は常に成立する。
\begin{equation}
\lambda_t^h q_{t, t+1} = \beta_t^H \pi \lambda_{t+1}^h \quad , \quad \lambda_t^{h'} q_{t, t+1} = \beta_t^H \pi \lambda_{t+1}^{h'}
\label{eq:foc_contingent_bonds_individual}
\end{equation}
ここで \( q_{t, t+1} \) は状態コンティンジェント債の価格であり \( \pi \) は状態の発生確率である。
これらの比をとることで、家計間の限界効用の比率が時間や状態に依存しない定数 \( k \) となることがわかる。
\begin{equation}
\frac{\lambda_t^h}{\lambda_t^{h'}} = \frac{\lambda_{t+1}^h}{\lambda_{t+1}^{h'}} = k
\label{eq:marginal_utility_ratio_constant}
\end{equation}

この定数 \( k \) の値を特定するためショックの発生時点 \( s \) に注目する。
すべての家計が同一の効用関数をもち、かつ第\ref{sec:model_steady_state}節で仮定したとおり
 $d_s^h = b_s^h = 0$ であることから、
ショックの発生時点 \( s \) において各家計が直面する最適化問題は数学的に完全に同一となる。
したがってその解として得られる所得の限界効用もすべての家計で一致しなければならない
(\( \lambda_s^h = \lambda_s^{h'} \))。

この条件を式 \eqref{eq:marginal_utility_ratio_constant} に適用することで \( k = 1 \) が特定される。
一度 \( k = 1 \) が確定すれば、国内債券市場が完備であるかぎり、将来にわたっていかなるショックが発生しても
すべての家計の所得の限界効用は常に一致し続けることが結論付けられる。
\begin{equation}
\lambda_t^h = \lambda_t^{h'} \quad (\text{for all } h, h' \in H, \text{ all } t)
\label{eq:marginal_utility_commonality}
\end{equation}

\subsection{総消費指数 \(c_t^{h \to W}\) の共通化}
所得の限界効用がすべての家計で共通であるという上記の結果を前節で導出した消費に関する
一階の条件 \eqref{eq:foc_consumption_home} に適用する。
\begin{equation}
\lambda_t^h = \frac{1}{p_t^{H \to W} c_t^{h \to W}} \tag{\ref{eq:foc_consumption_home}}
\end{equation}
この式において左辺の \(\lambda_t^h\) は今や全家計で共通の値をとる。
また右辺の消費者物価指数 (CPI) \( p_t^{H \to W} \) もマクロ変数であるためすべての家計にとって共通である。

したがってこの等式がすべての家計 \( h \) について成立するためには、総消費指数 \( c_t^{h \to W} \) もまた
すべての家計間で完全に一致しなければならない。
\begin{equation}
c_t^{h \to W} = c_t^{h' \to W} \quad (\text{for all } h, h' \in H, \text{ all } t)
\label{eq:consumption_commonality}
\end{equation}
この結果、すべての家計の総消費指数は同一の値をとることが示された。
カルボ型価格設定の価格改定の可否によって各家計の生産所得は同一とならない。
しかし国内完備市場の仮定により家計はコンティンジェント債券の取引を通じて完全に保険しあうことができ
その結果として各家計は同一の可処分所得を得ることとなる。
このことから
すべての家計の所得の限界効用 \(\lambda_t^h\)と総消費指数 \( c_t^{h \to W} \)  が一致するのである。
一方で留意点として、この共通の消費を実現するために
各家計の貯蓄行動(資産ポートフォリオ \( d_{t+1}^h, b_{t+1}^h \) の構成)は
個別の所得ショックに応じて異質的に変動する。
つまり消費は共通化されるが貯蓄や資産保有額の経路は家計ごとに異なる。
% !TeX root = ../../main.tex
% sections/chap3/sec_model_producers.tex

\section{生産者}
\label{sec:model_producers}
本モデルはヨーマン・ファーマ・モデルのため家計は生産者でもある。
本節では生産と価格設定の側面を記述する。
まず \( t \) 期における個別財の生産 \( y_t^h \) と労働 \( l_t^h \) の関係、
および個別財の市場均衡について記述する。

\paragraph{個別生産関数}
家計 \( h \) が生産する個別財の生産 \( y_t^h \) は
国全体の生産性 \( a_t^H \) と個人の労働 \( l_t^h \) を用いて以下のように定義される。
\begin{equation}
y_t^h = a_t^H l_t^h
\label{eq:individual_production_function}
\end{equation}
同様に外国の家計 \( f \) の個別生産関数は以下のように定義される。
\begin{equation}
y_t^f = a_t^F l_t^f
\label{eq:individual_production_function_foreign}
\end{equation}

\paragraph{個別財の市場均衡}
家計 \( h \) が生産する財 \( y_t^h \) への総需要は
自国家計 \( h' \) からの需要と外国家計 \( f \) からの需要の合計である。
財市場均衡では、この総需要が生産と一致する。
\begin{equation}
y_t^h = \sum_{h' \in H} c_t^{h' \to h} + \sum_{f \in F} c_t^{f \to h}
\label{eq:individual_goods_market_equilibrium}
\end{equation}
この均衡条件に個別の需要関数 \eqref{eq:demand_h_h_prime} と \eqref{eq:demand_f_h_prime} を代入し
共通項 \( (p_t^h/p_t^H)^{-\theta^H} \) で括ると個別生産 \( y_t^h \) は以下のように表される。
\begin{equation}
y_t^h = \left( \frac{p_t^h}{p_t^H} \right)^{-\theta^H} \left[ \sum_{h' \in H} c_t^{h' \to H} + \sum_{f \in F} c_t^{f \to H} \right]
\label{eq:demand_relation_yh_world_demand}
\end{equation}
ここで大括弧の中は世界全体からの自国消費指数への総需要を表す。

同様に外国財 \( f \) への需要にもとづき個別生産は以下のように表される。
\begin{equation}
y_t^f = \left( \frac{p_t^{f*}}{p_t^{F*}} \right)^{-\theta^F} \left[ \sum_{h \in H} c_t^{h \to F} + \sum_{f' \in F} c_t^{f' \to F} \right]
\label{eq:demand_relation_yf_world_demand}
\end{equation}

\subsection{生産関数と価格分散}
\label{subsec:model_production_dispersion}
本節では3.3節の冒頭で記述した個別レベルの関係式を集計しマクロレベルの生産関数を導出する。

\paragraph{マクロ生産関数の導出と集計量}
国全体の総労働 \( L_t^H \equiv \sum_{h \in H} l_t^h \) の定義から出発する。
\begin{enumerate}
    \item 個別生産関数 \eqref{eq:individual_production_function} を \( l_t^h \) について解き
    総労働の定義に代入する。
    \[
    L_t^H = \sum_{h \in H} \frac{y_t^h}{a_t^H} = \frac{1}{a_t^H} \sum_{h \in H} y_t^h
    \]
    \item 個別財の均衡式 \eqref{eq:demand_relation_yh_world_demand} を \( \sum y_t^h \) に代入する。
    \begin{align*}
    L_t^H &= \frac{1}{a_t^H} \sum_{h \in H} \left\{ \left( \frac{p_t^h}{p_t^H} \right)^{-\theta^H} \left[ \sum_{h' \in H} c_t^{h' \to H} + \sum_{f \in F} c_t^{f \to H} \right] \right\} \\
    &= \frac{1}{a_t^H} \left[ \sum_{h' \in H} c_t^{h' \to H} + \sum_{f \in F} c_t^{f \to H} \right] \sum_{h \in H} \left( \frac{p_t^h}{p_t^H} \right)^{-\theta^H}
    \end{align*}
    \item 価格の非効率性を表す価格分散項 \( \Delta_t^H \) を以下のように定義する。
    \begin{equation}
    \Delta_t^H \equiv \sum_{h \in H} \left(\frac{p_t^h}{p_t^H}\right)^{-\theta^H} \label{eq:dispersion_definition_home}
    \end{equation}
    \item これにより、総労働 \( L_t^H \)、世界総需要 \( [\sum c + \sum c] \)、
    価格分散 \( \Delta_t^H \) のあいだに以下の関係式が導かれる。
    \begin{equation}
    L_t^H = \frac{1}{a_t^H} \left[ \sum_{h' \in H} c_t^{h' \to H} + \sum_{f \in F} c_t^{f \to H} \right] \Delta_t^H
    \label{eq:labor_demand_dispersion_relation}
    \end{equation}
\end{enumerate}
ここでマクロの集計生産指数 \( Y_t^H \) を用いて
\begin{equation}
Y_t^H = \frac{a_t^H L_t^H}{\Delta_t^H} \label{eq:aggregate_production_function_home}
\end{equation}
という簡潔なマクロ生産関数を書きたいという動機を考える。
この目標とする式 \eqref{eq:aggregate_production_function_home} を \( L_t^H \) について解くと
 \( L_t^H = (Y_t^H / a_t^H) \Delta_t^H \) となる。
この式とステップ4で導出した関係式 \eqref{eq:labor_demand_dispersion_relation} を比較すると、
このマクロ生産関数が成立するためには集計生産指数 \( Y_t^H \) が必然的に
世界全体からの国内消費指数への総需要と一致しなければならないことがわかる。
\begin{equation}
Y_t^H = \sum_{h' \in H} c_t^{h' \to H} + \sum_{f \in F} c_t^{f \to H}
\label{eq:aggregate_goods_market_equilibrium}
\end{equation}
この集計レベルの財市場均衡が成立するとき
個別財の均衡式 \eqref{eq:demand_relation_yh_world_demand} は
\begin{equation}
y_t^h = \left( \frac{p_t^h}{p_t^H} \right)^{-\theta^H} Y_t^H \label{eq:demand_relation_yh_YH}
\end{equation}
と書き換えられる。
さらにこの関係式 \eqref{eq:demand_relation_yh_YH} と
価格指数 \( p_t^H \) の定義 \eqref{eq:define_pH} を用いると(詳細は付録参照)
集計生産指数 \( Y_t^H \) は個々の生産 \( y_t^h \) の
CES集計量として定義されなければならないことが数学的に示される。
\begin{equation}
Y_t^H \equiv \left[ \sum_{h \in H} (y_t^h)^{\frac{\theta^H-1}{\theta^H}} \right]^{\frac{\theta^H}{\theta^H-1}} \label{eq:CES_agg_Yh}
\end{equation}
この一連の導出により
マクロ生産関数 \eqref{eq:aggregate_production_function_home} を導くという動機から出発し、
そのために必要な集計レベルの財市場均衡 \eqref{eq:aggregate_goods_market_equilibrium} が論理的に要請され、
その結果として \( Y_t^H \) のCES集計という定義 \eqref{eq:CES_agg_Yh} が
モデルの内部整合性を保つために必然となる、という関係性が明らかになった。

\paragraph{外国の生産関数}
同様の論理展開により外国の生産関数は以下のように記述される。
\begin{gather}
Y_t^F = \frac{a_t^F L_t^F}{\Delta_t^F} \label{eq:aggregate_production_function_foreign} \\
Y_t^F \equiv \left[ \sum_{f \in F} (y_t^f)^{\frac{\theta^F-1}{\theta^F}} \right]^{\frac{\theta^F}{\theta^F-1}} \label{eq:CES_agg_Yf} \\
L_t^F \equiv \sum_{f \in F} l_t^f \label{eq:total_labor_foreign_def} \\
\Delta_t^F \equiv \sum_{f \in F} \left(\frac{p_t^{f*}}{p_t^{F*}}\right)^{-\theta^F} \label{eq:dispersion_definition_foreign}
\end{gather}
また集計レベルの財市場均衡と個別財と集計財の関係は以下の通りである。
\begin{gather}
Y_t^F = \sum_{h \in H} c_t^{h \to F} + \sum_{f' \in F} c_t^{f' \to F} \label{eq:aggregate_goods_market_equilibrium_foreign} \\
y_t^f = \left( \frac{p_t^{f*}}{p_t^{F*}} \right)^{-\theta^F} Y_t^F \label{eq:demand_relation_yf_YF}
\end{gather}

\subsection{価格設定の動学}
\label{subsec:model_price_setting}
財の価格設定は \textcite{Calvo1983} に従う。
すなわち毎期 \( 1-\xi^H \) の割合の家計のみが価格を改定することができ、
残りの \( \xi^H \) の割合の家計は価格を据え置く。
価格を改定する家計は将来にわたって価格を据え置く可能性を考慮し、
期待効用の割引現在価値を最大化する単一の最適価格 \( p_t^h \) を選択する。

\paragraph{自国の価格設定}
この価格決定問題は 3.3.1節で述べた効用最大化問題のうち
価格改定家計 \( h \) が時点 \( t \) で選択する価格 \( p_t^h \) に依存する項のみを抜き出し
その期待割引現在価値を最大化する問題として定式化される。
カルボ型価格設定の仮定の下では時点 \( t \) で設定した価格 \( p_t^h \) が \( k \) 期先まで維持される確率は
 \( (\xi^H)^k \) となる。
価格改定家計が最大化する期待効用は付録に示すように以下のように書ける。
\begin{equation}
\begin{aligned}
\max_{p_t^h} \operatorname{E}_t \sum_{k=0}^{\infty} (\xi^H)^k \left(\prod_{i=0}^{k-1} \beta_{t+i}^H \right) \Biggl[ & \left( \log c_{t+k}^{h \to W} - \frac{\phi^H}{2}(l_{t+k}^{h})^2 \right) \\
& + \lambda_{t+k}^{h} \biggl( \Bigl( d_{t+k}^{h \to H} + (1+i_{t+k-1}^{F}) b_{t+k}^{h \to F} + (1 - \tau_{t+k}^{H}) p_t^h y_{t+k}^{h} + t_{t+k}^{H} \Bigr) \\
& \quad - \Bigl( \sum_{j' \in J} q_{t+k, t+k+1} d_{t+k+1}^{h \to H} + b_{t+k+1}^{h \to F} + p_{t+k}^{H \to W} c_{t+k}^{h \to W} \Bigr) \biggr) \Biggr]
\end{aligned}
\label{eq:profit_max_problem_home}
\end{equation}
この目的関数を \( p_t^h \) について偏微分し1階の条件を求める際に以下の点を考慮する。
\begin{itemize}
    \item 個別家計 \( h \) の価格 \( p_t^h \) がマクロ変数(\( i_{t+k}^F, e_{t+k}, \tau_{t+k}^H, t_{t+k}^H, q_{t+k, t+k+1}, p_{t+k}^{H \to W} \))に与える影響はごく小さいため、ないものと仮定する。
    \item 価格 \( p_t^h \) が所得を通じて消費 \( c_{t+k}^{h \to W} \) と各種債権保有 \( d_{t+k+1}^{h \to H}, b_{t+k+1}^{h \to F} \) に与える影響は、予算制約として織り込まれている。そのため 消費 \( c_{t+k}^{h \to W} \) と各種債権保有 \( d_{t+k+1}^{h \to H}, b_{t+k+1}^{h \to F} \) は \( p_t^h \) から独立しているものとして扱えばよい。
    \item 一方で、制約 \( l_{t+k}^h = y_{t+k}^h / a_{t+k}^H \) と需要関数 \( y_{t+k}^h = (p_t^h / p_{t+k}^H)^{-\theta^H} Y_{t+k}^H \) (式 \eqref{eq:demand_relation_yh_YH} 参照) は、効用関数内の労働 \( l_{t+k}^h \) と予算制約内の生産 \( y_{t+k}^h \) に反映される。これにより \( l_{t+k}^h \) と \( y_{t+k}^h \) は \( p_t^h \) の関数となり、微分対象になる。
\end{itemize}
これらの仮定のもとで1階の条件を計算すると効用関数の \( \log c_{t+k}^{h \to W} \) の項と、
予算制約の大部分の項(債券、移転、消費支出)の微分がゼロとなり、
労働の非効用と税引き後所得の項のみが残る。
結果として最適化問題の1階の条件は以下のように得られる。
\begin{equation}
\operatorname{E}_t \sum_{k=0}^{\infty} (\xi^H)^k \left(\prod_{i=0}^{k-1} \beta_{t+i}^H \right) \left[ \frac{\partial}{\partial p_t^h} \left\{ \left( - \frac{\phi^H}{2}(l_{t+k}^h)^2 \right) + \lambda_{t+k}^h \left( (1 - \tau_{t+k}^H) p_t^h y_{t+k}^h \right) \right\} \right] = 0
\label{eq:foc_price_setting_home}
\end{equation}
この式は価格 \( p_t^h \) の微小な変化がそれが維持される可能性のある将来の各期において
「労働の非効用(限界費用に対応)」と「税引き後限界収入」に与える影響の、
期待割引現在価値の合計がゼロになる点に最適価格があることを示している。(偏微分の詳細な計算は付録参照。)

この1階の条件を計算すると以下の式が得られる(詳細は付録参照)。
\begin{equation}
(p_t^h)^{1+\theta^H} = \frac{\theta^H}{\theta^H-1} \frac{ \operatorname{E}_t \sum_{k=0}^{\infty} (\xi^H)^k \left(\prod_{j=0}^{k-1} \beta_{t+j}^H \right) \left[ \phi^H \frac{(Y_{t+k}^H)^2}{(a_{t+k}^H)^2} (p_{t+k}^H)^{2\theta^H} \right] }{ \operatorname{E}_t \sum_{k=0}^{\infty} (\xi^H)^k \left(\prod_{j=0}^{k-1} \beta_{t+j}^H \right) \left[ \lambda_{t+k}^h (1 - \tau_{t+k}^{H}) Y_{t+k}^H (p_{t+k}^H)^{\theta^H} \right] }
\label{eq:optimal_price_explicit_home}
\end{equation}
この式の右辺に出てくる全ての変数は \( t \) 期が所与のもとで
価格改定を行う全ての家計 \( h \) ににとって共通である。
特に \( \lambda_{t+k}^h \) は国内完備市場の仮定により
すべての家計で同一になることが保証されている(3.4節および付録参照)。
したがってこの方程式の解である \( p_t^h \) もすべての価格改定を行う家計 \( h \) にとって完全に同一の値となる。

しかし無限和を含む方程式は数値計算に使用できない。
そこで分子を \( v_t \)、分母を \( w_t \) とおき、
これら \( v_t, w_t \) を再帰的な形に書き直すことで無限和を含む元の方程式を以下の連立方程式群に分割する。
\begin{align}
(p_t^h)^{1+\theta^H} &= \frac{\theta^H}{\theta^H-1} \frac{v_t}{w_t} \label{eq:optimal_price_home} \\
v_t &= \phi^H \frac{(Y_t^H)^2}{(a_t^H)^2} (p_t^H)^{2\theta^H} + \beta_t^H\xi^H \operatorname{E}_t[v_{t+1}] \label{eq:v_recursive_home} \\
w_t &= \lambda_t^h (1 - \tau_t^{H}) Y_t^H (p_t^H)^{\theta^H} + \beta_t^H\xi^H \operatorname{E}_t[w_{t+1}] \label{eq:w_recursive_home}
\end{align}
これらの方程式がニュー・ケインジアン・フィリップス曲線の非線形な表現となる。

\paragraph{外国の価格設定}
同様に外国の価格設定は以下の連立方程式群で記述される。
外国家計 \( f \) が価格 \( p_t^{f*} \) を選択する際の最大化問題は以下のように書ける。
\begin{equation}
\begin{aligned}
\max_{p_t^{f*}} \operatorname{E}_t \sum_{k=0}^{\infty} (\xi^F)^k \left(\prod_{i=0}^{k-1} \beta_{t+i}^F \right) \Biggl[ & \left( \log c_{t+k}^{f \to W} - \frac{\phi^F}{2}(l_{t+k}^{f})^2 \right) \\
& + \lambda_{t+k}^{f/*} \biggl( \Bigl( d_{t+k}^{f \to F*} + (1+i_{t+k-1}^{H}) b_{t+k}^{f \to H*} + (1 - \tau_{t+k}^{F}) p_t^{f*} y_{t+k}^{f} + t_{t+k}^{F*} \Bigr) \\
& \quad - \Bigl( \sum_{j' \in J} q_{t+k, t+k+1}^* d_{t+k+1}^{f \to F*} + b_{t+k+1}^{f \to H*} + p_{t+k}^{F \to W*} c_{t+k}^{f \to W} \Bigr) \biggr) \Biggr]
\end{aligned}
\label{eq:profit_max_problem_foreign}
\end{equation}
これより導かれる最適価格 \( p_t^{f*} \) は、以下の再帰的な方程式群によって決定される。
\begin{align}
(p_t^{f*})^{1+\theta^F} &= \frac{\theta^F}{\theta^F-1} \frac{v_t^F}{w_t^F} \label{eq:optimal_price_foreign} \\
v_t^F &= \phi^F \frac{(Y_t^F)^2}{(a_t^F)^2} (p_t^{F*})^{2\theta^F} + \beta_t^F\xi^F \operatorname{E}_t[v_{t+1}^F] \label{eq:v_recursive_foreign} \\
w_t^F &= \lambda_t^{f/*} (1 - \tau_t^{F}) Y_t^F (p_t^{F*})^{\theta^F} + \beta_t^F\xi^F \operatorname{E}_t[w_{t+1}^F] \label{eq:w_recursive_foreign}
\end{align}
% !TeX root = ../../main.tex
% sections/chap3/sec_model_policies.tex

\section{政府と金融政策}
\label{sec:model_policies} % labelを微調整

\subsection{財政政策}
\label{subsec:model_fiscal_policy} % labelを微調整
\paragraph{自国の財政政策}
政府は均衡予算を達成し税収のすべてを家計への一括移転 \( t_t^H \) として還元する。
所得税率 \( \tau_t^H \) はその定常状態の値 \( \tau_{ss}^H \) の対数周囲での外生的な確率過程に従う。
\begin{equation}
\log(\tau_t^H) = (1-\rho_{\tau}^H)\log(\tau_{ss}^H) + \rho_{\tau}^H \log(\tau_{t-1}^H) + \varepsilon_t^{\tau,H}
\label{eq:tax_rule_H}
\end{equation}
政府の予算制約式は \( N t_t^H = \sum_{h \in H} \tau_t^H p_t^h y_t^h \) である。
ここで付録A.4(定理4)において証明されるように総名目所得は以下の関係式で表すことができる。
\begin{equation}
\sum_{h \in H} p_t^h y_t^h = p_t^H Y_t^H
\label{eq:nominal_income_identity}
\end{equation}
これらより集計レベルの予算制約式は以下のように決定される。
\begin{equation}
N t_t^H = \tau_t^H p_t^H Y_t^H
\label{eq:final_gov_budget_agg}
\end{equation}
これより一人当たり移転額は以下のように決定される。
\begin{equation}
t_t^H = \tau_t^H \frac{p_t^H Y_t^H}{N}
\label{eq:transfer_H}
\end{equation}

\paragraph{外国の財政政策}
同様に外国政府の税率 \( \tau_t^F \) は外生的な確率過程に従い、一人当たり移転額は以下のように決定される。
\begin{equation}
t_t^{F*} = \tau_t^F \frac{p_t^{F*} Y_t^F}{M}
\label{eq:transfer_F}
\end{equation}
\begin{equation}
\log(\tau_t^F) = (1-\rho_{\tau}^F)\log(\tau_{ss}^{F*}) + \rho_{\tau}^F \log(\tau_{t-1}^F) + \varepsilon_t^{\tau,F}
\label{eq:tax_rule_F}
\end{equation}

\subsection{金融政策}
\label{subsec:model_monetary_policy} % labelを微調整
中央銀行は利子率を \( i_t^H \) を操作する。
本稿ではシミュレーションにおいて自国が従う以下の3つの政策ルールを比較分析する。
すべての変数は一人当たりの変数で記述される。

本稿の分析においてハット( \( \hat{} \) )付き変数はその変数の対数をとったものと
その変数の定常状態での対数値との乖離(対数乖離)を表す
(例:\( \hat{y}_t^H = \log(y_t^H) - \log(y_{ss}^H) \))。
この対数乖離は、乖離が微小である場合に限り
その変化率( \( (y_t^H - y_{ss}^H) / y_{ss}^H \) )の良い近似となるため広く用いられる。

特にグロス・インフレ率(\( \pi_t^H \equiv \bar{p}_t^H / \bar{p}_{t-1}^H \))の
(対数)乖離 \( \hat{\pi}_t^H \) は、\( \log(\pi_t^H) - \log(\pi_{ss}^H) \) で定義される。
定常状態においてはネット・インフレ率は 0\%(すなわちグロス・インフレ率 \( \pi_{ss}^H = 1 \))となるため
\begin{align*}
\hat{\pi}_t^H &\equiv \log(\pi_t^H) - \log(\pi_{ss}^H) && \text{(ハットの定義)} \\
&= \log(\pi_t^H) - \log(1) && \text{(グロス・インフレ率 \(\pi_{ss}^H = 1\) を代入)} \\
&= \log(\pi_t^H) && \text{(\(\log(1) = 0\) のため)}
\end{align*}
と計算される。
最後にグロス・インフレ率の定義と対数の性質(\(\log(A/B) = \log(A) - \log(B)\))を用いると、
インフレ・ギャップの式が導出される。
\begin{equation}
\hat{\pi}_t^H = \log\left(\frac{\bar{p}_t^H}{\bar{p}_{t-1}^H}\right) = \log(\bar{p}_t^H) - \log(\bar{p}_{t-1}^H)
\label{eq:pi_hat_dynamics_def}
\end{equation}

\paragraph{1. インフレ目標 (IT)}
利子率はグロス・インフレ率(\( \pi_t^H \))の定常状態からの乖離に反応する。
\begin{equation}
i_t^H = i_{ss}^H + \phi_{\pi}^H (\pi_t^H - \pi_{ss}^H) + \varepsilon_{t}^{i,H}
\label{eq:mp_it_H}
\end{equation}

\paragraph{2. 名目GDP水準目標 (NGDPLT)}
利子率は一人当たりの名目GDP \( \bar{p}_t^H y_t^H \) のあらかじめ定められた目標パス \( \chi_t^H \) からの
対数乖離(\( \log(\bar{p}_t^H y_t^H) - \log(\chi_t^H) \))に反応し、
また過去の乖離の累積 \( \gamma_t^H \) にも反応する。
ここで \( i_{ss}^H \) は利子率の定常状態の値である。
\begin{align}
i_t^H &= i_{ss}^H + \phi_{gap} ( \log(\bar{p}_t^H y_t^H) - \log(\chi_t^H) ) + \phi_{level} \gamma_t^H \label{eq:mp_ngdplt_1} \\
\gamma_t^H &= \gamma_{t-1}^H + ( \log(\bar{p}_{t-1}^H y_{t-1}^H) - \log(\chi_{t-1}^H) ) \label{eq:mp_ngdplt_2}
\end{align}
目標パス \( \chi_t^H \) はその定常状態の値 \( \chi_{ss}^H \) の対数周囲での外生的な確率過程に従う。
\begin{equation}
\log(\chi_t^H) = (1-\rho_{\chi}^H)\log(\chi_{ss}^H) + \rho_{\chi}^H \log(\chi_{t-1}^H) + \varepsilon_{t}^{\chi,H}
\label{eq:mp_ngdplt_3}
\end{equation}

\paragraph{3. 名目消費水準目標 (NCLT)}
利子率は一人当たりの名目総消費 \( p_t^{H \to W} c_t^{H \to W} \) の
目標パス \( \chi_t^H \) からの対数乖離(\( \log(p_t^{H \to W} c_t^{H \to W}) - \log(\chi_t^H) \))
に反応する。
\begin{align}
i_t^H &= i_{ss}^H + \phi_{gap} ( \log(p_t^{H \to W} c_t^{H \to W}) - \log(\chi_t^H) ) + \phi_{level} \gamma_t^H \label{eq:mp_nclt_1} \\
\gamma_t^H &= \gamma_{t-1}^H + ( \log(p_{t-1}^{H \to W} c_{t-1}^{H \to W}) - \log(\chi_{t-1}^H) ) \label{eq:mp_nclt_2}
\end{align}
目標パス \( \chi_t^H \) は名目GDP水準目標と同様の確率過程に従う。
\begin{equation}
\log(\chi_t^H) = (1-\rho_{\chi}^H)\log(\chi_{ss}^H) + \rho_{\chi}^H \log(\chi_{t-1}^H) + \varepsilon_{t}^{\chi,H}
\label{eq:mp_nclt_3}
\end{equation}

\paragraph{外国の金融政策}
外国の中央銀行は本稿の分析を通じて正規化された生産者物価指数を用いたインフレ目標(IT PPI)に従うものと仮定する。
シミュレーションで比較するのは自国の政策ルールのみである。
外国の利子率 \( i_t^F \) は、以下の形式に従う。
\begin{equation}
i_t^F = i_{ss}^F + \phi_{\pi}^F (\pi_t^{F*} - \pi_{ss}^{F*}) + \varepsilon_{t}^{i,F}
\label{eq:mp_it_F}
\end{equation}
ここで \( \pi_t^{F*} \) は外国のグロス・インフレ率であり \( i_{ss}^F \) と \( \pi_{ss}^{F*} \) は
それぞれの定常状態の値である。
% !TeX root = ../../main.tex
% sections/chap3/sec_model_resource_constraints.tex

\section{国全体の資源制約式}
\label{sec:model_resource_constraints} % labelを微調整

\paragraph{自国の資源制約}
任意の \( t \) 期において成立するすべての自国家計(\( h \in H \))の予算制約式を足し合わせ、
自国の債券市場と政府部門の均衡条件を適用することで国全体の集計的な資源制約式が得られる
(導出の詳細は付録A.7参照)。
\begin{equation}
p_t^{H \to W} C_t^{H \to W} + B_{t+1}^{H} = p_t^H Y_t^H + (1+i_{t-1}^F) \frac{e_t^{/*}}{e_{t-1}^{/*}} B_t^{H}
\label{eq:aggregate_resource_constraint_H}
\end{equation}
ここで
\begin{equation}
C_t^{H \to W} = N c_t^{H \to W}
\label{eq:aggregate_consumption_def_H}
\end{equation}
は国全体の総消費、
\begin{equation}
B_t^{H} = N b_t^{H}
\label{eq:aggregate_assets_def_H}
\end{equation}
は国全体の対外純資産(自国通貨建て)である。
この式は国の消費と対外純資産の増加が国の生産と既存の対外資産からの収益によって賄われることを示している。

\paragraph{外国の資源制約(外国通貨建て)}
同様に外国の資源制約式は以下のように記述される。
\begin{equation}
p_t^{F \to W*} C_t^{F \to W} + B_{t+1}^{F*} = p_t^{F*} Y_t^F + (1+i_{t-1}^H) \frac{e_{t-1}^{/*}}{e_t^{/*}} B_t^{F*}
\label{eq:aggregate_resource_constraint_F}
\end{equation}
ここで
\begin{equation}
C_t^{F \to W} = M c_t^{F \to W}
\label{eq:aggregate_consumption_def_F}
\end{equation}
は外国の総消費、
\begin{equation}
B_t^{F*} = M b_t^{F*}
\label{eq:aggregate_assets_def_F}
\end{equation}
は外国の対外純資産(外国通貨建て)である。
% !TeX root = ../../main.tex
% sections/chap3/sec_model_final_equations.tex

\section{代表的家計モデルの最終的な方程式体系}
\label{sec:model_final_equations}
本章で導出された個々の最適化条件と均衡条件を集約し、シミュレーション分析で用いる代表的家計モデルの最終的な方程式体系を以下にまとめる。

本節では、家計の最適化条件(FOC)をもとに、代表的家計の方程式を導出する。

\subsection{市場均衡}
\subsubsection{財市場の均衡}
\paragraph{① 出発点:集計レベルの需給均衡}
3.5.1節 \eqref{eq:aggregate_goods_market_equilibrium} で示された通り、自国の総生産指数 \( Y_t^H \) は、国内からの総需要と、外国からの総需要の合計に等しくなる。
\begin{equation}
Y_t^H = \sum_{h \in H} c_t^{h \to H} + \sum_{f \in F} c_t^{f \to H} \tag{\ref{eq:aggregate_goods_market_equilibrium}}
\end{equation}

\paragraph{② 代表的家計モデルの方程式}
\( c_t^{h \to H} = c_t^{H \to H} \) と \( c_t^{f \to H} = c_t^{F \to H} \)より①式の総和は以下のように書き換えられる。
\begin{equation*}
Y_t^H = N c_t^{H \to H} + M c_t^{F \to H}
\end{equation*}
この式の両辺を自国の人口 \( N \) で割り、自国の一人当たり生産指数 \( y_t^H \equiv Y_t^H / N \) を用いると、一人当たりの均衡式が得られる。
\begin{equation}
y_t^H = c_t^{H \to H} + \frac{M}{N} c_t^{F \to H}
\label{eq:final_goods_market_eq_H}
\end{equation}
同様に、外国の一人当たり生産指数 \( y_t^F \equiv Y_t^F / M \) についても、以下の均衡式が得られる。
\begin{equation}
y_t^F = c_t^{F \to F} + \frac{N}{M} c_t^{H \to F}
\label{eq:final_goods_market_eq_F}
\end{equation}

\subsection{家計の最適化行動}
\subsubsection{所得の限界効用(消費のFOC)}
\paragraph{① 出発点:家計のFOC}
消費に関するFOCは、家計の所得の限界効用 \( \lambda_t^h \) を用いて次のように書ける。(3.3.2.3節 \eqref{eq:foc_consumption_home} 再掲)
\begin{equation}
\lambda_t^h = \frac{1}{p_t^{H \to W} c_t^{h \to W}} \tag{\ref{eq:foc_consumption_home}}
\end{equation}

\paragraph{② 代表的家計モデルの方程式}
3.4節で示された通り、全ての家計 \(h\) で \( \lambda_t^h \) と \( c_t^{h \to W} \) は共通の値をとる。そこで、代表的家計の変数を以下のように定義する。
\begin{equation}
\lambda_t^H \equiv \lambda_t^h \quad (\text{※3.4節より全家計で共通のため})
\label{eq:ra_lambda_H_final_def}
\end{equation}
\begin{equation}
c_t^{H \to W} \equiv c_t^{h \to W} \quad (\text{※3.4節より全家計で共通のため})
\label{eq:ra_consumption_W_final_def}
\end{equation}
これらを①式に代入すると、次式が得られる。
\begin{equation}
\lambda_t^H = \frac{1}{p_t^{H \to W} c_t^{H \to W}}
\label{eq:final_lambda_H}
\end{equation}
外国についても同様に、代表的家計の所得の限界効用 \( \lambda_t^{F/*} \) が定義される。
\begin{equation}
\lambda_t^{F/*} = \frac{1}{p_t^{F \to W*} c_t^{F \to W}}
\label{eq:final_lambda_F}
\end{equation}

\subsubsection{オイラー方程式(国内債券)}
\paragraph{① 出発点:家計のFOC}
個々の家計 \( h \) の国内債券保有に関する最適化の一階条件(FOC)は、次のように表現される。(3.3.2.3節 \eqref{eq:euler_domestic_home} 再掲)
\begin{equation}
\lambda_t^h = \beta_t^H (1+i_t^H) E_t [ \lambda_{t+1}^h ] \tag{\ref{eq:euler_domestic_home}}
\end{equation}

\paragraph{② 代表的家計モデルの方程式}
①式に、前項で定義した代表的家計の所得の限界効用 \( \lambda_t^h = \lambda_t^H \)、および \( \lambda_{t+1}^h = \lambda_{t+1}^H \) を代入すると、そのまま代表的家計の変数で記述されたオイラー方程式が得られる。
\begin{equation}
\lambda_t^H = \beta_t^H (1+i_t^H) E_t [ \lambda_{t+1}^H ]
\label{eq:final_euler_H}
\end{equation}
外国についても同様に、外国通貨建て債券に関するオイラー方程式が得られる。
\begin{equation}
\lambda_t^{F/*} = \beta_t^F (1+i_t^F) E_t [ \lambda_{t+1}^{F/*} ]
\label{eq:final_euler_F}
\end{equation}

\subsubsection{UIP条件(国際債券)}
\paragraph{① 出発点:家計のFOC}
自国の家計の国際リスクフリー債券に関する最適化条件(FOC)は、3.3.2.3節 \eqref{eq:uip_condition_home} で示された。
\begin{equation}
\lambda_t^h e_t^{/*} = (1+i_t^F) \beta_t^H \operatorname{E}_t \left[ \lambda_{t+1}^h e_{t+1}^{/*} \right] \tag{\ref{eq:uip_condition_home}}
\end{equation}

\paragraph{② 代表的家計モデルの方程式}
①式に \( \lambda_t^h = \lambda_t^H \)、および \( \lambda_{t+1}^h = \lambda_{t+1}^H \) を代入すると、代表的家計のUIP条件式が得られる。
\begin{equation}
\lambda_t^H e_t^{/*} = (1+i_t^F) \beta_t^H E_t \left[ \lambda_{t+1}^H e_{t+1}^{/*} \right]
\label{eq:final_uip_H}
\end{equation}
参考として、外国家計の視点からのUIP条件(自国通貨建て債券への投資)も記述しておく(3.3.2.3節 \eqref{eq:uip_condition_foreign} 対応)。
\begin{equation}
\frac{\lambda_t^{F/*}}{e_t^{/*}} = (1+i_t^H) \beta_t^F E_t \left[ \frac{\lambda_{t+1}^{F/*}}{e_{t+1}^{/*}} \right]
\label{eq:final_uip_F}
\end{equation}

\subsubsection{需要関数と消費者物価指数(CPI)の定義}
\paragraph{① 出発点:家計の需要関数}
家計 \( h \) の費用最小化問題から自国財消費と外国財消費への需要は以下のようになる。(3.3.2.2節 \eqref{eq:demand_domestic_basket}, \eqref{eq:demand_foreign_basket} 再掲)
\begin{equation}
c_t^{h \to H} = \alpha^H \frac{p_t^{H \to W}}{p_t^H} c_t^{h \to W} \quad , \quad c_t^{h \to F} = (1-\alpha^H) \frac{p_t^{H \to W}}{p_t^{F}} c_t^{h \to W} \tag{\ref{eq:demand_domestic_basket}, \ref{eq:demand_foreign_basket}}
\end{equation}

\paragraph{② 代表的家計モデルの方程式}
3.4節で示された通り、これらの消費指数も全家計で共通となる。そこで、1.1節と同様に代表的家計の変数を定義する。
\begin{equation}
c_t^{H \to H} \equiv c_t^{h \to H} \quad (\text{※共通のため})
\label{eq:ra_cHtoH_final_def}
\end{equation}
\begin{equation}
c_t^{H \to F} \equiv c_t^{h \to F} \quad (\text{※共通のため})
\label{eq:ra_cHtoF_final_def}
\end{equation}
これらと 1.1節で定義した \( c_t^{H \to W} \) を①式に代入すると、以下の代表的家計の需要関数が得られる。
\begin{equation}
c_t^{H \to H} = \alpha^H \frac{p_t^{H \to W}}{p_t^H} c_t^{H \to W}
\label{eq:final_demand_HH}
\end{equation}
\begin{equation}
c_t^{H \to F} = (1-\alpha^H) \frac{p_t^{H \to W}}{p_t^F} c_t^{H \to W}
\label{eq:final_demand_HF}
\end{equation}
CPIの定義(3.3.2.2節 \eqref{eq:cpi_definition} 再掲)も併記する。
\begin{equation}
p_t^{H \to W} = (p_t^H)^{\alpha^H} (p_t^F)^{1-\alpha^H}
\label{eq:final_cpi_H}
\end{equation}
外国についても同様に、代表的家計の需要関数とCPIが定義される。
\begin{equation}
c_t^{F \to F} = \alpha^F \frac{p_t^{F \to W*}}{p_t^{F*}} c_t^{F \to W}
\label{eq:final_demand_FF}
\end{equation}
\begin{equation}
c_t^{F \to H} = (1-\alpha^F) \frac{p_t^{F \to W*}}{p_t^{H*}} c_t^{F \to W}
\label{eq:final_demand_FH}
\end{equation}
\begin{equation}
p_t^{F \to W*} = (p_t^{F*})^{\alpha^F} (p_t^{H*})^{1-\alpha^F}
\label{eq:final_cpi_F}
\end{equation}

\subsection{生産と価格設定}
\subsubsection{代表的家計の生産関数}
\paragraph{① 出発点:集計レベルの関係式(厳密な形式)}
3.5.1節 \eqref{eq:aggregate_production_function_home} で示された通り、国全体の総生産指数 \( Y_t^H \)、総労働 \( L_t^H \)、および総価格分散 \( \Delta_t^H \) の間には、以下の厳密な関係が成り立つ。
\begin{equation}
Y_t^H = \frac{a_t^H L_t^H}{\Delta_t^H} \tag{\ref{eq:aggregate_production_function_home}}
\end{equation}

\paragraph{② 代表的家計モデルの方程式への変換}
代表的家計の生産 \( y_t^H \) と労働を \( l_t^H \) を以下のように定義する。
\begin{equation}
y_t^H \equiv Y_t^H / N
\label{eq:ra_yH_final_def}
\end{equation}
\begin{equation}
l_t^H \equiv L_t^H / N
\label{eq:ra_lH_final_def}
\end{equation}
これらを①式に代入する。
\begin{equation*}
N y_t^H = \frac{a_t^H (N l_t^H)}{\Delta_t^H}
\end{equation*}
両辺を \( N \) で割ると、一人当たりの生産関数が得られる。この式は、代表的家計の生産指数 \( y_t^H \) が、代表的家計の労働 \( l_t^H \) だけでなく、国全体の価格分散 \( \Delta_t^H \) にも影響を受けることを示している。
\begin{equation}
y_t^H = \frac{a_t^H l_t^H}{\Delta_t^H}
\label{eq:final_prod_H}
\end{equation}
外国についても同様に、一人当たりの生産関数が導出される。
\begin{equation}
y_t^F = \frac{a_t^F l_t^F}{\Delta_t^F}
\label{eq:final_prod_F}
\end{equation}

\subsubsection{価格設定の動学(非線形形式)}
\paragraph{① 出発点:家計のFOC}
3.5.2節で導出された通り、価格改定家計 \( h \) が設定する最適価格 \( \widetilde{p}_t^h \) は、以下の補助変数 \( v_t, w_t \) を用いた方程式群 \eqref{eq:optimal_price_home}-\eqref{eq:w_recursive_home} によって決定される。
\begin{align}
(\widetilde{p}_t^h)^{1+\theta^H} &= \frac{\theta^H}{\theta^H-1} \frac{v_t}{w_t} \tag{\ref{eq:optimal_price_home}} \\
v_t &= \phi^H \frac{(Y_t^H)^2}{(a_t^H)^2} (p_t^H)^{2\theta^H} + \beta_t^H\xi^H \operatorname{E}_t[v_{t+1}] \tag{\ref{eq:v_recursive_home}} \\
w_t &= \lambda_t^h (1 - \tau_t^{H}) Y_t^H (p_t^H)^{\theta^H} + \beta_t^H\xi^H \operatorname{E}_t[w_{t+1}] \tag{\ref{eq:w_recursive_home}}
\end{align}

\paragraph{② 代表的家計モデルの方程式}
3.5.2節で示したように、全ての価格改定家計 \(h\) は共通の最適価格 \( \widetilde{p}_t^h \) を設定する。そこで、代表的家計の最適価格 \( \widetilde{p}_t^H \) を以下のように定義する。
\begin{equation}
\widetilde{p}_t^H \equiv \widetilde{p}_t^h \quad (\text{※3.5.2節より全価格改定家計で共通のため})
\label{eq:ra_optimal_price_H_final_def}
\end{equation}
この \( \widetilde{p}_t^H \) と、3.8.2.1節で定義した \( \lambda_t^H \)、および3.8.3.1節で定義した \( Y_t^H = N y_t^H \) を、①の式群に代入し整理する。

\begin{equation}
({\widetilde{p}}_t^H)^{1+\theta^H} = \frac{\theta^H}{\theta^H-1} \frac{v_t}{w_t}
\label{eq:final_optimal_price_H}
\end{equation}
\begin{equation}
v_t = \phi^H N^2 \frac{(y_t^H)^2}{(a_t^H)^2} (p_t^H)^{2\theta^H} + \beta_t^H\xi^H \operatorname{E}_t[v_{t+1}]
\label{eq:final_v_recursive_H}
\end{equation}
\begin{equation}
w_t = \lambda_t^H N (1 - \tau_t^{H}) y_t^H (p_t^H)^{\theta^H} + \beta_t^H\xi^H \operatorname{E}_t[w_{t+1}]
\label{eq:final_w_recursive_H}
\end{equation}

\paragraph*{物価指数の動学:}
\( p_t^h \)は、当期の価格改定組(割合 \( 1-\xi^H \)、全員が \( \widetilde{p}_t^H \) を設定)と、価格据え置き組(割合 \( \xi^H \)、前期の物価指数 \( \bar{p}_{t-1}^H \))の二つのグループに分解できる。この集計計算を行うと(詳細は付録参照)、\( \bar{p}_t^H(j)^{1-\theta^H} (=\frac{1}{N}\sum_{h} (p_t^h)^{1-\theta^H}) \)が、二つのグループの価格の \( 1-\theta^H \) 乗の加重平均として以下のように表される。
\begin{equation}
(\bar{p}_t^H)^{1-\theta^H} = (1-\xi^H)({\widetilde{p}}_t^H)^{1-\theta^H} + \xi^H(\bar{p}_{t-1}^H)^{1-\theta^H}
\label{eq:final_price_index_dynamics_H}
\end{equation}

\paragraph*{価格分散の動学:}
価格分散 \(\Delta_t^H\) は価格改定家計と非改定家計の価格の乖離を集計したものであり、計算すると以下のように再帰的に記述できる(詳細は付録参照)。
\begin{equation}
\Delta_t^H = (1-\xi^H) N \left( \frac{{\widetilde{p}}_t^H}{p_t^H} \right)^{-\theta^H} + \xi^H \left(\frac{p_t^H}{p_{t-1}^H}\right)^{\theta^H} \Delta_{t-1}^H
\label{eq:final_dispersion_dynamics_H}
\end{equation}

\paragraph*{外国の価格設定:}
外国についても同様に、最適価格 \( \widetilde{p}_t^{F*} \)、補助変数 \( v_t^F, w_t^F \)、物価指数、価格分散の動学が定義される。
\begin{equation}
({\widetilde{p}}_t^{F*})^{1+\theta^F} = \frac{\theta^F}{\theta^F-1} \frac{v_t^F}{w_t^F}
\label{eq:final_optimal_price_F}
\end{equation}
\begin{equation}
v_t^F = \phi^F M^2 \frac{(y_t^F)^2}{(a_t^F)^2} (p_t^{F*})^{2\theta^F} + \beta_t^F\xi^F \operatorname{E}_t[v_{t+1}^F]
\label{eq:final_v_recursive_F}
\end{equation}
\begin{equation}
w_t^F = \lambda_t^{F/*} M (1 - \tau_t^{F}) y_t^F (p_t^{F*})^{\theta^F} + \beta_t^F\xi^F \operatorname{E}_t[w_{t+1}^F]
\label{eq:final_w_recursive_F}
\end{equation}
\begin{equation}
(\bar{p}_t^{F*})^{1-\theta^F} = (1-\xi^F)({\widetilde{p}}_t^{F*})^{1-\theta^F} + \xi^F(\bar{p}_{t-1}^{F*})^{1-\theta^F}
\label{eq:final_price_index_dynamics_F}
\end{equation}
\begin{equation}
\Delta_t^F = (1-\xi^F) M \left( \frac{{\widetilde{p}}_t^{F*}}{p_t^{F*}} \right)^{-\theta^F} + \xi^F \left( \frac{p_t^{F*}}{p_{t-1}^{F*}}\right)^{\theta^F} \Delta_{t-1}^F
\label{eq:final_dispersion_dynamics_F}
\end{equation}

\subsection{国全体の資源制約式}
\paragraph{① 出発点:集計レベルの関係式}
3.7節 \eqref{eq:aggregate_resource_constraint_H} で示された通り、全ての家計の予算制約式を足し合わせ、国内市場の均衡条件を適用すると、国全体の資源制約式が得られる。
\begin{equation}
p_t^{H \to W} C_t^{H \to W} + B_{t+1}^H = p_t^H Y_t^H + (1+i_{t-1}^F) \frac{e_t^{/*}}{e_{t-1}^{/*}} B_t^H \tag{\ref{eq:aggregate_resource_constraint_H}}
\end{equation}

\paragraph{② 代表的家計モデルの方程式への変換}
ここで、代表的家計の(一人当たりの)対外純資産を定義する。
\begin{equation}
b_t^{H} \equiv B_t^{H} / N
\label{eq:ra_bH_final_def}
\end{equation}
①式 \eqref{eq:aggregate_resource_constraint_H} に \( C_t^{H \to W} = N c_t^{H \to W} \)、 \( Y_t^H = N y_t^H \)、および上記の \( B_t^{H} \) を代入する。
\begin{equation*}
p_t^{H \to W} (N c_t^{H \to W}) + (N b_{t+1}^H) = p_t^H (N y_t^H) + (1+i_{t-1}^F) \frac{e_t^{/*}}{e_{t-1}^{/*}} (N b_t^H)
\end{equation*}
この式の両辺を人口 \( N \) で割ることで、代表的家計の資源制約式が得られる。
\begin{equation}
p_t^{H \to W} c_t^{H \to W} + b_{t+1}^H = p_t^H y_t^H + (1+i_{t-1}^F) \frac{e_t^{/*}}{e_{t-1}^{/*}} b_t^H
\label{eq:final_resource_constraint_H}
\end{equation}
外国についても同様に、代表的家計(人口 \( M \))の資源制約式が得られる。ここで \( b_t^F \equiv B_t^{F*} / M \) と定義する。
\begin{equation}
p_t^{F \to W*} c_t^{F \to W} + b_{t+1}^{F*} = p_t^{F*} y_t^F + (1+i_{t-1}^H) \frac{e_{t-1}^{/*}}{e_t^{/*}} b_t^{F*}
\label{eq:final_resource_constraint_F}
\end{equation}
ここで、外国の代表的家計の対外純資産 \( b_t^{F*} \) は、以下のように定義される。
\begin{equation}
b_t^{F*} \equiv -(N/M)(b_t^H / e_t^{/*})
\label{eq:ra_bFstar_final_def}
\end{equation}
この定義は、世界全体の債券市場均衡条件 \( N b_t^{h \to F} + M e_t^{/*} b_t^{f \to F*} = 0 \) (自国家計が保有する外国債券と外国家計が発行する外国債券の総和がゼロ)から導かれる。

\subsection{政策ルール}
\subsubsection{財政政策}
\paragraph{① 出発点:政府の集計予算制約}
3.6.1節 \eqref{eq:final_gov_budget_agg} で示された通り、政府の税収と移転の関係は、以下の集計レベルの予算制約式で厳密に表される。
\begin{equation}
N t_t^H = \tau_t^H p_t^H Y_t^H \tag{\ref{eq:final_gov_budget_agg}}
\end{equation}

\paragraph{② 代表的家計モデルの方程式への変換}
3.8.3.1 節で定義した \( y_t^H \equiv Y_t^H / N \) を上記に代入する。
\begin{equation*}
N t_t^H = \tau_t^H p_t^H (N y_t^H)
\end{equation*}
この式の両辺を人口 \( N \) で割ることで、一人当たりの一括移転 \( t_t^H \) が得られる(なお、移転は \( h \) によらず共通 \( t_t^H \equiv t_t^h \) である)。
\begin{equation}
t_t^H = \tau_t^H p_t^H y_t^H
\label{eq:final_transfer_H}
\end{equation}

\paragraph{税率の決定ルール}
税率は 3.6.1節 \eqref{eq:tax_rule_H} で定義した確率過程に従う。
\begin{equation}
\log(\tau_t^H) = (1-\rho_{\tau}^H)\log(\tau_{ss}^H) + \rho_{\tau}^H \log(\tau_{t-1}^H) + \varepsilon_t^{\tau,H} \tag{\ref{eq:tax_rule_H}}
\end{equation}
外国についても同様に定義される。
\begin{equation}
t_t^{F*} = \tau_t^F p_t^{F*} y_t^F
\label{eq:final_transfer_F}
\end{equation}
\begin{equation}
\log(\tau_t^F) = (1-\rho_{\tau}^F)\log(\tau_{ss}^{F*}) + \rho_{\tau}^F \log(\tau_{t-1}^F) + \varepsilon_t^{\tau,F} \tag{\ref{eq:tax_rule_F}}
\end{equation}

\subsubsection{金融政策ルール}
\label{sec:final_mp_rules}
中央銀行が従う金融政策ルールは、3.6.2節において既に一人当たりの変数を用いて定義されているため、本節では式の再掲を省略する。

\paragraph{1. インフレ目標 (IT)}
政策ルールは 3.6.2 節で定義したインフレ目標のルールに従う。
\begin{equation}
i_t^H = i_{ss}^H + \phi_{\pi}^H (\pi_t^{H} - \pi_{ss}^{H}) + \varepsilon_{t}^{i,H}
\label{eq:mp_it_H_final}
\end{equation}

\paragraph{2. 名目GDP水準目標 (NGDPLT)}
政策ルールは 3.6.2 節で定義した名目GDP水準目標のルール(式 \eqref{eq:mp_ngdplt_1}–\eqref{eq:mp_ngdplt_3})に従う。

\paragraph{3. 名目消費水準目標 (NCLT)}
政策ルールは 3.6.2 節で定義した名目消費水準目標のルール(式 \eqref{eq:mp_nclt_1}–\eqref{eq:mp_nclt_3})に従う。

\paragraph{外国の金融政策}
外国の中央銀行は、本稿の分析を通じて、常に標準的なインフレ目標(IT)に従うものと仮定する(3.6.2節参照)。政策ルールは 3.6.2 節で定義した式 \eqref{eq:mp_it_F} に従う。
\begin{equation}
i_t^F = i_{ss}^F + \phi_{\pi}^F (\pi_t^{F*} - \pi_{ss}^{F*}) + \varepsilon_{t}^{i,F}
\label{eq:mp_it_F_final}
\end{equation}

% !TeX root = ../main.tex
% sections/chapter4.tex

\chapter{シミュレーションの準備}
\label{chap:preparation}

% !TeX root = ../../main.tex
% sections/chap4/sec_preparation_overview.tex

本章では前章で構築したモデルを用いてシミュレーションをおこなうための準備をする。
第\ref{sec:preparation_method}節では、シミュレーション手法について述べる。
第\ref{sec:preparation_parameters}節では、パラメータ値の設定とその根拠について説明する。
最後に第\ref{sec:preparation_criteria}節では、本シミュレーションにおける最適政策とは何か議論する。
% !TeX root = ../../main.tex
% sections/chap4/sec_preparation_method.tex

\section{シミュレーション手法}
\label{sec:preparation_method}

本稿のシミュレーションではDynare 6.3を使用する。
Dynareを用いることによりショックに対する各変数の動的な反応をグラフとデータにより出力することができる。
またゼロ金利制約の再現のために\textcite{GuerrieriIacoviello2015}によって開発されたライブラリOccbinを用いる。
しかし初めに開発されたOccbinはDynare 6.3に対応していないため、
Johannes Pfeifer氏が最新のDynareとの互換性を維持するために修正した更新版リポジトリOccbin\_updateを使用する。

\subsection{Dynareによる数値計算}
\label{subsec:preparation_dynare}
シミュレーションにあたり3章の動学方程式系、定常状態、パラメータおよびショックをDynareに与える。
するとDynareはショックによって定常状態から飛ばされた経済が
動学方程式系をみたしながら元の定常状態へ戻ってくる様子をグラフに描いてくれる。
ここではこのシミュレーションの原理を直観的に説明する
動学方程式系 \( \mathbf{f} \) について、
定常状態 \( \mathbf{x}_{ss} \) はショックがない \( \mathbf{\epsilon} = 0 \) ときに
経済が永続的に留まる点として以下を満たすものと定義される。
\[ \mathbf{f}( \mathbf{x}_{ss}, \mathbf{x}_{ss}, 0 ) = 0 \]
この定常状態においてヤコビ行列が正則であれば陰関数定理によって点
\( ( \mathbf{x}_{t+1}, \mathbf{x}_t, \mathbf{\epsilon}_t ) = 
( \mathbf{x}_{ss}, \mathbf{x}_{ss}, 0 ) \)
の近傍において
\[ \mathbf{x}_{t+1} = \mathbf{h}( \mathbf{x}_t, \mathbf{\epsilon}_t ) \]
というベクトル値関数 \( \mathbf{h} \) の存在が数学的に保証される。
また定常状態の定義から \( \mathbf{x}_{ss} = \mathbf{h}( \mathbf{x}_{ss}, 0 ) \) という関係が成立する。
この関係を直観的に理解するため、経済の状態を一変数のスカラー \( x_t \) と仮定し、
ショックの値を固定した関数 \( h_{\epsilon}( x_t ) \equiv h( x_t, \epsilon ) \) を定義して
 \( (x_t, x_{t+1}) \) 平面上でその挙動を考察する。
ショックがない定常状態において、経済は \( x_{ss} = h_0( x_{ss} ) \) を満たす点に静止している。
これは2次元平面上では \( h_0 \) が45度線と交差している地点に相当する。
\( t=0 \) においてショック \( \epsilon_0 \) が発生すると、
グラフが \( h_{\epsilon_0} \) へと一時的に移動し、経済は定常状態 \( x_{ss} \) から
移動したグラフ上の点 \( x_0 = h_{\epsilon_0}( x_{ss} ) \) へとジャンプする。
ショックが収まる \( t \ge 1 \) 以降、グラフは元の \( h_0 \) に戻る。
経済は \( x_0 \) という地点から、以下の帰納的なプロセスによって定常状態へと戻っていく。
\[ x_1 = h_0( x_0 ), \quad x_2 = h_0( x_1 ), \quad \dots \]
この反復計算の結果 \( x_t \) は \( x_{ss} \) へと収束する。
この階段状の移動の軌跡を時間軸に沿ってプロットしたものがインパルス応答関数である。
実用上の計算においてはこの曲線 \( h_0 \) そのものを非線形のまま直接求めることは困難である。
そのため定常状態 \( (x_{ss}, x_{ss}) \) において曲線 \( h_0 \) に接する接線を導出する1次近似をおこなう。
この接線の傾きが1より小さければ経済は定常状態へと収束する。
ここで多次元の一般論へと議論を広げると以下の数理的な課題が生じる。
まず非線形なベクトル値関数 \( \mathbf{h} \) を厳密に特定することは実用上不可能である。
さらに9本の方程式系を満たしながら定常状態へ向かう関数 \( \mathbf{h} \) は数学的に複数存在する可能性があり、
その場合、定常状態の周りにおける1次近似( 接線の傾き )も一意に定まらない。
そのため1次近似を一意に決定するためのBlanchard-Kahn( BK )条件がある。
これは経済のジャンプ変数の数と
係数行列(接線の傾きに相当する)の固有値のうち1より大きいもの( 不安定根 )の数が
一致することを要求する条件である。
この条件が満たされるとき、発散する経路を排除し定常状態へと収束する唯一のサドルパス安定解として
接平面の傾きが一意に確定する。
シミュレーションにおいて、経済はこの確定した接線( 接平面 )の上を
ショックによって決まった初期値から帰納的に移動していくことで最終的な動学パスが算出されるのである。

\subsection{Occbinの役割}
\label{subsec:preparation_occbin}
前述の関数 \( \mathbf{h} \) は経済の自然な移動を記述するものであるが、
ゼロ金利制約は特定の区間においてその移動を壁のように遮断してしまう。
したがってすべての区間において \( \mathbf{h} \) により描写することは不可能である。
そこでOccbinは経済が自然に移動する区間とゼロ金利制約にかかる区間を別々のレジームとして定義する。
そしてそれぞれのレジームにおいて1次近似をおこない得られた結果を繋ぎ合わせることでこの問題を解決している。

\subsection{自国金融政策パラメータの決定に際するゲーム的状況の回避}
\label{subsec:preparation_game_theory}
シミュレーションをおこなうにあたり、
まず外国金融政策は消費者物価指数を用いたインフレ目標に固定し政策パラメータも標準的な値に固定する。
そのうえで自国の各金融政策については政策パラメータは
 \( \beta \) ショックのシミュレーションにおいて自国の厚生を最大化するものとする。
これらのパラメータを用いておこなった \( \beta \) ショックおよび \( a \) ショックのシミュレーション結果が
5章で表示するグラフである。
なお5章において説明されるが、本稿モデルのシミュレーションにおいては
自国金融政策は外国の生産者物価指数などの主要な外国関連変数に影響を与えない(遮断効果)。
そのため外国金融政策を生産者物価指数を用いたインフレ目標に固定することで
外国金融政策の最適パラメータが変わってしまうゲーム理論的状況を回避している。
これにより外国金融政策についてはパラメータを固定したまま自国金融政策の最適パラメータを探索することが正当化される。
% !TeX root = ../../main.tex
% sections/chap4/sec_preparation_parameters.tex

\section{パラメータの設定}
\label{sec:preparation_parameters}

本稿のシミュレーションで用いるパラメータの値は
日本( \( H \) )と米国( \( F \) )の近年の実証データと整合的になるように設定する。
パラメータは家計の選好、生産性、価格の硬直性、人口、および財政政策、金融政策に関するものに大別される。
表 \ref{tab:preparation_parameters} はパラメータ設定の一覧とその設定根拠である。

% 3列目の幅を自動計算し、右端切れを防止しつつ他のページと余白を統一
\begin{longtable}{@{}p{2.3cm}p{1.8cm}p{\dimexpr\textwidth-4.1cm-4\tabcolsep\relax}@{}}
  \caption{パラメータの設定} \label{tab:preparation_parameters} \\
  \toprule
  \textbf{パラメータ} & \textbf{設定値} & \textbf{内容・根拠} \\
  \midrule
\endfirsthead

  \caption[]{(続き) パラメータの設定} \\
  \toprule
  \textbf{パラメータ} & \textbf{設定値} & \textbf{内容・根拠} \\
  \midrule
\endhead

  \bottomrule
\endlastfoot

    \multicolumn{3}{l}{\textit{家計の選好}} \\
    \begin{tabular}[t]{@{}l@{}} \( \beta_{ss}^H \) \\ \( \beta_{ss}^F \) \end{tabular} 
    & 
    \begin{tabular}[t]{@{}c@{}} 0.995 \\ 0.995 \end{tabular} 
    & 
    定常状態の主観的割引因子。
    年率実質利子率( \( r^* \) )が 2\% であることを意味し、
    両国経済が近年の構造的低下に陥る以前の平時の値として設定する。
    \textbf{日本( \( H \) ):}
    \textcite{OkazakiSudo2018}は
    日本の \( r^* \) が1980年代の約 4\% から近年は 0.3\% まで著しく低下したことを示しており、
    その平時の値として 2\% を仮定することは妥当であると考えられる。
    \textbf{米国( \( F \) ):}
    \textcite{HolstonLaubachWilliams2017}らによる世界金融危機以前の \( r^* \) 推計値 2\% を採用する。
    \\
    \noalign{\smallskip}
    \begin{tabular}[t]{@{}l@{}} \( \rho_{\beta}^H \) \\ \( \rho_{\beta}^F \) \end{tabular} 
    & 
    \begin{tabular}[t]{@{}c@{}} 0.977 \\ 0.9 \end{tabular} 
    & 
    \( \beta \)ショックの持続性。
    \textbf{日本( \( H \) ):}
    価格の硬直性 \( \xi^H = 0.99 \) との組み合わせでDynareが解を見つけられる上限値。
    \textbf{米国( \( F \) ):}
    \textcite{BenignoNistico2013} などで用いられる標準的な需要ショックの持続性の値。
    \\
    \noalign{\smallskip}
    \begin{tabular}[t]{@{}l@{}} \( \phi^H \) \\ \( \phi^F \) \end{tabular} 
    & 
    \begin{tabular}[t]{@{}c@{}} 8/11 \\ 8/11 \end{tabular} 
    & 
    労働非効用の重み。
    \textcite{GarinLesterSims2016} にならい定常状態の労働が「1日8時間労働」に相当するように設定する。
    \\
    \noalign{\smallskip}
    \begin{tabular}[t]{@{}l@{}} \( \alpha^H \) \\ \( \alpha^F \) \end{tabular} 
    & 
    \begin{tabular}[t]{@{}c@{}} 0.85 \\ 0.85 \end{tabular} 
    & 
    消費における自国財への選好度(ホームバイアス)。
    \( 1 - \alpha = 0.15 \) は輸入シェア 15\% を意味し、
    World Bankのデータにおける日米両国の
    「Imports of goods and services (\% of GDP)」の平均的な値と整合的である。
    \\
    \noalign{\smallskip}
    \begin{tabular}[t]{@{}l@{}} \( \theta^H \) \\ \( \theta^F \) \end{tabular} 
    & 
    \begin{tabular}[t]{@{}c@{}} 11.0 \\ 11.0 \end{tabular} 
    & 
    国内財の間の代替の弾力性。
    独占的競争下において企業が設定する価格と限界費用の比であるマークアップが1.1となることを含意する。 
    \textbf{日本( \( H \) ):}
    この値は日本のミクロデータを用いた\textcite{AokiHogenItoKanaiTakatomi2024}の実証推定値と一致する。
    \textbf{米国( \( F \) ):}米国のマークアップは上昇傾向にあるとの研究もあるが
    その推定手法には深刻な欠陥があることも指摘されている(\textcite{Basu2019}, \textcite{Raval2023})。
    そこで米国においても伝統的とされてきた1.1を採用する。
    \\
    \midrule

    \multicolumn{3}{l}{\textit{生産性}} \\
    \begin{tabular}[t]{@{}l@{}} \( a_{ss}^H \) \\ \( a_{ss}^F \) \end{tabular} 
    & 
    \begin{tabular}[t]{@{}c@{}} 1.0 \\ 1.0 \end{tabular} 
    & 
    定常状態の生産性。
    \textcite{GarinLesterSims2016}などにおける正規化手法にもとづき1.0に設定する。
    \\
    \noalign{\smallskip}
    \begin{tabular}[t]{@{}l@{}} \( \rho_{a}^H \) \\ \( \rho_{a}^F \) \end{tabular} 
    & 
    \begin{tabular}[t]{@{}c@{}} 0.95 \\ 0.95 \end{tabular} 
    & 
    生産性ショックの持続性。
    \textcite{SmetsWouters2007}による米国経済の推計値にもとづき設定する。
    \\
    \midrule

    \multicolumn{3}{l}{\textit{価格の硬直性}} \\
    \begin{tabular}[t]{@{}l@{}} \( \xi^H \) \\ \( \xi^F \) \end{tabular} 
    & 
    \begin{tabular}[t]{@{}c@{}} 0.99 \\ 0.75 \end{tabular} 
    & 
    カルボ価格粘着パラメータ(価格を改定できない企業の割合)。
    \textbf{日本( \( H \) ):}
    フィリップス曲線の傾きがほぼゼロ( \( \kappa^H \approx 0 \) )であることを意味し、
    \textcite{KishabaOkuda2023}が日本の地域データを用いて発見した
    「2000年代以降、構造的傾きがゼロに崩壊した」という実証結果と整合的である。
    \textbf{米国( \( F \) ):}
    傾きが平坦だがゼロではない( \( \kappa_F \approx 0.0070 \) )であることを意味し、
    \textcite{HazellHerrenoNakamuraSteinsson2022}が米国の地域データから推定した
    安定的かつ有意にプラスの傾き( \( \kappa_F \approx 0.0062 \) )と整合的である。
    \\
    \midrule

    \multicolumn{3}{l}{\textit{人口}} \\
    \begin{tabular}[t]{@{}l@{}} \( N \) \\ \( M \) \end{tabular} 
    & 
    \begin{tabular}[t]{@{}c@{}} 1.0 \\ 1.0 \end{tabular} 
    & 
    日本( \( H \) )および米国( \( F \) )の人口。 
    \textcite{ClaridaGaliGertler2002, GaliMonacelli2005}など日米を含む2国モデルの標準的な慣行にしたがい、
    簡略化のため1.0に設定する。
    \\
    \midrule

    \multicolumn{3}{l}{\textit{財政政策}} \\
    \begin{tabular}[t]{@{}l@{}} \( \tau_{ss}^H \) \\ \( \tau_{ss}^F \) \end{tabular} 
    & 
    \begin{tabular}[t]{@{}c@{}} 0.2 \\ 0.2 \end{tabular} 
    & 
    定常状態の所得税率。
    World Bank のデータによれば日米の税収のGDP比は 10\% 程度だが、
    これは社会保障負担を含まないため家計や企業が直面する実際の税負担を網羅していない。
    本稿のモデルではこの \( \tau \) は社会保障も含む広義の所得税率を意味するため
    \textcite{GarinLesterSims2016}などにならい 20\% に設定する。
    \\
    \midrule

\multicolumn{3}{l}{\textit{米国( \( F \) )の金融政策}} \\
    \noalign{\smallskip}
    \multicolumn{3}{l}{\rule[0.4ex]{1em}{0.4pt} \textbf{消費者物価インフレ目標(IT-CPI)} \hrulefill} \\
    \( \phi_{\pi}^F \) & 1.5 & インフレへの反応係数。\textcite{Taylor1993}が提唱した標準的な値。 \\
    \( \phi_{y}^F \) & 0.5 & 生産ギャップへの反応係数。\textcite{Taylor1993}が提唱した標準的な値。 \\
    \( \rho_{i}^F \) & 0.8 & 利子率平滑化係数。\textcite{SmetsWouters2007}の推定値と整合的な値。 \\
    \midrule

    \multicolumn{3}{l}{\textit{日本( \( H \) )の金融政策}} \\
    \noalign{\smallskip}
    \multicolumn{3}{l}{\rule[0.4ex]{1em}{0.4pt} \textbf{総消費水準目標(CLT)} \hrulefill} \\
    \( \phi_{gap}^H \) & 0.17 & 目標乖離への反応係数 \\
    \( \phi_{level}^H \) & 5.06 & 過去の乖離累積への反応係数 \\
    \noalign{\smallskip}
    \multicolumn{3}{l}{\rule[0.4ex]{1em}{0.4pt} \textbf{消費者物価水準目標(CPLT)} \hrulefill} \\
    \( \phi_{gap}^H \) & 51000 & 目標乖離への反応係数 \\
    \( \phi_{level}^H \) & 100000 & 過去の乖離累積への反応係数 \\
    \noalign{\smallskip}
    \multicolumn{3}{l}{\rule[0.4ex]{1em}{0.4pt} \textbf{消費者物価インフレ目標(IT-CPI)} \hrulefill} \\
    \( \phi_{\pi}^H \) & 300 & インフレへの反応係数 \\
    \noalign{\smallskip}
    \multicolumn{3}{l}{\rule[0.4ex]{1em}{0.4pt} \textbf{生産者物価インフレ目標(IT-PPI)} \hrulefill} \\
    \( \phi_{\pi}^H \) & 300 & インフレへの反応係数 \\
    \noalign{\smallskip}
    \multicolumn{3}{l}{\rule[0.4ex]{1em}{0.4pt} \textbf{名目総消費水準目標(NCLT)} \hrulefill} \\
    \( \phi_{gap}^H \) & 129.9 & 目標乖離への反応係数 \\
    \( \phi_{level}^H \) & 140 & 過去の乖離累積への反応係数 \\
    \noalign{\smallskip}
    \multicolumn{3}{l}{\rule[0.4ex]{1em}{0.4pt} \textbf{名目GDP水準目標(NGDPLT)} \hrulefill} \\
    \( \phi_{gap}^H \) & 6.5 & 目標乖離への反応係数 \\
    \( \phi_{level}^H \) & 13.8 & 過去の乖離累積への反応係数 \\
    \noalign{\smallskip}
    \multicolumn{3}{l}{\rule[0.4ex]{1em}{0.4pt} \textbf{生産水準目標(OLT)} \hrulefill} \\
    \( \phi_{gap}^H \) & 85 & 目標乖離への反応係数 \\
    \( \phi_{level}^H \) & 106 & 過去の乖離累積への反応係数 \\
    \noalign{\smallskip}
    \multicolumn{3}{l}{\rule[0.4ex]{1em}{0.4pt} \textbf{潜在生産水準目標(POLT)} \hrulefill} \\
    \( \phi_{gap}^H \) & 0.2 & 目標乖離への反応係数 \\
    \( \phi_{level}^H \) & 0.06 & 過去の乖離累積への反応係数 \\
    \noalign{\smallskip}
      \multicolumn{3}{l}{\rule[0.4ex]{1em}{0.4pt} \textbf{生産者物価水準目標(PPLT)} \hrulefill} \\
    \( \phi_{gap}^H \) & 100 & 目標乖離への反応係数 \\
    \( \phi_{level}^H \) & 500 & 過去の乖離累積への反応係数 \\
    \noalign{\smallskip}
    \multicolumn{3}{l}{\rule[0.4ex]{1em}{0.4pt} \textbf{消費者物価テイラー・ルール(TR CPI)} \hrulefill} \\
    \( \phi_{\pi}^H \) & 83 & インフレへの反応係数 \\
    \( \phi_{y}^H \) & 0.5 & 生産ギャップへの反応係数 \\
    \noalign{\smallskip}
    \multicolumn{3}{l}{\rule[0.4ex]{1em}{0.4pt} \textbf{生産者物価テイラー・ルール(TR PPI)} \hrulefill} \\
    \( \phi_{\pi}^H \) & 0 & インフレへの反応係数 \\
    \( \phi_{y}^H \) & 2.94 & 生産ギャップへの反応係数 \\
\end{longtable}
% !TeX root = ../../main.tex
% sections/chap4/sec_preparation_shocks.tex

\section{ショックの設定}
\label{sec:preparation_shocks}

本稿のシミュレーションでは自国家計が貯蓄を好むようになる \( \beta \) ショック、
および生産性が低下する \( a \) ショックを経済すなわち第 \ref{chap:model} 章の動学方程式系に対して与える。

主観的割引因子 \( \beta_t^H \) の上昇は家計が我慢強くなり、
現在の消費よりも将来の消費をより高く評価するようになることを意味する。
その結果、家計は現在の消費を減らし将来に備えて貯蓄を増やそうとする。
これは負の需要ショックの一種である。

生産性 \( a_t^H \) の低下は生産に下落圧力を与え、
もし同じ生産を保とうとするならばより多くの労働が必要となる。
これは負の供給ショックの一種である。

シミュレーションではこれらのショックが第0期 ( \( t=0 \) ) に発生し、
その後は \( \beta \) および \( a \) の自己回帰プロセスにしたがって徐々に定常状態へと減衰していく。
\begin{align*}
\log( \beta_t^H ) &= ( 1-\rho_{\beta}^H ) \log( \beta_{ss}^H ) + \rho_{\beta}^H \log( \beta_{t-1}^H ) + \varepsilon_t^{\beta,H} \\
\log( a_t^H ) &= ( 1-\rho_{a}^H ) \log( a_{ss}^H ) + \rho_{a}^H \log( a_{t-1}^H ) + \varepsilon_t^{a,H}
\end{align*}
自己回帰係数についてはパラメータ設定の表 \ref{tab:preparation_parameters} でも示したとおり
\( \rho_{\beta}^H \) および \( \rho_{a}^H \) はともに 1 より小さい値に設定されており、
ショックの影響が時間とともに消失する定常的な過程となっている。
また外生的なショック項は第0期 ( \( t=0 \) ) においてのみ
\( \varepsilon_0^{\beta,H} = 0.02 \) および \( \varepsilon_0^{a,H} = -0.1 \) となり、
それ以降の期間 ( \( t > 0 \) ) では 0 となる。

% !TeX root = ../main.tex
% sections/chapter5.tex

\chapter{シミュレーション結果}
\label{chap:results}

% --- 各セクションの読み込み ---

% 5.1 本章の概要
% !TeX root = ../../main.tex
% sections/chap5/sec_results_overview.tex

本章では前章までの議論にもとづきシミュレーションをおこない結果を詳細に分析する。

本章の構成は以下のとおりである。
まず第 \ref{sec:results_asad} 節では、シミュレーション結果を解釈するための理論的枠組みとして
第 \ref{chap:model} 章の動学方程式系をもとに総供給(AS)曲線および総需要(AD)曲線を導出する。
続く第 \ref{sec:results_beta_shock} 節と第 \ref{sec:results_a_shock} 節においては、
家計が貯蓄志向を高める \( \beta \) ショックおよび
生産性が低下する \( a \) ショックのシミュレーションをおこなう。
それぞれの分析においては各変数のインパルス応答グラフを提示するとともに、
第 \ref{sec:results_asad} 節で構築したAS-AD曲線を用いて
各政策がどのような仕組みにより経済に影響を与えたかをみていく。

本章で提示するインパルス応答グラフの読み方について一点補足する。
各グラフにおいて横軸は期を表しており便宜的にショック発生期を \( t=0 \) と設定して描写している。
縦軸は各変数の定常状態からの水準乖離を示している。
これは対数乖離(ハット変数)による近似表示とは異なるため、
乖離の式とグラフの数値を照らし合わせる際には注意が必要である。

% 5.2 分析の理論的枠組み:AS-ADモデルの導出(新設)
% !TeX root = ../../main.tex
% sections/chap5/sec_results_asad.tex

\section{分析の理論的枠組み:AS-ADモデルの導出}
\label{sec:results_asad}

本節ではシミュレーション結果を解釈するための理論的枠組みとして、
第 \ref{chap:model} 章の動学方程式系から\textbf{総供給( AS )曲線}および\textbf{総実需( AD )曲線}を導出する。
AS曲線は代表的家計の供給側の全条件を、AD曲線は需要側の全条件をそれぞれ集約したものである。
これにより各金融政策がいかなる効果を経済に及ぼすのかを( \( y, p \) )平面上の
均衡点の移動として視覚的に理解することが可能となる。

\subsection{総供給(AS)曲線}
\label{sec:results_asad_as}

供給側は代表的家計の価格の最適化を記述する以下の4本の非線形方程式系によって構成される。
\begin{itemize}
    \item 代表的家計の最適価格決定式 \eqref{eq:final_optimal_price_H}
    \item 代表的家計の補助変数 \( v_t \) の再帰動学式 \eqref{eq:final_v_recursive_H}
    \item 代表的家計の補助変数 \( w_t \) の再帰動学式 \eqref{eq:final_w_recursive_H}
    \item 代表的家計の物価指数動学方程式 \eqref{eq:final_price_index_dynamics_H}
\end{itemize}
これらの方程式系を定常状態の周りで対数線形近似し補助変数を消去して整理することにより、
以下の線形化されたフィリップス曲線が得られる(付録 \ref{chap:appendix_optimal_price_derivation})。
\begin{equation}
\hat{\pi}_t^H = \beta_{ss}^H E_t [\hat{\pi}_{t+1}^H] + \kappa^H (\hat{y}_t^H - 2\hat{a}_t^H - \hat{p}_t^H - \hat{\lambda}_t^H - \hat{\tau}_t^H)
\label{eq:results_asad_phillips}
\end{equation}
ここで \( \kappa^H \) は価格の粘着性に依存する正の定数である。
パラメータ設定の表 \ref{tab:preparation_parameters} でも述べたとおり、
本稿では日本経済の実証データにもとづき、自国の価格は極めて硬直的に設定している( \( \xi^H = 0.99 \) )。
この条件下では \( \kappa^H \) が極めて小さいため、本稿の分析では単純化して \( \kappa^H = 0 \) と置く。
このときインフレ率は期待インフレ率にのみ依存する形式となる。

さらにインフレ率の定義より \( \hat{\pi}_t^H = \hat{p}_t^H - \hat{p}_{t-1}^H \)、
期待インフレ率の定義より \( E_t [\hat{\pi}_{t+1}^H] = E_t [\hat{p}_{t+1}^H] - \hat{p}_t^H \) を代入して
 \( \hat{p}_t^H \) について整理すると以下の\textbf{AS曲線}が得られる。
\begin{equation}
\hat{p}_t^H = \frac{1}{1 + \beta_{ss}^H} \hat{p}_{t-1}^H + \frac{\beta_{ss}^H}{1 + \beta_{ss}^H} E_t [\hat{p}_{t+1}^H]
\label{eq:results_asad_as_curve}
\end{equation}
式 \eqref{eq:results_asad_as_curve} には生産 \( \hat{y}_t^H \) が含まれていない。
したがって( \( y, p \) )平面においてAS曲線は\textbf{水平な直線}として描かれる。
この水平線の高さは前期の生産者物価指数 \( p_{t-1}^H \) と期待生産者物価指数 \( E_t [p_{t+1}^H] \) のみによって決定される。

\subsection{総需要(AD)曲線}
\label{sec:results_asad_ad}
需要側は代表的家計の価格以外の最適化を記述するすべての方程式によって構成される。
注意点としては、個別財への需要関数 \eqref{eq:final_demand_HH} および \eqref{eq:final_demand_HF} を導出した家計の費用最小化問題において、
すでに名目予算制約式 \eqref{eq:nominal_budget_home_original} が用いられていた。
したがってそれらを集計した代表的家計の資源制約式 \eqref{eq:final_resource_constraint_H} は、
代表的家計の需要関数 \eqref{eq:final_demand_HH} および \eqref{eq:final_demand_HF} に内包されている。
よって代表的家計の資源制約式 \eqref{eq:final_resource_constraint_H} は以下の構成要素一覧に含めない。

\begin{itemize}
    \item 代表的家計の財市場均衡条件 \eqref{eq:final_goods_market_eq_H}
    \item 代表的家計の所得の限界効用の式 \eqref{eq:final_lambda_H}
    \item 代表的家計のオイラー方程式 \eqref{eq:final_euler_H}
    \item 代表的家計の UIP 条件 \eqref{eq:final_uip_H}
    \item 代表的家計の需要関数 \eqref{eq:final_demand_HH} および \eqref{eq:final_demand_HF}
    \item 代表的家計の消費者物価指数 ( CPI ) の定義 \eqref{eq:final_cpi_H}
    \item 一物一価の法則 \eqref{eq:lop}
\end{itemize}

まず需要関数 \eqref{eq:final_demand_HH} および \eqref{eq:final_demand_HF} 、一物一価の法則 \eqref{eq:lop} 、UIP条件 \eqref{eq:final_uip_H} から導かれた為替レート決定式 \eqref{eq:appendix_exchange_rate_final} を、
財市場均衡条件 \eqref{eq:final_goods_market_eq_H} に代入する。
これにより名目GDPと名目総消費を結びつける以下の関係式が得られる(付録 \ref{chap:appendix_exchange_rate})。
\begin{equation*}
p_t^H y_t^H = \left[ \alpha^H + (1-\alpha^F) E_t [\mathcal{B}_t \Lambda_{\infty}] \right] p_t^{H \to W} c_t^{H \to W} \label{eq:results_asad_gdp_identity}
\tag{\ref{eq:appendix_exchange_rate_gdp_substituted}}
\end{equation*}
この関係式は需要側の方程式系のうち、所得の限界効用の式 \eqref{eq:final_lambda_H} とオイラー方程式 \eqref{eq:final_euler_H} を除くすべての方程式を集約したものである。

次に \eqref{eq:appendix_exchange_rate_gdp_substituted} の名目総消費 \( p_t^{H \to W} c_t^{H \to W} \) に対し、所得の限界効用の式 \( \lambda_t^H = 1/(p_t^{H \to W} c_t^{H \to W}) \) \eqref{eq:final_lambda_H} を名目総消費について解き代入する。
さらにそうして出てきた所得の限界効用 \( \lambda_t^H \) に、
オイラー方程式 \( \lambda_t^H = \beta_t^H (1+i_t^H) E_t [ \lambda_{t+1}^H ] \) \eqref{eq:final_euler_H} を代入する。
最後に \( y_t^H \) を右辺に移項することで以下の\textbf{AD曲線}が導出される。
\begin{equation}
p_t^H = \frac{1}{y_t^H} \times \frac{\alpha^H + (1-\alpha^F) E_t [\mathcal{B}_t \Lambda_{\infty}]}{\beta_t^H (1+i_t^H) E_t [\lambda_{t+1}^H]}
\label{eq:results_asad_ad_curve}
\end{equation}
生産者物価指数 \( p_t^H \) が生産 \( y_t^H \) に反比例していることから、
AD曲線は右下がりの\textbf{直角双曲線}となっていることがわかる。

ここでAD曲線の分子に含まれる \( E_t [\mathcal{B}_t \Lambda_{\infty}] \) は自国金融政策の影響を受けない。
なぜなら \( \mathcal{B}_t \equiv \prod_{j=0}^{\infty} (\beta_{t+j}^H / \beta_{t+j}^F) \) は
両国の主観的割引因子の系列のみから構成され、
また \( \Lambda_{\infty} \equiv \lambda_{\infty}^{H/*} / \lambda_{\infty}^{F/*} \) を構成する
外国通貨建て所得の限界効用 \( \lambda^{/*} \) は遮断効果により自国金融政策の影響を受けないためである。
したがって金融政策の分析においてはこれらの項の動きは考慮しなくてよい。

\subsection{政策による均衡の制御と期待の役割}
\label{sec:results_asad_policy_control}

本モデルの均衡は水平なAS曲線 \eqref{eq:results_asad_as_curve} と、
右下がりのAD曲線 \eqref{eq:results_asad_ad_curve} の交点として描かれる。
先に述べたように \( \mathcal{B}_t \) と \( \Lambda_{\infty} \) は金融政策の影響を受けないため、
金融政策の分析上考慮すべき変数は
 \( p_t^H, y_t^H, i_t^H, E_t[\lambda_{t+1}^H], E_t[p_{t+1}^H] \)
の5つである。
以下では \( i_t^H, E_t[\lambda_{t+1}^H], E_t[p_{t+1}^H] \) を仮想的に単独で変化させ
( \( y, p \) )平面上の均衡点の動きを観察する。


\paragraph{1. 名目利子率 \( i_t^H \) の低下}
名目利子率 \( i_t^H \) が低下するとAD曲線の分母が小さくなり右辺が大きくなるため、
\textbf{AD曲線は右上へ移動}する。
これにより生産 \( y_t^H \) は増加する。


\paragraph{2. 所得の期待限界効用 \( E_t[\lambda_{t+1}^H] \) の低下}
所得の期待限界効用 \( E_t[\lambda_{t+1}^H] \) が低下すると
AD曲線の分母が小さくなり右辺が大きくなるため、\textbf{AD曲線は右上へ移動}する。
これにより生産 \( y_t^H \) は増加する。
これは名目利子率を下げる余地がないゼロ金利制約下にあっても、
中央銀行の約束が家計の期待に働きかけることにより
生産 \( y_t^H \) を回復させることが可能であることを示している。


\paragraph{3. 期待生産者物価指数 \( E_t[p_{t+1}^H] \) の上昇}
期待生産者物価指数 \( E_t[p_{t+1}^H] \) が上昇した場合、
AS曲線の右辺第2項が増大するため、\textbf{AS曲線は上へ移動}する。
これにより生産者物価指数 \( p_t^H \) は上昇し、生産 \( y_t^H \) は下落する。
しかし、このとき期待限界効用 \( E_t[\lambda_{t+1}^H] \) は
所得の限界効用の式 \eqref{eq:final_lambda_H} を通じて減少するため、
先に述べたように\textbf{AD曲線は右上へ移動}し、これは生産 \( y_t^H \) を増加させる方向に働く。

名目総消費水準目標はこれら3つの指標に総合的に働きかけることにより、
ゼロ金利制約に直面した経済を効率的に回復させる。

% 5.3 需要ショック(Betaショック)の分析(AS-AD解釈を追加予定)
% !TeX root = ../../main.tex
% sections/chap5/sec_results_beta_shock.tex

\section{ \( \beta \) ショックのシミュレーション結果}
\label{sec:results_beta_shock}
本節では主要なマクロ経済変数のインパルス応答グラフを順に示し各金融政策の効果を比較する。

\subsection{効用と厚生}
\label{sec:results_beta_shock_welfare_eval}
結果のグラフを見るにあたり、各金融政策の効果を評価する指標となる厚生の計算手法について解説する。
本稿のシミュレーションは Dynare と Occbin により1次近似を用いておこなわれ、
その際に2次以上の項は消失してしまう。
そのため2次の項である価格分散( \( \Delta_t \) )も消えてしまい、
価格分散の厚生に与える影響が考慮されなくなってしまう。
しかし価格分散の厚生に与える影響は重要であるため \textcite{Woodford2003} にならいこれを手計算で復活させる。
そこで付録 \ref{chap:appendix_welfare_correction} のように補正をおこなうと
効用の定常状態からの水準乖離が以下のように得られる。

\begin{equation}
    U(c_t, l_t) - U(c_{ss}, l_{ss}) \approx \hat{c}_t^{H \to W} - \phi^H (l_{ss}^H)^2 (\hat{y}_t^H - \hat{a}_t^H) - \phi^H (l_{ss}^H)^2 \left( \xi^H \hat{\Delta}_{t-1}^H + \frac{\theta^H \xi^H}{2(1-\xi^H)} (\hat{\pi}_t^H)^2 \right) \label{eq:results_beta_shock_period_utility}
\end{equation}

ここで右辺第3項の括弧内が手計算により復元した価格分散である。
厚生 ( \( W \) ) は、この期間効用の乖離を、変動する主観的割引因子 ( \( \beta_t^H \) ) を用いて
将来にわたって割り引き、その期待値を合計することで算出される。
本稿では第3章の生涯効用関数 \eqref{eq:lifetime_utility_home_original} の定義に基づき、
\( t=0 \) から \( t=100 \) までの合計値を以下の式で計算する。

\begin{equation}
    W = E_0 \sum_{t=0}^{100} \left( \prod_{k=0}^{t-1} \beta_{k}^H \right) \left[ U(c_t^{H \to W}, l_t^H) - U(c_{ss}^{H \to W}, l_{ss}^H) \right] \label{eq:results_beta_shock_social_welfare}
\end{equation}
ここで、 \( t=0 \) 期における割引項は \(\prod_{k=0}^{-1} \beta_k^H = 1\) と定義される。

\( \beta \) ショックにおける効用のグラフと厚生の数値を以下に表示する。

\begin{figure}[H]
    \centering
    \includegraphics[width=0.9\textwidth]{comparison_graphs/beta_shock/beta_shock_utility_H_without_delta.png}
    \caption{自国の効用( \( \text{utility}^H \) )と厚生( \( \Delta \) 無し)}
    \label{fig:results_beta_shock_utility_without_delta}
\end{figure}

\begin{figure}[H]
    \centering
    \includegraphics[width=0.9\textwidth]{comparison_graphs/beta_shock/beta_shock_utility_H_with_delta.png}
    \caption{自国の効用( \( \text{utility}^H \) )と厚生( \( \Delta \) 有り)}
    \label{fig:results_beta_shock_utility_with_delta}
\end{figure}

\begin{figure}[H]
    \centering
    \includegraphics[width=0.9\textwidth]{comparison_graphs/beta_shock/beta_shock_Delta_H.png}
    \caption{自国の価格分散( \( \Delta^H \) ) }
    \label{fig:results_beta_shock_delta_H}
\end{figure}

\paragraph{価格分散コストの影響と政策評価の構造変化}
図 \ref{fig:results_beta_shock_utility_without_delta} ( \( \Delta \) なし)と
図 \ref{fig:results_beta_shock_utility_with_delta} ( \( \Delta \) あり)を比較すると、
価格分散( \( \Delta_t \) )の導入が各政策の評価にどのような影響を与えるかがわかる。
まずグラフから読み取れるように、
すべての政策において \( \Delta \) ありの場合の厚生は \( \Delta \) なしの場合よりも低下している。
価格分散 \( \Delta_t \) は定義上 1 以上の値をとり大きいほど生産効率を悪化させる機能するため、
これを考慮すれば厚生は必ず押し下げられるためである。
しかし重要なのはその低下の程度が政策によって大きく異なる点である。
この違いを分析するために各政策を以下の2つの政策群に分類して考察する。

\begin{enumerate}
    \item \textbf{物価を目標に含めない政策群:} 実質消費水準目標(CLT)、生産水準目標(OLT)、潜在生産水準目標(POLT) 
    \item \textbf{物価を目標に含める政策群:} 消費者物価水準目標(CPLT)、インフレ目標(IT)、名目総消費水準目標(NCLT)、名目GDP水準目標(NGDPLT)、生産者物価水準目標(PPLT) 
\end{enumerate}

物価を目標に含めない政策(CLT, OLT, POLT)ではインフレ率の変動幅が大きくなり、
それにしたがい価格分散 \( \Delta_t \) も大きくなる。
そのため厚生も低下も大きくなる。
物価を目標に含める政策(CPLT、IT、NCLT、NGDPLT、PPLT)ではインフレ率が抑制される。
その結果、相対的に厚生の低下は軽微となる。
\( \beta^H \) ショックにおいてNCLTが高い評価を得た理由の一つは
実質消費を支えつつも物価を目標に組み込むことにより、
この価格分散を小さく抑えることに成功している点にある。

\subsection{遮断効果}
\label{sec:results_beta_shock_insulation_effect}
次に国際的な波及効果について考察する。
本稿のシミュレーションにおいては
自国の金融政策の違いが主要な外国関連変数 \( \{ \lambda_t^{H/*}, \lambda_t^{F/*}, p_t^{F*}, i_t^F, c_t^{H \to F}, c_t^{F \to F}, y_t^F, l_t^F, \bar{p}_t^{F*}, \pi_t^{F*}, \widetilde{p}_t^{F*}, v_t^F, w_t^F, t_t^{F*}, b_t^H, \Delta_t^F \} \) 
にまったく影響を与えないという結果が得られた。
これは遮断効果として知られている。
ただし完全伸縮物価( \( IT-PPI \) )についてはグラフが異なっている点に注意する。
これは遮断効果のもととなる方程式系の中に価格硬直性パラメータが含まれており、
そのパラメータの値が他の政策群とは異なるためである。
遮断効果の詳しい説明は付録 \ref{chap:appendix_insulation_effect} に記す。

\begin{figure}[H]
    \centering
    \includegraphics[width=0.9\textwidth]{comparison_graphs/beta_shock/beta_shock_e_slash_star.png}
    \caption{自国通貨建て名目為替レート( \( e^{/*} \) ) }
    \label{fig:results_beta_shock_exchange_rate}
\end{figure}

\begin{figure}[H]
    \centering
    \phantomcaption % 図番号を確保
    \label{fig:results_beta_shock_foreign_vars}
    \makebox[\textwidth][c]{
        \begin{minipage}{1.2\textwidth}
            \centering
            \begin{subfigure}{0.48\linewidth}
                \includegraphics[width=\linewidth]{comparison_graphs/beta_shock/beta_shock_lambda_H_slash_star.png}
                \caption{自国の外国通貨建て所得の限界効用( \( \lambda^{H/*} \) ) }
                \label{fig:results_beta_shock_lambda_H_slash_star}
            \end{subfigure}
            \hfill
            \begin{subfigure}{0.48\linewidth}
                \includegraphics[width=\linewidth]{comparison_graphs/beta_shock/beta_shock_lambda_F_slash_star.png}
                \caption{外国の所得の限界効用( \( \lambda^{F/*} \) ) }
                \label{fig:results_beta_shock_lambda_F_slash_star}
            \end{subfigure}
        \end{minipage}
    }
\end{figure}

\begin{figure}[H]
    \ContinuedFloat
    \centering
    \makebox[\textwidth][c]{
        \begin{minipage}{1.2\textwidth}
            \centering
            \begin{subfigure}{0.48\linewidth}
                \includegraphics[width=\linewidth]{comparison_graphs/beta_shock/beta_shock_p_F_star.png}
                \caption{外国の生産者物価指数(PPI)( \( p^{F*} \) ) }
                \label{fig:results_beta_shock_p_F_star_sub}
            \end{subfigure}
            \hfill
            \begin{subfigure}{0.48\linewidth}
                \includegraphics[width=\linewidth]{comparison_graphs/beta_shock/beta_shock_i_F.png}
                \caption{外国の名目利子率( \( i^F \) ) }
                \label{fig:results_beta_shock_i_F_sub}
            \end{subfigure}
        \end{minipage}
    }
\end{figure}

\begin{figure}[H]
    \ContinuedFloat
    \centering
    \makebox[\textwidth][c]{
        \begin{minipage}{1.2\textwidth}
            \centering
            \begin{subfigure}{0.48\linewidth}
                \includegraphics[width=\linewidth]{comparison_graphs/beta_shock/beta_shock_c_H_F.png}
                \caption{自国の外国財消費指数( \( c^{H \to F} \) ) }
                \label{fig:results_beta_shock_c_H_F_sub}
            \end{subfigure}
            \hfill
            \begin{subfigure}{0.48\linewidth}
                \includegraphics[width=\linewidth]{comparison_graphs/beta_shock/beta_shock_c_F_F.png}
                \caption{外国の外国財消費指数( \( c^{F \to F} \) ) }
                \label{fig:results_beta_shock_c_F_F_sub}
            \end{subfigure}
        \end{minipage}
    }
\end{figure}

\begin{figure}[H]
    \ContinuedFloat
    \centering
    \makebox[\textwidth][c]{
        \begin{minipage}{1.2\textwidth}
            \centering
            \begin{subfigure}{0.48\linewidth}
                \includegraphics[width=\linewidth]{comparison_graphs/beta_shock/beta_shock_y_F.png}
                \caption{外国の生産( \( y^F \) ) }
                \label{fig:results_beta_shock_y_F_sub}
            \end{subfigure}
            \hfill
            \begin{subfigure}{0.48\linewidth}
                \includegraphics[width=\linewidth]{comparison_graphs/beta_shock/beta_shock_l_F.png}
                \caption{外国の労働( \( l^F \) ) }
                \label{fig:results_beta_shock_l_F_sub}
            \end{subfigure}
        \end{minipage}
    }
\end{figure}

\begin{figure}[H]
    \ContinuedFloat
    \centering
    \makebox[\textwidth][c]{
        \begin{minipage}{1.2\textwidth}
            \centering
            \begin{subfigure}{0.48\linewidth}
                \includegraphics[width=\linewidth]{comparison_graphs/beta_shock/beta_shock_p_F_star_bar.png}
                \caption{外国の正規化された生産者物価指数(PPI)( \( \bar{p}^{F*} \) ) }
                \label{fig:results_beta_shock_p_F_star_bar_sub}
            \end{subfigure}
            \hfill
            \begin{subfigure}{0.48\linewidth}
                \includegraphics[width=\linewidth]{comparison_graphs/beta_shock/beta_shock_pi_F_star.png}
                \caption{外国の正規化された生産者物価指数(PPI)を用いたグロス・インフレ率( \( \pi^{F*} \) ) }
                \label{fig:results_beta_shock_pi_F_star_sub}
            \end{subfigure}
        \end{minipage}
    }
\end{figure}

\begin{figure}[H]
    \ContinuedFloat
    \centering
    \makebox[\textwidth][c]{
        \begin{minipage}{1.2\textwidth}
            \centering
            \begin{subfigure}{0.48\linewidth}
                \includegraphics[width=\linewidth]{comparison_graphs/beta_shock/beta_shock_p_F_star_tilde.png}
                \caption{外国の最適生産者物価( \( \widetilde{p}^{F*} \) ) }
                \label{fig:results_beta_shock_p_F_star_tilde_sub}
            \end{subfigure}
            \hfill
            \begin{subfigure}{0.48\linewidth}
                \includegraphics[width=\linewidth]{comparison_graphs/beta_shock/beta_shock_v_F.png}
                \caption{外国の最適生産者物価の補助変数1( \( v^F \) ) }
                \label{fig:results_beta_shock_v_F_sub}
            \end{subfigure}
        \end{minipage}
    }
\end{figure}

\begin{figure}[H]
    \ContinuedFloat
    \centering
    \makebox[\textwidth][c]{
        \begin{minipage}{1.2\textwidth}
            \centering
            \begin{subfigure}{0.48\linewidth}
                \includegraphics[width=\linewidth]{comparison_graphs/beta_shock/beta_shock_w_F.png}
                \caption{外国の最適生産者物価の補助変数2( \( w^F \) ) }
                \label{fig:results_beta_shock_w_F_sub}
            \end{subfigure}
            \hfill
            \begin{subfigure}{0.48\linewidth}
                \includegraphics[width=\linewidth]{comparison_graphs/beta_shock/beta_shock_t_F.png}
                \caption{外国の一括移転( \( t^F \) ) }
                \label{fig:results_beta_shock_t_F_sub}
            \end{subfigure}
        \end{minipage}
    }
\end{figure}

\begin{figure}[H]
    \ContinuedFloat
    \centering
    \makebox[\textwidth][c]{
        \begin{minipage}{1.2\textwidth}
            \centering
            \begin{subfigure}{0.48\linewidth}
                \includegraphics[width=\linewidth]{comparison_graphs/beta_shock/beta_shock_b_H.png}
                \caption{自国の対外純資産( \( b^H \) ) }
                \label{fig:results_beta_shock_b_H_sub}
            \end{subfigure}
            \begin{subfigure}{0.48\linewidth}
                \includegraphics[width=\linewidth]{comparison_graphs/beta_shock/beta_shock_Delta_F.png}
                \caption{外国の価格分散( \( \Delta^F \) ) }
                \label{fig:results_beta_shock_delta_F_sub}
            \end{subfigure}
        \end{minipage}
    }
    % 最後にまとめてキャプションを表示
    \caption{外国経済変数のインパルス応答関数( 全政策で不変 ) }
\end{figure}

なお価格分散( \( \Delta^F \) )のグラフにおいて縦軸の左上に \( 10^{-13} \) という表記がある。
これはグラフの変動幅が極めて微小な計算誤差レベルであることを示しており、
各金融政策について価格分散( \( \Delta^F \) )グラフは同一であるとみなせる。

\subsection{AS-AD分析}
\label{sec:results_beta_shock_asad_analysis}
本節では効用や為替レートも含む主要な自国関連変数のグラフを掲載し、
各金融政策についてその効果をAS-AD分析により解釈する。

\begin{figure}[H]
    \centering
    \includegraphics[width=0.9\textwidth]{comparison_graphs/beta_shock/beta_shock_utility_H_with_delta.png}
    \caption{自国の効用( \( \text{utility}^H \) )と厚生( \( \Delta \) 有り) }
    \label{fig:results_beta_shock_utility_with_delta_asad}
\end{figure}

\begin{figure}[H]
    \centering
    \includegraphics[width=0.9\textwidth]{comparison_graphs/beta_shock/beta_shock_y_H.png}
    \caption{自国の生産( \( y^H \) ) }
    \label{fig:results_beta_shock_y_H}
\end{figure}

\begin{figure}[H]
    \centering
    \includegraphics[width=0.9\textwidth]{comparison_graphs/beta_shock/beta_shock_l_H.png}
    \caption{自国の労働( \( l^H \) ) }
    \label{fig:results_beta_shock_l_H}
\end{figure}

\begin{figure}[H]
    \centering
    \includegraphics[width=0.9\textwidth]{comparison_graphs/beta_shock/beta_shock_c_H_W.png}
    \caption{自国の総消費指数( \( c^{H \to W} \) ) }
    \label{fig:results_beta_shock_c_H_W}
\end{figure}

\begin{figure}[H]
    \centering
    \includegraphics[width=0.9\textwidth]{comparison_graphs/beta_shock/beta_shock_p_H_W.png}
    \caption{自国の消費者物価指数( \( p^{H \to W} \) ) }
    \label{fig:results_beta_shock_p_H_W}
\end{figure}

\begin{figure}[H]
    \centering
    \includegraphics[width=0.9\textwidth]{comparison_graphs/beta_shock/beta_shock_p_H.png}
    \caption{自国の生産者物価指数(PPI)( \( p^H \) ) }
    \label{fig:results_beta_shock_p_H}
\end{figure}

\begin{figure}[H]
    \centering
    \includegraphics[width=0.9\textwidth]{comparison_graphs/beta_shock/beta_shock_pi_H.png}
    \caption{自国の生産者物価指数(PPI)を用いたグロス・インフレ率( \( \pi^H \) ) }
    \label{fig:results_beta_shock_pi_H}
\end{figure}

\begin{figure}[H]
    \centering
    \includegraphics[width=0.9\textwidth]{comparison_graphs/beta_shock/beta_shock_p_H_bar.png}
    \caption{自国の正規化された生産者物価指数(PPI)( \( \bar{p}^H \) ) }
    \label{fig:results_beta_shock_p_H_bar}
\end{figure}

\begin{figure}[H]
    \centering
    \includegraphics[width=0.9\textwidth]{comparison_graphs/beta_shock/beta_shock_lambda_H.png}
    \caption{自国の所得の限界効用( \( \lambda^H \) ) }
    \label{fig:results_beta_shock_lambda_H}
\end{figure}

\begin{figure}[H]
    \centering
    \includegraphics[width=0.9\textwidth]{comparison_graphs/beta_shock/beta_shock_p_H_W_c_H_W.png}
    \caption{自国の名目総消費( \( p^{H \to W} c^{H \to W} \) ) }
    \label{fig:results_beta_shock_p_H_W_c_H_W}
\end{figure}

\begin{figure}[H]
    \centering
    \includegraphics[width=0.9\textwidth]{comparison_graphs/beta_shock/beta_shock_p_H_bar_y_H.png}
    \caption{自国の名目GDP( \( \bar{p}^H y^H \) ) }
    \label{fig:results_beta_shock_p_H_bar_y_H}
\end{figure}

\begin{figure}[H]
    \centering
    \includegraphics[width=0.9\textwidth]{comparison_graphs/beta_shock/beta_shock_polt_y_H_and_y_H_potential.png}
    \caption{自国の生産と潜在生産( \( y^H \) and \( y^{H,pot} \) )( 潜在生産水準目標 ) }
    \label{fig:results_beta_shock_y_H_potential}
\end{figure}

\begin{figure}[H]
    \centering
    \includegraphics[width=0.9\textwidth]{comparison_graphs/beta_shock/beta_shock_i_H.png}
    \caption{自国の名目利子率( \( i^H \) ) }
    \label{fig:results_beta_shock_i_H}
\end{figure}

\begin{figure}[H]
    \centering
    \includegraphics[width=0.9\textwidth]{comparison_graphs/beta_shock/beta_shock_gamma_H.png}
    \caption{自国の目標未達分の累積( \( \gamma^H \) ) }
    \label{fig:results_beta_shock_gamma_H}
\end{figure}

\subsubsection{政策群に分けてのAS-AD分析}
まず各政策をショックを緩和できるか否かによって以下の2つの群に分けて全体の分析をおこなう。
\begin{itemize}
    \item \textbf{ショックを緩和できない政策群:} 消費者物価水準目標(CPLT)、生産者物価水準目標(PPLT)、インフレ目標(IT)
    \item \textbf{ショックを緩和できる政策群:} 実質消費水準目標(CLT)、名目総消費水準目標(NCLT)、名目GDP水準目標(NGDPLT)、生産水準目標(OLT)、潜在生産水準目標(POLT)
\end{itemize}
このような政策の成否は以下の2点が合わさったことにより生じている。
\begin{enumerate}
    \item 価格の硬直性が極めて高いこと
    \item ショックを緩和できない政策群は物価指数のみを目標に入れているのに対し、
    緩和できる政策群は実質変数も目標に入れていること
\end{enumerate}
\ref{sec:preparation_parameters}節で設定したとおり
本稿のモデルでは価格の硬直性が \( \xi^H = 0.99 \) と極めて高い。
\( \beta \) ショックが発生すると家計は現在の消費よりも将来の消費を評価するようになり
生産 \( y_t^H \) が減少する。
このときもし物価が伸縮的であれば生産者物価指数 \( p_t^H \) が下落し生産 \( y_t^H \) の下落は緩和されるが、
物価が極めて高い本稿のモデルではこの価格調整原理が機能せず、
ショックは生産 \( y_t^H \) の大幅な減少によってのみ吸収される。
このとき緩和できない政策は物価のみを目標としているため変数が目標から乖離しない。
したがって中央銀行は動くことがなく図 \ref{fig:results_beta_shock_y_H} などが示すとおり
経済が大きく落ち込む。
対照的に緩和できる政策は目標に消費や生産などの実質変数を含んでいる。
そのため物価が動かなくとも経済の落ち込みを実質変数の目標からの乖離として検知できる。
それにより中央銀行は名目利子率を下げ、またこうした緩和に対する期待も醸成されることにより
経済の落ち込みが抑えられた。


以上の説明をAS-AD分析により整理する。
正の \( \beta \) ショックが発生するとAD曲線\eqref{eq:results_asad_ad_curve}は分母が大きくなるため
左下へ移動する。
AS曲線\eqref{eq:results_asad_as_curve}は水平であったから、
生産者物価指数 \( p_t^H \) は変わらず生産 \( y_t^H \) は大きく減少する。
またこのように生産者物価指数 \( p_t^H \) は変わらないため
期待生産者物価指数 \( E_t [p_{t+1}^H] \) も上昇せずAS曲線はそのままの位置で固定される。
このとき緩和できない政策群は物価指数のみを目標としているため利子率を動かすことはなく
AD曲線\eqref{eq:results_asad_ad_curve}はその場にとどまる。
そのため図 \ref{fig:results_beta_shock_y_H} などが示すとおり経済は大きく落ち込んでいる。
対照的に緩和できる政策は目標に消費や生産などの実質変数を含んでいるため利子率を下げることとなり、
AD曲線\eqref{eq:results_asad_ad_curve}は分母が小さくなるため


\subsubsection{各政策の分析}
以下では各政策について詳しいAS-AD分析をおこなう。

\paragraph{A. 消費者物価水準目標(CPLT)}
「ショックを緩和できない政策群」に属するCPLTは
図 \ref{fig:results_beta_shock_p_H_W} において
消費者物価指数 \( p_t^{H \to W} \) を定常状態の線上で完璧に固定している。
この極めて高い制御性の背景には目標を構成する各変数の動学的な性質の差がある。
まず目標である消費者物価指数は次のように定義される。

\begin{equation}
\ln p_t^{H \to W} = \alpha^H \ln p_t^H + (1-\alpha^H) (\ln e_t^{/*} + \ln p_t^{F*}) \label{eq:results_beta_shock_analysis_cpi_final}
\end{equation}

ここで、国内財価格 \( p_t^H \) および外国財価格 \( p_t^{F*} \) は過去の蓄積によって決まる\textbf{状態変数}であり、ショックが発生した直後の \( t=0 \) においては定常状態から動くことはない。CPLT のターゲットはこれら「動きの遅い」状態変数を主たる構成要素としているため、ショック発生の瞬間には目標からの乖離が生じず、結果として名目利子率 \( i_t^H \) も定常状態の \( i_{ss}^H \) から始動することになる(図 \ref{fig:results_beta_shock_i_H})。

一方で、名目為替レート \( e_t^{/*} \) は以下の UIP(等価利回り)条件に従う\textbf{ジャンプ変数}である。

\begin{equation}
1 + i_t^H = (1 + i_t^F) E_t [ e_{t+1}^{/*} / e_t^{/*} ] \label{eq:results_beta_shock_analysis_uip_final}
\end{equation}

式 \eqref{eq:results_beta_shock_analysis_uip_final} が示す通り、為替レートは名目利子率操作に対して即座に反応する性質を持つ。時間が経過し、状態変数である \( p_t^H \) や \( p_t^{F*} \) が低下し始めると、中央銀行はそれを検知して名目利子率操作を行う。このとき、反応の遅い物価の下落圧力に対し、反応の極めて速い為替レートを対置させることで、式 \eqref{eq:results_beta_shock_analysis_cpi_final} の合計値を常に一定に保つような精密な相殺が可能となる。これが、CPLT において消費者物価指数のグラフが一直線になる数理的な理由である。

しかし、この制御の容易さは実体経済の安定を意味しない。AD 曲線を右上へ復元させ、需要の蒸発を防ぐためには、ショック直後の \( t=0 \) における大胆な名目利子率低下と為替のジャンプが必要である。物価という遅行指標のみをターゲットとする CPLT は、初動において「まだ物価は動いていない」と判断して緩和シグナルを発することができない。結果として、物価の数字上の安定と引き換えに、ジャンプ変数である実質消費 \( c_t^{H \to W} \) の深刻な急落を放置することになり、厚生を大きく損なう結果となっている。

\paragraph{B. 生産者物価水準目標(PPLT)}
「ショックを緩和できない政策群」の二つ目として、生産者物価水準目標(PPLT、水色点線)の挙動を分析する。PPLT は自国財価格指数 \( p_t^H \) のみをターゲットとする政策ルールであり、その式は以下のように記述される。

\begin{equation}
i_{t, notional}^H = i_{ss}^H + \phi_{gap}^H ( \ln p_t^H - \ln \chi^H ) + \phi_{level}^H \gamma_t^H \label{eq:results_beta_shock_analysis_plt_policy}
\end{equation}

PPLT において特筆すべきは、図 \ref{fig:results_beta_shock_i_H} に示される通り、名目利子率 \( i_t^H \) がショック直後の \( t=0 \) においてジャンプせず定常状態から始動し、その後も他の緩和的な政策群(NCLT 等)とは対照的に、ゼロ名目利子率制約(ZLB)に達するほどの強力な緩和を行っていない点である。この不十分な初動の背景には、ターゲット変数の「純粋な状態変数性」がある。

先述の CPLT のターゲット(CPI)にはジャンプ変数である為替レート \( e^{/*} \) が含まれていたが、PPLT が凝視する自国財価格 \( p_t^H \) は、為替の影響すら受けない純粋な\textbf{状態変数}である。本モデルのように価格粘着性が極めて高い設定下では、物価はショック直後には一切動かず、時間の経過とともに極めて緩やかに低下していく。式 \eqref{eq:results_beta_shock_analysis_plt_policy} に基づけば、ターゲットである \( p_t^H \) が \( t=0 \) で動かない以上、中央銀行はショックの直撃を検知できず、名目利子率を引き下げる根拠を持たない。

この「初動の遅れ」は、AD 曲線の復元において決定的な足かせとなる。

\begin{enumerate}
    \item 為替の不在: CPLT では為替レート \( e^{/*} \) を操作することで CPI の数値を「無理やり」固定できたが、PPLT のターゲットには為替が含まれない。そのため、名目利子率操作が為替を動かしても、それが直接ターゲットの安定に寄与せず、むしろ物価が実際に下落し始めるのを待つしかない。
    \item 需要の放置: オイラー方程式において、名目利子率 \( i_t^H \) が十分に低下せず、かつ将来の価格上昇へのコミットメントも物価が動き出すまで発動しないため、AD 曲線は左方に沈んだままとなる。これが、図 \ref{fig:results_beta_shock_y_H} 等で示される深刻な生産の落ち込みを招く。
\end{enumerate}

図 \ref{fig:results_beta_shock_p_H} が示す通り、PPLT は「水準目標」としての歴史依存性を持つため、長期的には物価を定常状態へ戻す能力において全政策中で最も高い精度を見せる。しかし、価格という「極めて動きの遅い敵」が動き出すのを待ってからしか反応できないという性質上、ショック初期の数年間においては実体経済を深刻な不況にさらすこととなる。厚生の観点からは、長期的な物価の正確な復元よりも、初動でのジャンプによる需要の下支えが重要であり、PPLT が CPLT 同様に低い評価に留まる理由はここにある。

\paragraph{C. インフレ目標(IT)}
「ショックを緩和できない政策群」の中で、最も深刻な景気後退を招いたのがインフレ目標(IT、赤点線)である。図 \ref{fig:results_beta_shock_pi_H} が示す通り、IT 下でのインフレ率は初期に全政策の中で最も大きく下落している。「物価の安定(インフレの維持)」を掲げる政策が、皮肉にも最も深刻なデフレを許容してしまった理由は、本モデルの動学的な構造から次のように説明できる。

まず、自国財のグロス・インフレ率 \( \pi_t^H \) の定義を確認する。自国の全企業が同一の価格設定を行う本モデルの設定下では、インフレ率は自国財価格を用いて次のように記述される。

\begin{equation}
\ln \pi_t^H = \ln p_t^H - \ln p_{t-1}^H \label{eq:results_beta_shock_analysis_it_pi_def}
\end{equation}

図 \ref{fig:results_beta_shock_pi_H} においてインフレ率の絶対的な変動幅が極めて小さいのは、価格粘着性が極端に高い( \( \xi^H = 0.99 \) )ために、その原資である国内物価指数 \( p_t^H \) の動き自体が抑制されているためである。

IT においてインフレ率が全政策で最も低くなった原因は、水準目標が持つ「歴史依存性」の欠如にある。AS 曲線(フィリップス曲線)に基づけば、現在のインフレ率は将来のインフレ期待 \( E_t [ \pi_{t+1}^H ] \) に強く依存する。

\begin{enumerate}
    \item 期待の崩壊: NCLT や PPLT のような水準目標は、ショックで物価が下がれば、将来それを埋め合わせるために「目標を上回るインフレ」を生成することを約束する。しかし、IT は過去の未達分を考慮しないため、需要ショックに直面した家計は「物価は下がったままで、将来も中央銀行が物価を押し戻すことはない」と予想する。この期待インフレ率の低下が、現在のインフレ率をさらに押し下げるという負のスパイラルを生んでいる。
    \item 不十分な名目利子率反応: IT の政策ルールは現在のインフレ率のみに反応する。
    \begin{equation}
    i_{t, notional}^H = i_{ss}^H + \phi_{\pi}^H ( \ln \pi_t^H - \ln \pi_{ss}^H ) \label{eq:results_beta_shock_analysis_it_policy}
    \end{equation}
    図 \ref{fig:results_beta_shock_i_H} が示す通り、IT 下の名目利子率は初期にわずかに低下するものの、水準目標群のようにゼロ名目利子率制約(ZLB)に張り付くほどの大胆な緩和を見せない。これは、ターゲットであるインフレ率が「水準(Level)」に比べて変化率として小さく見えるため、中央銀行が危機の深刻さを過小評価してしまうからである。
\end{enumerate}

最終的にインフレ率が定常状態に向かって回復しているのは、需要ショック \( \beta_t^H \) 自体が時間とともに減衰するという外生的な要因によるものであり、政策の力ではない。オイラー方程式を通じて将来の物価上昇をコミットできない IT は、実質利子率の高止まりを許容し、AD 曲線 \eqref{eq:results_asad_ad_curve} を左方に放置し続ける。これが、厚生を全政策で最も悪化させた数理的な帰結である。

\paragraph{D. 実質消費水準目標(CLT)}
「ショックを緩和できる政策群」の筆頭として、実質消費水準目標(CLT、オレンジ点線)の動学的な挙動を分析する。CLT は、物価等の名目変数ではなく、家計の効用に直結する実質総消費指数 \( c_t^{H \to W} \) そのものをターゲットとする政策ルールである。

\begin{equation}
i_{t, notional}^H = i_{ss}^H + \phi_{gap}^H ( \ln c_t^{H \to W} - \ln \chi^H ) + \phi_{level}^H \gamma_t^H \label{eq:results_beta_shock_analysis_clt_policy}
\end{equation}

図 \ref{fig:results_beta_shock_c_H_W} および図 \ref{fig:results_beta_shock_i_H} が示す通り、CLT は需要ショックに対して極めて迅速かつ強力な回復力を示している。この「初動の速さ」と「復元の早さ」は、ターゲットである消費 \( c_t^{H \to W} \) が\textbf{ジャンプ変数}であることに起因する。

\begin{enumerate}
    \item 即時の ZLB 到達(初動の強さ):
    前述の CPLT や PPLT がターゲットとしていた「状態変数(物価)」とは異なり、実質消費 \( c_t^{H \to W} \) はショックが発生した瞬間に将来の予測を織り込んで即座にジャンプする性質を持つ。需要ショックの直撃により消費が急落しようとした瞬間、式 \eqref{eq:results_beta_shock_analysis_clt_policy} はその乖離を \( t=0 \) の時点で直ちに検知する。その結果、図 \ref{fig:results_beta_shock_i_H} の通り、名目利子率 \( i_t^H \) はショック発生と同時にゼロ名目利子率制約(ZLB)まで一気にジャンプする。この初動の速さが、AD 曲線を即座に右上へと押し戻し、消費の初期落ち込みを最小限に留めている。
    \item 期待を通じた AD 曲線の復元:
    CLT の下では、中央銀行が「実質消費の水準」を維持することを家計にコミットしている。家計は、現在の消費が落ち込んでも中央銀行がそれを必ず元の水準へ戻すと確信するため、オイラー方程式における将来の所得の限界効用の期待値 \( E_t [ \lambda_{t+1}^H ] \) が低下(将来の期待消費が潤沢になると予想)する。この期待のアンカーが、名目利子率の ZLB 継続と相まって、図 \ref{fig:results_beta_shock_c_H_W} に見られるような急速な V 字回復を実現させている。
    \item 最速の ZLB 出口(目標達成の早さ):
    図 \ref{fig:results_beta_shock_i_H} において、CLT は 80 期付近で他の政策よりも早くゼロ名目利子率を脱却している。これは、ターゲットである実質消費 \( c_t^{H \to W} \) が物価のような遅行指標(状態変数)ではなく、政策に対して最も感応度の高いジャンプ変数であるため、目標とする定常状態への復帰が早期に完了したことを意味する。
\end{enumerate}

以上の分析から、CLT は実体経済の痛みを直接検知し、ジャンプ変数としての消費の性質を活かして AD 曲線に強力な復元力を与える、極めて有効なショック緩和策であることが確認できる。しかし、CLT は物価をターゲットに含んでいないため、物価の安定性やそれに伴う価格分散コストの抑制という観点では、次に述べる NCLT に対して一歩譲ることになる。

\paragraph{E. 名目総消費水準目標(NCLT)}
本稿の厚生評価において最良の結果を得た名目総消費水準目標(NCLT、青実線)のメカニズムを、AS-AD フレームワークおよび各変数の動学的な性質から分析する。NCLT は名目総消費支出 \( p_t^{H \to W} c_t^{H \to W} \) をターゲットとしており、その政策ルールは次のように記述される。

\begin{equation}
i_{t, notional}^H = i_{ss}^H + \phi_{gap}^H ( \ln p_t^{H \to W} + \ln c_t^{H \to W} - \ln \chi^H ) + \phi_{level}^H \gamma_t^H \label{eq:results_beta_shock_analysis_nclt_policy}
\end{equation}

図 \ref{fig:results_beta_shock_p_H_W_c_H_W} (名目総消費)において、NCLT は CLT に次ぐ高い復元力を見せつつ、定常状態への収束速度においては CLT を上回る極めて優れた安定性を示している。この挙動の背後には、名目支出をターゲットにすることによる「価格と実質の高度なバランス」が存在する。

\begin{enumerate}
    \item 為替レートを介した消費者物価( \( p_t^{H \to W} \) )の即時ジャンプ:
    図 \ref{fig:results_beta_shock_p_H_W} において、NCLT はショック直後の \( t=0 \) で消費者物価指数を即座に上昇させている。これは、ターゲットにジャンプ変数である実質消費 \( c_t^{H \to W} \) が含まれているため、中央銀行が初動で大胆な名目利子率引き下げ(ZLB への突入)を行い、それに応答して名目為替レート \( e^{/*} \) が即座に減価(上昇)した結果である。式 \eqref{eq:results_beta_shock_analysis_cpi_final} で示した通り、状態変数である国内価格が動かない初動において、ジャンプ変数である為替が「名目支出」を下支えする役割を果たしている。
    \item \( \lambda_t^H \) と名目支出の反転関係:
    所得の限界効用 \( \lambda_t^H \) は、定義により名目総消費支出 \( p_t^{H \to W} c_t^{H \to W} \) の逆数となる。図 \ref{fig:results_beta_shock_lambda_H} において CLT の方が \( \lambda_t^H \) を低く抑えられている(=実質消費をより拡大させている)ように見えるが、これは CLT が物価の安定を完全に度外視して実質変数のみを強力に押し戻しているためである。しかし、この CLT の「過剰な」復元力は、図 \ref{fig:results_beta_shock_y_H} における大きなオーバーシュート(生産の過熱)を招く。
    \item 生産のオーバーシュートと負効用の抑制:
    図 \ref{fig:results_beta_shock_y_H} を見ると、CLT は復興期において NCLT よりも高く生産(=労働供給)を跳ね上げている。これは家計にとって過度な労働による負効用の増大を意味する。NCLT は、ターゲットに「価格」が含まれていることで、実質変数の過剰な拡大に伴うインフレ圧力を自動的に検知し、ブレーキをかける機能を持つ。このため、NCLT は CLT よりもオーバーシュートを小さく抑え、生産と労働をより安定的に定常状態へと帰還させている。
    \item 物価のドリフト防止(名目アンカーの効果):
    図 \ref{fig:results_beta_shock_p_H} (自国財価格)において、CLT や OLT といった物価を目標に含まない政策は、物価水準が定常状態から離れていく「ドリフト」現象を起こしやすくなる。これに対し、NCLT は名目支出を目標に据えることで、長期的には物価を定常状態付近へと引き戻す強力な名目アンカーとして機能する。
\end{enumerate}

以上の分析から、NCLT がなぜ最高評価を得たのかが明らかになる。NCLT は、CLT のような「実体経済への素早い初動(ジャンプ変数の活用)」を維持しつつも、名目支出という枠組みを通じて「過剰な労働供給の抑制」と「長期的な物価の安定(名目アンカー)」を同時に達成しているのである。この「実質と名目のハイブリッドな安定化能力」こそが、価格分散コストと実体経済の乖離の両面を最小化し、厚生を最大化させた本質的な要因である。

\paragraph{F. 名目GDP水準目標(NGDPLT)}
「ショックを緩和できる政策群」の三つ目として、名目GDP水準目標(NGDPLT、緑一点鎖線)の分析を行う。NGDPLT は国内財価格(PPI) \( \bar{p}_t^H \) と実質生産 \( y_t^H \) の積をターゲットとするルールであり、次のように定義される。

\begin{equation}
i_{t, notional}^H = i_{ss}^H + \phi_{gap}^H ( \ln \bar{p}_t^H + \ln y_t^H - \ln \chi^H ) + \phi_{level}^H \gamma_t^H \label{eq:results_beta_shock_analysis_ngdplt_policy}
\end{equation}

図 \ref{fig:results_beta_shock_p_H_bar_y_H} 等に示される通り、NGDPLT は NCLT や CLT と並んで強力な復元力を示すが、それらと比較して「オーバーシュートが小さく、定常状態への収束が早い」という穏やかな動学特性を持つ。この差異は、ターゲットに含まれる変数の構成、特に「物価指数の範囲」と「実数項の定義」の違いに由来する。

\begin{enumerate}
    \item ターゲット変数の感度の差:
    NCLT がターゲットとする名目消費には、ジャンプ変数である為替レート \( e^{/*} \) が CPI を通じて直接含まれている。一方、NGDPLT のターゲットである名目 GDP( \( \bar{p}_t^H y_t^H \) )に含まれる国内価格は、為替の影響を直接受けない純粋な状態変数である。このため、NGDPLT は NCLT に比べてショック直後のターゲットの変動が抑制され、それが政策利子率の早期脱却(図 \ref{fig:results_beta_shock_i_H})と、復興期における景気過熱(オーバーシュート)の回避につながっている。
    \item 消費者物価指数の乖離に関するパラドックス:
    図 \ref{fig:results_beta_shock_p_H_W} において、NCLT の方が NGDPLT よりも消費者物価指数の定常状態からのプラスジャンプが大きくなっている。これは、NCLT が「CPI を含む名目支出」の維持を至上命題としているため、需要ショックによる実質消費の減少を相殺すべく、大胆に名目利子率を下げて為替を減価させ、CPI を強く押し上げている結果である。対して NGDPLT は国内指標を重視するため、為替を介した輸入物価の押し上げには相対的に消極的であり、その結果として CPI のジャンプ幅が小さく留まっている。
    \item 収束の早さと安定性:
    NGDPLT が全政策中で最も早く定常状態へ収束している点は注目に値する。これは、国内生産 \( y_t^H \) をターゲットに据えることで、国内の需給バランスを直接的にアンカーしているためである。消費 \( c_t^{H \to W} \) は異時点間の最適化によって大きく変動しやすい性質を持つが、生産 \( y_t^H \) は経済全体の安定化を図る上では名目 GDP を指標とする方が、動学的な揺れを最小限に抑えやすい傾向がある。
\end{enumerate}

以上の比較分析から、NGDPLT は NCLT ほどの需要回復力は持たないものの、実体経済の過熱を抑制し、最も安定的かつ早期に経済を平時へと帰還させる能力に長けていると言える。この性質が、厚生評価において NCLT に次ぐ高い評価を支える要因となっている。

\paragraph{G. 生産水準目標(OLT)}
「ショックを緩和できる政策群」の一つである生産水準目標(OLT、黄一点鎖線)の挙動を分析する。OLT は実質生産 \( y_t^H \) のみをターゲットとする政策ルールであり、次のように定義される。

\begin{equation}
i_{t, notional}^H = i_{ss}^H + \phi_{gap}^H ( \ln y_t^H - \ln \chi^H ) + \phi_{level}^H \gamma_t^H \label{eq:results_beta_shock_analysis_olt_policy}
\end{equation}

図 \ref{fig:results_beta_shock_y_H} において、OLT の生産パスは名目GDP水準目標(NGDPLT)とほぼ重なる軌道を描いている。この類似性は、本稿の重要な設定である価格の硬直性( \( \kappa^H = 0 \) )に起因する。NGDPLT がターゲットとする名目GDPにおいて、国内財価格 \( p_t^H \) が極めて安定しているため、名目値の変化は実質生産の変化とほぼ同義となり、結果として OLT と NGDPLT は近似的に同一の政策として機能している。

OLT の動学特性における注目点は、物価の「上昇ドリフト」と「利子率の早期脱却」である。

\begin{enumerate}
    \item 名目アンカーの欠如による物価の推移:
    図 \ref{fig:results_beta_shock_p_H} において、国内財価格 \( p_t^H \) は定常状態から乖離し、右肩上がりの上昇傾向を見せる。需要ショック自体は本来物価に下落圧力をかけるが、OLT の下では中央銀行が生産の回復を最優先し、強力な金融緩和を継続する。OLT は物価水準をターゲットに含んでいないため、実体経済を回復させる過程で生じたインフレ期待を抑制する「ブレーキ」が働かない。その結果、AS 曲線が将来の物価上昇期待 \( E_t [ \hat{p}_{t+1}^H ] \) によって上方へシフトし続け、物価水準が定常状態から離れていくドリフト現象が生じている。
    \item 目標達成の早さと名目利子率の挙動:
    図 \ref{fig:results_beta_shock_i_H} において、OLT は実質消費水準目標(CLT)に次いで早い時期にゼロ名目利子率制約(ZLB)から離脱している。これは、OLT が実質生産 \( y_t^H \) を直接の目標としているため、ショックによって左方へシフトした AD 曲線 \eqref{eq:results_asad_ad_curve} を元の位置へ復元させる力が、ターゲット変数に対して最もダイレクトに作用したためである。
\end{enumerate}

以上の通り、OLT は生産の安定化において非常に強力な能力を発揮する。しかし、ターゲットに名目変数(物価や名目支出)を一切含まないため、長期的な物価水準を一定に保つ能力(名目アンカー効果)を欠いている。この「物価の放任」が、回復期において生産のオーバーシュートを招き、厚生の評価を NCLT 等に譲る要因となっている。

\paragraph{H. 潜在生産水準目標(POLT)}
「ショックを緩和できる政策群」の最後として、潜在生産水準目標(POLT、黒破線)の動態を分析する。POLT は実際の生産 \( y_t^H \) と潜在生産 \( y_t^{H,pot} \) との乖離(生産ギャップ)をターゲットとするルールであり、次のように記述される。

\begin{equation}
i_{t, notional}^H = i_{ss}^H + \phi_{gap}^H ( \ln y_t^H - \ln y_t^{H,pot} ) + \phi_{level}^H \gamma_t^H \label{eq:results_beta_shock_analysis_polt_policy}
\end{equation}

POLT の挙動において最も特徴的な点は、ターゲットである潜在生産自体が外生ショックに反応して動いている点である。

\begin{enumerate}
    \item ターゲットの動的変化と生産ギャップの拡大:
    図 \ref{fig:results_beta_shock_y_H_potential} に示される通り、需要ショックが発生した直後、家計の貯蓄選好の高まりを反映して潜在生産 \( y_t^{H,pot} \) は上昇する。一方で、実際の生産 \( y_t^H \) は価格硬直性により需要の減退を直接受け、大幅に下落する。つまり、POLT では目標(潜在生産)が上がり、実態(実際の生産)が下がるという二重の乖離が発生する。この巨大な生産ギャップの発生が、強力な緩和圧力を生む原動力となっている。
    \item 最長のゼロ名目利子率継続と履歴依存性の役割:
    図 \ref{fig:results_beta_shock_i_H} において、POLT は 90 期付近までゼロ名目利子率制約(ZLB)を継続しており、全政策の中で最も出口が遅い。式 \eqref{eq:results_beta_shock_analysis_polt_policy} に含まれる累計乖離項 \( \gamma_t^H \) が、初期に蓄積された巨大な負のギャップの記憶を保持しているため、過去の未達分を相殺するために生産が潜在水準を上回る「過熱状態」を長期間維持する必要が生じ、これが ZLB の継続をもたらしている。
    \item 物価の著しい上昇ドリフト:
    図 \ref{fig:results_beta_shock_p_H} において、国内財価格 \( p_t^H \) は全政策で最も高い上昇を見せている。これは、上述の長期間にわたる過剰な金融緩和が将来の物価上昇期待を強力に押し上げ、水平な AS 曲線 \eqref{eq:results_asad_as_curve} を絶え間なく上方へシフトさせた結果である。
\end{enumerate}

以上の分析から、POLT は実体経済の過熱と物価の不安定化を招きやすい性質を持つことがわかる。厚生の観点からは、この過剰な労働供給と物価上昇に伴う価格分散コストがペナルティとなり、実体経済を適度に制御しつつ名目アンカーを維持する NCLT に評価を譲る結果となっている。


\subsection{厚生の決定要因:価格分散と実体経済のトレードオフ}
\label{sec:results_beta_shock_welfare_factors}
本節の締めくくりとして、なぜ名目総消費水準目標(NCLT)が厚生において最高評価を得たのかを、図 \ref{fig:results_beta_shock_delta_H} の価格分散( \( \Delta^H \) )の視点を交えて総括する。各政策のパフォーマンスは、以下の 3 つの要因のバランスによって決定されている。

\begin{enumerate}
    \item \textbf{初動の速さ(ジャンプ変数の活用)}:
    IT や CPLT、PPLT といった「ショックを緩和できない政策群」は、ターゲットが状態変数(物価)のみ、あるいは期待のアンカーを欠いているため、ショック直後の \( t=0 \) で名目利子率を十分に下げられない。これに対し、NCLT や CLT はジャンプ変数である実質消費をターゲットに含むことで、ショックと同時に名目利子率を ZLB までジャンプさせ、不況の深刻化を未然に防いでいる。
    \item \textbf{名目アンカーによる物価ドリフトの抑制}:
    CLT、OLT、POLT といった実質変数のみを追う政策は、実体経済を強力に回復させる一方で、物価水準を定常状態へ戻す力が弱い。図 \ref{fig:results_beta_shock_p_H} に見られる物価の上昇ドリフトは、長期的には経済の不確実性を高める要因となる。NCLT は名目支出をターゲットとすることで、不況期には強力な緩和を行いながらも、回復期には物価上昇を検知して適切にブレーキをかけ、物価を定常状態へと収束させている。
    \item \textbf{価格分散コスト( \( \Delta^H \) )の最小化}:
    厚生を最も大きく損なう要因の一つが、インフレ率の変動に伴う価格分散である。図 \ref{fig:results_beta_shock_delta_H} を見ると、IT(インフレ目標)が突出して高い価格分散を記録している。これは IT が水準目標を持たないためにデフレ期待のアンカーに失敗し、物価の大きな下落とその後の揺り戻しを許容してしまったためである。
    対照的に、NCLT は図 \ref{fig:results_beta_shock_delta_H} において最も低い水準で \( \Delta^H \) を安定させている。これは、将来の名目支出を保証することで期待インフレ率を一定に保ち、実際のインフレ率の変動幅を最小限に抑え込んだ結果である。
\end{enumerate}

以上の分析から、 NCLT の優位性は、(1) 消費というジャンプ変数を検知する「鋭い初動」、(2) 名目支出という枠組みによる「物価の安定(名目アンカー)」、および (3) それらがもたらす「価格分散コストの抑制」という 3 点を、単一のルールで最も高次元にバランスさせた点にあると結論付けられる。

% 5.4 生産性ショック(Aショック)の分析(AS-AD導出部分を削除し整理予定)
% !TeX root = ../../main.tex
% sections/chap5/sec_results_a_shock.tex

\section{ \( a \) ショックのシミュレーション結果 }
\label{sec:results_a_shock}
前節までは、需要ショック( ( \( \beta^H \) ) の上昇 )に対して名目総消費水準目標( NCLT )が優れた安定化効果を持つことを論じてきた。本節では、負の供給ショックとして自国の生産性( ( \( a_t^H \) ) )の一時的な低下ショックが発生した場合の経済変動を分析し、各政策ルールの「頑健性( Robustness )」を検証する。


\subsection{厚生評価}
\label{sec:results_a_shock_welfare_eval}
まず、各政策ルール下の厚生を確認する。図 \ref{fig:results_a_shock_utility_without_delta} ( 価格分散コスト調整なし )および図 \ref{fig:results_a_shock_utility_with_delta} ( 調整済み )は、供給ショックに対する各政策の評価を示している。

\begin{figure}[H]
    \centering
    \includegraphics[width=0.9\textwidth]{comparison_graphs/a_shock/a_shock_utility_H_without_delta.png}
    \caption{自国の効用( ( \( \text{utility}^H \) ) )と厚生( ( \( \Delta \) 無し ) )}
    \label{fig:results_a_shock_utility_without_delta}
\end{figure}

\begin{figure}[H]
    \centering
    \includegraphics[width=0.9\textwidth]{comparison_graphs/a_shock/a_shock_utility_H_with_delta.png}
    \caption{自国の効用( ( \( \text{utility}^H \) ) )と厚生( ( \( \Delta \) 有り ) )}
    \label{fig:results_a_shock_utility_with_delta}
\end{figure}

( \( \Delta \) ) の考慮を加えたときの厚生の落ち込みは、たしかに物価を目標に含む政策群の方が小さいが、その差はごくわずかである。これは ( \( a \) ) ショックが起きたときの物価変動圧力が小さいためであり、本ショックにおける厚生の序列は、実体経済の安定性、すなわち所得の限界効用 ( \( \lambda_t^H \) ) の安定性に支配されている。特に後述する POLT 等の下では、不適切な需要抑制により所得の限界効用が突出して上昇しており、これが深刻な厚生損失を招いていることが確認できる。


\subsection{遮断効果}
\label{sec:results_a_shock_insulation_effect}
前節では外国関連主要変数が自国金融政策の影響を受けないことを確認した。それらの外国関連主要変数は ( \( \beta \) ) ショック自体の影響は受けたが、ここでは対照的に生産性 ( \( a \) ) ショックの影響は受けないことをみる。ただし、完全伸縮物価( ( \( IT-PPI \) ) )ケースについては、遮断効果を構成する方程式系の中に価格硬直性パラメータそのものが含まれているため、他の政策群とは異なる推移を示す点に留意が必要である。それら外国関連変数を決定する方程式群には自国内生変数が含まれていないことを説明したが、加えて生産性を表す変数 ( \( a \) ) も含まれていないため ( \( a \) ) ショックの影響は受けないことになる( 付録 \ref{chap:appendix_insulation_effect} を参照 )。シミュレーション結果においてもこのことを確認するため、以下に変動する名目為替レートとこれら外国関連主要変数のグラフを表示する。

グラフ描画領域の左上に ( \( 10^{-15} \) ) などときわめて小さな単位が書かれている場合、グラフの変動はコンピュータによる計算誤差と考えられ、理論的には変動がないことを示している。このことより外国関連主要変数のグラフはいずれも理論的な変動がないことがわかる。

\begin{figure}[H]
    \centering
    \includegraphics[width=0.9\textwidth]{comparison_graphs/a_shock/a_shock_e_slash_star.png}
    \caption{自国通貨建て名目為替レート( ( \( e^{/*} \) ) )}
    \label{fig:results_a_shock_exchange_rate}
\end{figure}

\begin{figure}[H]
    \centering
    \phantomcaption % 図番号を確保
    \label{fig:results_a_shock_foreign_vars}
    \makebox[\textwidth][c]{
        \begin{minipage}{1.2\textwidth}
            \centering
            \begin{subfigure}{0.48\linewidth}
                \includegraphics[width=\linewidth]{comparison_graphs/a_shock/a_shock_lambda_H_slash_star.png}
                \caption{自国の外国通貨建て所得の限界効用( ( \( \lambda^{H/*} \) ) )}
                \label{fig:results_a_shock_lambda_H_slash_star}
            \end{subfigure}
            \hfill
            \begin{subfigure}{0.48\linewidth}
                \includegraphics[width=\linewidth]{comparison_graphs/a_shock/a_shock_lambda_F_slash_star.png}
                \caption{外国の所得の限界効用( ( \( \lambda^{F/*} \) ) )}
                \label{fig:results_a_shock_lambda_F_slash_star}
            \end{subfigure}
        \end{minipage}
    }
\end{figure}

\begin{figure}[H]
    \ContinuedFloat
    \centering
    \makebox[\textwidth][c]{
        \begin{minipage}{1.2\textwidth}
            \centering
            \begin{subfigure}{0.48\linewidth}
                \includegraphics[width=\linewidth]{comparison_graphs/a_shock/a_shock_p_F_star.png}
                \caption{外国の生産者物価指数( PPI )( ( \( p^{F*} \) ) )}
                \label{fig:results_a_shock_p_F_star_sub}
            \end{subfigure}
            \hfill
            \begin{subfigure}{0.48\linewidth}
                \includegraphics[width=\linewidth]{comparison_graphs/a_shock/a_shock_i_F.png}
                \caption{外国の名目利子率( ( \( i^F \) ) )}
                \label{fig:results_a_shock_i_F_sub}
            \end{subfigure}
        \end{minipage}
    }
\end{figure}

\begin{figure}[H]
    \ContinuedFloat
    \centering
    \makebox[\textwidth][c]{
        \begin{minipage}{1.2\textwidth}
            \centering
            \begin{subfigure}{0.48\linewidth}
                \includegraphics[width=\linewidth]{comparison_graphs/a_shock/a_shock_c_H_F.png}
                \caption{自国の外国財消費指数( ( \( c^{H \to F} \) ) )}
                \label{fig:results_a_shock_c_H_F_sub}
            \end{subfigure}
            \hfill
            \begin{subfigure}{0.48\linewidth}
                \includegraphics[width=\linewidth]{comparison_graphs/a_shock/a_shock_c_F_F.png}
                \caption{外国の外国財消費指数( ( \( c^{F \to F} \) ) )}
                \label{fig:results_a_shock_c_F_F_sub}
            \end{subfigure}
        \end{minipage}
    }
\end{figure}

\begin{figure}[H]
    \ContinuedFloat
    \centering
    \makebox[\textwidth][c]{
        \begin{minipage}{1.2\textwidth}
            \centering
            \begin{subfigure}{0.48\linewidth}
                \includegraphics[width=\linewidth]{comparison_graphs/a_shock/a_shock_y_F.png}
                \caption{外国の生産( ( \( y^F \) ) )}
                \label{fig:results_a_shock_y_F_sub}
            \end{subfigure}
            \hfill
            \begin{subfigure}{0.48\linewidth}
                \includegraphics[width=\linewidth]{comparison_graphs/a_shock/a_shock_l_F.png}
                \caption{外国の労働( ( \( l^F \) ) )}
                \label{fig:results_a_shock_l_F_sub}
            \end{subfigure}
        \end{minipage}
    }
\end{figure}

\begin{figure}[H]
    \ContinuedFloat
    \centering
    \makebox[\textwidth][c]{
        \begin{minipage}{1.2\textwidth}
            \centering
            \begin{subfigure}{0.48\linewidth}
                \includegraphics[width=\linewidth]{comparison_graphs/a_shock/a_shock_p_F_star_bar.png}
                \caption{外国の正規化された生産者物価指数( PPI )( ( \( \bar{p}^{F*} \) ) )}
                \label{fig:results_a_shock_p_F_star_bar_sub}
            \end{subfigure}
            \hfill
            \begin{subfigure}{0.48\linewidth}
                \includegraphics[width=\linewidth]{comparison_graphs/a_shock/a_shock_pi_F_star.png}
                \caption{外国の正規化された生産者物価指数を用いたグロス・インフレ率( ( \( \pi^{F*} \) ) )}
                \label{fig:results_a_shock_pi_F_star_sub}
            \end{subfigure}
        \end{minipage}
    }
\end{figure}

\begin{figure}[H]
    \ContinuedFloat
    \centering
    \makebox[\textwidth][c]{
        \begin{minipage}{1.2\textwidth}
            \centering
            \begin{subfigure}{0.48\linewidth}
                \includegraphics[width=\linewidth]{comparison_graphs/a_shock/a_shock_p_F_star_tilde.png}
                \caption{外国の最適生産者物価( ( \( \widetilde{p}^{F*} \) ) )}
                \label{fig:results_a_shock_p_F_star_tilde_sub}
            \end{subfigure}
            \hfill
            \begin{subfigure}{0.48\linewidth}
                \includegraphics[width=\linewidth]{comparison_graphs/a_shock/a_shock_v_F.png}
                \caption{外国の最適生産者物価の補助変数1( ( \( v^F \) ) )}
                \label{fig:results_a_shock_v_F_sub}
            \end{subfigure}
        \end{minipage}
    }
\end{figure}

\begin{figure}[H]
    \ContinuedFloat
    \centering
    \makebox[\textwidth][c]{
        \begin{minipage}{1.2\textwidth}
            \centering
            \begin{subfigure}{0.48\linewidth}
                \includegraphics[width=\linewidth]{comparison_graphs/a_shock/a_shock_w_F.png}
                \caption{外国の最適生産者物価の補助変数2( ( \( w^F \) ) )}
                \label{fig:results_a_shock_w_F_sub}
            \end{subfigure}
            \hfill
            \begin{subfigure}{0.48\linewidth}
                \includegraphics[width=\linewidth]{comparison_graphs/a_shock/a_shock_t_F.png}
                \caption{外国の一括移転( ( \( t^F \) ) )}
                \label{fig:results_a_shock_t_F_sub}
            \end{subfigure}
        \end{minipage}
    }
\end{figure}

\begin{figure}[H]
    \ContinuedFloat
    \centering
    \makebox[\textwidth][c]{
        \begin{minipage}{1.2\textwidth}
            \centering
            \begin{subfigure}{0.48\linewidth}
                \includegraphics[width=\linewidth]{comparison_graphs/a_shock/a_shock_b_H.png}
                \caption{自国の対外純資産( ( \( b^H \) ) )}
                \label{fig:results_a_shock_b_H_sub}
            \end{subfigure}
            \hfill
            \begin{subfigure}{0.48\linewidth}
                \includegraphics[width=\linewidth]{comparison_graphs/a_shock/a_shock_Delta_F.png}
                \caption{外国の価格分散( ( \( \Delta^F \) ) )}
                \label{fig:results_a_shock_delta_F_sub}
            \end{subfigure}
        \end{minipage}
    }
    % 最後にまとめてキャプションを表示
    \caption{外国経済変数のインパルス応答関数( 全政策で不変 )}
\end{figure}


\subsection{AS-AD分析}
\label{sec:results_a_shock_asad_analysis}
次にインパルス応答のグラフを示し、金融政策ごとにAS-AD分析をおこなう。

\begin{figure}[H]
    \centering
    \includegraphics[width=0.9\textwidth]{comparison_graphs/a_shock/a_shock_utility_H_with_delta.png}
    \caption{自国の効用( ( \( \text{utility}^H \) ) )と厚生( ( \( \Delta \) 無し ) )}
    \label{fig:results_a_shock_utility_with_delta_asad}
\end{figure}

\begin{figure}[H]
    \centering
    \includegraphics[width=0.9\textwidth]{comparison_graphs/a_shock/a_shock_y_H.png}
    \caption{自国の生産( ( \( y^H \) ) )}
    \label{fig:results_a_shock_y_H}
\end{figure}

\begin{figure}[H]
    \centering
    \includegraphics[width=0.9\textwidth]{comparison_graphs/a_shock/a_shock_l_H.png}
    \caption{自国の労働( ( \( l^H \) ) )}
    \label{fig:results_a_shock_l_H}
\end{figure}

\begin{figure}[H]
    \centering
    \includegraphics[width=0.9\textwidth]{comparison_graphs/a_shock/a_shock_c_H_W.png}
    \caption{自国の総消費指数( ( \( c^{H \to W} \) ) )}
    \label{fig:results_a_shock_c_H_W}
\end{figure}

\begin{figure}[H]
    \centering
    \includegraphics[width=0.9\textwidth]{comparison_graphs/a_shock/a_shock_p_H.png}
    \caption{自国の生産者物価指数(PPI)( \( p^H \) ) }
    \label{fig:results_a_shock_p_H}
\end{figure}

\begin{figure}[H]
    \centering
    \includegraphics[width=0.9\textwidth]{comparison_graphs/a_shock/a_shock_p_H_W.png}
    \caption{自国の消費者物価指数( ( \( p^{H \to W} \) ) )}
    \label{fig:results_a_shock_p_H_W}
\end{figure}

\begin{figure}[H]
    \centering
    \includegraphics[width=0.9\textwidth]{comparison_graphs/a_shock/a_shock_p_H_bar.png}
    \caption{自国の正規化された生産者物価指数( ( \( \bar{p}^H \) ) )}
    \label{fig:results_a_shock_p_H_bar}
\end{figure}

\begin{figure}[H]
    \centering
    \includegraphics[width=0.9\textwidth]{comparison_graphs/a_shock/a_shock_pi_H.png}
    \caption{自国の正規化された生産者物価指数を用いたグロス・インフレ率( ( \( \pi^H \) ) )}
    \label{fig:results_a_shock_pi_H}
\end{figure}

\begin{figure}[H]
    \centering
    \includegraphics[width=0.9\textwidth]{comparison_graphs/a_shock/a_shock_lambda_H.png}
    \caption{自国の所得の限界効用( ( \( \lambda^H \) ) )}
    \label{fig:results_a_shock_lambda_H}
\end{figure}

\begin{figure}[H]
    \centering
    \includegraphics[width=0.9\textwidth]{comparison_graphs/a_shock/a_shock_p_H_W_c_H_W.png}
    \caption{自国の名目総消費( ( \( p^{H \to W} c^{H \to W} \) ) )}
    \label{fig:results_a_shock_p_H_W_c_H_W}
\end{figure}

\begin{figure}[H]
    \centering
    \includegraphics[width=0.9\textwidth]{comparison_graphs/a_shock/a_shock_p_H_bar_y_H.png}
    \caption{自国の名目GDP( ( \( \bar{p}^H y^H \) ) )}
    \label{fig:results_a_shock_p_H_bar_y_H}
\end{figure}

\begin{figure}[H]
    \centering
    \includegraphics[width=0.9\textwidth]{comparison_graphs/a_shock/a_shock_polt_y_H_and_y_H_potential.png}
    \caption{自国の生産と潜在生産( ( \( y^H \) and \( y^{H,pot} \) ) )( 潜在生産水準目標 )}
    \label{fig:results_a_shock_polt_y_H_and_y_H_potential}
\end{figure}

\begin{figure}[H]
    \centering
    \includegraphics[width=0.9\textwidth]{comparison_graphs/a_shock/a_shock_i_H.png}
    \caption{自国の名目利子率( ( \( i^H \) ) )}
    \label{fig:results_a_shock_i_H}
\end{figure}

\begin{figure}[H]
    \centering
    \includegraphics[width=0.9\textwidth]{comparison_graphs/a_shock/a_shock_gamma_H.png}
    \caption{自国の目標未達分の累積( ( \( \gamma^H \) ) )}
    \label{fig:results_a_shock_gamma_H}
\end{figure}

負の供給ショック( 生産性 ( \( a_t^H \) ) の低下 )に対する各政策の挙動は、AS-AD フレームワークを用いることで視覚的に理解できる 。
本モデルの価格硬直性( ( \( \kappa^H = 0 \) ) )の下では、AS 曲線 \eqref{eq:results_asad_as_curve} は ( \( y, p \) ) 平面において水平である。
生産性の低下は企業の限界費用を押し上げるため、水平な AS 曲線は上方( 物価上昇方向 )へとシフトする。
この供給制約に対し、AD 曲線 \eqref{eq:results_asad_ad_curve} がどのように反応するかが政策の成否を分ける。

\begin{itemize}
    \item \textbf{ショックを緩和できない政策群:} 生産者物価水準目標( PPLT )、潜在生産水準目標( POLT )
    \item \textbf{ショックを緩和できる政策群:} 消費者物価水準目標( CPLT )、名目総消費水準目標( NCLT )、名目GDP水準目標( NGDPLT )、生産水準目標( OLT )、インフレ目標( IT )
\end{itemize}

\paragraph{1. ショックを緩和できない政策( グループ1 )の挙動}
グループ1が不況を増幅させる理由は、AS 曲線の上方シフトに対し、ターゲットを維持するために AD 曲線を強力に左方へシフトさせてしまう点にある。物価水準を一定に保とうとする PPLT や、生産を下落した潜在水準へ合わせようとする POLT は、利子率を引き上げて需要を抑制する。しかし、AS 曲線が水平であるため、この需要抑制は物価を下げる効果を持たず、生産 ( \( y_t^H \) ) の大幅な減少と所得の限界効用 ( \( \lambda_t^H \) ) の急騰を招く。

\paragraph{2. ショックを緩和できる政策( グループ2 )の挙動}
対照的に、NCLT を含むグループ2は、供給制約を物価と生産のトレードオフとして柔軟に処理する。名目支出の維持を目標とするこれらの政策は、生産減少に伴う物価の上昇を許容することで、AD 曲線( 直角双曲線 )をほぼ元の位置に固定する。物価の上昇と実質支出の減少が名目額において相殺されるため、所得の限界効用 ( \( \lambda_t^H \) ) は安定し、グループ1のような深刻な不況を回避できる。

\subsubsection{各政策の分析}

\paragraph{A. 生産者物価水準目標( PPLT )}
負の生産性ショック( ( \( a_t^H \) ) の低下 )に対する PPLT ( 水色点線 )の反応を検討する。本モデルの価格硬直性( ( \( \kappa^H = 0 \) ) )の下では、生産性の低下は企業の限界費用を直接的に押し上げるため、水平な AS 曲線は ( \( y, p \) ) 平面において上方へとシフトする。

図 \ref{fig:results_a_shock_p_H} が示す通り、PPLT は物価の上昇を全政策中で最小限に抑え込み、最も速やかに定常状態へと収束させている。ただし、自国の価格粘着性パラメータを実証データに基づき極めて高く設定( ( \( \xi^H = 0.99 \) ) )しているため 、供給ショックによる自国財価格 ( \( p_t^H \) ) の実際の上昇幅は、他の政策を含め全体として極めて微小なものに留まっている点には留意が必要である。

ここで注目すべきは、図 \ref{fig:results_a_shock_i_H} において政策利子率 ( \( i_t^H \) ) が定常状態付近に留まっているにもかかわらず、PPLT が強力な物価抑制を実現できている点である。このメカニズムは、AS 曲線における将来の価格期待 ( \( E_t [ p_{t+1}^H ] \) ) のアンカー効果によって説明される。

\begin{enumerate}
    \item \textbf{期待を通じた抑制}: AS 曲線によれば、現在の価格水準は将来の価格期待に強く依存する。PPLT は物価の「水準」を目標とするため、ショックによって物価がわずかに上昇した際、中央銀行は将来それを打ち消すためにデフレ的な政策をとることを自動的にコミットすることになる。この「将来は物価が下落する」という家計・企業の期待が、現在の物価上昇圧力を事前に相殺するため、実際の利子率を大きく引き上げるまでもなく、期待形成のみによって物価水準が強固に安定化される。
    \item \textbf{実体経済への影響}: 以上の物価抑制プロセスは、実体経済にとっては需要の過度な抑制として作用する。物価上昇を許容しないためには、AD 曲線 \eqref{eq:results_asad_ad_curve} を大幅に左方へシフトさせ、需要を生産性の低下に見合う以上に押し下げる必要がある。その結果、図 \ref{fig:results_a_shock_c_H_W} および図 \ref{fig:results_a_shock_y_H} が示す通り、消費と生産の初期の落ち込みは POLT に次いで深刻なものとなっている。
\end{enumerate}

なお、需要ショック( ( \( \beta \) ) ショック )においては政策によって物価の動く方向がまちまちであったが、生産性ショックにおいては全政策で物価上昇圧力が一貫して確認される。これは、 ( \( \beta \) ) ショックが需要側を直接叩く( AD 曲線のシフト )性質を持つのに対し、 ( \( a \) ) ショックは供給側のコストを直接引き上げる( AS 曲線のシフト )という構造的な差に由来する。PPLT はこの供給側からのコストプッシュ圧力に対し、実体経済を犠牲にすることで物価目標を死守する挙動を示したと結論付けられる。

\paragraph{B. 潜在生産水準目標( POLT )}
負の生産性ショック( ( \( a_t^H \) ) の低下 )に対する POLT ( 黒破線 )の動態を検討する。本モデルにおける POLT の政策ルールは、実際の生産 ( \( y_t^H \) ) を、価格が伸縮的な場合の効率的な生産水準である潜在生産 ( \( y_t^{H,pot} \) ) に一致させるよう機能する 。

図 \ref{fig:results_a_shock_polt_y_H_and_y_H_potential} において、生産 ( \( y_t^H \) ) と潜在生産 ( \( y_t^{H,pot} \) ) の推移が完全に重なり、かつ初期において大幅な負の値をとっている点は、本ショックの性質と政策目的を端的に示している。

\begin{enumerate}
    \item \textbf{潜在生産の低下とターゲットの移動}: 生産性ショック( ( \( a_t^H \) ) の低下 )が発生すると、経済の供給能力そのものが減退するため、自然な生産水準である ( \( y_t^{H,pot} \) ) は即座に低下する。需要ショック( ( \( \beta \) ) ショック )時に潜在生産が上昇していたのとは対照的であり、これは供給側の制約が経済の実力を直接的に引き下げたことを意味する。
    \item \textbf{厳格なターゲット追随}: 図において生産が潜在生産と一致しているのは、政策ルールにおける乖離への反応係数 ( \( \phi_{gap}^H \) ) が十分に機能し、実際の生産を低下した潜在水準へと強制的に誘導しているためである。
\end{enumerate}

このターゲット追随を実現するための手段が、図 \ref{fig:results_a_shock_i_H} に見られる名目利子率 ( \( i_t^H \) ) の急激な上昇である。他の政策群がゼロ金利制約( ZLB )に直面、あるいは低金利を維持して需要を下支えしようとするのに対し、POLT のみが利子率を大幅に引き上げている。これは、供給能力が低下した経済において過剰需要( インフレ圧力 )が発生するのを防ぐため、中央銀行がオイラー方程式を通じて意図的に現在の消費・生産を押し下げていることを示している。利子率の急騰と生産の急落が鏡像関係にあるのは、この「需要抑制」による潜在生産への強制的な適応が、POLT の安定化メカニズムそのものであるからである。

物価水準 ( \( p_t^H \) ) については、図 \ref{fig:results_a_shock_p_H} が示す通り、供給ショックに伴うコストプッシュ圧力により上昇している。ただし、 ( \( \xi^H = 0.99 \) ) という高い価格粘着性の下では、実際の上昇幅は極めて微小に留まっている。POLT は生産ギャップを一定に保つことで、実体経済側からの追加的なインフレ圧力を排除しているため、インフレ率は時間の経過とともに速やかに定常状態へと収束していく。

以上の分析から、POLT は供給能力の低下という負のショックに対し、需要を自ら抑制することで均衡を図る政策であると言える。実体経済を潜在水準に一致させるという点では論理的に一貫しているが、図 \ref{fig:results_a_shock_c_H_W} が示す通り、全政策中で最も深刻な消費の落ち込みと所得の限界効用 ( \( \lambda_t^H \) ) の急騰を招いている。社会的厚生の観点からは、この過度な需要抑制が損失を拡大させ、実体経済を適度に維持する NCLT 等に比べて低い評価に留まる結果となっている。

\paragraph{C. 消費者物価水準目標( CPLT )}
「ショックを緩和できる政策群」の一つとして、消費者物価水準目標( CPLT、マゼンタ実線 )の動態を分析する。CPLT は自国の消費者物価指数( CPI, ( \( p_t^{H \to W} \) ) )を目標変数とするルールであり、その政策式は以下のように記述される 。

\begin{equation}
i_{t, notional}^H = i_{ss}^H + \phi_{gap}^H ( \ln p_t^{H \to W} - \ln \chi^H ) + \phi_{level}^H \gamma_t^H \label{eq:results_a_shock_analysis_cplt_rule}
\end{equation}

図 \ref{fig:results_a_shock_p_H_W} において、CPLT の下で消費者物価指数が定常状態の線上で完璧に固定されている点は、需要ショック時と同様、ターゲット変数の動学的な制御が極めて容易であることを示している。このメカニズムを、以下の 3 つの視点から解明する。

\begin{enumerate}
    \item \textbf{消費者物価指数( CPI )の制御容易性}: 消費者物価指数の定義式 \( \ln p_t^{H \to W} = \alpha^H \ln p_t^H + (1-\alpha^H) ( \ln e_t^{/*} + \ln p_t^{F*} ) \)  に基づき、物価の安定化プロセスを検討する。生産性ショック( ( \( a \) ) ショック )により自国の生産コストが上昇すると、自国財価格 ( \( p_t^H \) ) には上昇圧力がかかる( 図 \ref{fig:results_a_shock_p_H} )。このとき、中央銀行はジャンプ変数である名目為替レート ( \( e_t^{/*} \) ) を増価( 低下 )させることで、輸入物価の下落を通じて国内価格の上昇分を相殺し、CPI の値を目標値に保つ。図 \ref{fig:results_a_shock_exchange_rate} で為替レートが負の方向に膨らんでいる( 通貨高 )のは、まさにこの相殺メカニズムの結果である。
    \item \textbf{外国財価格 ( \( p_t^{F*} \) ) に関する遮断効果の再確認}: 図 \ref{fig:results_a_shock_p_F_star_sub} において外国財価格が政策によって微動しているように見えるが、縦軸の単位が ( \( 10^{-15} \) ) である点に注目する必要がある。付録 B.2 で証明した通り 、自国の内生変数は外国の変数群から構造的に切り離されている( 遮断効果 )。したがって、マクロ経済学的な意味において、この変動は計算機上の誤差に過ぎず、理論的には外国財価格が不変に維持されていると解釈される。この遮断効果により、自国の中央銀行は外国の物価変動を考慮することなく、為替レートのみを調整手段として CPI を制御することが可能となっている。
    \item \textbf{利子率の安定性と期待管理}: 図 \ref{fig:results_a_shock_i_H} において、CPLT の名目利子率が定常状態の線上でほぼ一直線であるにもかかわらず為替レートを誘導できている理由は、本モデルにおける高い価格粘着性( ( \( \xi^H = 0.99 \) ) )に由来する。供給ショック下でも自国財価格の実際の上昇幅が極めて微小であるため、これを相殺するために必要な為替の調整幅も必然的に小さくなる。結果として、UIP 条件を通じた将来の為替期待をアンカーするだけで十分であり、現行の名目利子率を定常状態から大きく動かす必要がないのである。
\end{enumerate}

以上の分析から、CPLT は自国財価格 ( \( p_t^H \) ) の上昇を為替の増価によって打ち消すことで、名目的な CPI の安定を達成したと言える。自国財価格の水準維持を優先して需要を直接抑制する PPLT に対し、CPLT は為替レートをクッションとして利用できるため、図 \ref{fig:results_a_shock_y_H} における生産の落ち込みは PPLT や POLT よりも緩和されている。このように、CPLT は供給ショックに際して一定の柔軟性を発揮する政策であると評価できる。

\paragraph{D. 名目総消費水準目標( NCLT )}
「ショックを緩和できる政策群」の核心である名目総消費水準目標( NCLT、青実線 )を、負の生産性ショック( ( \( a_t^H \) ) の低下 )の下で分析する。NCLT は名目総消費支出 ( \( p_t^{H \to W} c_t^{H \to W} \) ) をターゲットとしており、その政策ルールは以下の通りである 。

\begin{equation}
i_{t, notional}^H = i_{ss}^H + \phi_{gap}^H ( \ln p_t^{H \to W} + \ln c_t^{H \to W} - \ln \chi_t^H ) + \phi_{level}^H \gamma_t^H \label{eq:results_a_shock_analysis_nclt_policy}
\end{equation}

図 \ref{fig:results_a_shock_p_H_W_c_H_W} が示す通り、NCLT の下で名目総消費支出は定常状態の線上で完璧に固定されている。これは、供給側の制約( 生産性の低下 )によるコストプッシュ圧力に対し、NCLT が名目支出の枠組みを通じて極めて安定的な制御を行っていることを意味する。

本分析において特筆すべきは、供給ショック下における NCLT と名目GDP水準目標( NGDPLT )の同値性である。

\begin{enumerate}
    \item \textbf{理論的同値性}: 本モデルのように自国と外国の主観的割引因子が等しい供給ショック下では、為替レートの調整を介して自国の名目総消費支出と名目GDPは恒等的に一致する( ( \( p_t^H y_t^H = p_t^{H \to W} c_t^{H \to W} \) ) )ことが数学的に証明される。図 \ref{fig:results_a_shock_c_H_W} ( 実質消費 )、図 \ref{fig:results_a_shock_y_H} ( 実質生産 )、図 \ref{fig:results_a_shock_i_H} ( 名目利子率 )、および図 \ref{fig:results_a_shock_lambda_H} ( 所得の限界効用 )において、NCLT と NGDPLT が完全に一致して重なっている点は、この数理的な性質を実証している。
    \item \textbf{所得の限界効用の安定化}: 生産性が低下しても、名目支出を一定に保つことで所得の限界効用 ( \( \lambda_t^H \) ) は定常状態( 対数水準で 0 )に維持される。その結果、家計は消費を抑制する必要がなく、供給側の負のショックが実体経済の不況( 需要の減退 )へと波及する経路がほぼ完全に遮断されている。
\end{enumerate}

物価指数の挙動については、図 \ref{fig:results_a_shock_p_H} および図 \ref{fig:results_a_shock_p_H_W} が示す通り、全政策で共通してわずかな上昇が確認される。これは生産性の低下が企業の限界費用を押し上げ、水平な AS 曲線 \eqref{eq:results_asad_as_curve} を上方へシフトさせるためである。しかし、自国の価格粘着性が極めて高い( ( \( \xi^H = 0.99 \) ) )設定下では、実際の上昇幅は 0.001 未満と極めて低水準に抑えられている。NCLT はこの微小な物価上昇を許容することで AD 曲線 \eqref{eq:results_asad_ad_curve} をほぼ元の位置に固定し、実質変数への悪影響を最小限に食い止め、不況の深刻化を防いでいる。

以上の分析から、名目総消費水準目標( NCLT )は、供給ショックに際して名目GDP水準目標( NGDPLT )と同様の頑健性を発揮し、実体経済の安定を維持する極めて有効な政策枠組みであることが確認された。

\paragraph{E. 名目GDP水準目標( NGDPLT )}
「ショックを緩和できる政策群」の三つ目として、名目GDP水準目標( NGDPLT、緑一点鎖線 )の動態を分析する。NGDPLT は自国財価格( PPI, ( \( \bar{p}^H_t \) ) )と実質生産 ( \( y_t^H \) ) の積をターゲットとするルールであり、その政策式は以下のように記述される 。

\begin{equation}
i_{t, notional}^H = i_{ss}^H + \phi_{gap}^H ( \ln \bar{p}_t^H + \ln y_t^H - \ln \chi_t^H ) + \phi_{level}^H \gamma_t^H \label{eq:results_a_shock_analysis_ngdplt_rule}
\end{equation}

生産性ショック( ( \( a \) ) ショック )下における NGDPLT の挙動は、前節で述べた名目総消費水準目標( NCLT )と極めて高い整合性を示す。この特徴を以下の 2 つの視点から整理する。

\begin{enumerate}
    \item \textbf{名目支出の恒等的な一致と理論的同値性}: 本モデルにおいて、自国と外国の主観的割引因子が等しい状況下で生産性ショックが発生した場合、為替レートの調整を通じて自国の名目総消費支出と名目GDPは以下の通り恒等的に一致する。
    \begin{equation}
    p_t^{H \to W} c_t^{H \to W} = p_t^H y_t^H \label{eq:results_a_shock_nominal_expenditure_identity}
    \end{equation}
    この関係式の詳細な導出については、付録 \ref{chap:appendix_exchange_rate} を参照されたい。この数理的な性質により、生産性ショックにおいて NGDPLT は NCLT と完全に同値な政策として機能し、高い厚生を実現している。
    \item \textbf{シミュレーション結果の整合性}: 図 \ref{fig:results_a_shock_p_H_W_c_H_W} ( 名目総支出 )および図 \ref{fig:results_a_shock_p_H_bar_y_H} ( 名目GDP )のグラフを比較すると、両政策のパスは完全に定常状態の線上で重なっている。また、実質消費 ( \( c_t^{H \to W} \) ) ( 図 \ref{fig:results_a_shock_c_H_W} )、実質生産 ( \( y_t^H \) ) ( 図 \ref{fig:results_a_shock_y_H} )、所得の限界効用 ( \( \lambda_t^H \) ) ( 図 \ref{fig:results_a_shock_lambda_H} )といった主要な実体経済変数においても、NCLT と NGDPLT は一致した軌跡を辿っている。これは、本モデルの均衡決定メカニズムが、ショックの種類に応じて両政策を理論通り同一の解へと収束させていることを実証している。
\end{enumerate}

以上の通り、名目GDP水準目標( NGDPLT )は、生産性ショックに際して名目総消費水準目標( NCLT )と同様の優れた安定化能力を発揮する。名目額の安定を通じて所得の限界効用を一定に保つメカニズムは、供給側のショックが実体経済の深刻な不況を招くのを防ぐ上で、有効な障壁として機能していると言える。

\paragraph{F. 生産水準目標( OLT )}
「ショックを緩和できる政策群」の一つとして、生産水準目標( OLT、黄一点鎖線 )の動態を分析する。OLT は実質生産 ( \( y_t^H \) ) のみをターゲットとする政策ルールであり、次のように定義される 。

\begin{equation}
i_{t, notional}^H = i_{ss}^H + \phi_{gap}^H ( \ln y_t^H - \ln \chi^H ) + \phi_{level}^H \gamma_t^H \label{eq:results_a_shock_analysis_olt_policy}
\end{equation}

図 \ref{fig:results_a_shock_y_H} において、OLT 下の生産パスは負の生産性ショックに対しても定常状態の線上でほぼ完璧に固定されている。これは、本政策が生産水準の維持において極めて強力な制御能力を持っていることを示している。この挙動の背後には、以下のメカニズムが存在する。

\begin{enumerate}
    \item \textbf{ターゲットの直接制御}: OLT は生産 ( \( y_t^H \) ) の目標からの乖離を直接的な操作対象としている。生産性ショック( ( \( a_t^H \) ) の低下 )は企業の限界費用を押し上げ、生産を減少させる方向に働くが、中央銀行は式 \eqref{eq:results_a_shock_analysis_olt_policy} に基づき、生産のわずかな落ち込みに対しても敏感に反応する。その結果、生産は定常状態から殆ど乖離することなく維持される。
    \item \textbf{物価の上昇とインフレの収束}: 図 \ref{fig:results_a_shock_p_H} において、自国財価格 ( \( p_t^H \) ) はインフレ目標( IT )と並び、全政策の中で最も高い上昇を示している。生産性低下によって供給能力が落ちているにもかかわらず、OLT が需要を抑制せずに生産を旧来の水準で維持しようとするため、経済には過剰な需要圧力が生じ、それが物価を押し上げる。ただし、自国の価格粘着性が極めて高い設定であるため、物価の絶対的な上昇幅は 0.001 未満と非常に小さく抑えられている。また、時間の経過とともにインフレ率が 0 に収束しているのは、水準目標としての性質により、物価が新たな均衡水準へと緩やかに落ち着いていくためである。
    \item \textbf{利子率の安定性}: 図 \ref{fig:results_a_shock_i_H} において、OLT の名目利子率 ( \( i_t^H \) ) がほぼ一直線になっているのは、生産 ( \( y_t^H \) ) の制御が極めて高い精度で行われている結果である。ターゲットである生産が殆ど 0 から動かないため、政策ルール内のギャップ項や累積項に大きな乖離が発生せず、名目利子率を大きく動かす必要が生じない。
\end{enumerate}

以上の通り、OLT は生産性ショック下においても実体経済を定常状態に留める上で非常に有効な政策である。しかし、潜在生産が低下している中で無理に生産を維持するため、物価水準が恒久的に高い位置へシフトする「物価のドリフト」を許容している。社会的厚生の観点からは、この物価上昇がもたらす価格分散コストが、NCLT などの名目支出目標に一歩及ばない要因となっている。

\paragraph{G. インフレ目標( IT )}
生産性ショック( ( \( a \) ) ショック )における最後の政策として、インフレ目標( IT、赤点線 )の動態を分析する 。IT はインフレ率の定常状態からの乖離にのみ反応するルールであり、次のように定義される 。

\begin{equation}
i_{t, notional}^H = i_{ss}^H + \phi_{\pi}^H ( \ln \pi_t^H - \ln \pi_{ss}^H ) + \epsilon_t^{i,H} \label{eq:results_a_shock_analysis_it_policy}
\end{equation}

需要ショック( ( \( \beta \) ) ショック )において、IT は期待のアンカーに失敗し厚生を著しく損なう結果となったが、生産性ショック下においては、一転して良好なパフォーマンスを示している。この対照的な挙動の背景には、本モデルの高い価格粘着性が深く関わっている。

\begin{enumerate}
    \item \textbf{インフレ率の極小な変動}: 図 \ref{fig:results_a_shock_pi_H} によれば、自国のグロス・インフレ率 ( \( \pi_t^H \) ) は全政策を通じて極めて微小な変動に留まっている。これは、自国の価格粘着性が極めて高いため、生産性低下による限界費用の増大が実際の価格改定に結びつきにくい「水平な AS 曲線」の性質を反映している。
    \item \textbf{受動的な利子率操作と実体経済の維持}: IT ルール \eqref{eq:results_a_shock_analysis_it_policy} は、インフレ率が動かない限り政策利子率を変更しない。図 \ref{fig:results_a_shock_i_H} が示す通り、IT 下の名目利子率は定常状態の線上でほぼ一直線となっており、中央銀行が積極的な引き締めを行っていないことがわかる。物価水準を一定に保とうとして需要を無理に抑制する PPLT や POLT とは異なり、IT は物価が動き出さない限り需要を叩かないという受動的なスタンスをとる。このため、図 \ref{fig:results_a_shock_y_H} において生産 ( \( y_t^H \) ) は定常状態付近で極めて安定しており、不必要な不況の発生を回避できている。
    \item \textbf{物価水準のシフト( ドリフト )}: 一方で、図 \ref{fig:results_a_shock_p_H} において、自国財価格 ( \( p_t^H \) ) は OLT と並んで全政策中で最も高い上昇を示している。IT はインフレ率( 変化率 )のみを凝視し、過去の上昇分を埋め合わせる水準の概念を持たない。そのため、供給ショックによって一度上方へシフトした物価水準を元の位置( 0 )へ引き戻す力が働かず、高い位置に留まることを許容する。しかし、供給ショック時においては、この「物価を無理に戻さない」という柔軟さが、結果として生産の安定を支える要因となっている。
\end{enumerate}

以上の分析から、生産性ショック下におけるインフレ目標( IT )は、価格硬直性が極めて高いという条件下において、事実上の利子率据え置きに近い効果をもたらし、それが実体経済の安定化に寄与したと言える。社会的厚生の観点からは、物価の上昇幅がごくわずかであるため、価格分散コストの影響も限定的であり、NCLT や OLT と並んで高い評価を得る結果となった。

\paragraph{生産性ショックにおける厚生の総括:名目アンカーのトレードオフと今後の課題}
本節の締めくくりとして、負の生産性ショック( ( \( a \) ) ショック )における各政策の厚生評価を総括する。図 \ref{fig:results_a_shock_utility_with_delta} が示す通り、本シミュレーションの条件下では IT、OLT、CLT が NCLT を僅差で上回る結果となった。この結果の解釈と、そこから導出される理論的な課題について以下に整理する。

\begin{enumerate}
    \item \textbf{物価と消費の負の相関による「静観」のメカニズム}: 生産性低下に伴うコストプッシュ圧力により物価が上昇する一方、実体経済には下落圧力がかかる供給ショック下では、物価( ( \( p \) ) )と実質消費( ( \( c \) ) )は負の相関を持つ。このとき、名目総消費支出( ( \( p \cdot c \) ) )をターゲットとする NCLT の下では、物価の上昇と実質支出の減少が名目額において相殺されるため、中央銀行は利子率を大きく操作せず経済の動態を静観するスタンスをとる。本シミュレーションにおいて NCLT の厚生評価が実質変数目標( OLT, CLT )に一歩譲った主因は、この名目支出の維持という枠組みが、物価上昇を許容する代償として実質消費のわずかな減退を容認した点にある。
    \item \textbf{物価安定化による厚生改善の可能性}: しかし、NCLT によるこうした静観的な対応は、自国財価格の際限のない上昇を抑え込む名目アンカーとしても機能している。社会的厚生の観点からは、物価変動の抑制は価格分散コスト( ( \( \Delta^H \) ) )を減少させ、資源配分の歪みを是正するプラスの側面を持つ。本モデルの設定( ( \( \xi^H = 0.99 \) ) )では物価の変動圧力が極めて小さいため、価格分散によるペナルティが殆ど発生せず、実質値を固定した OLT 等が相対的に有利な結果となった。しかし、もし供給ショックによって物価がより大きく動くような経済環境であれば、NCLT による物価安定化の恩恵が実質消費抑制のコストを上回り、NCLT が実質変数目標を凌駕する可能性は十分に考えられる。
    \item \textbf{頑健性の評価と今後の課題}: 以上の分析から、NCLT は需要ショックにおいて圧倒的な優位性を示す一方で、供給ショック下では「物価の安定」と「実体経済の維持」の間に複雑なトレードオフを抱えていることが明らかになった。本稿の結果のみを以て供給ショック下で NCLT が実質変数目標に劣ると断定することはできず、両者の優劣はショックの性質や価格粘着性の度合いに強く依存すると推察される。したがって、生産性ショックによる価格変動がより激しいケースや、異なるパラメータ条件下での厚生への影響を詳細に検証することは、名目総消費水準目標の更なる頑健性を明らかにする上での今後の重要な研究課題である。
\end{enumerate}

以上の比較分析を総合すると、名目総消費水準目標( NCLT )は、需要・供給の両ショックに直面する現実の経済環境において、致命的な厚生損失を回避しつつ、経済を安定軌道に留めることができる極めてバランスの良い政策体系であると評価できる。

% --- 第5章の終わり ---
% !TeX root = ../main.tex
% sections/chapter6.tex

\chapter{考察}
\label{chap:discussion}
第5章のシミュレーション分析はゼロ金利制約が存在する物価が硬直的な経済において
名目総消費水準目標がマクロ経済を安定させることを明らかにした。
本章ではこの名目総消費水準目標について AS-AD 分析によるまとめの解釈を与え実務面から補足もおこなう。

\section{AS-AD 分析による名目総消費水準目標の働きのまとめ}
\label{sec:discussion_as_ad}
ここでは AS-AD 分析を用いて名目総消費水準目標の効果について総括する。
AS 曲線は期待生産者物価指数 \( E_t [p_{t+1}^H] \) によって動かされ、
AD 曲線は名目利子率 \( i_t^H \) と所得の期待限界効用 \( E_t [\lambda_{t+1}^H] \) によって動かされるのであった。

\subsection{\( \beta \) ショック(需要ショック)への反応}
\label{sec:discussion_as_ad_beta_shock}
負の \( \beta^H \) ショックにおいて名目総消費水準目標が優れた成績を残した要因として
名目利子率 \( i_t^H \) と所得の期待限界効用 \( E_t [\lambda_{t+1}^H] \) を
効果的に下落させたことが考えられる。

\paragraph{1. 名目利子率 \( i_t^H \)}
負の \( \beta^H \) ショックは自国家計の財布のひもを固くするショックであるから
とりわけ総消費指数 \( c_t^{H \to W} \) を大幅に低下させる。
これにより名目総消費 \( p_t^{H \to W} c_t^{H \to W} \) も大幅に低下するため
名目総消費水準目標により自国の利子率 \( i_t^H \) は大きく下げられ
ゼロ金利に達してからの目標未達分の累積も大きくなる。
これにより AD 曲線は右方に移動する。
\paragraph{2. 所得の期待限界効用 \( E_t [\lambda_{t+1}^H] \)}
1階の条件(FOC)より名目総消費 \( p_{t+1}^{H \to W} c_{t+1}^{H \to W} \) は
所得の限界効用 \( \lambda_{t+1}^H \) の逆数であった。
付録Aの「割引因子ショックを含む為替レート決定式の導出」の期待値の議論より、
この逆数の関係はこれらの期待値についても近似的に成立する。
したがって名目総消費水準目標は期待名目総消費 \( E_t [p_{t+1}^{H \to W} c_{t+1}^{H \to W}] \) を上昇、
よって所得の期待限界効用 \( E_t [\lambda_{t+1}^H] \) を下落させる。
これにより AD 曲線は右方へと移動する。
\paragraph{3. 期待生産者物価指数 \( E_t [p_{t+1}^H] \)}
名目総消費指数 \( p_t^{H \to W} c_t^{H \to W} \) には消費者物価指数 \( p_t^{H \to W} \) の一部として
生産者物価指数 \( p_t^H \) が含まれる。
付録Aの「割引因子ショックを含む為替レート決定式の導出」の期待値の議論より、
この逆数の関係はこれらの期待値についても近似的に成立する。
そのため名目総消費水準目標は期待名目総消費 \( E_t [p_{t+1}^{H \to W} c_{t+1}^{H \to W}] \) を上昇、
よって期待生産者物価指数 \( E_t [p_{t+1}^H] \) の下落を抑えた。
これにより AS 曲線はほとんど動かない。


こうして AD 曲線を大きく右方移動させた一方で、AS 曲線はおおよそ初期位置にとどめたため
経済を強力に回復へと向かった。


\subsection{\( a \) ショック(供給ショック)のまとめ}
\label{sec:discussion_as_ad_a_shock}
第 5 章の分析結果から負の \( a^H \) ショックに対しても名目総消費水準目標は高い頑健性を示すことが明らかになった。
付録 \ref{app:appendix_exchange_rate} 式 \eqref{eq:appendix_exchange_rate_final} で証明したとおり
本稿のモデルでは \( a^H \) ショック時には名目総消費と名目 GDP は一致する。
これにより名目総消費水準目標は名目 GDP 水準目標と同値な政策となる。
\paragraph{1. 名目利子率  \( i_t^H \)}
負の \( a^H \) ショックは生産性を低下させるショックであるから生産 \( y_t^H \) を低下させるかのようにみえる。
しかしながら本稿モデルの価格硬直性の仮定により AS 曲線の傾き \( \kappa = 0 \) となり
\( a^H \) 自体の変動は AS 曲線を動かさない。
これにより生産 \( y_t^H \) や生産者物価指数 \( p_t^H \) は動かない。
したがって名目総消費水準目標は名目金利 \( i_t^H \) を動かさない。
これにより AD 曲線は動かない。
\paragraph{2. 所得の期待限界効用 \( E_t [\lambda_{t+1}^H] \)}
前節において名目消費目標に関する主要な研究を概観した段落同様、負の \( a^H \) ショックでは生産 \( y_t^H \) や生産者物価指数 \( p_t^H \) はほとんど動かず
したがって名目総消費水準目標は名目金利 \( i_t^H \) を動かさない。
よって生産 \( y_t^H \) および生産者物価指数 \( p_t^H \) は将来においても動かないことが予想されるため
期待生産 \( E_t [y_t^H] \) および期待生産者物価指数 \( E_t [p_{t+1}^H] \) も変動しない。
したがって期待名目総消費 \( E_t [p_{t+1}^{H \to W} c_{t+1}^{H \to W} = \bar{p}_{t+1}^H y_{t+1}^H] \) も変動しないから
その逆数である所得の期待限界効用 \( E_t [\lambda_{t+1}^H] \) も変動しない。
これにより AD 曲線は動かない。
\paragraph{3. 期待生産者物価指数 \( E_t [p_{t+1}^H] \)}
全段落で述べたように、本稿の価格が硬直的なモデルにおいては
\( a^H \) ショックでは生産 \( y_t^H \) や生産者物価指数 \( p_t^H \) は動かず
したがって名目総消費水準目標は名目金利 \( i_t^H \) を動かさない。
よって生産者物価指数 \( p_t^H \) は将来においても動かないことが予想されるため
期待生産者物価指数 \( E_t [p_{t+1}^H] \) も変動しない。
これにより AS 曲線は動かない。


このように名目総消費水準目標は負の \( a^H \) ショックに際しては不必要な政策介入をおこなわず
生産の維持された経済をただ静観する。


\section{実務面からみた名目総消費水準目標の実現可能性}
\label{sec:discussion_practicality}
実務において金融政策を採用できるかどうかは目標とする指標の速報性および正確性に依存する。
本節では指標作成の実務的観点から名目総消費水準目標の実現可能性を考察する。

\subsection{主要統計の作成主体と現状の課題}
まず本稿が比較対象とする主要3指標の作成過程を整理する(表 \ref{tab:stat_comparison})。
\begin{table}[H]
\centering
\caption{主要マクロ統計の特性比較}
\label{tab:stat_comparison}
\small
\begin{tabular}{lllp{20em}}
\toprule
指標 & 作成主体 & 公表頻度 & 特徴と実務上の課題 \\
\midrule
名目 GDP & 内閣府 & 四半期 & 確定に時間を要し、大幅な遡及改定が頻発する。 \\
消費者物価指数 (CPI) & 総務省 & 月次 & 品質調整(ヘドニック法)等の統計的加工が複雑。 \\
消費活動指数(名目) & 日本銀行 & 月次 & 速報性は高いが、既存統計に基づく推計値の側面が強い。 \\
\bottomrule
\end{tabular}
\end{table}
日本銀行が公表する「消費活動指数」は、供給側(商業動態統計等)と需要側(家計調査等)のデータを統合した速報性の高い指標であるが、現時点では後に算出される国民経済計算(GDP 統計)の確定値に比して正確性で劣るという課題がある。

\subsection{デジタル技術による正確性と速報性の両立}
しかし近年のデジタル技術の発展は統計の正確性を劇的に向上させる可能性を秘めている。
\begin{itemize}
    \item \textbf{ビッグデータの活用:} クレジットカードや電子マネーの決済データ(キャッシュレスデータ)のリアルタイム集計により、サンプル調査に頼らない全数に近い消費動向の把握が可能となる。
    \item \textbf{POS データの連携:} 小売店の販売時点情報管理(POS)データは、従来のアンケート方式よりも迅速かつ正確に「実際に支払われた名目額」を記録する。
\end{itemize}
これらの技術により現在の日銀統計が持つ「速報性」を維持したまま「正確性」が補完されることで
名目総消費水準目標を現実的な政策ルールとして運用するための技術的基盤が整いつつある。

\subsection{他政策に対する集計上の優位性}
以上の実務的背景を踏まえ、名目総消費水準目標がもつ相対的な優位性を以下の通り整理する。
\paragraph{1. 名目 GDP 水準目標(NGDPLT)に対する優位性}
GDP は消費以外に投資、政府支出、在庫変動、純輸出を網羅する必要がある。
特に設備投資や在庫の推計は測定誤差が大きく、速報値と確報値の乖離(遡及改定)の主因となる。
対して名目総消費は、取引頻度が高くデジタル決済との親和性が強いため、GDP 全体よりも早期かつ正確な集計が可能である。
\paragraph{2. インフレ目標(IT)およびテーラールール(TR)に対する優位性}
名目総消費水準目標は伝統的なインフレ目標(IT)やテーラールール(TR)に対しても集計上の優位性をもつ。
まず実務上のインフレ目標やテイラールールが参照する消費者物価指数(CPI)の算出には
製品の性能向上を価格下落とみなす「品質調整」や「代替バイアス」の処理など高度に専門的な統計的調整を要する。
このような統計的加工の複雑さはデータの公表に際して不可避なタイムラグを生じさせるだけでなく
推計の前提となる仮定の置き方次第で数値が変動しうるという不確実性を伴う。
これに対して名目総消費は取引された名目額そのものであり確実性が高い。
さらにテーラールールは潜在生産(potential output)という推定の困難な情報にも依存している。
第 2 章でも述べた通り潜在生産の誤推計は政策判断を誤らせ
経済を不安定化させる主因となりうる \parencite{Orphanides2003, BeckworthHendrickson2019}。
対して名目総消費は実際に観測可能な指標であるため、潜在産出のような誤推計の問題は生じにくい。


\section{政策意図の浸透}
\label{sec:discussion_communication}
実務における金融政策の効果は中央銀行の意図が家計に正しく伝わり適切な期待を形成できるかに依存する。
この観点からも名目総消費水準目標は名目 GDP 水準目標に対して明確な優位性をもつ。
元 FRB 局長の Nathan Sheets が指摘したように
 GDP は家計にとってはなじみが浅いためその概念が十分に理解されているとは言えない。
対照的に消費は家計の生活に直結した指標である。
そのため「中央銀行が皆さんの名目的な消費水準を将来にわたって保証する」という約束は
家計の貯蓄・消費判断に対して極めて具体的かつ直感的な指針を与える。
こうして消費を標的とすることは、とりわけゼロ金利下において
中央銀行の景気回復への約束を家計に浸透させ期待を形成するための強力な手段となりうる。


\section{モデルの限界と今後の研究課題}
\label{sec:discussion_limitations}
本稿の分析はゼロ金利制約下における名目総消費水準目標の優位性を理論的、実務的に明らかにしたが、
理論的評価の厳密性と実務的適用可能性をさらに高めるためには以下の課題を検討していく必要がある。

\subsection{効用関数の一般化}
\label{sec:discussion_limitations_preference}
本稿では家計の自国財と外国財の代替弾力性 \( \eta = 1 \) および
対数効用(相対的リスク回避度 \( \sigma = 1 \))を仮定した。
これらの仮定により 1 階の条件(FOC)において所得の限界効用 \( \lambda^H \) が
名目総消費支出 \( p^{H \to W} c^{H \to W} \) の逆数となり、さらに外国関連変数への遮断効果が生じた。
しかしこれらのパラメータが 1 ではない効用関数の下では
名目総消費目標のもつ期待への働きかけの効果が変化し、外国関連変数の動きも思わぬ波及をもたらす可能性がある。
これらの効用パラメータを一般化した場合の政策評価は理論の汎用性を高める上での課題となる。

\subsection{家計の異質性の導入}
\label{sec:discussion_limitations_heterogeneity}
効用関数の一般化と関連して家計の異質性の導入が考えられる。
本稿は家計パラメータの同一性と国内債券市場の完備性を仮定することで
消費や最適生産者物価を全家計で共通化し代表的家計モデルによる分析をおこなった。
しかし現実には家計の選好や物価改定確率は一様ではなく、
またあらゆるリスクに対して債券が完備されているわけでもない。
そこでたとえば主観的割引因子 \( \beta \) を高めた我慢強い家計、
低めたその日暮らしの家計が混在するモデルを作ることもできる。
また国内債券市場が完備でないモデルを考えれば
価格改定機会の運により生まれた生産所得の格差が保険によって均一化されないため
可処分所得が一致せず、消費に違いが生まれる。
こうした多様な家計が存在する経済における政策評価も今後の課題である。

\subsection{経済の硬直性の緩和}
\label{sec:discussion_limitations_rigidity}
本分析では日本経済の実証データ等にもとづき物価の強い硬直性( \( \xi^H = 0.99 \) )や
ショックの強い持続性を仮定している。
これは名目総消費目標のもつ期待への強い働きかけの能力を最大限に活かすための設定であった。
経済にこうした硬さがないのであれば
名目総消費目標のもつ \( \beta \) ショックへの優位性は薄れる可能性がある。
また物価が硬直的でないとすると \( a \) ショックが起こった際に価格はより上昇するだろうから
名目総水準目標においては生産の下落が予想される。
このようなときに名目利子率に動きは出るのか、あるいは依然として静観するのだろうか。
こうした検証をおこなうには、より広範なパラメータ空間で分析をおこなう必要がある。

% !TeX root = ../main.tex
% sections/chapter6.tex

\chapter{結論}
\label{chap:conclusion}
本稿はゼロ金利制約に直面した経済において名目総消費水準目標がいかに効率的に家計の期待に働きかけ
経済を迅速に回復させるかを負の \( \beta \) (主観的割引因子)ショックのシミュレーションにより確認した。
そこにおいて名目消費水準目標は名目利子率 \( i_t^H \) と
所得の期待限界効用 \( E_0[\lambda_t^H] \) に最も効果的に下落させ厚生の落ち込みを最小限にとどめた。
また負の \( a \) (生産性)ショックに対しても名目総消費水準目標の頑健性が示された。
本稿の物価硬直性の仮定においては \( a \) (生産性)ショックが起こっても物価と生産はほとんど動かない。
このとき名目総消費目標は静観の姿勢をとることで経済に余計な刺激を与えなかったのである。


実務面では指標集計と政策意図の浸透について名目総消費水準目標の優位性を論じた。
集計面においては、指標としての名目総消費は消費者物価指数や潜在生産に比べて正確な予測が容易であり、
また速報性の高い数値が既に提供されている点も適時適切な政策判断を下す上で大きな利点となる。
正確性と速報性の両立については現時点で依然として課題が残るものの、
決済データのリアルタイム集計やビッグデータの活用といったデジタル技術の進展により
段階的に解消されることが期待される。
くわえて家計にとって馴染み深い消費を目標に据えることは
中央銀行の政策意図を広く浸透させ円滑な期待形成を図る上で極めて有効である。


今後の本研究の課題としては効用関数の一般化、家計の異質性の導入、および経済の硬直性の緩和の3点が挙げられる。
効用関数の一般化や家計の異質性の導入はより多様な家計からなる現実的な経済の分析を可能にする。
また本稿の分析は物価が硬直的でありショックの持続性が強いという硬直的な自国の仮定にもとづいていた。
これは自国を日本と想定し設定したものであったが、
この仮定を緩めたときに依然として名目総消費水準目標の優位性が保たれるかの検証は今後の課題となる。


% ===== 後付 (参考文献・付録) =====
\printbibliography[title=参考文献]

\appendix
% !TeX root = ../main.tex
% sections/appendix.tex

\appendix
\label{chap:appendix} % 付録セクション全体の開始点

% =============================================================
% 1. プログラムと構成( 先頭に配置 )
% =============================================================
% !TeX root = ../../main.tex
% sections/app/appendix_programs_overview.tex

\chapter{数値計算プログラムの構成と実行環境}
\label{chap:appendix_programs_overview}

本付録では名目総消費水準目標のシミュレーションに使用したプログラムを提示する。
なお本稿のシミュレーションにおいて用いたプログラムの全体は以下のリポジトリにおいて公開している。

\url{https://github.com/nanazou/masters_thesis}

\section*{数値計算の実行環境とアルゴリズム}

シミュレーションの実行にあたっては、Dynare 6.3 および Octave 9.4.0 を使用した。特に、名目金利のゼロ金利制約( ZLB )を厳密に考慮するため、Guerrieri and Iacoviello (2015) によって開発された \texttt{Occbin} ツールキットを導入し、レジームスイッチングを伴う非線形動学を算出している。

なお、Octave 上で Dynare 等のパスを通すためのセットアップ・スクリプト( \texttt{env\_setup} 等 )を実行する場合は、まずコマンドウィンドウ上で該当するスクリプトが保存されているフォルダーまでカレントディレクトリを移動してから実行することに留意されたい。

\section*{プログラムの構成とディレクトリ構造}

本プロジェクト( \texttt{nclt\_project} )の主要なプログラムのディレクトリ構造は以下の通りである。なお、Dynare が自動生成する中間ファイルやシミュレーション結果の出力データについては、記述の簡潔さのため省略している。

\begin{verbatim}
nclt_project/
├── nclt_main.m                   % メイン・実行制御スクリプト
├── nclt_find_optimal_params.m    % 最適政策パラメータ探索スクリプト
└── src/                          % モデル定義・ソルバー関連
    ├── nclt_model.mod            % Dynareモデル定義(通常時)
    ├── nclt_model_zlb.mod        % Dynareモデル定義(ZLB考慮時)
    ├── nclt_model_declarations.inc      % 変数・パラメータの宣言
    ├── nclt_model_equations_common.inc  % モデルの共通方程式
    └── nclt_solve_core.m         % 均衡解法およびシミュレーション基幹部
\end{verbatim}

\section*{ソースコードの依存関係}

提示するプログラムは、以下の 3 つの役割に大別される。
\begin{enumerate}
    \item \textbf{モデル定義と共通方程式}: \texttt{src/} フォルダ内の \texttt{.mod} ファイルおよび \texttt{.inc} ファイルに、第 3 章で導出した非線形方程式体系を記述している。
    \item \textbf{シミュレーション実行スクリプト}: \texttt{nclt\_main.m} を通じて \texttt{nclt\_solve\_core.m} を呼び出し、Dynare を用いて各ショックに対するインパルス応答関数( IRF )を算出する。
    \item \textbf{政策パラメータの最適化}: \texttt{nclt\_find\_optimal\_params.m} を用い厚生を最大化する最適な政策パラメータの値をグリッドサーチによって特定する。
\end{enumerate}

次節以降に、各ソースコードの具体的内容を順次掲載する。

% =============================================================
% 2. 第 3 章( モデル構築 )で参照される基礎的な導出
% =============================================================
% !TeX root = ../../main.tex
% sections/app/appendix_cost_minimization.tex

\chapter{費用最小化問題からの需要関数と価格指数の導出( 第 1 段階 )}
\label{chap:appendix_cost_minimization}

本付録の目的は、総名目生産額が、集計産出量と真の価格指数の単純な積で表せること、そしてその関係を担保するために集計産出量が必然的にCES集計の形でなければならないことを証明することである。

\section*{定理 2:個別財への需要関数と真の価格指数}

家計 \( h \) が、ある時点 \( t \) において、所与の量の国内財バスケット \( c_t^{h \to H} = \left[ \sum_{h' \in H} (c_t^{h \to h'})^{\frac{\theta^H-1}{\theta^H}} \right]^{\frac{\theta^H}{\theta^H-1}} \) を、個別財の価格 \( \{p_t^{h'}\} \) の下で費用最小化的に購入するとき、以下の関係が成立する。
\begin{enumerate}
    \item 家計 \( h \) の個別財 \( h' \) への需要関数は、次式で与えられる。
    \[ c_t^{h \to h'} = \left( \frac{p_t^{h'}}{p_t^H} \right)^{-\theta^H} c_t^{h \to H} \]
    \item 上記の需要関数に現れる \( p_t^H \) は、国内財バスケットの「 真の価格指数 」であり、個々の財価格から以下のように定義される。この値は全ての国内家計で共通である。
    \[ p_t^H = \left[ \sum_{h' \in H} (p_t^{h'})^{1-\theta^H} \right]^{\frac{1}{1-\theta^H}} \]
\end{enumerate}

\section*{証明}

この定理を、自国家計 \( h \) の場合について証明する。外国の家計 \( f \) についても全く対称的な手順で証明可能である。

\subsection*{1. 費用最小化問題の設定}
家計が直面する問題は、時点 \( t \) において、所与の国内財バスケット \( c_t^{h \to H} \) を達成するための総費用を最小化することである。
\begin{itemize}
    \item \textbf{最小化対象}:
    \[ \min_{\{c_t^{h \to h'}\}_{h' \in H}} \sum_{h' \in H} p_t^{h'} c_t^{h \to h'} \]
    \item \textbf{制約条件}:
    \[ c_t^{h \to H} = \left[ \sum_{h' \in H} (c_t^{h \to h'})^{\frac{\theta^H-1}{\theta^H}} \right]^{\frac{\theta^H}{\theta^H-1}} \]
\end{itemize}

\subsection*{2. ラグランジュ関数と一階の条件( FOC )}
この問題を解くために、ラグランジュ関数 \( \mathcal{L}^{h \to H} \) を設定する。ラグランジュ乗数を \( \mu_t^{h \to H} \) とする。
\[
\mathcal{L}^{h \to H} = \sum_{h' \in H} p_t^{h'} c_t^{h \to h'} - \mu_t^{h \to H} \left( \left[ \sum_{h' \in H} (c_t^{h \to h'})^{\frac{\theta^H-1}{\theta^H}} \right]^{\frac{\theta^H}{\theta^H-1}} - c_t^{h \to H} \right)
\]
このラグランジュ関数を、任意の個別財 \( c_t^{h \to h'} \) で偏微分し、ゼロと置くことで一階の条件( FOC )が得られる。
\begin{align*}
\frac{\partial \mathcal{L}^{h \to H}}{\partial c_t^{h \to h'}} = p_t^{h'} - \mu_t^{h \to H} \cdot \frac{\theta^H}{\theta^H-1} \left[ \sum_{i \in H} (c_t^{h \to i})^{\frac{\theta^H-1}{\theta^H}} \right]^{\frac{\theta^H}{\theta^H-1}-1} \cdot \frac{\theta^H-1}{\theta^H} (c_t^{h \to h'})^{\frac{\theta^H-1}{\theta^H}-1} &= 0 \\
p_t^{h'} &= \mu_t^{h \to H} \cdot \left( c_t^{h \to H} \right)^{\frac{1}{\theta^H}} \cdot (c_t^{h \to h'})^{-\frac{1}{\theta^H}} \\
p_t^{h'} &= \mu_t^{h \to H} \left( \frac{c_t^{h \to H}}{c_t^{h \to h'}} \right)^{\frac{1}{\theta^H}}
\end{align*}

\subsection*{3. FOC からの需要関数の導出}
上記で得られた FOC を、個別財への需要量 \( c_t^{h \to h'} \) について解く。
\begin{align*}
    \frac{p_t^{h'}}{\mu_t^{h \to H}} &= \left( \frac{c_t^{h \to H}}{c_t^{h \to h'}} \right)^{\frac{1}{\theta^H}} \\
    \left(\frac{p_t^{h'}}{\mu_t^{h \to H}}\right)^{\theta^H} &= \frac{c_t^{h \to H}}{c_t^{h \to h'}}
\end{align*}
これにより、ラグランジュ乗数 \( \mu_t^{h \to H} \) を含む需要関数が導出される。
\[ c_t^{h \to h'} = \left( \frac{p_t^{h'}}{\mu_t^{h \to H}} \right)^{-\theta^H} c_t^{h \to H} \]

\subsection*{4. 真の価格指数 \( p_t^H \) の導出}
次に、ラグランジュ乗数 \( \mu_t^{h \to H} \) の具体的な形を導出する。ステップ 3 で得た需要関数を、制約条件である CES 集計式に代入する。
\[
c_t^{h \to H} = \left[ \sum_{h' \in H} \left( \left( \frac{p_t^{h'}}{\mu_t^{h \to H}} \right)^{-\theta^H} c_t^{h \to H} \right)^{\frac{\theta^H-1}{\theta^H}} \right]^{\frac{\theta^H}{\theta^H-1}}
\]
式を整理していく。
\begin{align*}
    c_t^{h \to H} &= \left[ \sum_{h' \in H} \left( \frac{p_t^{h'}}{\mu_t^{h \to H}} \right)^{1-\theta^H} (c_t^{h \to H})^{\frac{\theta^H-1}{\theta^H}} \right]^{\frac{\theta^H}{\theta^H-1}} \\
    &= \left[ (c_t^{h \to H})^{\frac{\theta^H-1}{\theta^H}} (\mu_t^{h \to H})^{\theta^H-1} \sum_{h' \in H} (p_t^{h'})^{1-\theta^H} \right]^{\frac{\theta^H}{\theta^H-1}} \\
    &= (c_t^{h \to H}) \cdot (\mu_t^{h \to H})^{\theta^H} \cdot \left[ \sum_{h' \in H} (p_t^{h'})^{1-\theta^H} \right]^{\frac{\theta^H}{\theta^H-1}}
\end{align*}
両辺の \( c_t^{h \to H} \) を消去し、 \( \mu_t^{h \to H} \) について解く。
\begin{align*}
    1 &= (\mu_t^{h \to H})^{\theta^H} \left[ \sum_{h' \in H} (p_t^{h'})^{1-\theta^H} \right]^{\frac{\theta^H}{\theta^H-1}} \\
    (\mu_t^{h \to H})^{-\theta^H} &= \left[ \sum_{h' \in H} (p_t^{h'})^{1-\theta^H} \right]^{\frac{\theta^H}{\theta^H-1}} \\
    \mu_t^{h \to H} &= \left[ \sum_{h' \in H} (p_t^{h'})^{1-\theta^H} \right]^{\frac{1}{1-\theta^H}}
\end{align*}
このラグランジュ乗数 \( \mu_t^{h \to H} \) は、個々の家計 \( h \) に依存しない共通の価格リストのみで決定されるため、全ての国内家計で共通の値をとる。
本稿では、この家計が直面する真の価格指数を \( p_t^H \) と定義する。
\[ p_t^H \equiv \mu_t^{h \to H} = \left[ \sum_{h' \in H} (p_t^{h'})^{1-\theta^H} \right]^{\frac{1}{1-\theta^H}} \]
この \( p_t^H \) をステップ 3 の需要関数に代入することで、定理の項目 1 が証明される( 証明終 )。      % 家計の需要関数
% !TeX root = ../../main.tex
% sections/app/appendix_cpi_derivation.tex

\chapter{総消費価格指数と財バスケットへの需要関数の導出( 第 2 段階 )}
\label{chap:appendix_cpi_derivation}

\section*{定理 3:財バスケットへの需要関数と総消費価格指数}

家計 \( h \) が、ある時点 \( t \) において、所与の量の総消費バスケット \( c_t^{h \to W} \) を、国内財バスケットの真の価格指数 \( p_t^H \) と外国財バスケットの真の価格指数 \( p_t^{F*} \) の下で費用最小化的に購入するとき、以下の関係が成立する。
\begin{enumerate}
    \item 国内財バスケット \( c_t^{h \to H} \) および外国財バスケット \( c_t^{h \to F} \) への需要関数は、それぞれ次式で与えられる。
    \[ c_t^{h \to H} = \alpha^H \frac{p_t^{H \to W}}{p_t^H} c_t^{h \to W} \]
    \[ c_t^{h \to F} = (1-\alpha^H) \frac{p_t^{H \to W}}{e_t p_t^{F*}} c_t^{h \to W} \]
    \item 上記の需要関数に現れる \( p_t^{H \to W} \) は、総消費バスケットの価格指数( CPI )であり、各財バスケットの真の価格指数から以下のように定義される。
    \[ p_t^{H \to W} = (p_t^H)^{\alpha^H} (e_t p_t^{F*})^{1-\alpha^H} \]
\end{enumerate}

\section*{証明}

この定理を、自国家計 \( h \) の場合について証明する。

\subsection*{1. 費用最小化問題の設定}
この第 2 段階では、家計が時点 \( t \) において、所与の総消費量 \( c_t^{h \to W} \) を達成するために、国内財バスケット \( c_t^{h \to H} \) と海外財バスケット \( c_t^{h \to F} \) をどのように組み合わせれば総費用を最小化できるかを分析する。
\begin{itemize}
    \item \textbf{最小化すべき総費用}:
    \[ \min_{c_t^{h \to H}, c_t^{h \to F}} \quad p_t^H c_t^{h \to H} + e_t p_t^{F*} c_t^{h \to F} \]
    \item \textbf{制約条件}( 目標とする正規化されたバスケットの量 ):
    \[ c_t^{h \to W} \equiv \frac{(c_t^{h \to H})^{\alpha^H} (c_t^{h \to F})^{1-\alpha^H}}{(\alpha^H)^{\alpha^H} (1-\alpha^H)^{1-\alpha^H}} \]
\end{itemize}

\subsection*{2. ラグランジュ関数と一階の条件( FOC )}
制約式の対数を取ると扱いやすい。ラグランジュ関数 \( \mathcal{L}^{h \to W} \) を設定する( この段階のラグランジュ乗数を \( \eta_t^h \) とする )。
\[ \mathcal{L}^{h \to W} = p_t^H c_t^{h \to H} + e_t p_t^{F*} c_t^{h \to F} - \eta_t^h \left( \alpha^H \ln c_t^{h \to H} + (1-\alpha^H) \ln c_t^{h \to F} - \ln c_t^{h \to W} - \text{const.} \right) \]
これを \( c_t^{h \to H} \) と \( c_t^{h \to F} \) でそれぞれ偏微分してゼロと置くと、以下の一階の条件( FOC )が得られる。
\begin{align*}
\frac{\partial \mathcal{L}^{h \to W}}{\partial c_t^{h \to H}} = p_t^H - \eta_t^h \frac{\alpha^H}{c_t^{h \to H}} = 0 \quad &\implies \quad p_t^H c_t^{h \to H} = \alpha^H \eta_t^h \\
\frac{\partial \mathcal{L}^{h \to W}}{\partial c_t^{h \to F}} = e_t p_t^{F*} - \eta_t^h \frac{1-\alpha^H}{c_t^{h \to F}} = 0 \quad &\implies \quad e_t p_t^{F*} c_t^{h \to F} = (1-\alpha^H) \eta_t^h
\end{align*}

\subsection*{3. 需要関数の導出( \( \eta_t^h \) を含む形 )}
上記の一階の条件をそれぞれ \( c_t^{h \to H} \) と \( c_t^{h \to F} \) について解くと、ラグランジュ乗数 \( \eta_t^h \) を含む形で各バスケットへの需要関数が得られる。
\[ c_t^{h \to H} = \frac{\alpha^H \eta_t^h}{p_t^H} \quad , \quad c_t^{h \to F} = \frac{(1-\alpha^H) \eta_t^h}{e_t p_t^{F*}} \]

\subsection*{4. ラグランジュ乗数 \( \eta_t^h \) の導出}
ステップ 3 で得た需要関数を、制約条件である総消費バスケットの定義式に代入する。
\begin{align*}
c_t^{h \to W} &= \frac{1}{(\alpha^H)^{\alpha^H} (1-\alpha^H)^{1-\alpha^H}} \left( \frac{\alpha^H \eta_t^h}{p_t^H} \right)^{\alpha^H} \left( \frac{(1-\alpha^H) \eta_t^h}{e_t p_t^{F*}} \right)^{1-\alpha^H} \\
&= \frac{(\eta_t^h)^{\alpha^H + (1-\alpha^H)}}{(\alpha^H)^{\alpha^H} (1-\alpha^H)^{1-\alpha^H}} \cdot \frac{(\alpha^H)^{\alpha^H} (1-\alpha^H)^{1-\alpha^H}}{(p_t^H)^{\alpha^H} (e_t p_t^{F*})^{1-\alpha^H}} \\
c_t^{h \to W} &= \eta_t^h \cdot \frac{1}{(p_t^H)^{\alpha^H} (e_t p_t^{F*})^{1-\alpha^H}}
\end{align*}
この式をラグランジュ乗数 \( \eta_t^h \) について解く。
\[ \eta_t^h = c_t^{h \to W} (p_t^H)^{\alpha^H} (e_t p_t^{F*})^{1-\alpha^H} \]

\subsection*{5. 総合物価指数 \( p_t^{H \to W} \) と最終的な需要関数の導出}
名目総消費は、最小化すべき総費用 \( p_t^H c_t^{h \to H} + e_t p_t^{F*} c_t^{h \to F} \) として定義される。ステップ 2 の FOC より、これは \( \alpha^H \eta_t^h + (1-\alpha^H) \eta_t^h = \eta_t^h \) に等しい。一方で、名目総消費は \( p_t^{H \to W} c_t^{h \to W} \) とも定義されるため、 \( p_t^{H \to W} c_t^{h \to W} = \eta_t^h \) という関係が成立しなければならない。

この関係式にステップ 4 で導出した \( \eta_t^h \) の式を代入する。
\[ p_t^{H \to W} c_t^{h \to W} = c_t^{h \to W} (p_t^H)^{\alpha^H} (e_t p_t^{F*})^{1-\alpha^H} \]
両辺の \( c_t^{h \to W} \) を消去することで、定理の項目 2 で示された\textbf{総合物価指数 \( p_t^{H \to W} \)} が導かれる。
\[ p_t^{H \to W} = (p_t^H)^{\alpha^H} (e_t p_t^{F*})^{1-\alpha^H} \]
最後に、この \( \eta_t^h = p_t^{H \to W} c_t^{h \to W} \) という関係をステップ 3 で得た需要関数に代入することで、定理の項目 1 で示された最終的な財バスケットへの需要関数が完成する。
\[ c_t^{h \to H} = \frac{\alpha^H (p_t^{H \to W} c_t^{h \to W})}{p_t^H} = \alpha^H \frac{p_t^{H \to W}}{p_t^H} c_t^{h \to W} \]
\[ c_t^{h \to F} = \frac{(1-\alpha^H) (p_t^{H \to W} c_t^{h \to W})}{e_t p_t^{F*}} = (1-\alpha^H) \frac{p_t^{H \to W}}{e_t p_t^{F*}} c_t^{h \to W} \]
( 証明終 )         % CPI の定義
% !TeX root = ../../main.tex
% sections/app/appendix_production_aggregation.tex

\chapter{生産関数の集計と価格分散}
\label{chap:appendix_production_aggregation}

本付録では、個々の家計の生産関数から出発し、価格の異質性が存在する経済における国全体の集計生産関数を導出する。

\section*{定理 5:価格分散を考慮した集計生産関数}

個人の生産関数が \( y_t^h = a_t^H l_t^h \) であり、個別財への需要が \( y_t^h = ( p_t^h / p_t^H )^{-\theta^H} Y_t^H \) で与えられる経済において、国全体の集計生産関数は以下のように表される。
\[
Y_t^H = \frac{a_t^H L_t^H}{\Delta_t^H}
\]
ここで、 \( L_t^H \) は総労働投入量 \( \sum_{h \in H} l_t^h \) であり、 \( \Delta_t^H \) は経済全体の非効率性を示す価格分散項であり、次式で定義される。
\[
\Delta_t^H \equiv \sum_{h \in H} \left( \frac{p_t^h}{p_t^H} \right)^{-\theta^H}
\]

\section*{証明}

この定理を証明するために、総労働投入量 \( L_t^H \) の定義式から出発し、集計生産関数を導出する。

\subsection*{1. 総労働の定義式からの展開}
国全体の総労働投入量 \( L_t^H \) は、全ての家計の労働投入量の合計である。
\[
L_t^H = \sum_{h \in H} l_t^h
\]
この式に、個人の生産関数 \( y_t^h = a_t^H l_t^h \) を \( l_t^h \) について解いた \( l_t^h = y_t^h / a_t^H \) を代入する。国全体の生産性 \( a_t^H \) は全ての家計で共通であるため、総和の外に出すことができる。
\begin{align*}
L_t^H &= \sum_{h \in H} \frac{y_t^h}{a_t^H} \\
&= \frac{1}{a_t^H} \sum_{h \in H} y_t^h
\end{align*}
次に、個別財への需要関数 \( y_t^h = \left( p_t^h / p_t^H \right)^{-\theta^H} Y_t^H \) を代入する。集計産出量 \( Y_t^H \) は個々の家計 \( h \) に依存しないため、これも総和の外に出すことができる。
\begin{align*}
L_t^H &= \frac{1}{a_t^H} \sum_{h \in H} \left[ \left( \frac{p_t^h}{p_t^H} \right)^{-\theta^H} Y_t^H \right] \\
&= \frac{Y_t^H}{a_t^H} \sum_{h \in H} \left( \frac{p_t^h}{p_t^H} \right)^{-\theta^H}
\end{align*}

\subsection*{2. 価格分散項の定義と結論}
ここで、総和の部分を価格分散項 \( \Delta_t^H \) として定義する。
\[
\Delta_t^H \equiv \sum_{h \in H} \left( \frac{p_t^h}{p_t^H} \right)^{-\theta^H}
\]
この定義をステップ 1 で得られた式に代入すると、総労働 \( L_t^H \) と総生産 \( Y_t^H \) の間に以下の厳密な関係式が成立する。
\[
L_t^H = \frac{Y_t^H}{a_t^H} \Delta_t^H
\]
この式を \( Y_t^H \) について解くことで、定理で示された集計生産関数が得られる( 証明終 )。
\[
Y_t^H = \frac{a_t^H L_t^H}{\Delta_t^H}
\] % 生産関数の集計
% !TeX root = ../../main.tex
% sections/app/appendix_aggregate_output.tex

\chapter{集計産出量と名目所得の関係}
\label{chap:appendix_aggregate_output}

本付録の目的は、総名目生産額が、集計産出量と真の価格指数の単純な積で表せること、そしてその関係を担保するために集計産出量が必然的にCES集計の形でなければならないことを証明することである。

\section*{定理4:総名目所得の分解と集計産出量の整合性}

財市場の均衡において、ある時点 \( t \) で、以下の 2 つの関係が成立する。
\begin{enumerate}
    \item 国全体の総名目所得は、集計産出量 \( Y_t^H \) と真の価格指数 \( p_t^H \) の積に等しい。
    \[ \sum_{h \in H} p_t^h y_t^h = p_t^H Y_t^H \]
    \item 上記の関係式が成立するためには、集計産出量 \( Y_t^H \) は、個々の財の生産量 \( y_t^h \) のCES集計関数でなければならない。
    \[ Y_t^H = \left[ \sum_{h \in H} (y_t^h)^{\frac{\theta^H-1}{\theta^H}} \right]^{\frac{\theta^H}{\theta^H-1}} \]
\end{enumerate}

\section*{証明}

\subsection*{1. 個別生産量と集計量の関係式の導出}
まず、個別財 \( h \) の生産量 \( y_t^h \) と集計産出量 \( Y_t^H \) の間の関係を厳密に導出する。財市場の均衡では、個別財 \( h \) の生産量( 供給 )は、その財への全世界からの総需要と等しい。
\[ y_t^h = \sum_{h' \in H} c_t^{h' \to h} + \sum_{f \in F} c_t^{f \to h} \]
ここに、各消費者( 国内家計 \( h' \)、外国の家計 \( f \) )の個別財への需要関数を代入する。買い手が国内か国外かに関わらず、彼らが直面する自国財の価格体系は共通であるため、需要関数 \( c_t^{\cdot \to h} = (p_t^h/p_t^H)^{-\theta^H} c_t^{\cdot \to H} \) の価格に関する項は共通となる。
\begin{align*}
y_t^h &= \sum_{h' \in H} \left[ \left( \frac{p_t^h}{p_t^H} \right)^{-\theta^H} c_t^{h' \to H} \right] + \sum_{f \in F} \left[ \left( \frac{p_t^h}{p_t^H} \right)^{-\theta^H} c_t^{f \to H} \right] \\
&= \left( \frac{p_t^h}{p_t^H} \right)^{-\theta^H} \left[ \sum_{h' \in H} c_t^{h' \to H} + \sum_{f \in F} c_t^{f \to H} \right]
\end{align*}
ここで、大括弧の中の項は「 世界全体からの国内財バスケットへの総需要 」を意味する。財市場全体の均衡において、これは国内財の集計産出量 \( Y_t^H \) と定義上一致する。
\[ Y_t^H \equiv \sum_{h' \in H} c_t^{h' \to H} + \sum_{f \in F} c_t^{f \to H} \]
したがって、以下の厳密な関係式が導かれる。
\[ y_t^h = \left( \frac{p_t^h}{p_t^H} \right)^{-\theta^H} Y_t^H \]

\subsection*{2. 総名目所得の計算}
次に、総名目所得 \( \sum_{h \in H} p_t^h y_t^h \) を計算する。ステップ 1 で導出した関係式を代入する。
\begin{align*}
    \sum_{h \in H} p_t^h y_t^h &= \sum_{h \in H} p_t^h \left[ \left( \frac{p_t^h}{p_t^H} \right)^{-\theta^H} Y_t^H \right] \\
    &= Y_t^H (p_t^H)^{\theta^H} \sum_{h \in H} (p_t^h)^{1-\theta^H}
\end{align*}
ここで、付録 \ref{chap:appendix_cost_minimization} で導出した真の価格指数 \( p_t^H \) の定義式の両辺を \( 1-\theta^H \) 乗すると、 \( \sum_{h \in H} (p_t^h)^{1-\theta^H} = (p_t^H)^{1-\theta^H} \) という関係が得られる。これを上式に代入する。
\begin{align*}
    \sum_{h \in H} p_t^h y_t^h &= Y_t^H (p_t^H)^{\theta^H} (p_t^H)^{1-\theta^H} \\
    &= p_t^H Y_t^H
\end{align*}
これにより、定理の項目 1 が証明された。

\subsection*{3. 集計産出量 \( Y_t^H \) の定義の整合性}
最後に、この関係式が成り立つために、集計産出量 \( Y_t^H \) が必然的にCES集計の形でなければならないことを示す。ステップ 1 の関係式の両辺を \( \frac{\theta^H-1}{\theta^H} \) 乗し、全ての \( h \) について合計を取る。
\begin{align*}
\sum_{h \in H} (y_t^h)^{\frac{\theta^H-1}{\theta^H}} &= \sum_{h \in H} \left[ \left( \frac{p_t^h}{p_t^H} \right)^{-\theta^H} Y_t^H \right]^{\frac{\theta^H-1}{\theta^H}} \\
&= \sum_{h \in H} \left( \frac{p_t^h}{p_t^H} \right)^{1-\theta^H} (Y_t^H)^{\frac{\theta^H-1}{\theta^H}} \\
&= \frac{(Y_t^H)^{\frac{\theta^H-1}{\theta^H}}}{(p_t^H)^{1-\theta^H}} \sum_{h \in H} (p_t^h)^{1-\theta^H}
\end{align*}
再び価格指数の定義 \( (p_t^H)^{1-\theta^H} = \sum_{h \in H} (p_t^h)^{1-\theta^H} \) を使うと、右辺の価格項は相殺される。
\[
\sum_{h \in H} (y_t^h)^{\frac{\theta^H-1}{\theta^H}} = (Y_t^H)^{\frac{\theta^H-1}{\theta^H}}
\]
この両辺の \( \frac{\theta^H}{\theta^H-1} \) 乗を取ると、 \( Y_t^H \) が定理の項目 2 で示されたCES集計関数でなければならないことが示される( 証明終 )。      % 集計生産量と資源制約
% !TeX root = ../../main.tex
% sections/app/appendix_equivalence.tex

\chapter{効用最大化問題と 2 段階最適化の等価性}
\label{chap:appendix_equivalence}

本付録では、本稿のモデル構築において採用している「 個別財の費用最小化問題を解き、そこで得られた価格指数を用いて効用最大化問題を解く 」という 2 段階の最適化アプローチが、全ての個別財の消費量を一度に選択するという単一の効用最大化問題と数学的に完全に等価であることを証明する。



\section*{定理 1:最適化問題の等価性}

家計 \( h \) が、ある時点 \( t \) において所与の総支出額 \( E_t^h \) の下で、全ての個別財 \( \{c_t^{h \to h'}\} \), \( \{c_t^{h \to f'}\} \) の消費量を直接選択する単一の効用最大化問題は、以下の 2 段階の最適化問題と等価である。
\begin{enumerate}
    \item \textbf{第 1 段階( 費用最小化 )}: 所与の財バスケット量 \( c_t^{h \to H} \), \( c_t^{h \to F} \) を達成するための最小費用と、その際の個別財への需要を求める。この過程で、真の価格指数 \( p_t^H \), \( p_t^{F*} \) が導出される。
    \item \textbf{第 2 段階( 効用最大化 )}: 第 1 段階で導出された価格指数を所与として、予算制約 \( p_t^H c_t^{h \to H} + e_t p_t^{F*} c_t^{h \to F} = E_t^h \) の下で、財バスケット \( c_t^{h \to H} \), \( c_t^{h \to F} \) の最適な組み合わせを選択し、効用を最大化する。
\end{enumerate}

\section*{証明}

この定理を証明するために、出発点となる厳密な単一の効用最大化問題を、数学的に等価な変形を施すことで、 2 段階の最適化問題へと帰着させる。

\subsection*{1. 出発点:単一の効用最大化問題}
厳密な問題設定は、時点 \( t \) において、以下の通りである。
\begin{itemize}
    \item \textbf{最大化対象}:
    \[
    \max_{\{c_t^{h \to h'}\}, \{c_t^{h \to f'}\}} \ln(c_t^{h \to W})
    \]
    ただし、 \( c_t^{h \to W} \), \( c_t^{h \to H} \), \( c_t^{h \to F} \) は各個別財消費量の関数である。

    \item \textbf{制約条件}:
    \[
    \sum_{h' \in H} p_t^{h'} c_t^{h \to h'} + \sum_{f' \in F} p_t^{f'} c_t^{h \to f'} = E_t^h
    \]
\end{itemize}

\subsection*{2. 問題の分解}
この最大化問題は、総支出 \( E_t^h \) を国内財への支出 \( E_t^{h \to H} \) と海外財への支出 \( E_t^{h \to F} \) に分割する、以下のネストした( 入れ子構造の )問題と等価である。
\[
\max_{E_t^{h \to H}, E_t^{h \to F}} \left( \max_{\{c_t^{h \to h'}\}, \{c_t^{h \to f'}\}} \ln(c_t^{h \to W}) \quad \text{s.t.} \sum p_t^{h'}c_t^{h \to h'} = E_t^{h \to H}, \sum p_t^{f'}c_t^{h \to f'} = E_t^{h \to F} \right)
\]
\[
\text{subject to} \quad E_t^{h \to H} + E_t^{h \to F} = E_t^h
\]

\subsection*{3. 双対性( Duality )の利用}
ここで、内側の最大化問題に注目する。例えば国内財については、「 所与の支出 \( E_t^{h \to H} \) で、バスケット \( c_t^{h \to H} \) の量を最大化する問題 」である。
\[
\max_{\{c_t^{h \to h'}\}} c_t^{h \to H}(\{c_t^{h \to h'}\}) \quad \text{s.t.} \quad \sum_{h' \in H} p_t^{h'}c_t^{h \to h'} = E_t^{h \to H}
\]
ミクロ経済学の双対性( duality )の原理により、この問題は、「 所与のバスケット量 \( \bar{c}_t^{h \to H} \) を、最小の費用で達成する問題 」と完全に等価である。
\[
\min_{\{c_t^{h \to h'}\}} \sum_{h' \in H} p_t^{h'}c_t^{h \to h'} \quad \text{s.t.} \quad c_t^{h \to H}(\{c_t^{h \to h'}\}) = \bar{c}_t^{h \to H}
\]
この費用最小化問題こそが、 2 段階アプローチにおける第 1 段階に他ならない。本稿の付録 \ref{chap:appendix_cost_minimization} で示したように、この問題を解くことで、所与のバスケット量 \( c_t^{h \to H} \) を達成するための最小費用( 支出関数 )は \( p_t^H c_t^{h \to H} \) となることが導かれる。同様に、外国財バスケットの最小費用は \( e_t p_t^{F*} c_t^{h \to F} \) となる。

\subsection*{4. 問題の再定式化と結論}
この支出関数の関係 \( E_t^{h \to H} = p_t^H c_t^{h \to H} \) と \( E_t^{h \to F} = e_t p_t^{F*} c_t^{h \to F} \) を、ステップ 2 で分解した問題に代入する。すると、問題は以下のように書き換えられる。
\begin{itemize}
    \item \textbf{最大化}:
    \[
    \max_{c_t^{h \to H}, c_t^{h \to F}} \ln \left( \frac{(c_t^{h \to H})^{\alpha^H} (c_t^{h \to F})^{1-\alpha^H}}{(\alpha^H)^{\alpha^H} (1-\alpha^H)^{1-\alpha^H}} \right)
    \]
    \item \textbf{制約条件}:
    \[
    p_t^H c_t^{h \to H} + e_t p_t^{F*} c_t^{h \to F} = E_t^h
    \]
\end{itemize}
この書き換えられた問題は、まさしく 2 段階アプローチにおける第 2 段階そのものである。

以上により、厳密な単一の効用最大化問題が、本稿で採用した 2 段階の最適化アプローチと数学的に完全に等価であることが証明された( 証明終 )。           % 変数の同値関係( 再帰的価格等 )
% !TeX root = ../../main.tex
% sections/app/appendix_resource_constraint.tex

\chapter{国全体の資源制約式の導出}
\label{chap:appendix_resource_constraint}

本付録では、個々の家計の予算制約式を集計し、国内市場の均衡条件を適用することで、国全体の資源制約式を導出する。

\section*{定理 7:国全体の資源制約式}

国内金融市場が完備であり、政府が均衡予算を達成する経済において、全ての個人の予算制約式を集計すると、以下の国全体の資源制約式が得られる。
\[
p_t^{H \to W} C_t^{H \to W} + B_{t+1}^{H} = p_t^H Y_t^H + (1+i_{t-1}^F) \frac{e_t}{e_{t-1}} B_t^{H}
\]
ここで、 \( C_t^{H \to W} \) は国全体の総消費、 \( Y_t^H \) は集計産出量、 \( B_{t+1}^{H} \) は期末の対外純資産( 自国通貨建て )を表す。

\section*{証明}

\subsection*{1. 国全体での集計}
まず、家計 \( h \) の期ごとの名目予算制約式を、国内の全ての家計 \( h \in H \) について足し合わせる( \( \sum_{h \in H} \) )。全ての変数は時点 \( t \) に依存する。
\[
\begin{split}
    & \sum_{h \in H} \left( \sum_{j' \in J} q_{t+1}(j') d_{t+1}^{h \to H}(j') + b_{t+1}^{h \to F} + p_t^{H \to W} c_t^{h \to W} \right) \\
    & \qquad = \sum_{h \in H} \left( d_t^{h \to H} + (1+i_{t-1}^F) \frac{e_t}{e_{t-1}} b_t^{h \to F} + (1-\tau_t^H) p_t^h y_t^h + t_t^H \right)
\end{split}
\]

\subsection*{2. 国内取引の相殺}
次に、国全体で集計するとゼロになる国内完結の取引を相殺する。

\paragraph{国内債券市場の均衡}
国内で取引される状態コンティンジェント債券は、国内の誰かの負債が他の誰かの資産となるゼロサム取引であるため、純供給はゼロである。
\[ \sum_{h \in H} d_{t}^{h \to H} = 0 \quad \text{and} \quad \sum_{h \in H} \sum_{j'} q_{t+1}(j') d_{t+1}^{h \to H}(j') = 0 \]
これにより、集計した式から国内債券に関する項( \( d \) )はすべて消去される。

\paragraph{政府部門の予算制約}
政府は均衡予算を達成し、税収のすべてを家計への移転に使うため、移転の総額と税収の総額は一致する。
\[
\sum_{h \in H} t_t^H = \sum_{h \in H} \tau_t^H p_t^h y_t^h
\]
この関係を用いると、集計後の予算制約式の右辺にある移転項 \( \sum t_t^H \) は、所得項の一部である税金部分 \( \sum \tau_t^H p_t^h y_t^h \) と相殺される。

\subsection*{3. 集計変数への書き換え}
国内取引が相殺された結果、式に残るのは以下の項のみである。
\[
\sum_{h \in H} \left( b_{t+1}^{h \to F} + p_t^{H \to W} c_t^{h \to W} \right) = \sum_{h \in H} \left( (1+i_{t-1}^F) \frac{e_t}{e_{t-1}} b_t^{h \to F} + p_t^h y_t^h \right)
\]
この式を、国全体の集計変数( 大文字の変数 )を使って書き換える。

\paragraph{消費と対外資産の集計}
価格指数 \( p_t^{H \to W} \) は全ての家計で共通であるため、総和 \( \sum_{h \in H} \) の外に出すことができる。
\[
\sum_{h \in H} p_t^{H \to W} c_t^{h \to W} = p_t^{H \to W} \sum_{h \in H} c_t^{h \to W} \equiv p_t^{H \to W} C_t^{H \to W}
\]
対外純資産( 自国通貨建て )については、単純な総和として定義される。
\[
\sum_{h \in H} b_{t+1}^{h \to F} \equiv B_{t+1}^{H}
\]
同様に、前期からの対外資産の償還額も \( (1+i_{t-1}^F) \frac{e_t}{e_{t-1}} B_t^{H} \) となる。

\paragraph{名目総生産の集計}
付録 \ref{chap:appendix_aggregate_output} で証明した通り、総名目所得は \( \sum_{h \in H} p_t^h y_t^h = p_t^H Y_t^H \) という関係が厳密に成立する。

\subsection*{4. 結論}
以上の集計結果をまとめると、定理で示された国全体の資源制約式が導出される( 証明終 )。
\[
p_t^{H \to W} C_t^{H \to W} + B_{t+1}^{H} = p_t^H Y_t^H + (1+i_{t-1}^F) \frac{e_t}{e_{t-1}} B_t^{H}
\]    % 資源制約の線形化
% !TeX root = ../../main.tex
% sections/app/appendix_risk_sharing.tex

\chapter{完備市場における消費の共通化と貯蓄の個別化}
\label{chap:appendix_risk_sharing}

本付録では、国内完備市場とカルボ型価格設定を仮定したモデルにおいて、なぜ全ての家計の消費が共通化される一方、貯蓄( 資産ポートフォリオ )は家計ごとに個別化されるのかを厳密に証明する。

\section*{定理 9:リスク共有の結果}

国内金融市場が完備である経済において、以下の関係が成立する。
\begin{enumerate}
    \item 全ての国内家計 \( h \in H \) の所得の限界効用は、いかなる時点 \( t \) 、いかなる状態 \( j \) においても常に一致する。
    \[ \lambda_t^h = \lambda_t^{h'} \quad \forall h, h' \in H \]
    \item 全ての国内家計 \( h \in H \) の総消費指数は、常に一致する。
    \[ c_t^{h \to W} = c_t^{h' \to W} \quad \forall h, h' \in H \]
    \item 各家計が購入する次期のための資産ポートフォリオの総価値は、家計が当期に価格改定の機会を得たか否かによって異なるため、一般に一致しない。
\end{enumerate}

\section*{証明}

\subsection*{1. 所得の限界効用 \( \lambda_t \) の一致( 定理 9-1 の証明 )}

\paragraph{ステップ A:限界効用の比率の不変性}
家計の最適化行動は全ての時点・状態で成立するため、任意の 2 つの国内家計 \( h \) と \( h' \) の国内コンティンジェント債券に関する一階の条件( FOC )は、常に成立する。
\[
\begin{aligned}
\lambda_t^h q_{t+1} &= \beta_t^H \pi \lambda_{t+1}^h \\
\lambda_t^{h'} q_{t+1} &= \beta_t^H \pi \lambda_{t+1}^{h'}
\end{aligned}
\]
これら 2 式の比を取ると共通項が消去され、以下の関係が得られる。
\[
\frac{\lambda_t^h}{\lambda_t^{h'}} = \frac{\lambda_{t+1}^h}{\lambda_{t+1}^{h'}} = k
\]
この比率 \( k \) は、時間や状態に依存しない普遍的な定数である。

\paragraph{ステップ B:定数 \( k \) の値の特定}
第 \ref{sec:model_overview} 節で定義した「 事前的対称性 」の仮定を用いる。初期時点 \( s \) において、全ての家計は同一の選好をもち、かつ初期貯蓄がゼロ( \( d_s^h = d_s^{h'} = 0 \) )である。このとき、各家計が直面する最適化問題は数学的に完全に同一であるため、初期の所得の限界効用は全ての家計で一致しなければならない。
\[
\lambda_s^h = \lambda_s^{h'} \quad \Longrightarrow \quad k = \frac{\lambda_s^h}{\lambda_s^{h'}} = 1
\]

\paragraph{ステップ C:結論}
普遍的な定数が \( k=1 \) であることから、将来のすべての時点・すべての状態において \( \lambda_t^h = \lambda_t^{h'} \) が成立する。

\subsection*{2. 総消費指数 \( c_t^{h \to W} \) の一致( 定理 9-2 の証明 )}
全ての家計の \( \lambda_t \) が一致するという結果を、消費に関する FOC( 式 \ref{eq:model_foc_consumption_home} )に適用する。
\[
\lambda_t = \frac{1}{p_t^{H \to W} c_t^{h \to W}}
\]
左辺の \( \lambda_t \) と右辺の物価指数 \( p_t^{H \to W} \) は全ての家計で共通であるため、総消費指数 \( c_t^{h \to W} \) もまた、全ての家計間で完全に一致しなければならない。

\subsection*{3. ポートフォリオ購入総額の個別化( 定理 9-3 の証明 )}
家計 \( h \) の予算制約式を、次期のために購入する資産ポートフォリオの総価値 \( v_{t+1}^h \equiv \sum_{j'} q_{t+1} d_{t+1}^{h \to H} + e_t b_{t+1}^{h \to F*} \) について整理する。
\[
v_{t+1}^h = \underbrace{\left( d_t^{h \to H} + (1+i_{t-1}^F)e_t b_t^{h \to F*} + (1 - \tau_t^H) p_t^h y_t^h + t_t^H \right)}_{\text{当期の総収入}} - \underbrace{p_t^{H \to W} c_t^{h \to W}}_{\text{当期の消費支出}}
\]
定理 9-2 より消費支出は共通であるが、当期の総収入、特に労働所得 \( (1 - \tau_t^H) p_t^h y_t^h \) はカルボ型の価格設定( \( p_t^h \) の異質性 )によって家計ごとに異なる。支出が共通で収入が異なる以上、その差額を埋める資産保有額 \( v_{t+1}^h \) は、各家計が直面したショックに応じて個別化される。( 証明終 )          % リスクシェアリング条件

% =============================================================
% 2. 第 4 章( モデル構築 )で参照される基礎的な導出
% =============================================================
% !TeX root = ../../main.tex
% sections/app/appendix_optimal_price_derivation.tex

\chapter{ニューケインジアン・フィリップス曲線の詳細な導出}
\label{chap:appendix_optimal_price_derivation}

本付録では、本文第 \ref{chap:results} 章で用いたニューケインジアン・フィリップス曲線を一切の省略なく導出する。

\section{価格決定家計のラグランジアンと最適化の前提}
\label{sec:appendix_optimal_price_derivation_lagrangian}

価格改定の機会を得た家計は、任意の時点 \( t \) において以下の目的関数を最大化する価格 \( \widetilde{p}_t^h \) を選択する。

\[
\begin{aligned}
\max_{\widetilde{p}_t^h} & \Biggl[ \left( \log c_t^{h \to W} - \frac{\phi^H}{2}(l_t^{h})^2 \right) \\
& \quad + \lambda_{t}^{H} \biggl( \Bigl( d_{t}^{h \to H} + (1+i_{t-1}^F)e_t^{/*} b_{t}^{h \to F} + (1 - \tau_{t}^{H}) \widetilde{p}_t^h y_{t}^{h} + t_{t}^{H} \Bigr) \\
& \quad - \Bigl( \sum_{j' \in J} q_{t,t+1}(j') d_{t+1}^{h \to H}(j') + e_{t}^{/*} b_{t+1}^{h \to F} + p_t^{H \to W} c_t^{h \to W} \Bigr) \biggr) \Biggr] \\
& + E_t \Biggl[ \sum_{k=1}^{\infty} (\xi^H)^k \left(\prod_{j=0}^{k-1} \beta_{t+j}^H \right) \Biggl\{ \left( \log c_{t+k}^{h \to W} - \frac{\phi^H}{2}(l_{t+k}^{h})^2 \right) \\
& \quad + \lambda_{t+k}^{H} \biggl( \Bigl( d_{t+k}^{h \to H} + (1+i_{t+k-1}^F)e_{t+k}^{/*} b_{t+k}^{h \to F} + (1 - \tau_{t+k}^{H}) \widetilde{p}_t^h y_{t+k}^{h} + t_{t+k}^{H} \Bigr) \\
& \quad - \Bigl( \sum_{j' \in J} q_{t+k,t+k+1}(j') d_{t+k+1}^{h \to H}(j') + e_{t+k}^{/*} b_{t+k+1}^{h \to F} + p_{t+k}^{H \to W} c_{t+k}^{h \to W} \Bigr) \biggr) \Biggr\} \Biggr]
\end{aligned}
\]

この目的関数を \( \widetilde{p}_t^h \) について偏微分し、 1 階の条件を求める際に、以下の点を考慮する。

\begin{itemize}
    \item 個別家計 \( h \) の価格 \( \widetilde{p}_t^h \) がマクロ変数( \( p_{t+k}^{H \to W}, \lambda_{t+k}^H, \bar{p}_{t+k}^H \) など )に与える影響は無視できるほど小さいと仮定する。
    \item 価格 \( \widetilde{p}_t^h \) が所得を通じて消費 \( c_{t+k}^{h \to W} \) に与える影響は、予算制約として織り込まれているため、効用関数内の \( c_{t+k}^{h \to W} \) は \( \widetilde{p}_t^h \) から独立しているものとして扱う。
    \item 一方で、制約 \( l_{t+k}^h = y_{t+k}^h / a_{t+k}^H \) と \( y_{t+k}^h = (\frac{\widetilde{p}_t^h}{\bar{p}_{t+k}^H})^{-\theta^H} y_{t+k}^H \) は、効用関数内の労働 \( l_{t+k}^h \) と所得項の生産 \( y_{t+k}^h \) に反映される。これにより \( l_{t+k}^h \) と \( y_{t+k}^h \) は \( \widetilde{p}_t^h \) の関数となり、微分対象になる。
\end{itemize}

これらの仮定のもとで 1 階の条件を計算すると、効用関数の \( \log c_{t+k}^{h \to W} \) の項と、予算制約の大部分の項の微分がゼロとなり、労働の非効用と収入の項のみが残る。

\section{時点 \( t \) における価格 \( \widetilde{p}_t^h \) の最適化条件の導出}
\label{sec:appendix_optimal_price_derivation_foc}

カルボ型の価格設定では、価格改定家計は、一度設定した価格が将来全ての期間にわたり有効であり続ける可能性を考慮し、期待効用の割引現在価値の合計を最大化する単一の価格 \( \widetilde{p}_t^h \) を選択する。

この最適化問題の 1 階の条件は、目的関数を \( \widetilde{p}_t^h \) で偏微分し、ゼロと置くことで得られる。
\[
E_t \sum_{k=0}^{\infty} (\xi^H)^k \left(\prod_{j=0}^{k-1} \beta_{t+j}^H \right) \left[ \frac{\partial}{\partial \widetilde{p}_t^h} \left\{ \left( - \frac{\phi^H}{2}(l_{t+k}^h)^2 \right) + \lambda_{t+k}^H \left( (1 - \tau_{t+k}^H) \widetilde{p}_t^h y_{t+k}^h \right) \right\} \right] = 0
\]

次に、中括弧内の各項の偏微分を、前提となる関係式を用いて詳細に計算する。

\subsection*{前提となる関係式}
\begin{itemize}
    \item \textbf{生産関数}: \( y_{t+k}^h = a_{t+k}^H l_{t+k}^h \implies l_{t+k}^h = y_{t+k}^h / a_{t+k}^H \)
    \item \textbf{需要関数}: \( y_{t+k}^h = \left( \frac{\widetilde{p}_t^h}{\bar{p}_{t+k}^H} \right)^{-\theta^H} y_{t+k}^H \)
\end{itemize}

\subsection*{ステップ 1:第 1 項( 労働の非効用 )の偏微分}
連鎖律( chain rule )を用いると、
\[
\frac{\partial}{\partial \widetilde{p}_t^h} \left( - \frac{\phi^H}{2}(l_{t+k}^h)^2 \right) = -\phi^H l_{t+k}^h \frac{\partial l_{t+k}^h}{\partial \widetilde{p}_t^h}
\]
ここで、 \( \frac{\partial l_{t+k}^h}{\partial \widetilde{p}_t^h} \) を求めるために、前提となる関係式から順に計算する。

\begin{enumerate}
    \item まず、生産関数の関係から、 \( l_{t+k}^h \) の \( \widetilde{p}_t^h \) に関する偏微分は次のようになる。
    \[
    \frac{\partial l_{t+k}^h}{\partial \widetilde{p}_t^h} = \frac{1}{a_{t+k}^H} \frac{\partial y_{t+k}^h}{\partial \widetilde{p}_t^h}
    \]

    \item 次に、需要関数を \( \widetilde{p}_t^h \) で偏微分する。
    \[
    \begin{aligned}
    \frac{\partial y_{t+k}^h}{\partial \widetilde{p}_t^h} &= (-\theta^H) \left( \frac{\widetilde{p}_t^h}{\bar{p}_{t+k}^H} \right)^{-\theta^H-1} \cdot \frac{1}{\bar{p}_{t+k}^H} \cdot y_{t+k}^H \\
    &= -\frac{\theta^H}{\widetilde{p}_t^h} \cdot \left( \frac{\widetilde{p}_t^h}{\bar{p}_{t+k}^H} \right)^{-\theta^H} \cdot y_{t+k}^H \\
    &= -\theta^H \frac{y_{t+k}^h}{\widetilde{p}_t^h} \quad ( \because \text{元の需要関数の定義より} )
    \end{aligned}
    \]
\end{enumerate}

上記 2 つの関係を組み合わせると、 \( \frac{\partial l_{t+k}^h}{\partial \widetilde{p}_t^h} \) は以下のように求められる。
\[
\frac{\partial l_{t+k}^h}{\partial \widetilde{p}_t^h} = \frac{1}{a_{t+k}^H} \left( -\theta^H \frac{y_{t+k}^h}{\widetilde{p}_t^h} \right) = -\theta^H \frac{l_{t+k}^h}{\widetilde{p}_t^h} \quad ( \because y_{t+k}^h/a_{t+k}^H = l_{t+k}^h )
\]
これを最初の式に代入すると、第 1 項の偏微分は、
\[
-\phi^H l_{t+k}^h \left( -\theta^H \frac{l_{t+k}^h}{\widetilde{p}_t^h} \right) = \frac{\phi^H \theta^H (l_{t+k}^h)^2}{\widetilde{p}_t^h}
\]
となる。

\subsection*{ステップ 2:第 2 項( 所得 )の偏微分}
積の微分法則( product rule )を用いると、
\[
\frac{\partial}{\partial \widetilde{p}_t^h} \left( \widetilde{p}_t^h y_{t+k}^h \right) = 1 \cdot y_{t+k}^h + \widetilde{p}_t^h \frac{\partial y_{t+k}^h}{\partial \widetilde{p}_t^h}
\]
ステップ 1 で求めた需要関数の微分 \( \frac{\partial y_{t+k}^h}{\partial \widetilde{p}_t^h} = -\theta^H \frac{y_{t+k}^h}{\widetilde{p}_t^h} \) を代入すると、
\[
y_{t+k}^h + \widetilde{p}_t^h \left( -\theta^H \frac{y_{t+k}^h}{\widetilde{p}_t^h} \right) = y_{t+k}^h - \theta^H y_{t+k}^h = (1-\theta^H)y_{t+k}^h
\]
したがって、第 2 項の偏微分は、
\[
\lambda_{t+k}^H (1 - \tau_{t+k}^H) (1-\theta^H) y_{t+k}^h
\]
となる。

\subsection*{ステップ 3: 1 階の条件式と共通最適価格 \( \widetilde{p}_t^H \) の定義}

ステップ 1 とステップ 2 で求めた偏微分を元の 1 階の条件式に代入し、結合すると以下のようになる。
\[
E_t \sum_{k=0}^{\infty} (\xi^H)^k \left(\prod_{j=0}^{k-1} \beta_{t+j}^H \right) \left[ \frac{\phi^H \theta^H (l_{t+k}^h)^2}{\widetilde{p}_t^h} + \lambda_{t+k}^H (1 - \tau_{t+k}^H) (1-\theta^H) y_{t+k}^h \right] = 0
\]
この 1 階の条件式を \( \widetilde{p}_t^h \) について解くと、個々の家計 \( h \) の最適価格が得られる。
\[
\widetilde{p}_t^h = \frac{\theta^H}{\theta^H-1} \frac{ E_t \sum_{k=0}^{\infty} (\xi^H)^k \left(\prod_{j=0}^{k-1} \beta_{t+j}^H \right) \left[ \phi^H (l_{t+k}^{h})^2 \right] }{ E_t \sum_{k=0}^{\infty} (\xi^H)^k \left(\prod_{j=0}^{k-1} \beta_{t+j}^H \right) \left[ \lambda_{t+k}^H (1 - \tau_{t+k}^{H}) y_{t+k}^{h} \right] }
\]
右辺に含まれる個別変数 \( l_{t+k}^h \) と \( y_{t+k}^h \) を、需要関数と生産関数の関係式を用いて \( \widetilde{p}_t^h \) の関数として展開する。

\begin{itemize}
    \item \( y_{t+k}^h = \left( \frac{\widetilde{p}_t^h}{\bar{p}_{t+k}^H} \right)^{-\theta^H} y_{t+k}^H \)
    \item \( l_{t+k}^h = y_{t+k}^h / a_{t+k}^H = \frac{1}{a_{t+k}^H} \left( \frac{\widetilde{p}_t^h}{\bar{p}_{t+k}^H} \right)^{-\theta^H} y_{t+k}^H \)
\end{itemize}

これらを代入して \( \widetilde{p}_t^h \) の項を整理すると、以下の関係式が得られる。
\[
(\widetilde{p}_t^h)^{1+\theta^H} = \frac{\theta^H}{\theta^H-1} \frac{ E_t \sum_{k=0}^{\infty} (\xi^H)^k \left(\prod_{j=0}^{k-1} \beta_{t+j}^H \right) \left[ \phi^H \frac{(y_{t+k}^H)^2}{(a_{t+k}^H)^2} (\bar{p}_{t+k}^H)^{2\theta^H} \right] }{ E_t \sum_{k=0}^{\infty} (\xi^H)^k \left(\prod_{j=0}^{k-1} \beta_{t+j}^H \right) \left[ \lambda_{t+k}^H (1 - \tau_{t+k}^{H}) y_{t+k}^H (\bar{p}_{t+k}^H)^{\theta^H} \right] }
\]
この式の右辺に出てくる全ての変数は、価格改定を行う全ての家計 \( h \) にとって共通である( 完備市場の仮定より \( \lambda_{t+k}^h = \lambda_{t+k}^H \) )。

したがって、この方程式の解である \( \widetilde{p}_t^h \) も、全ての価格改定を行う家計 \( h \) にとって完全に同一の値となる。この全ての家計にとって共通の最適価格を \( \widetilde{p}_t^H \) と表記する。

\subsection*{ステップ 4:最適価格式の再帰形式への変換( 詳細導出 )}

分子と分母の無限和の部分をそれぞれ新しい補助変数 \( v_t \) と \( w_t \) で定義する。

\paragraph{分子 \( v_t \) の再帰式の導出}

\begin{enumerate}
    \item \textbf{定義式から出発する。}
    \[ v_t = E_t \sum_{k=0}^{\infty} (\xi^H)^k \left(\prod_{j=0}^{k-1} \beta_{t+j}^H \right) \left[ \phi^H \frac{(y_{t+k}^H)^2}{(a_{t+k}^H)^2} (\bar{p}_{t+k}^H)^{2\theta^H} \right] \]
    \item \textbf{無限和を「 今日の項( k = 0 ) 」と「 明日以降の項( k \ge 1 ) 」に分解する。}
    ( \( k=0 \) のとき、 \( \prod \) の部分は空積なので 1 となる )
    \[ v_t = \underbrace{\phi^H \frac{(y_t^H)^2}{(a_t^H)^2} (\bar{p}_t^H)^{2\theta^H}}_{k=0 \text{の項}} + E_t \underbrace{\sum_{k=1}^{\infty} (\xi^H)^k \left(\prod_{j=0}^{k-1} \beta_{t+j}^H \right) \left[ \dots \right]}_{k \ge 1 \text{の項}} \]
    \item \textbf{明日以降の項の和から、共通の因子である \( \beta_t^H \xi^H \) を外に出す。}
    \[ v_t = \phi^H \frac{(y_t^H)^2}{(a_t^H)^2} (\bar{p}_t^H)^{2\theta^H} + E_t \left[ \xi^H \beta_t^H \sum_{k=1}^{\infty} (\xi^H)^{k-1} \left(\prod_{j=1}^{k-1} \beta_{t+j}^H \right) \left[ \dots \right] \right] \]
    \item \textbf{総和の添え字を \( k' = k-1 \) と置き換える。} すると \( k=1 \) は \( k'=0 \) に対応し、累積積も \( \prod_{j=1}^{k-1} \beta_{t+j}^H = \prod_{j'=0}^{k'-1} \beta_{t+1+j'}^H \) となる。
    \[ v_t = \phi^H \frac{(y_t^H)^2}{(a_t^H)^2} (\bar{p}_t^H)^{2\theta^H} + \xi^H E_t \left[ \beta_t^H \sum_{k'=0}^{\infty} (\xi^H)^{k'} \left(\prod_{j'=0}^{k'-1} \beta_{t+1+j'}^H \right) \left[ \dots \right]_{t+1+k'} \right] \]
    \item \textbf{期待値の法則( Law of Iterated Expectations )を適用する。}
    時点 \( t \) の期待値の中にある \( \beta_t^H \) と明日以降の項の和は、 \( t+1 \) 時点の期待値 \( E_{t+1} \) で書き換えられる。
    \[ v_t = \phi^H \frac{(y_t^H)^2}{(a_t^H)^2} (\bar{p}_t^H)^{2\theta^H} + \xi^H E_t \left[ \beta_t^H E_{t+1} \left[ \sum_{k'=0}^{\infty} (\xi^H)^{k'} \left( \dots \right) \right] \right] \]
    \( \beta_t^H \) は \( t \) 期の情報なので \( E_t \) の外に出せ、また \( E_t[E_{t+1}[\cdot]] = E_t[\cdot] \) なので、角括弧の中身は \( v_{t+1} \) の定義そのものになる。
    \[ v_t = \phi^H \frac{(y_t^H)^2}{(a_t^H)^2} (\bar{p}_t^H)^{2\theta^H} + \beta_t^H \xi^H E_t[v_{t+1}] \]
\end{enumerate}

\paragraph{分母 \( w_t \) の再帰式の導出}

\begin{enumerate}
    \item \textbf{定義式から出発し、「 今日の項 」と「 明日以降の項 」に分解する。}
    \[ w_t = \lambda_{t}^H (1 - \tau_{t}^{H}) y_{t}^H (\bar{p}_{t}^H)^{\theta^H} + E_t \sum_{k=1}^{\infty} (\xi^H)^k \left(\prod_{j=0}^{k-1} \beta_{t+j}^H \right) \left[ \dots \right] \]
    \item \textbf{分子と同様の手順で、共通因子 \( \beta_t^H \xi^H \) の括り出しと添え字の変換を行う。}
    \[ w_t = \lambda_{t}^H (1 - \tau_{t}^{H}) y_{t}^H (\bar{p}_{t}^H)^{\theta^H} + \beta_t^H \xi^H E_t \left[ \sum_{k'=0}^{\infty} (\xi^H)^{k'} \left(\prod_{j'=0}^{k'-1} \beta_{t+1+j'}^H \right) \left[ \dots \right]_{t+1+k'} \right] \]
    \item \textbf{角括弧の中身を \( w_{t+1} \) に置き換える。}
    \[ w_t = \lambda_t^H (1 - \tau_t^{H}) y_t^H (\bar{p}_t^H)^{\theta^H} + \beta_t^H \xi^H E_t[w_{t+1}] \]
\end{enumerate}

以上の詳細な導出により、最適価格は以下の 3 本の連立方程式に集約される。
\begin{align}
(\widetilde{p}_t^H)^{1+\theta^H} &= \frac{\theta^H}{\theta^H-1} \frac{v_t}{w_t} \\
v_t &= \phi^H \frac{(y_t^H)^2}{(a_t^H)^2} (\bar{p}_t^H)^{2\theta^H} + \beta_t^H \xi^H E_t[v_{t+1}] \\
w_t &= \lambda_t^H (1 - \tau_t^{H}) y_t^H (\bar{p}_t^H)^{\theta^H} + \beta_t^H \xi^H E_t[w_{t+1}]
\end{align}

\section{対数線形化の手法:定義と一般法則}
\label{sec:appendix_optimal_price_derivation_log_linearization_rules}

\subsection*{1. ハット変数の定義}
ある変数 \( x_t \) の定常状態の値を \( x_{ss} \) とするとき、対数乖離 \( \hat{x}_t \) を次のように定義する。
\[ \hat{x}_t \equiv \log(x_t) - \log(x_{ss}) = \log\left(\frac{x_t}{x_{ss}}\right) \]
 \( x=0 \) の周りでの \( \log(1+x) \approx x \) というマクローリン展開を用いることで、乖離が小さい場合にはパーセント乖離率とほぼ等しくなる。
\[ \hat{x}_t = \log\left(1 + \frac{x_t - x_{ss}}{x_{ss}}\right) \approx \frac{x_t - x_{ss}}{x_{ss}} \]

\subsection*{2. 対数線形化の一般法則}
\begin{itemize}
    \item \textbf{積のルール}: \( z_t = x_t y_t \implies \hat{z}_t = \hat{x}_t + \hat{y}_t \)
    \item \textbf{商のルール}: \( z_t = x_t / y_t \implies \hat{z}_t = \hat{x}_t - \hat{y}_t \)
    \item \textbf{べき乗のルール}: \( z_t = x_t^a \implies \hat{z}_t = a \hat{x}_t \)
    \item \textbf{定数倍のルール}: \( z_t = a x_t \implies \hat{z}_t = \hat{x}_t \)
    \item \textbf{和のルール}: \( z_t = a x_t + b y_t \implies z_{ss} \hat{z}_t = a x_{ss} \hat{x}_t + b y_{ss} \hat{y}_t \)
\end{itemize}

\section{最適価格式の対数線形化}
\label{sec:appendix_optimal_price_derivation_optimal_price_log_linearization}

\subsection*{ステップ 1:出発点となる方程式群}
前章の方程式を、一般法則を用いてそれぞれ対数線形化する。

\begin{enumerate}
    \item \textbf{最適価格の定義式の対数線形化}
    \[ (1+\theta^H) \hat{\widetilde{p}}_t^H = \hat{v}_t - \hat{w}_t \]

    \item \textbf{\( v_t \) の再帰式の対数線形化}
    和のルールを適用する。定常状態では \( v_{flow,ss} = (1-\beta_{ss}^H \xi^H)v_{ss} \), \( v_{stock,ss} = \beta_{ss}^H \xi^H v_{ss} \) の関係があるため、
    \[ \hat{v}_t = (1-\beta_{ss}^H \xi^H) \left( 2\hat{y}_t^H - 2\hat{a}_t^H + 2\theta^H \hat{\bar{p}}_t^H \right) + \beta_{ss}^H \xi^H \left( \hat{\beta}_t^H + E_t[\hat{v}_{t+1}] \right) \]

    \item \textbf{\( w_t \) の再帰式の対数線形化}
    \[ \hat{w}_t = (1-\beta_{ss}^H \xi^H) \left( \hat{\lambda}_t^H + \widehat{(1-\tau_t^H)} + \hat{y}_t^H + \theta^H \hat{\bar{p}}_t^H \right) + \beta_{ss}^H \xi^H \left( \hat{\beta}_t^H + E_t[\hat{w}_{t+1}] \right) \]
\end{enumerate}

\subsection*{ステップ 2:式の結合と最終的な関係式の導出}

\begin{enumerate}
    \item \textbf{\( \hat{v}_t - \hat{w}_t \) を計算し、最適価格 \( \hat{\widetilde{p}}_t^H \) を代入する}
    \[ (1+\theta^H) \hat{\widetilde{p}}_t^H = (1-\beta_{ss}^H \xi^H) \left[ \dots \right] + \beta_{ss}^H \xi^H E_t[(1+\theta^H) \hat{\widetilde{p}}_{t+1}^H] \]
    ここで \( \left[ \dots \right] \) の部分は、 \( \hat{v}_t \) のフロー部分から \( \hat{w}_t \) のフロー部分を引いた、以下の駆動項 \( \hat{\psi}_t \) となる。
    \[ \hat{\psi}_t = \hat{y}_t^H - 2\hat{a}_t^H + \theta^H \hat{\bar{p}}_t^H - \hat{\lambda}_t^H - \widehat{(1-\tau_t^H)} \]

    \item \textbf{最終的な関係式}
    両辺を \( 1+\theta^H \) で割ることで、以下の式が導かれる。
    \[ \hat{\widetilde{p}}_t^H = \frac{1-\beta_{ss}^H \xi^H}{1+\theta^H} \hat{\psi}_t + \beta_{ss}^H \xi^H E_t[\hat{\widetilde{p}}_{t+1}^H] \]
\end{enumerate}

\section{価格設定の導出( 詳細解説 )}
\label{sec:appendix_optimal_price_derivation_pricing_dynamics}

\subsection*{ステップ 1:全体の物価指数の対数線形化}
物価指数 PPI の動学を表す次式を対数線形化する。
\[ (\bar{p}_t^H)^{1-\theta^H} = (1-\xi^H)(\widetilde{p}_t^H)^{1-\theta^H} + \xi^H(\bar{p}_{t-1}^H)^{1-\theta^H} \]
和のルールとべき乗のルールを適用し整理すると、以下の関係が得られる。
\[ \hat{\bar{p}}_t^H = (1-\xi^H) \hat{\widetilde{p}}_t^H + \xi^H \hat{\bar{p}}_{t-1}^H \]

\textbf{インフレと相対価格の関係式の導出:}
インフレ率 \( \hat{\pi}_t^H \equiv \hat{\bar{p}}_t^H - \hat{\bar{p}}_{t-1}^H \) について変形する。まず、両辺から \( \hat{\bar{p}}_{t-1}^H \) を引く。
\[ \hat{\bar{p}}_t^H - \hat{\bar{p}}_{t-1}^H = (1-\xi^H) \hat{\widetilde{p}}_t^H + \xi^H \hat{\bar{p}}_{t-1}^H - \hat{\bar{p}}_{t-1}^H \]
右辺を整理し \( \hat{\bar{p}}_{t-1}^H = \hat{\bar{p}}_t^H - \hat{\pi}_t^H \) を代入する。
\[ \hat{\pi}_t^H = (1-\xi^H)(\hat{\widetilde{p}}_t^H - (\hat{\bar{p}}_t^H - \hat{\pi}_t^H)) \]
項を整理し、インフレ \( \hat{\pi}_t^H \) の項を左辺にまとめると、以下の関係式が得られる。
\[ (1-(1-\xi^H)) \hat{\pi}_t^H = (1-\xi^H)(\hat{\widetilde{p}}_t^H - \hat{\bar{p}}_t^H) \]
\[ \xi^H \hat{\pi}_t^H = (1-\xi^H)(\hat{\widetilde{p}}_t^H - \hat{\bar{p}}_t^H) \quad \text{--- ( 式 A )} \]

\subsection*{ステップ 2: 2 つの式の結合と整理}
最適価格設定のルールを \( \hat{\widetilde{p}}_t^H = (1-\beta_{ss}^H \xi^H) \hat{\psi}_t' + \beta_{ss}^H \xi^H E_t[\hat{\widetilde{p}}_{t+1}^H] \) ( ただし \( \hat{\psi}_t' = \frac{\hat{\psi}_t}{1+\theta^H} \) )とする。

\paragraph{手順 A:式の準備}
式 A を変形して得られる相対価格の式を代入できるよう、最適価格ルールの両辺から \( \hat{\bar{p}}_t^H \) を引く。
\[ \hat{\widetilde{p}}_t^H - \hat{\bar{p}}_t^H = (1-\beta_{ss}^H \xi^H) \hat{\psi}_t' - \hat{\bar{p}}_t^H + \beta_{ss}^H \xi^H E_t[\hat{\widetilde{p}}_{t+1}^H] \]

\paragraph{手順 B:期待値の展開}
右辺の期待値 \( E_t[\hat{\widetilde{p}}_{t+1}^H] \) に \( E_t[\hat{\bar{p}}_{t+1}^H] \) を足して引く。
\[ \hat{\widetilde{p}}_t^H - \hat{\bar{p}}_t^H = (1-\beta_{ss}^H \xi^H) \hat{\psi}_t' - \hat{\bar{p}}_t^H + \beta_{ss}^H \xi^H E_t\left[(\hat{\widetilde{p}}_{t+1}^H - \hat{\bar{p}}_{t+1}^H) + \hat{\bar{p}}_{t+1}^H\right] \]
 \( \hat{\bar{p}}_{t+1}^H = \hat{\bar{p}}_t^H + \hat{\pi}_{t+1}^H \) を代入し、 \( \hat{\bar{p}}_t^H \) は時点 \( t \) で既知であるため期待値の外に出す。
\[ \hat{\widetilde{p}}_t^H - \hat{\bar{p}}_t^H = (1-\beta_{ss}^H \xi^H) \hat{\psi}_t' - \hat{\bar{p}}_t^H + \beta_{ss}^H \xi^H E_t\left[\hat{\widetilde{p}}_{t+1}^H - \hat{\bar{p}}_{t+1}^H\right] + \beta_{ss}^H \xi^H(\hat{\bar{p}}_t^H + E_t[\hat{\pi}_{t+1}^H]) \]

\paragraph{手順 C:式の整理}
相対価格をインフレ率で書き換え、期待値の項をまとめる。
\[ \frac{\xi^H}{1-\xi^H} \hat{\pi}_t^H = (1-\beta_{ss}^H \xi^H)(\hat{\psi}_t' - \hat{\bar{p}}_t^H) + \frac{\beta_{ss}^H \xi^H}{1-\xi^H} E_t[\hat{\pi}_{t+1}^H] \]
両辺に \( \frac{1-\xi^H}{\xi^H} \) を掛けることで、インフレの方程式が得られる。
\[ \hat{\pi}_t^H = \beta_{ss}^H E_t[\hat{\pi}_{t+1}^H] + \frac{(1-\xi^H)(1-\beta_{ss}^H \xi^H)}{\xi^H}(\hat{\psi}_t' - \hat{\bar{p}}_t^H) \]

\subsection*{最終ステップ:ニューケインジアン・フィリップス曲線の導出}
中間式に \( \hat{\psi}_t' \) および \( \hat{\psi}_t \) の定義を代入し、係数 \( \kappa^H = \frac{(1-\xi^H)(1-\beta_{ss}^H \xi^H)}{\xi^H(1+\theta^H)} \) を定義して整理すると、最終的なフィリップス曲線が得られる。
\[ \hat{\pi}_t^H = \beta_{ss}^H E_t[\hat{\pi}_{t+1}^H] + \kappa^H \left( \hat{y}_t^H - 2\hat{a}_t^H - \hat{\bar{p}}_t^H - \hat{\lambda}_t^H - \widehat{(1-\tau_t^H)} \right) \]
( 証明終 )          % フィリップス曲線

% =============================================================
% 3. 第 5 章( シミュレーション結果 )で参照される詳細な分析
% =============================================================
% !TeX root = ../../main.tex
% sections/app/appendix_exchange_rate.tex

\chapter{割引因子ショックを含む為替レート決定式の導出}
\label{chap:appendix_exchange_rate}

本稿の第 5 章における考察では、自国と外国で家計の時間選好( 割引因子 \( \beta \) )が異なる場合、為替レートの決定式に将来の \( \beta \) の格差が累積的に影響することを論じた。本付録では、その数理的な導出過程を、統計学的な恒等式と対数線形近似の性質、および前方展開( Forward Solving )の手法に焦点を当て詳述する。

\section*{1. 期待値の比に関する線形近似の証明}

本稿のような線形近似モデルにおいて、期待値の比 \( E_t[X_t] / E_t[Y_t] \) が「 比の期待値 」 \( E_t[X_t / Y_t] \) で近似できる理由を、一般的な確率変数の性質に基づいて数学的に証明する。

\subsection*{一般論としての変数の定義}
ある任意の確率変数 \( X_t \) について、その期待値( 平均 )を \( \bar{X} \equiv E[X_t] \) とし、期待値からの対数乖離( 変化率 )を \( \hat{x}_t \) と定義する。
\begin{equation*}
    \hat{x}_t \equiv \frac{X_t - \bar{X}}{\bar{X}} \quad \Longleftrightarrow \quad X_t = \bar{X}(1+\hat{x}_t)
\end{equation*}
本稿のような線形近似モデルでは、この乖離 \( \hat{x}_t \) を「 1 次の微小量 」として扱い、その 2 乗以上( \( \hat{x}_t^2 \) や \( \hat{x}_t \hat{y}_t \) )を無視できるほど小さい項( \( \approx 0 \) )として計算を行う。

\subsection*{ステップ 1:積の期待値の近似( 共分散項の評価 )}

一般に、 2 つの確率変数 \( X, Y \) の積の期待値は、「 期待値の積 」と「 共分散( Covariance ) 」の和に分解できる。
\begin{equation*}
    E_t[XY] = E_t[X]E_t[Y] + \text{Cov}_t(X, Y)
\end{equation*}
ここで、共分散の項を上記の定義に従って展開する。
\begin{align*}
    \text{Cov}_t(X, Y) &= E_t \left[ (X - \bar{X}) (Y - \bar{Y}) \right] \\
    &= E_t \left[ \left( \bar{X}(1+\hat{x}) - \bar{X} \right) \left( \bar{Y}(1+\hat{y}) - \bar{Y} \right) \right] \\
    &= \bar{X}\bar{Y} E_t \left[ \hat{x}\hat{y} \right]
\end{align*}
ここで、右辺の期待値の中身は、微小変動同士の積( \( \hat{x} \times \hat{y} \) のオーダー )となっている。 1 次近似の枠組みでは、 1 次の微小量同士の積は 2 次の微小量となるため、無視することができる( \( \approx 0 \) )。したがって、以下の近似が成立する。
\begin{equation}
    \text{Cov}_t(X, Y) \approx 0 \quad \Longrightarrow \quad E_t[XY] \approx E_t[X]E_t[Y]
    \label{eq:appendix_exchange_rate_approx_product}
\end{equation}

\subsection*{ステップ 2:逆数の期待値の近似( 分散項の評価 )}

次に、変数 \( Z \) の逆数の期待値 \( E_t[1/Z] \) を考える。一般に、イェンゼンの不等式により \( E[1/Z] \neq 1/E[Z] \) であるが、その乖離幅は分散に依存する。
関数 \( f(Z) = 1/Z \) を、期待値 \( \bar{Z} = E_t[Z] \) の周りで 2 次までテイラー展開し、期待値をとる。
\begin{align*}
    E_t\left[\frac{1}{Z}\right] &\approx E_t \left[ \frac{1}{\bar{Z}} - \frac{1}{\bar{Z}^2}(Z - \bar{Z}) + \frac{1}{\bar{Z}^3}(Z - \bar{Z})^2 \right] \\
    &= \frac{1}{\bar{Z}} - \frac{1}{\bar{Z}^2}\underbrace{E_t[Z - \bar{Z}]}_{0} + \frac{1}{\bar{Z}^3}\underbrace{E_t[(Z - \bar{Z})^2]}_{\text{分散 Var}(Z)} \\
    &= \frac{1}{E_t[Z]} + \frac{1}{(E_t[Z])^3} \text{Var}_t(Z)
\end{align*}
ここで分散 \( \text{Var}_t(Z) = \bar{Z}^2 E_t \left[ \hat{z}^2 \right] \) は微小変動の 2 乗を含むため、 1 次近似においてはこれを無視できる。
したがって、以下の近似式が成立する。
\begin{equation}
    E_t\left[\frac{1}{Z}\right] \approx \frac{1}{E_t[Z]}
    \label{eq:appendix_exchange_rate_approx_inverse}
\end{equation}

\subsection*{結論:比の期待値への変換}

式 \eqref{eq:appendix_exchange_rate_approx_product} と 式 \eqref{eq:appendix_exchange_rate_approx_inverse} の結果を組み合わせることで、期待値の比を比の期待値として扱うことが正当化される。
\begin{align*}
    \frac{E_t [X]}{E_t [Y]} &= E_t [X] \cdot \frac{1}{E_t [Y]} \\
    &\approx E_t [X] \cdot E_t \left[ \frac{1}{Y} \right] \quad ( \because \text{分散項} \approx 0 ) \\
    &\approx E_t \left[ X \cdot \frac{1}{Y} \right] \quad ( \because \text{共分散項} \approx 0 ) \\
    &= E_t \left[ \frac{X}{Y} \right]
\end{align*}
本付録の以降の導出では、この一般的な近似式 \( E_t[X_t]/E_t[Y_t] \approx E_t[X_t/Y_t] \) を利用する。

\section*{2. 修正UIP条件とオイラー方程式の比}

名目為替レートの決定プロセスを明らかにするため、まず両国の基礎方程式を整理する。

\begin{itemize}
    \item \textbf{自国代表的家計の修正UIP条件:}
    本文第 5 章( 式 \ref{eq:results_asad_modified_uip_H} )で確認した通り、自国の UIP 条件 \eqref{eq:uip_condition_home} に対し、自国家計にとっての外貨 1 単位の限界効用の定義式 \( \lambda_t^{H/*} \equiv e_t^{/*} \lambda_t^H \) を適用すると、以下の「 修正UIP条件 」が得られる。
    \begin{equation}
        \lambda_t^{H/*} = \beta_t^H (1+i_t^F) E_t [\lambda_{t+1}^{H/*}]
        \tag{\ref{eq:results_asad_modified_uip_H}}
    \end{equation}

    \item \textbf{外国代表的家計の修正UIP条件:}
    同様の手順を、外国代表的家計の UIP 条件 \eqref{eq:model_uip_F} に対しても適用する。外国代表的家計にとっての外貨( この場合、自国通貨 1 単位 )の限界効用は \( \lambda_t^{F/*} \) である。外国のオイラー方程式から、以下の関係が得られる。
    \begin{equation}
        \lambda_t^{F/*} = \beta_t^F (1+i_t^F) E_t [\lambda_{t+1}^{F/*}]
        \label{eq:appendix_exchange_rate_modified_uip_F}
    \end{equation}
\end{itemize}

ここで、自国代表的家計と外国代表的家計による外貨評価の比率を \( \Lambda_t \) と定義する。
\begin{equation}
    \Lambda_t \equiv \frac{\lambda_t^{H/*}}{\lambda_t^{F/*}}
    \label{eq:appendix_exchange_rate_lambda_def}
\end{equation}
式 \eqref{eq:results_asad_modified_uip_H} と 式 \eqref{eq:appendix_exchange_rate_modified_uip_F} の辺々を割ると、共通項である外国の名目金利 \( (1+i_t^F) \) が相殺され、以下の関係が得られる。
\begin{equation}
    \Lambda_t = \frac{\beta_t^H}{\beta_t^F} \frac{E_t [\lambda_{t+1}^{H/*}]}{E_t [\lambda_{t+1}^{F/*}]}
    \label{eq:appendix_exchange_rate_lambda_ratio_raw}
\end{equation}

第 1 節で証明した一般的な近似関係を用いることで、式 \eqref{eq:appendix_exchange_rate_lambda_ratio_raw} の右辺にある「 期待値の比 」を「 比の期待値 」へと変換し、以下の再帰式を得る。
\begin{equation}
    \Lambda_t \approx \frac{\beta_t^H}{\beta_t^F} E_t \left[ \frac{\lambda_{t+1}^{H/*}}{\lambda_{t+1}^{F/*}} \right] = \frac{\beta_t^H}{\beta_t^F} E_t [\Lambda_{t+1}]
    \label{eq:appendix_exchange_rate_lambda_recursive}
\end{equation}

\section*{3. 将来に向けた前方展開と極限}

前節で導出した再帰式 \eqref{eq:appendix_exchange_rate_lambda_recursive} を将来に向かって逐次代入( Forward Solving )することで、現在の比率 \( \Lambda_t \) を決定する。

\paragraph{① 再帰的代入と一般形の導出}
式 \eqref{eq:appendix_exchange_rate_lambda_recursive} の右辺にある \( \Lambda_{t+1} \) に対して、 1 期先の関係式 \( \Lambda_{t+1} = \frac{\beta_{t+1}^H}{\beta_{t+1}^F} E_{1+t}[\Lambda_{t+2}] \) を代入する。この操作を \( k \) 回繰り返すと、反復期待値の法則により、以下の一般形が得られる。
\begin{equation}
    \Lambda_t = E_t \left[ \left( \prod_{j=0}^{k} \frac{\beta_{t+j}^H}{\beta_{t+j}^F} \right) \Lambda_{t+k+1} \right]
    \label{eq:appendix_exchange_rate_lambda_iterative}
\end{equation}

\paragraph{② 極限の適用}
次に、式 \eqref{eq:appendix_exchange_rate_lambda_iterative} の両辺について \( k \to \infty \) の極限をとる。
本稿のモデルにおいては、全ての変数が定常状態近傍で安定的に推移( 有界性 )することを前提としている。このとき、ルベーグの優収束定理( Dominated Convergence Theorem )により、期待値演算子 \( E_t \) と極限操作 \( \lim \) の順序を入れ替えることが正当化される。すなわち、期待値の極限は極限の期待値に等しい。

\begin{align*}
    \Lambda_t &= \lim_{k \to \infty} E_t \left[ \left( \prod_{j=0}^{k} \frac{\beta_{t+j}^H}{\beta_{t+j}^F} \right) \Lambda_{t+k+1} \right] \\
    &= E_t \left[ \lim_{k \to \infty} \left( \left( \prod_{j=0}^{k} \frac{\beta_{t+j}^H}{\beta_{t+j}^F} \right) \Lambda_{t+k+1} \right) \right]
\end{align*}

ここで、経済の安定性条件より、無限遠方において一時的なショックの影響は消失し、変数はショック後の新たな定常状態へ収束すると仮定する。したがって、評価の比率 \( \Lambda_{t+k+1} \) は、定常状態における比率 \( \Lambda_{\infty} \) へと収束する。
\begin{equation*}
    \lim_{k \to \infty} \Lambda_{t+k+1} = \Lambda_{\infty}
\end{equation*}
不完備市場モデルにおいては、この \( \Lambda_{\infty} \) はショックの履歴に依存する確率変数としての性質を持つため、期待値演算子の中に留まる。

ここで、資本移動要因 \( \mathcal{B}_t \) を以下のように定義する。
\begin{equation}
    \mathcal{B}_t \equiv \prod_{j=0}^{\infty} \frac{\beta_{t+j}^H}{\beta_{t+j}^F}
    \label{eq:appendix_exchange_rate_bt_definition}
\end{equation}

以上より、現在の \( \Lambda_t \) は以下の形式で確定する。
\begin{equation}
    \Lambda_t = E_t \left[ \mathcal{B}_t \Lambda_{\infty} \right]
    \label{eq:appendix_exchange_rate_lambda_final_solution}
\end{equation}

\section*{4. 結論:為替レートの決定式}

為替レートの定義式 \( e_t^{/*} = \lambda_t^{H/*} / \lambda_t^H \) に、 \( \lambda_t^{H/*} = \Lambda_t \lambda_t^{F/*} \) を代入する。
さらに、代表的家計の所得の限界効用の定義 \eqref{eq:model_lambda_H} および外国の同様の関係式を用いる。
\begin{equation*}
    \lambda_t^H = \frac{1}{p_t^{H \to W} c_t^{H \to W}}, \quad \lambda_t^{F/*} = \frac{1}{p_t^{F \to W*} c_t^{F \to W}}
\end{equation*}
これらを代入して整理すると、最終的な為替レート決定式が得られる。

\begin{align}
    e_t^{/*} &= \Lambda_t \times \frac{\lambda_t^{F/*}}{\lambda_t^H} \nonumber \\
    &= \Lambda_t \times \frac{1 / (p_t^{F \to W*} c_t^{F \to W})}{1 / (p_t^{H \to W} c_t^{H \to W})} \nonumber \\
    &= E_t \left[ \mathcal{B}_t \Lambda_{\infty} \right] \times \frac{p_t^{H \to W} c_t^{H \to W}}{p_t^{F \to W*} c_t^{F \to W}}
    \label{eq:appendix_exchange_rate_final}
\end{align}

ここで、期待値項 \( E_t \left[ \mathcal{B}_t \Lambda_{\infty} \right] \) について、定常状態における割引因子の設定が決定的な役割を果たす。

\begin{itemize}
    \item \textbf{定常状態で割引因子が等しい場合( \( \beta_{ss}^H = \beta_{ss}^F \) ):}
    本稿の基本設定である。一時的なショックにより \( \beta_t \) が変動しても、長期的には元の水準に戻るため、無限乗積項 \( \mathcal{B}_t \) は有限の値に収束する。
    特に、全期間において \( \beta_t^H = \beta_t^F \) であるならば、式 \eqref{eq:appendix_exchange_rate_final} の期待値項は \( E_t [\Lambda_{\infty}] \) となり、為替レートは純粋に両国の名目支出比率のみで決定される。
    \begin{equation}
        e_t^{/*} = E_t [\Lambda_{\infty}] \times \frac{p_t^{H \to W} c_t^{H \to W}}{p_t^{F \to W*} c_t^{F \to W}} 
        \label{eq:appendix_exchange_rate_simplified}
    \end{equation}
    
    \item \textbf{定常状態で割引因子が異なる場合( \( \beta_{ss}^H \neq \beta_{ss}^F \) ):}
    もし恒久的に \( \beta \) が異なると仮定すると、無限乗積項は発散または消失し、安定的な均衡が存在しなくなる。したがって、本モデルのような無限期間モデルにおいて安定解を得るためには、定常状態において両国の割引因子が一致していることが前提条件となる。
\end{itemize}

\section*{5. 資源制約式と対外純資産の恒等的なゼロ均衡}

上記の為替レート決定メカニズムが、対外純資産( \( b_t^H \) )の動学に与える影響を確認する。

\subsection*{1. 名目GDP方程式の導出と為替レート決定式の代入}

自国の一人当たり財市場均衡条件 \eqref{eq:model_goods_market_eq_H} から出発する。
\begin{equation}
    y_t^H = c_t^{H \to H} + \frac{M}{N} c_t^{F \to H}
    \label{eq:appendix_exchange_rate_goods_market_raw}
\end{equation}
この両辺に自国財価格 \( p_t^H \) を乗じると、名目GDPは各主体の需要内訳として以下のように展開される。
\begin{align*}
    p_t^H y_t^H &= p_t^H c_t^{H \to H} + p_t^H \frac{M}{N} c_t^{F \to H} \\
    &= p_t^H \left( \alpha^H \frac{p_t^{H \to W}}{p_t^H} c_t^{H \to W} \right) + p_t^H \frac{M}{N} \left( (1 - \alpha^F) \frac{p_t^{F \to W*}}{p_t^{H*}} c_t^{F \to W} \right) \\
    &= \alpha^H (p_t^{H \to W} c_t^{H \to W}) + \frac{M}{N} (1 - \alpha^F) \frac{p_t^H}{p_t^{H*}} (p_t^{F \to W*} c_t^{F \to W})
\end{align*}
ここで、一物一価の法則 \( p_t^H = e_t^{/*} p_t^{H*} \) \eqref{eq:model_lop} および、人口比 \( N=M=1 \) の設定を適用すると、以下の一般的な名目GDPの方程式が得られる。
\begin{equation}
    p_t^H y_t^H = \alpha^H(p_t^{H \to W} c_t^{H \to W}) + (1 - \alpha^F) e_t^{/*} (p_t^{F \to W*} c_t^{F \to W})
    \label{eq:appendix_exchange_rate_gdp_general}
\end{equation}

さらに、式 \eqref{eq:appendix_exchange_rate_gdp_general} の右辺にある為替レート \( e_t^{/*} \) に、先に導出した決定式 \eqref{eq:appendix_exchange_rate_final} を代入する。
\begin{align}
    p_t^H y_t^H &= \alpha^H(p_t^{H \to W} c_t^{H \to W}) + (1 - \alpha^F) \left( E_t \left[ \mathcal{B}_t \Lambda_{\infty} \right] \frac{p_t^{H \to W} c_t^{H \to W}}{p_t^{F \to W*} c_t^{F \to W}} \right) (p_t^{F \to W*} c_t^{F \to W}) \nonumber \\
    &= \alpha^H(p_t^{H \to W} c_t^{H \to W}) + (1 - \alpha^F) E_t \left[ \mathcal{B}_t \Lambda_{\infty} \right] (p_t^{H \to W} c_t^{H \to W}) \nonumber \\
    &= \left[ \alpha^H + (1 - \alpha^F) E_t \left[ \mathcal{B}_t \Lambda_{\infty} \right] \right] p_t^{H \to W} c_t^{H \to W}
    \label{eq:appendix_exchange_rate_gdp_substituted}
\end{align}

ここで第 \ref{sec:model_steady_state} 節において仮定されたとおり
初期時点およびそれ以前の対外純資産はゼロ( \( b_s^H = b_{s-1}^H = 0 \) )である。

自国の資源制約式 \eqref{eq:model_resource_constraint_H} においてショック発生前の \( t = s - 1 \) の時点を考える。
\begin{equation}
    p_{s-1}^{H \to W} c_{s-1}^{H \to W} + b_s^H = p_{s-1}^H y_{s-1}^H + (1+i_{s-2}^F) \frac{e_{s-1}^{/*}}{e_{s-2}^{/*}} b_{s-1}^H
    \label{eq:appendix_exchange_rate_resource_constraint_init}
\end{equation}
ここに仮定 \( b_s^H = b_{s-1}^H = 0 \) を代入すると、以下の貿易収支の均衡条件が導かれる。
\begin{equation}
    p_{s-1}^H y_{s-1}^H = p_{s-1}^{H \to W} c_{s-1}^{H \to W}
    \label{eq:appendix_exchange_rate_trade_balance_init}
\end{equation}
式 \eqref{eq:appendix_exchange_rate_gdp_substituted} を初期時点 \( t = s - 1 \) で評価し、貿易収支均衡 \eqref{eq:appendix_exchange_rate_trade_balance_init} と比較すると、右辺の名目消費にかかる係数部分は \( 1 \) でなければならないことがわかる。すなわち、
\begin{equation}
    1 = \alpha^H + (1 - \alpha^F) E_{s-1} \left[ \mathcal{B}_{s-1} \Lambda_{\infty} \right]
    \label{eq:appendix_exchange_rate_coefficient_unity_condition}
\end{equation}
が成立する。

\subsection*{2. a ショックが発生したときの均衡}

次に、全期間において自国と外国の割引因子が等しく( \( \beta_t^H = \beta_t^F \) )、生産性ショック( \( a \) ショック )のみが発生するケースを検討する。

まず、本稿の第 5 章 「 生産性ショックにおける国際的波及( 遮断効果 ) 」で論じたように、外貨建ての限界効用 \( \lambda_t^{H/*} \) および \( \lambda_t^{F/*} \) は生産性ショックの影響を受けない方程式系によって決定される。したがって、ショックの前後でこれらの値は変化せず、その比率である \( \Lambda_t = \lambda_t^{H/*} / \lambda_t^{F/*} \) も不変に維持される。すなわち、
\begin{equation}
    \Lambda_{s-1} = \Lambda_s = \dots = \Lambda_{\infty}
\end{equation}
が成立する。この性質により、評価の比率の極限値 \( \Lambda_{\infty} \) は、ショックの履歴に依存しない確定的定数とみなすことができ、期待値演算子 \( E_t \) の外に出すことが可能となる。

また、本ケースでは全期間において割引因子が一定であるため、定義式 \eqref{eq:appendix_exchange_rate_bt_definition} より資本移動要因は恒等的に \( \mathcal{B}_t = 1 \) となる。

この条件を式 \eqref{eq:appendix_exchange_rate_coefficient_unity_condition} に適用すると、
\begin{equation}
    1 = \alpha^H + (1 - \alpha^F) \Lambda_{\infty}
    \label{eq:appendix_exchange_rate_coefficient_unity_condition_substituted}
\end{equation}
となり、これを整理すると以下が得られる。
\begin{equation}
    \Lambda_{\infty} = \frac{1 - \alpha^H}{1 - \alpha^F}
    \label{eq:appendix_exchange_rate_lambda_steady_state_ratio}
\end{equation}

この \( \Lambda_{\infty} \) と \( \mathcal{B}_t = 1 \) を式 \eqref{eq:appendix_exchange_rate_gdp_substituted} に代入すると、以下の名目貿易収支の均衡式が得られる。
\begin{equation}
    p_t^H y_t^H = p_t^{H \to W} c_t^{H \to W}
    \label{eq:appendix_exchange_rate_nominal_trade_balance_identity}
\end{equation}

最後に、式 \eqref{eq:appendix_exchange_rate_nominal_trade_balance_identity} を自国の資源制約式 \eqref{eq:model_resource_constraint_H} に代入すると、資産蓄積に関する以下の差分方程式が得られる。
\begin{equation}
    b_{t+1}^H = (1+i_{t-1}^F) \frac{e_t^{/*}}{e_{t-1}^{/*}} b_t^H
    \label{eq:appendix_exchange_rate_nfa_difference_equation}
\end{equation}

初期条件 \( b_s^H = 0 \) を用いると、帰納的にすべての \( t \ge s \) において \( b_{t+1}^H = 0 \) であることが証明される。以上により、Cole-Obstfeld 条件下では為替レートの調整により貿易収支が常に均衡し、対外純資産が恒等的にゼロとなることが数学的に裏付けられた。          % 為替レートと GDP の恒等式
% !TeX root = ../../main.tex
% sections/app/appendix_log_linearization.tex

\chapter{対数線形近似の定義と幾何学的解釈}
\label{chap:appendix_log_linearization}

本付録では、動学的確率的一般均衡( DSGE )モデルの標準的な手法に従い、非線形の方程式系を定常状態近傍で対数線形近似( Log-linear approximation )することで、線形の連立差分方程式系へと変換する手法について詳述する。本節では、本稿で用いるハット変数( \( \hat{x}_t \) )の定義と、その数学的・幾何学的な意味について補足する。

\section{対数線形化の目的と定義}
\label{sec:appendix_log_linearization_purpose}

経済モデルの均衡条件は、通常、コブ=ダグラス型生産関数 \( Y_t = A_t K_t^\alpha L_t^{1-\alpha} \) のように、変数の積や冪乗を含む非線形方程式として記述される。これらの式をそのまま解くことは困難であるため、対数変換を行うことで乗算を加算( 線形 )の形に変換し、計算上の便宜を図ることが対数線形化の主たる目的である。

本稿では、ある変数 \( x_t \) の定常状態の値を \( x \) とするとき、定常状態からの対数乖離 \( \hat{x}_t \) を以下のように定義する。
\begin{equation}
    \hat{x}_t \equiv \ln x_t - \ln x
    \label{eq:appendix_log_linearization_hat_definition}
\end{equation}
ここで、 \( \ln x_t \) という値そのものには、具体的な経済学的単位( 円や数量など )としての意味はない。これはあくまで、積の形を和の形に変換するために導入された数学的な操作上の値に過ぎない。
しかし、この定義式に対して定常状態近傍での 1 次近似( テイラー展開 )を適用することで、 \( \hat{x}_t \) に具体的な経済学的意味( 変化率 )が付与される。

\section{変化率との関係}
\label{sec:appendix_log_linearization_rate}

関数 \( f(z) = \ln z \) を定常状態 \( z = x \) の周りで 1 次テイラー展開すると、以下の近似式が得られる。
\begin{equation}
    f(x_t) \approx f(x) + f'(x)(x_t - x)
\end{equation}
 \( f(z) = \ln z \) より \( f'(z) = 1/z \) であるため、上式は以下のように書き換えられる。
\begin{equation}
    \ln x_t \approx \ln x + \frac{1}{x}(x_t - x)
\end{equation}
これを変形すると、対数乖離 \( \hat{x}_t \) と変化率の関係が導かれる。
\begin{equation}
    \underbrace{\ln x_t - \ln x}_{\hat{x}_t} \approx \frac{x_t - x}{x}
    \label{eq:appendix_log_linearization_approx_formula}
\end{equation}
右辺は「 定常状態からの変化率( % ) 」そのものである。したがって、本稿における \( \hat{x}_t \) は、数学的には対数の差であるが、経済学的には「 定常状態からのパーセント乖離 」として解釈される。

\section{幾何学的解釈:曲線と接線の高さ}
\label{sec:appendix_log_linearization_geometry}

式 \eqref{eq:appendix_log_linearization_approx_formula} の両辺が近似的に等しい( \( \approx \) )ことの意味は、幾何学的には「 曲線の高さ 」と「 接線の高さ 」の関係として理解できる。



横軸に変数 \( z \)、縦軸に \( y = \ln z \) をとったグラフを考える。
\begin{itemize}
    \item \textbf{左辺( \( \ln x_t - \ln x \) ):}
    これは、対数曲線 \( y = \ln z \) 上における、定常状態 \( x \) から \( x_t \) までの\textbf{「 実際の高さ( 縦軸 )の変化量 」}を表している。
    
    \item \textbf{右辺( \( \frac{x_t - x}{x} \) ):}
    これは、定常状態 \( x \) において曲線に引いた\textbf{「 接線 」上での高さの変化量}を表している。
    接線の傾きは \( f'(x) = 1/x \) であるため、横方向の変化量 \( (x_t - x) \) に対する縦方向の変化量は、
    \[
    \text{縦の変化} = \text{傾き} \times \text{横の変化} = \frac{1}{x} \times (x_t - x)
    \]
    となる。
\end{itemize}

すなわち、対数線形近似とは、定常状態の近傍において「 対数曲線上の高さの変化( 左辺 ) 」を「 接線上の高さの変化( 右辺 ) 」で代用することに他ならない。
 \( x_t \) が \( x \) に十分に近ければ、曲線と接線はほぼ重なり合うため、この近似の精度は保たれる。シミュレーションにおいては、この近似関係を等号として扱うことで、巨大な連立方程式系を行列演算によって効率的に解くことが可能となる。

\section{積の線形化( \( \widehat{x_t y_t} = \hat{x}_t + \hat{y}_t \) の導出 )}
\label{sec:appendix_log_linearization_product}

変数の積 \( x_t y_t \) の対数線形近似が、それぞれのハット変数の和 \( \hat{x}_t + \hat{y}_t \) となることは、前述のハット変数の定義を用いることで、以下のように計算できる。

\begin{enumerate}
    \item \textbf{積の変数の定義:}
    まず、積の変数 \( x_t y_t \) について、その対数乖離 \( \widehat{x_t y_t} \) を定義に従って記述する。対数線形モデルにおいては、これを線形近似した厳密な等号として扱う。
    \begin{equation}
        \widehat{x_t y_t} = \ln(x_t y_t) - \ln(xy)
    \end{equation}
    
    \item \textbf{個別の変数の定義:}
    同様に、個別の変数 \( x_t, y_t \) についても、それぞれの対数乖離を定義する。
    \begin{align}
        \hat{x}_t &= \ln x_t - \ln x \\
        \hat{y}_t &= \ln y_t - \ln y
    \end{align}

    \item \textbf{展開と代入:}
    第 1 式の右辺に対し、対数の性質 \( \ln(AB) = \ln A + \ln B \) を適用して展開する。
    \begin{equation}
        \widehat{x_t y_t} = (\ln x_t + \ln y_t) - (\ln x + \ln y)
    \end{equation}
    項を並べ替えて整理する。
    \begin{equation}
        \widehat{x_t y_t} = (\ln x_t - \ln x) + (\ln y_t - \ln y)
    \end{equation}
    ここに第 2 項の個別変数の定義を代入すると、以下の関係が得られる。
    \begin{equation}
        \widehat{x_t y_t} = \hat{x}_t + \hat{y}_t
    \end{equation}
\end{enumerate}

このように、対数線形近似を用いることで、変数の乗算( 非線形 )の関係式は、ハット変数の加算( 線形 )の関係式へと変換される。      % 線形化フィリップス曲線の導出
% !TeX root = ../../main.tex
% sections/app/appendix_welfare_correction.tex

\chapter{厚生評価における近似の整合性と補正項の導出}
\label{chap:appendix_welfare_correction}

本付録では、本稿のシミュレーション分析で採用している「 1 次近似( 対数線形近似 )による動学算出 」と「 2 次近似的視点に基づく厚生評価 」の間の論理的整合性、および具体的な補正ロジックについて詳述する。

\section{近似による情報の欠落と整合性の問題}
\label{sec:appendix_welfare_correction_consistency}

本稿のシミュレーションは対数線形近似( 1 次近似 )を用いているが、この手法で得られた結果を厚生評価に用いる際には、「 非線形な関数 」と「 近似された変数 」の取り扱いに細心の注意が必要である。なぜなら、\textbf{「 非線形な関数 」に「 1 次近似された変数 」をそのまま代入して計算すると、計算の整合性が取れなくなる( 偽の精度が生じる )}からである。

この問題を理解するために、簡単な「 A + B 」の例を考えてみよう。
ある真の値 \( X \) が、主要な変動成分である「 1 次の項 \( A \) 」と、微細な補正成分である「 2 次の項 \( B \) 」から成るとする( \( X = A + B \) )。
一方、評価したい関数が \( F(X) = X + X^2 \) という非線形( 2 次 )の関数だとする。

真の値をこの関数に入力して、 2 次の精度まで正しく計算すると、以下のようになる( 3 次以上の微小項は無視する )。
\[
\begin{aligned}
F(A+B) &= (A+B) + (A+B)^2 \\
&= A + B + (A^2 + 2AB + B^2) \\
&\approx A + B + A^2
\end{aligned}
\]
ここで重要なのは、\textbf{「 入力に含まれる 2 次の項 \( B \) 」と「 関数によって生成される 2 次の項 \( A^2 \) 」の両方が、計算結果として残る}という点である。

しかし、もしシミュレーションが 1 次近似で行われていたらどうなるだろうか。シミュレーション結果 \( X^{sim} \) は、微細な \( B \) を無視して \( X^{sim} \approx A \) と出力される。これをそのまま非線形関数に代入してしまうと、
\[
F(X^{sim}) = A + A^2
\]
となる。一見もっともらしい値が出るが、ここには重大な欠陥がある。本来足されるべきだった \textbf{「 入力由来の 2 次の項 \( B \) 」が欠落している一方で、「 関数由来の 2 次の項 \( A^2 \) 」だけが計算されている}のである。同じ重要度を持つはずの要素のうち、片方だけを計算し、片方を無視するのは、計算として不整合であり、結果の信頼性を損なう。

本稿の分析においても、これと全く同じ問題が発生する。

\paragraph{① 指数関数によるレベル変数の復元を行ってはならない理由}
シミュレーション結果として得られる対数乖離 \( \hat{x}_t \) から、レベル変数 \( x_t \) を復元する際、定義式である指数関数 \( x_t = x_{ss}\exp(\hat{x}_t) \) を用いて計算してはいけない。
なぜなら、\textbf{非線形な指数関数をマクローリン展開すると} \( x_{ss}(1 + \hat{x}_t + \frac{1}{2}\hat{x}_t^2 + \dots) \) となり、\textbf{2 次以上の項が含まれる}からである。入力である \( \hat{x}_t \) 自体が 2 次の情報( 上例の \( B \) )を欠落させているにも関わらず、計算式だけで 2 次以上の項を生成するのは、上述の不整合を引き起こす。

\paragraph{② 線形復元した変数を非線形な効用関数に代入してはならない理由}
では、 2 次以上の項が出ないように \( x_t = x_{ss}(1 + \hat{x}_t) \) と線形で復元すればよいかというと、それも不十分である。
もし、そうして復元した \( x_t \) ( すなわち \( c_t, l_t \) )を、以下のような対数や二乗を含む非線形な効用関数
\[
U(c_t^{H\to W}, l_t^H) = \ln c_t^{H\to W} - \frac{\phi^H}{2}(l_t^H)^2
\]
に直接代入してしまえば、結局は効用関数側が 2 次以上の項( \( A^2 \) など )を生成することになり、入力情報の欠落( \( B \) の無視 )との不整合が生じるからである。

\paragraph{③ 正しい対処法:効用関数自体の線形近似}
整合性を保つための正しい手順は、変数を代入する前に、まず効用関数 \( U \) 自体を定常状態近傍で 1 次近似( 線形化 )し、その式に出てくる \( \hat{c}_t, \hat{l}_t \) に、シミュレーションで得られた 1 次近似値 \( \hat{c}_t, \hat{l}_t \) を代入することである。
このように関数を線形化しておけば、「 線形 」対「 線形 」の対応となり、整合性が保たれる。

\paragraph{④ 価格分散コスト \( \Delta \) の手動補正}
ただし、この線形化の手順をとると、本来は 2 次以上の項として評価されるべき重要な要素、すなわち「 価格分散コスト \( \Delta_t \) 」が式から消滅してしまう( 1 次近似の世界では \( \Delta_t \approx 0 \) となるため )。
そこで本稿では、この消えてしまったコスト \( \Delta_t \) を、 \textcite{Woodford2003} にならい別途手計算で導出し、線形近似された厚生の値から事後的に差し引くという補正を行う。これにより、シミュレーションの整合性を保つとともに、価格のばらつきによる経済的損失を正確に評価に反映させることが可能となる。

次節より、その具体的な導出過程を示す。

\section{価格分散コストの補正式の導出}
\label{sec:appendix_welfare_correction_derivation}

補正の根拠は、線形近似モデルにおいてゼロとなる価格分散の影響を、効用関数の計算に復元することにある。以下にその導出過程を示す。

\paragraph{1. 効用関数の線形近似}

まず、前節で示した期間効用関数 \( U(c_t^{H \to W}, l_t^H) \) を再掲する。
\begin{equation}
    U(c_t^{H \to W}, l_t^H) = \ln c_t^{H \to W} - \frac{\phi^H}{2}(l_t^H)^2
\end{equation}
この関数の値が、定常状態 \( (c_{ss}^{H \to W}, l_{ss}^H) \) からどの程度乖離するかを、 1 次テイラー展開( 線形近似 )を用いて評価する。
多変数関数の近似公式を適用すると、効用の変動部分は以下のように記述できる。
\begin{align}
    U(c_t^{H \to W}, l_t^H) - U(c_{ss}^{H \to W}, l_{ss}^H) &\approx \frac{\partial U}{\partial c_t^{H \to W}}(c_{ss}^{H \to W}, l_{ss}^H) \cdot (c_t^{H \to W} - c_{ss}^{H \to W}) \nonumber \\
    &\quad + \frac{\partial U}{\partial l_t^H}(c_{ss}^{H \to W}, l_{ss}^H) \cdot (l_t^H - l_{ss}^H)
\end{align}
ここで、定常状態における各偏微分係数は以下の通りである。
\begin{align}
    \frac{\partial U}{\partial c_t^{H \to W}}(c_{ss}^{H \to W}, l_{ss}^H) &= \frac{1}{c_{ss}^{H \to W}} \\
    \frac{\partial U}{\partial l_t^H}(c_{ss}^{H \to W}, l_{ss}^H) &= -\phi^H l_{ss}^H
\end{align}
また、ここでの変数 \( \hat{x}_t \) を定常状態からの乖離 \( \hat{x}_t = \frac{x_t - x_{ss}}{x_{ss}} \) と定義すると、レベル変数の乖離は \( x_t - x_{ss} = x_{ss} \hat{x}_t \) と変換できる。
これらを上式に代入して整理すると、以下の線形近似された効用の変動式が得られる。
\begin{align}
    U(c_t^{H \to W}, l_t^H) - U(c_{ss}^{H \to W}, l_{ss}^H) &\approx \frac{1}{c_{ss}^{H \to W}} \cdot (c_{ss}^{H \to W} \hat{c}_t^{H \to W}) + (-\phi^H l_{ss}^H) \cdot (l_{ss}^H \hat{l}_t^H) \nonumber \\
    &= \hat{c}_t^{H \to W} - \phi^H (l_{ss}^H)^2 \hat{l}_t^H \label{eq:appendix_welfare_correction_utility_base}
\end{align}

\paragraph{2. 労働投入量の近似と補正項の導出}

次に、労働投入量 \( \hat{l}_t^H \) を生産関数から導出する。
第 3 章で示した集計生産関数 \( y_t^H = a_t^H l_t^H / \Delta_t^H \) を、まず労働投入量 \( l_t^H \) について解く。
\begin{equation}
    l_t^H = \frac{y_t^H \Delta_t^H}{a_t^H}
\end{equation}
この式の両辺について定常状態からの乖離( ハット変数 )をとると、以下の関係が得られる。
\begin{equation}
    \hat{l}_t^H = \hat{y}_t^H - \hat{a}_t^H + \hat{\Delta}_t^H \label{eq:appendix_welfare_correction_labor_approx}
\end{equation}

この式 \eqref{eq:appendix_welfare_correction_labor_approx} を、そのまま式 \eqref{eq:appendix_welfare_correction_utility_base} に代入すると、以下の式が得られる。
\begin{align}
    U(c_t^{H \to W}, l_t^H) - U(c_{ss}^{H \to W}, l_{ss}^H) &\approx \hat{c}_t^{H \to W} - \phi^H (l_{ss}^H)^2 (\hat{y}_t^H - \hat{a}_t^H + \hat{\Delta}_t^H) \nonumber \\
    &= \hat{c}_t^{H \to W} - \phi^H (l_{ss}^H)^2 (\hat{y}_t^H - \hat{a}_t^H) - \phi^H (l_{ss}^H)^2 \hat{\Delta}_t^H \label{eq:appendix_welfare_correction_utility_combined}
\end{align}

シミュレーションにおいては、この第 3 項 \( -\phi^H (l_{ss}^H)^2 \hat{\Delta}_t^H \) が重要な役割を果たす。標準的な線形近似モデルではこの項が欠落してしまうため、本分析では特別に 2 次近似を用いて \( \hat{\Delta}_t^H \) を導出し、手動で厚生計算に反映させる。

以下に、その導出過程を記述する。

\paragraph{価格分散の関数の定義と近似方針}

まず、第 3 章で導出した\textbf{物価指数の動学}( 式 \ref{eq:final_price_index_dynamics_H} )と\textbf{価格分散の動学}( 式 \ref{eq:final_dispersion_dynamics_H} )を出発点とする。ここで議論を簡潔にするため人口を \( N=1 \) とする。
\begin{equation}
    1 = (1-\xi^H) \left( \frac{\widetilde{p}_t^H}{p_t^H} \right)^{1-\theta^H} + \xi^H (\pi_t^H)^{\theta^H-1} \label{eq:appendix_welfare_correction_pi_dynamics}
\end{equation}
\begin{equation}
    \Delta_t^H = (1-\xi^H) \left( \frac{\widetilde{p}_t^H}{p_t^H} \right)^{-\theta^H} + \xi^H (\pi_t^H)^{\theta^H} \Delta_{t-1}^H \label{eq:appendix_welfare_correction_delta_dynamics}
\end{equation}
式 \eqref{eq:appendix_welfare_correction_pi_dynamics} を相対価格 \( \widetilde{p}_t^H / p_t^H \) について解き、それを式 \eqref{eq:appendix_welfare_correction_delta_dynamics} に代入して相対価格を消去すると、価格分散 \( \Delta_t^H \) はインフレ率 \( \pi_t^H \) と前期の分散 \( \Delta_{t-1}^H \) のみの関数として、以下のように定義できる。
\begin{equation}
    \Delta_t^H = (1-\xi^H)^{\frac{1}{1-\theta^H}} \left( 1 - \xi^H (\pi_t^H)^{\theta^H-1} \right)^{\frac{\theta^H}{\theta^H-1}} + \xi^H (\pi_t^H)^{\theta^H} \Delta_{t-1}^H \label{eq:appendix_welfare_correction_delta_combined}
\end{equation}

この関数を、定常状態 \( (\pi_{ss}^H, \Delta_{ss}^H) = (1, 1) \) の周りで 2 次テイラー展開( 近似 )する。
一般に、 2 変数関数 \( f(x, y) \) の点 \( (x_0, y_0) \) 周りでの 2 次近似公式は以下のように記述される。
\begin{align}
    f(x, y) &\approx f(x_0, y_0) + \frac{\partial f}{\partial x}(x_0, y_0)(x - x_0) + \frac{\partial f}{\partial y}(x_0, y_0)(y - y_0) \nonumber \\
    &\quad + \frac{1}{2} \left[ \frac{\partial^2 f}{\partial x^2}(x_0, y_0)(x - x_0)^2 + 2\frac{\partial^2 f}{\partial x \partial y}(x_0, y_0)(x - x_0)(y - y_0) + \frac{\partial^2 f}{\partial y^2}(x_0, y_0)(y - y_0)^2 \right]
\end{align}
この公式を本モデルに適用する。ここで、インフレ率と価格分散の定常状態値は \( \pi_{ss}^H = 1, \Delta_{ss}^H = 1 \) であるため、レベル変数の乖離 \( x_t - 1 \) は、対数乖離 \( \hat{x}_t \) と以下のように一致する。
\begin{equation}
    x_t - 1 = \frac{x_t - 1}{1} = \frac{x_t - x_{ss}}{x_{ss}} = \hat{x}_t
\end{equation}
また、変数 \( \pi_t^H \) と \( \Delta_{t-1}^H \) は第 2 項において分離した形( 積の形 )で入っており、第 2 項は \( \Delta_{t-1}^H \) について 1 次式であるため、 \( \Delta_{t-1}^H \) に関する 2 階微分はゼロとなる。また交差項 \( \hat{\pi}_t^H \hat{\Delta}_{t-1}^H \) は 3 次の微小量となるため無視できる。

したがって、求める近似式は以下の形となる。
\begin{equation}
    \hat{\Delta}_t^H \approx \frac{\partial \Delta_t^H}{\partial \Delta_{t-1}^H}(1, 1) \hat{\Delta}_{t-1}^H + \frac{\partial \Delta_t^H}{\partial \pi_t^H}(1, 1) \hat{\pi}_t^H + \frac{1}{2} \frac{\partial^2 \Delta_t^H}{\partial (\pi_t^H)^2}(1, 1) (\hat{\pi}_t^H)^2
\end{equation}

\paragraph{偏微分係数の導出}

まず、式 \eqref{eq:appendix_welfare_correction_delta_combined} に基づき、全ての偏導関数を計算する。

\textbf{1. \( \Delta_{t-1}^H \) に関する 1 階偏微分} \\
第 2 項のみを微分する。
\begin{equation}
    \frac{\partial \Delta_t^H}{\partial \Delta_{t-1}^H} = \xi^H (\pi_t^H)^{\theta^H}
\end{equation}

\textbf{2. \( \pi_t^H \) に関する 1 階偏微分} \\
合成関数の微分公式を用いて計算する。
\begin{align}
    \frac{\partial \Delta_t^H}{\partial \pi_t^H} &= (1-\xi^H)^{\frac{1}{1-\theta^H}} \frac{\theta^H}{\theta^H-1} \left( 1 - \xi^H (\pi_t^H)^{\theta^H-1} \right)^{\frac{\theta^H}{\theta^H-1}-1} \left( -\xi^H (\theta^H-1) (\pi_t^H)^{\theta^H-2} \right) \nonumber \\
    &\quad + \xi^H \theta^H (\pi_t^H)^{\theta^H-1} \Delta_{t-1}^H \nonumber \\
    &= -\theta^H \xi^H (1-\xi^H)^{\frac{1}{1-\theta^H}} \left( 1 - \xi^H (\pi_t^H)^{\theta^H-1} \right)^{\frac{1}{\theta^H-1}} (\pi_t^H)^{\theta^H-2} \nonumber \\
    &\quad + \xi^H \theta^H (\pi_t^H)^{\theta^H-1} \Delta_{t-1}^H
\end{align}

\textbf{3. \( \pi_t^H \) に関する 2 階偏微分} \\
上記の 1 階偏微分をさらにもう一度 \( \pi_t^H \) で微分する。第 1 項には積の微分公式を適用する。
\begin{align}
    \frac{\partial^2 \Delta_t^H}{\partial (\pi_t^H)^2} &= -\theta^H \xi^H (1-\xi^H)^{\frac{1}{1-\theta^H}} \Bigg[ \frac{1}{\theta^H-1} \left( 1 - \xi^H (\pi_t^H)^{\theta^H-1} \right)^{ \frac{1}{\theta^H-1}-1} (-\xi^H (\theta^H-1) (\pi_t^H)^{\theta^H-2}) \cdot (\pi_t^H)^{\theta^H-2} \nonumber \\
    &\quad + \left( 1 - \xi^H (\pi_t^H)^{\theta^H-1} \right)^{\frac{1}{\theta^H-1}} \cdot (\theta^H-2)(\pi_t^H)^{\theta^H-3} \Bigg] \nonumber \\
    &\quad + \xi^H \theta^H (\theta^H-1) (\pi_t^H)^{\theta^H-2} \Delta_{t-1}^H
\end{align}

次に、計算したこれらの偏導関数を定常状態 \( (\pi_{ss}^H, \Delta_{ss}^H) = (1, 1) \) で評価する。

\textbf{1. ラグ項の係数の評価}
\begin{equation}
    \frac{\partial \Delta_t^H}{\partial \Delta_{t-1}^H}(1, 1) = \xi^H (1)^{\theta^H} = \xi^H
\end{equation}

\textbf{2. 1 次の係数の評価}
\begin{align}
    \frac{\partial \Delta_t^H}{\partial \pi_t^H}(1, 1) &= -\theta^H \xi^H (1-\xi^H)^{\frac{1}{1-\theta^H}} \left( 1 - \xi^H \right)^{\frac{1}{\theta^H-1}} (1) + \xi^H \theta^H (1) (1) \nonumber \\
    &= -\theta^H \xi^H (1-\xi^H)^{\frac{1}{1-\theta^H} + \frac{1}{\theta^H-1}} + \xi^H \theta^H \nonumber \\
    &= -\theta^H \xi^H (1-\xi^H)^0 + \xi^H \theta^H \nonumber \\
    &= -\theta^H \xi^H + \theta^H \xi^H = 0
\end{align}

\textbf{3. 2 次の係数の評価}
\begin{align}
    \frac{\partial^2 \Delta_t^H}{\partial (\pi_t^H)^2}(1, 1) &= -\theta^H \xi^H (1-\xi^H)^{\frac{1}{1-\theta^H}} \Bigg[ -\xi^H (1-\xi^H)^{\frac{1}{\theta^H-1}-1} (1) + (1-\xi^H)^{\frac{1}{\theta^H-1}} (\theta^H-2) \Bigg] \nonumber \\
    &\quad + \xi^H \theta^H (\theta^H-1) (1) (1) \nonumber \\
    &= \theta^H (\xi^H)^2 (1-\xi^H)^{-1} - \theta^H \xi^H (\theta^H-2) (1) + \xi^H \theta^H (\theta^H-1) \nonumber \\
    &= \frac{\theta^H (\xi^H)^2}{1-\xi^H} + \theta^H \xi^H \left[ -(\theta^H-2) + (\theta^H-1) \right] \nonumber \\
    &= \frac{\theta^H (\xi^H)^2}{1-\xi^H} + \theta^H \xi^H \nonumber \\
    &= \theta^H \xi^H \left( \frac{\xi^H}{1-\xi^H} + 1 \right) = \frac{\theta^H \xi^H}{1-\xi^H}
\end{align}

\paragraph{結論:価格分散の 2 次近似式と線形モデルでの含意}

以上の計算結果を近似式に代入することで、以下の最終的な関係式が得られる。
\begin{align}
    \hat{\Delta}_t^H &\approx \xi^H \hat{\Delta}_{t-1}^H + 0 \cdot \hat{\pi}_t^H + \frac{1}{2} \left( \frac{\theta^H \xi^H}{1-\xi^H} \right) (\hat{\pi}_t^H)^2 \nonumber \\
    &= \xi^H \hat{\Delta}_{t-1}^H + \frac{\theta^H \xi^H}{2(1-\xi^H)} (\hat{\pi}_t^H)^2
    \label{eq:appendix_welfare_correction_delta_approx_final}
\end{align}
これが最終的な価格分散の 2 次近似式である。

この結果( 式 \ref{eq:appendix_welfare_correction_delta_approx_final} )は、線形近似モデルにおいて価格分散項がゼロとなることの数学的な証明となっている。具体的には、上式の導出過程で確認した通り、インフレ率 \( \hat{\pi}_t^H \) に関する 1 次の係数( 偏微分係数 )は計算の結果ちょうど \( 0 \) となる。したがって、もし方程式系全体を 1 次のオーダーで近似( 線形化 )した場合、 2 次以上の微小量である右辺第 2 項( \( (\hat{\pi}_t^H)^2 \) を含む項 )は無視され、式は以下のように退化する。
\[
    \hat{\Delta}_t^H \approx \xi^H \hat{\Delta}_{t-1}^H
\]
定常状態から出発する場合、初期値は \( \hat{\Delta}_0^H = 0 \) であるため、この再帰式に従えば、将来にわたり常に \( \hat{\Delta}_t^H = 0 \) となる。これが、標準的な線形近似シミュレーションにおいて価格分散による厚生ロスが消失してしまう理由である。

\paragraph{厚生計算に用いる最終的な効用関数}

最後に、導出された価格分散の近似式 \eqref{eq:appendix_welfare_correction_delta_approx_final} を、先述の効用の近似式 \eqref{eq:appendix_welfare_correction_utility_combined} に代入することで、本分析のプログラムにおいて実際に計算される最終的な効用関数の式が得られる。

\begin{equation}
    U(c_t^{H \to W}, l_t^H) - U(c_{ss}^{H \to W}, l_{ss}^H) \approx \hat{c}_t^{H \to W} - \phi^H (l_{ss}^H)^2 (\hat{y}_t^H - \hat{a}_t^H) - \phi^H (l_{ss}^H)^2 \left( \xi^H \hat{\Delta}_{t-1}^H + \frac{\theta^H \xi^H}{2(1-\xi^H)} (\hat{\pi}_t^H)^2 \right)
    \label{eq:appendix_welfare_correction_final_utility}
\end{equation}     % 厚生補正( 価格分散の動学 )
\input{sections/app/appendix_insulation_effect.tex}      % 遮断効果の数学的証明

\end{document}